\title{Sheaf algorithms using the exterior algebra}
\titlerunning{Sheaf algorithms}
\toctitle{Sheaf algorithms using the exterior algebra}
\author{Wolfram Decker
        \inst 1
   \and David Eisenbud
        \inst 2
        }
\authorrunning{Decker and Eisenbud}
\institute{FB Mathematik, Universit\"at des Saarlandes, 66041 Saarbr\"ucken, Germany
           \and
           1000 Centennial Drive, Mathematical Sciences Research Institute, Berkeley, CA 94720, USA}

%%%%%%%%%%%%% EISENBUDS %%%%%%%%%%%%%%%%%%
%\input begin.tex
%\showlabels
%\showlabelsabove
% \input diagrams.tex
%\vsize=13.6truecm
%\hsize=19truecm
% \overfullrule=0pt
% Gothic fonts from AMSTeX 
\font\tengoth=eufm10  \font\fivegoth=eufm5
\font\sevengoth=eufm7
\newfam\gothfam  \scriptscriptfont\gothfam=\fivegoth 
\textfont\gothfam=\tengoth \scriptfont\gothfam=\sevengoth
\def\goth{\fam\gothfam\tengoth}
%
% Bold italic fonts 
\font\tenbi=cmmib10  \font\fivebi=cmmib5
\font\sevenbi=cmmib7
\newfam\bifam  \scriptscriptfont\bifam=\fivebi 
\textfont\bifam=\tenbi \scriptfont\bifam=\sevenbi
\def\bi{\fam\bifam\tenbi}
%
\font\hd=cmbx10 scaled\magstep1
%%%%%%%%%%%%%%%%%%%%%% EISENBUD ENDE %%%%%%%%%%%%%%%%%%

%\usepackage{amsmath,amscd,amsthm,amssymb,amsxtra,latexsym,epsfig,epic,eepic,graphics}

% \usepackage{amsmath,amscd,amsthm,amssymb,amsxtra,latexsym,epsfig,epic,graphics}

% \usepackage[matrix,arrow,curve]{xy}
%

\def \fix#1 {{\par \bf (( #1 ))\par}}
\def \Box {\hfill\hbox{}\nobreak \vrule width 1.6mm height 1.6mm
depth 0mm  \par \goodbreak \smallskip}
\def \reg {\mathop{\rm regularity}}
\def \coker {\mathop{\rm coker}}
\def \ker {\mathop{\rm ker}}
\def \im {\mathop{\rm im}}
\def \deg  {\mathop{\rm deg}}
\def \depth {\mathop{\rm depth}}
\def \span {\mathop{\rm span}}
\def \socle {\mathop{\rm socle}}
\def \dim{{\rm dim}}
\def \codim{{\rm codim}}
\def \Im  {\mathop{\rm Im}}
\def \ann  {\mathop{\rm ann}}
\def \rank {\mathop{\rm rank}}
\def \sing {\mathop{\rm Sing}}
\def \iso {\cong}
\def \tensor {\otimes}
\def \dsum {\oplus}
\def \intersect {\cap}
\def \Hom {{\mathop{\rm Hom}\nolimits}}
\def \hom {{\mathop{\rm Hom}\nolimits}}
\def \Ext {{\rm Ext}}
\def \ext{{\rm Ext}}
\def \Tor {{\rm Tor}}
\def \tor{{\rm Tor}}
\def \Sym {{\mathop{\rm Sym}\nolimits}}
\def \End {{\mathop{\rm End}\nolimits}}
\def \sym{{\rm Sym}}
\def \GL{{\rm GL}}
\def \Proj{{\rm Proj}}
\def \h {{\rm h}}

\def \coh{{\rm Coh}}
%\def \BGG{{\rm BGG}}
\def \lin{{\rm lin}}

\def \th {{^{\rm th}}}
\def \st {{^{\rm st}}}


\def \A {{\cal A}}
\def \AA {{\bf A}}
\def \B {{\cal B}}
\def \C {{\cal C}}
\def \CC {{\bf C}}
\def \DD  {{\bf D}}
\def \E  {{\cal E}}
\def \F {{\cal F}}
\def \FF {{\bf F}}
\def \G {{\cal G}}
\def \GG {{\bf G}}
\def \K {{\cal K}}
\def \H {{\rm H}}
\def \KK {{\bf K}}
\def \L {{\cal L}}
\def \LL {{\bf L}}
\def \MM{{\bf M}}
\def \N {{\cal N}}
\def \O {{\cal O}}
\def \P {{\bf P}}
\def \PP {{\bf P}}
\def \RR {{\bf R}}
\def \TT {{\bf T}}
\def \Z {{\bf Z}}

\def \gm {{\goth m}}



%
%%%%%%%%%%%%%%%%%%%%%%%%%%%%
%%%The black board font
%%%%%%%%%%%%%%%%%%%%%%%%%%%
%\newcommand{\A}{{\mathbb A}}
%\newcommand{\B}{{\mathbb B}}
%\newcommand{\C}{{\mathbb C}}
%\newcommand{\D}{{\mathbb D}}
%\newcommand{\E}{{\mathbb E}}
%\newcommand{\F}{{\mathbb F}}
%\newcommand{\G}{{\mathbb G}}
%\newcommand{\H}{{\mathbb H}}
%\newcommand{\I}{{\mathbb I}}
%\newcommand{\J}{{\mathbb J}}
%\newcommand{\K}{{\mathbb K}}
%\newcommand{\L}{{\mathbb L}}
%\newcommand{\M}{{\mathbb M}}
%\newcommand{\N}{{\mathbb N}}
%\newcommand{{\mathbb O}}
%\newcommand{{\mathbb P}}
\newcommand{\QQ}{{\mathbb Q}}
%\newcommand{\R}{{\mathbb R}}
%\newcommand�{{\mathbb S}}
%\newcommand{\T}{{\mathbb T}}
%\newcommand{\U}{{\mathbb U}}
%\newcommand{\V}{{\mathbb V}}
%\newcommand{\W}{{\mathbb W}}
%\newcommand{\X}{{\mathbb X}}
%\newcommand{\Y}{{\mathbb Y}}
%\newcommand{\Z}{{\mathbb Z}}

%\newcommand{\LL}{{\mathbb L}}
%\newcommand{\TT}{{\mathbb T}}
%\newcommand{\HH}{{\mathbb H}}

\def\pp#1{{\text{$\text{I\!P}_#1$}}}
%\newcommand{\PP}{{\mathbb P}}

%\DeclareMathOperator{\gin}{gin}

%\newcommand{\Syz}{{\rm{Syz}\;}}
%\newcommand{\sym}{{\rm{Sym}}}
%\newcommand{\SSyz}{{\rm{Syz}}}
%\newcommand{\spoly}{{\rm{spoly}}}
%\newcommand{\Spe}{{\rm{Sp}}}
%\newcommand{\openC}{{\mathbb C}}
%\newcommand{\ms}{{\rm{m}}}
\newcommand{\SL}{{\rm{SL}}}
%\newcommand{\IS}{{\rm{I}}}
%\newcommand{\Loc}{{\rm{Loc}\,}}
%\newcommand{\lcm}{{\rm{lcm}}}
%\newcommand{\lc}{{\rm{lc}}}
%\newcommand{\lm}{{\rm{lm}}}
%\newcommand{\con}{{\rm{c}}}
%\newcommand{\ext}{{\rm{e}}}
%\newcommand{\ec}{{\rm{ec}}}
%\newcommand{\ann}{{\rm{ann}}}
%\newcommand{\Ext}{{\rm{Ext}}}
%\newcommand{\equi}{{\rm{equi}}}
%\newcommand{\rk}{{\rm{rk}}}

%%%%%%%%%%%%%%%%%%%%%%%%%%%%%
%%% new commands for calligraphic characters with amsmath
%%%%%%%%%%%%%%%%%%%%%%%%%%%%

\newcommand{\ka}{{\mathcal A}}
\newcommand{\kb}{{\mathcal B}}
\newcommand{\kc}{{\mathcal C}}
\newcommand{\kd}{{\mathcal D}}
\newcommand{\ke}{{\mathcal E}}
\newcommand{\kf}{{\mathcal F}}
\newcommand{\kg}{{\mathcal G}}
\newcommand{\kh}{{\mathcal H}}
\newcommand{\ki}{{\mathcal I}}
\newcommand{\kj}{{\mathcal J}}
\newcommand{\kk}{{\mathcal K}}
\newcommand{\kl}{{\mathcal L}}
\newcommand{\km}{{\mathcal M}}
\newcommand{\kn}{{\mathcal N}}
\newcommand{\ko}{{\mathcal O}}
\newcommand{\kp}{{\mathcal P}}
\newcommand{\kq}{{\mathcal Q}}
\newcommand{\kr}{{\mathcal R}}
\newcommand{\ks}{{\mathcal S}}
\newcommand{\kt}{{\mathcal T}}
\newcommand{\ku}{{\mathcal U}}
\newcommand{\kv}{{\mathcal V}}
\newcommand{\kw}{{\mathcal W}}
\newcommand{\kx}{{\mathcal X}}
\newcommand{\ky}{{\mathcal Y}}
\newcommand{\kz}{{\mathcal Z}}
%%%%%%%%%%%%%%%%%%%%%%%%%%%%%%
%%%The mathscript for sheaves
%%%%%%%%%%%%%%%%%%%%% %%%%%%%%%
\newcommand{\s}{\mathscr}
\newcommand{\sA}{{\s A}}
\newcommand{\sB}{{\s B}}
\newcommand{\sC}{{\s C}}
\newcommand{\sD}{{\s D}}
\newcommand{\sE}{{\s E}}
\newcommand{\sF}{{\s F}}
\newcommand{\sG}{{\s G}}
\newcommand{\sH}{{\s H}}
\newcommand{\sI}{{\s I}}
\newcommand{\sJ}{{\s J}}
\newcommand{\sK}{{\s K}}
\newcommand{\sL}{{\s L}}
\newcommand{\sM}{{\s M}}
\newcommand{\sN}{{\s N}}
\newcommand{\sO}{{\s O}}
\newcommand{\sP}{{\s P}}
\newcommand{\sQ}{{\s Q}}
\newcommand{\sR}{{\s R}}
\newcommand{\sS}{{\s S}}
\newcommand{\sT}{{\s T}}
\newcommand{\sU}{{\s U}}
\newcommand{\sV}{{\s V}}
\newcommand{\sW}{{\s W}}
\newcommand{\sX}{{\s X}}
\newcommand{\sY}{{\s Y}}
\newcommand{\sZ}{{\s Z}}

\newcommand{\cO}{{\s O}}
\newcommand{\cI}{{\s I}}
% \newcommand{\cL}{{\s L}} %% \cL is used in \xy
% \newcommand{\cR}{{\s R}} %% \cR is used in \xy
\newcommand{\cN}{{\s N}}
\newcommand{\cT}{{\s T}}
\newcommand{\cX}{{\s X}}
%%%%%%%%%%%%%%%%%%%%%%%%%%%%%%%%
%% Arrows
%%%%%%%%%%%%%%%%%%%%%%%%%%%%%%%
\newcommand{\inj}{\hookrightarrow}
\newcommand{\surj}{\lra \lra}
\newcommand{\lra}{\longrightarrow}
\newcommand{\lla}{\longleftarrow}
%%%%%%%%%%%%%%%%%%%%%%%%%%%%%%%%%%%%
%\newcommand{\C}{\C}
\newcommand{\openP}�
\newcommand{\uf}{{\bf F}}
\newcommand{\uc}{{\bf C}}
%\newcommand{\tensor}{\otimes}
\newcommand{\mi}{{\bf m}}
\newcommand{\tX}{\widetilde{X}}
\newcommand{\punkt}{\hspace{-.3ex}\raise.15ex\hbox to1ex{\Huge.}}
\newcommand{\tpunkt}{\hspace{-.3ex}\hbox to1ex{\Huge.}}
\newlength{\br}
\newlength{\ho}
%\DeclareMathOperator{\GL}{GL}
%\DeclareMathOperator{\Aut}{Aut}
%\DeclareMathOperator{\Oo}{O}
%\DeclareMathOperator{\Spec}{Spec}
%\DeclareMathOperator{\Hom}{Hom}
%\DeclareMathOperator{\Tor}{Tor}
%\DeclareMathOperator{\syz}{syz}
%\DeclareMathOperator{\ord}{ord}
%\DeclareMathOperator{\word}{w\,ord}
%\DeclareMathOperator{\supp}{supp}
%\DeclareMathOperator{\Ker}{Ker}
%\DeclareMathOperator{\im}{im}
%\DeclareMathOperator{\wdeg}{w\,deg}
%\DeclareMathOperator{\depth}{depth}
%\DeclareMathOperator{\Coker}{Coker}
%\DeclareMathOperator{\NF}{NF}
%\DeclareMathOperator{\pd}{pd}
%\DeclareMathOperator{\SL}{SL}
%\DeclareMathOperator{\SO}{SO}
%\DeclareMathOperator{\Ort}{O}
%\DeclareMathOperator{\Spez}{Sp}
%\DeclareMathOperator{\PSL}{PSL}
%\DeclareMathOperator{\wdim}{wdim}
%\DeclareMathOperator{\cdim}{cdim}
%\DeclareMathOperator{\cha}{char}
%\DeclareMathOperator{\trdeg}{trdeg}
%\DeclareMathOperator{\codim}{codim}
%\DeclareMathOperator{\kdim}{kdim}
%\DeclareMathOperator{\height}{height}
%\DeclareMathOperator{\Ass}{Ass}
%\DeclareMathOperator{\Lie}{Lie}
\renewcommand{\labelenumi}{(\arabic{enumi})}
\newcommand{\Ndash}{\nobreakdash--}% for pages 1\Ndash 9
\newcommand{\somespace}{\hfill{}\\ \vspace{-0.7cm}}

%%%theosdefinitionen
%newcommand{\gm}{\mathfrak m}
%newcommand{\integer}{\Z}
%newcommand{\proj}
%newcommand{\complex}{\C}
%newcommand{\real}{\mathbb R}
%newcommand{\gp}{\mathfrak p}
%newcommand{\gq}{\mathfrak q}
%newcommand{\scr}{\cal}
%newcommand{\openF}{\F}
%\newcommand{\CC}{\mathbb C}
%newcommand{\ZZ}{\Z}
%newcommand{\QQ}{\Q}
%newcommand{\FF}{\F}

%%%%%%%%%%%%%%%BIBLIOGRAPHY
%newcommand{\by}{}
%newcommand{\paper}{: \begin{it}}
%newcommand{\jour }{, \end{it}}
%newcommand{\vol}{}
%newcommand{\pages}{}
%newcommand{\yr}{}
%\newcommand{\endref}{}
% Local Variables:
% mode: latex
% TeX-master: "tot"
% End:
%%%%%%%%%%%%%%%%%%%%%%%%%%%%%%%%%%%%%%%%%
%%     HOLGERs S�tze
%%%%%%%%%%%%%%%%%%%%%%%%%%%%%%%%%%%%%%%%%


\maketitle

\begin{abstract}
In this chapter we explain constructive methods for computing
the cohomology of a sheaf on a projective variety. We also
give a construction for the Beilinson monad, a tool for
studying the sheaf from partial knowledge of its cohomology.
Finally, we give some examples illustrating the use of the Beilinson
monad. 
\end{abstract}

\section{Introduction}

%\textbf{EISENBUD}

In this chapter $V$ denotes a vector space of finite dimension $n+1$ over
a field $K$ with dual space $W=V^*$, and $S=\sym_K(W)$ is the symmetric
algebra of $W$, isomorphic to the polynomial ring on a basis for $W$.
We write $E$ for the \ie{exterior algebra} on $V$. We grade
$S$ and $E$ by taking elements of $W$ to have
degree 1, and elements of $V$ to have degree $-1$.
We denote the projective space of 1-quotients of $W$
(or of lines in $V$) by $\P^n = \P(W)$.

Serre's sheafification functor $M\mapsto \tilde M$ 
allows one to consider a coherent sheaf on $\P(W)$ as an
equivalence class of 
finitely generated graded $S$-modules, where we identify two such modules
$M$ and $M'$ if, for some $r$, the truncated modules $M_{\geq r}$ and
$M'_{\geq r}$ are isomorphic. A free resolution of $M$,
sheafified, becomes a resolution of $\tilde M$ by sheaves that
are direct sums of line bundles on $\P(W)$ -- that is, a
description of $\tilde M$ in terms of homogeneous matrices
over $S$. Being able to compute
syzygies over
$S$ one can compute the cohomology of $\tilde M$ 
starting from the minimal free resolution of $M$ (see 
\cite{EA:MR1484973:eisenbud} in \cite{EA:VAS} for a sketch of an algorithm,
or \cite{EA:MR1769664} for a more systematic treatment of a generalization.)

The \ie{Bernstein-Gel'fand-Gel'fand correspondence} (\ie{BGG}) is an isomorphism
between the derived category of bounded complexes of finitely
ge\-nerated $S$-modules and the derived category of 
bounded complexes of finitely generated $E$-modules
or of certain ``\ie{Tate resolution}s'' of $E$-modules.
In this chapter we show how to effectively compute
the Tate resolution $\TT(\mathcal F)$
associated to a sheaf $\mathcal F$, and we use this construction
to give relatively cheap computations of the cohomology of $\mathcal F$.

It turns out that by applying a simple functor to the Tate resolution $\TT(\mathcal F)$
one gets a finite complex of sheaves  whose homology is the sheaf $\mathcal F$
itself. This complex is called a
{\it Beilinson monad\/}
\index{Beilinson monad} \index{monad!Beilinson}
for $\mathcal F$. The Beilinson monad
provides a powerful method for getting information about a sheaf from 
partial knowledge of its cohomology.
It is a representation of the sheaf in terms  of direct sums of
(suitably twisted) bundles of differentials and homomorphisms  between these
bundles, which are given by homogeneous matrices over $E$. 

The following recipe for computing the cohomology of a sheaf
is typical of our methods: Suppose that
$\F=\tilde M$ is the coherent sheaf on $\P(W)$ associated to a
finitely generated graded $S$-module $M=\oplus M_i$. To
compute the cohomology of $\F$ we consider a
sequence of free $E$-modules and maps
$$
\FF(M):\quad \cdots \rTo F^{i-1}\rTo^{\phi_{i-1}} 
F^i\rTo^{\phi_{i}} 
F^{i+1}\rTo\cdots.
$$
Here we set $F^i=M_i \otimes_K E$ and define $\phi_i:F^i\rTo F^{i+1}$ to be the
map taking $m\otimes 1\in M_{i}\otimes_K E$ to
$$
\sum_j x_jm\otimes e_j\in M_{i+1}\otimes V\subset F^{i+1},
$$
where $\{x_j\}$ and $\{e_j\}$ are dual bases of $W$ and $V$
respectively. It turns out that $\FF(M)$
is a complex; that is, $\phi_i\phi_{i-1}=0$ for
every $i$ (the reader may easily check this by direct
computation; a proof without indices is given in
\cite{EA:eis-sch:sheaf}). %Eisenbud and Schreyer [2000].
If we regard $M_i$ as a vector space concentrated
in degree $i$, so that $F^i$ is a direct sum of copies of $E(-i)$,
then these maps are homogeneous of degree 0.

We shall see that if $s$ is a sufficiently large integer then
the truncation of the Tate resolution
$$
F^s\rTo^{\phi_s} F^{s+1}\rTo\cdots
$$ 
is exact and is thus the minimal injective
resolution of the finitely generated graded $E$-module
$P_s=\ker \phi_{s+1}$. (In fact any value of $s$ greater
than the 
Castelnuovo-Mumford regularity of $M$ will do.) 

Because the number of monomials 
in $E$ in any given degree is small compared to the number of monomials of 
that degree in the symmetric algebra, it is relatively cheap to compute a free
resolution of 
$P_s$ over $E$, and thus to 
compute the graded vector
spaces $\Tor^E_t(P_s,K)$. Our algorithm exploits the fact, proved in
\cite{EA:eis-sch:sheaf}, %Eisenbud and Schreyer [2000],
that the $j^\th$ cohomology $\H^j\F$ of $\F$ 
in the Zariski topology is isomorphic to the degree $-n-1$ part of 
$\Tor^E_{s-j}(P_s,K)$; that is,
$$
\H^j\F\cong\Tor^E_{s-j}(P_s,K)_{-n-1}.
$$
In addition, the linear parts of the matrices
in the complex $\TT(\mathcal F)$ determine the
graded
$S$-modules
$$
\H^j_*\mathcal F := \oplus_{i\in\mathbb Z}\, \H^j  \mathcal F(i)\ .
$$
In many cases this is the fastest known method for computing cohomology.

The Beilinson monad has played an important role in the construction
and study of vector bundles and varieties. In the typical application
one constructs or classifies monads in order to  construct or classify
sheaves. In later sections of this chapter we give examples which
show how this process works. 

One application of the construction of the Beilinson
complex (in a slightly more general setting) is to compute
\ie{Chow form}s of varieties; see
\cite{EA:Eisenbud-Schreyer:ChowForms}. As for other applications,
perhaps the
situation is similar to that in the beginning of the 1980's when it
became clear that syzygies could be computed by a machine. Though
syzygies had been used theoretically for many years it took quite a
while until the practical computation of syzygies lead to applications,
too, mostly through the greatly increased ability to study examples. 

A good project in this direction might be to extend and make
more precise the very useful criterion given in Proposition \ref{critsur}.
Can the reader find a necessary and sufficient condition
to replace the necessary condition for surjectivity given there?
How about a criterion for exactness?

\section{Basics of the Bernstein-Gel'fand-Gel'fand correspondence}

\index{Bernstein-Gel'fand-Gel'fand correspondence}
In this section we describe the basic idea of the BGG correspondence,
introduced in \cite{EA:MR80c:14010a}. %Bernstein, Gel'fand, and Gel'fand [1978]. 
For a more complete treatment along the lines given here, see the first section
of \cite{EA:eis-sch:sheaf}. %Eisenbud and Schreyer [2000] .

As a simple example of the construction given in
%%% Referenz
Section~1, consider the case $M=S=\sym_K(W)$. The associated 
complex, made from the homogeneous components $\sym_i(W)$ of $S$, 
has the form
$$
\FF(S):\quad E\rTo W\otimes E \rTo \Sym_2(W)\otimes E\rTo\cdots,
$$
where we regard $\Sym_iW$ as concentrated in degree $i$.
It is easy to see that the kernel of the first map,
$E\rTo W\otimes E$, is exactly the socle $\bigwedge^{n+1} V\subset E$,
which is a 1-dimensional vector space concentrated in degree $-n-1$.
In fact $\FF(S)$ is the minimal injective resolution of
this vector space.  If we tensor with the dual vector space
$\bigwedge^{n+1} W$ (which is concentrated in degree $n+1$),
we obtain the minimal injective resolution of the
vector space $\bigwedge^{n+1} W\otimes \bigwedge^{n+1} V$,
which may be identified canonically
with the residue field $K$ of $E$.
This resolution is called the {\it Cartan resolution\/}
\index{Cartan resolution}
of $K$.  To write it conveniently, we
set $\omega_E=\bigwedge^{n+1}W\otimes E$. The socle of
$\omega_E$ is $K$. Since $E$ is
injective (as well as projective) as an $E$-module, 
the same goes for $\omega_E$, so $\omega_E$ is the
injective envelope of the residue class field $K$ and
we have $\omega_E=\Hom_K(E,K)$. Thus we
can write the injective resolution of the residue field as
$$
\RR(S):\quad \omega_E\rTo W\otimes \omega_E \rTo 
\Sym_2(W)\otimes \omega_E \rTo\cdots,
$$
or again as
$$
\Hom_K(E,K)\rTo \Hom_K(E,W) \rTo 
\Hom_K(E,\Sym_2(W)) \rTo\cdots.
$$
Taking our cue from this situation, 
our primary object of study in the case of
an arbitrary finitely generated graded $S$-module 
$M = \oplus M_i$ will be the complex
$$
\RR(M):\quad \cdots \rTo M_i\otimes \omega_E
\rTo M_{i+1}\otimes\omega_E\rTo \cdots,
$$ 
which will have a more natural grading than $\FF(M)$;
in any case, it differs from $\FF(M)$
only by tensoring over $K$ with the one-dimensional
$K$-vector space $\bigwedge^{n+1}W$, concentrated in degree $n+1$, and
thus has the same basic properties. 
(Writing $\RR(M)$ in terms of 
$\Hom$ as above suggests that the functor $\RR$
might have a left adjoint, and
indeed there is a left adjoint that produces linear free
complexes over $S$ from graded $E$-modules. $\RR$ and its
left adjoint are used to construct the isomorphisms of 
derived categories in the BGG correspondence; see
\cite{EA:eis-sch:sheaf} %Eisenbud and Schreyer [2000]
for a treatment in this spirit.)

An important fact for us is that the complex $\RR(M)$ is eventually 
exact (and thus 
$$
F^i\rTo^{\phi_i} F^{i+1}\rTo\cdots
$$
is the minimal injective resolution of $\ker \phi_i$ when $i \gg 0$).
It turns out that the point at which exactness sets in is
a well-known invariant, the \ie{Castelnuovo-Mumford regularity} of
$M$, whose definition we briefly recall:

If $M=\oplus M_i$ is a finitely generated graded $S$-module
then for all large integers $r$
the submodule $M_{\geq r}\subset M$ is generated in degree $r$ and has
a {\it linear free resolution\/}; 
\index{linear free resolution}
that is, its first syzygies are
generated in degree $r+1$, its second syzygies in degree $r+2$,
etc. (see \cite[chapter 20]{EA:MR97a:13001}%Eisenbud [1995]
).  The {\it Castelnuovo-Mumford regularity\/} of $M$ is the
least integer $r$ for which this occurs.

\begin{theorem}[\cite{EA:eis-sch:sheaf}]\label{regularity}
% \vskip0.3cm
% \noindent
% \textbf{Theorem 3.1.}
%\theorem{regularity} 
% {\em 
Let $M$ be a finitely generated graded
$S$-module of Cas\-tel\-nuovo-Mumford regularity $r$.
The complex $\RR(M)$ is exact at $\Hom_K(E,M_i)$
for all $i\geq s$ if and only if $s>r$.\qed
%}
%\vskip0.3cm
\end{theorem}

More generally, it is shown in \cite{EA:eis-sch:sheaf} that the components of the 
cohomology of the complex $\RR(M)$ can be identified with the Koszul cohomology
of $M$. An equivalent result was stated in
\cite{EA:MR89g:13005:appendix}. %Buchweitz [1985] .

For instance, it is not hard to show that if 
$M$ is of finite length, then the regularity of $M$
is the largest $i$ such that $M_i\neq 0$. Let us verify Theorem
\ref{regularity} directly in a simple example:

%\begin{Example}
\begin{example}
Let $S=K[x_0,x_1,x_2]$, and let
$M=S/(x_0^2,x_1^2,x_2^2)$. The module
$M_{\geq 3}= K\cdot x_0x_1x_2$ is a trivial $S$-module, and
its resolution is the Koszul complex on $x_0$, $x_1$ and $x_2$, which is linear.
Thus the Castelnuovo-Mumford regularity of $M$
is $\leq 3$. On the other hand $M_{\geq 2}$ is, up to twist,
isomorphic to the dual of $S/(x_0,x_1,x_2)^2$, and it follows that
the resolution of $M_{\geq 2}$ has the form
$$
0\rTo S(-6)\rTo 6\:\! S(-4)\rTo 8\:\! S(-3)\rTo 3\:\! S(-2),
$$
which is not linear, so the Castelnuovo-Mumford regularity of
$M$ is exactly 3. Note that the regularity is larger than the degrees
of the generators and relations of $M$---in general it
can be much larger.

Over $E$ the linear free complex 
corresponding to $M$ has the form
$$
\cdots\to  0\to
M_0\otimes\omega_E
\to 
M_1\otimes\omega_E
\to 
M_2\otimes\omega_E
\to
M_3\otimes\omega_E
\to 0\to
\cdots,
$$
where all the terms not shown are 0. Using the isomorphism
$\omega_E\cong E(-3)$ this
can be written (non-canonically) as
$$
0\rTo E(-3)
\rTo^{ \begin{pmatrix} e_0\\ e_1\\ e_2 \end{pmatrix}} 
3\:\! E(-2)
\rTo^{\begin{pmatrix} 0&e_2&e_1\\ e_2&0&e_0\\ e_1&e_0&0 \end{pmatrix}} 
3\:\! E(-1)
\rTo^{\begin{pmatrix}e_0& e_1& e_2 \end{pmatrix}} 
E\rTo 0.
$$
One checks easily that this complex is inexact at  every non-zero term
(despite its resemblance to a Koszul complex),
verifying Theorem \ref{regularity}.\qed
\end{example}
%\end{example}

Another case in which everything can be checked directly
occurs when $M$ is the homogeneous coordinate ring of a point:


%\begin{Example}
\begin{example} 
Take $M=S/I$ where $I$ is generated
by a codimension 1 space of linear forms in $W$, so that 
$I$ is the homogeneous ideal of a point $p\in \P(W)$.
The free resolution of $M$ is 
the Koszul complex on $n$ linear forms,
so $M$ is 0-regular. As $M_i$
is 1-dimensional for every $i$ the terms of the complex
$\RR(M)$ are all rank 1 free $E$-modules. One
easily checks that $\RR(M)$ takes the form
$$
\RR(M):\quad \omega_E\rTo^a \omega_E(-1)\rTo^a\omega_E(-2)\rTo^a\cdots\ ,
$$
where $a\in V = W^*$ is a linear functional that vanishes on all
the linear forms in $I$; that is, $a$ is a generator of 
the one-dimensional subspace of $V$ corresponding to the point $p$.
As for any linear form in $E$, the annihilator of $a$ 
is generated by $a$, and it follows directly that the complex
$\RR(M)$ is acyclic in this case.\qed
\end{example}
%\end{example}


We present two Macaulay2 functions, {\texttt{symExt}} and 
{\texttt{bgg}}, which compute a differential of the complex
$\RR(M)$ for a finitely generated graded module $M$ defined over 
some polynomial ring $S=K[x_0,\dots,x_n]$ with variables $x_i$ of 
degree 1. Both functions expect as an additional input the name of an 
exterior algebra $E$ with the same number $n+1$ of generators, also 
supposed to be of degree 1 (and NOT -1). This convention, which
makes the cohomology diagrams more naturally looking when printed in 
Macaulay2, necessitates the adjustment of degrees in the second half of 
the programs.

The first of the functions, {\texttt{symExt}}, takes as input a
matrix $m$ with linear entries, which we think of as
a presentation matrix for a positively graded $S$-module  
$M = \oplus _{i\geq 0} M_i$, and 
returns a matrix represen\-ting the map
$M_{0}\otimes\omega_E\to M_{1}\otimes\omega_E$
which is the first differential of the complex $\RR(M)$. 
\vskip0.3cm

<<<symExt = (m,E) ->(
     ev := map(E,ring m,vars E);
     mt := transpose jacobian m;
     jn := gens kernel mt;
     q  := vars(ring m)**id_(target m);
     ans:= transpose ev(q*jn);
     --now correct the degrees:
     map(E^{(rank target ans):1}, E^{(rank source ans):0}, 
         ans));>>>

\vskip0.3cm
\noindent
If $M$ is a module whose presentation is not linear
in the sense above, we can still
apply {\texttt{symExt}} to a high truncation of $M$:
\vskip0.3cm

<<<S=ZZ/32003[x_0..x_2];>>>
<<<E=ZZ/32003[e_0..e_2,SkewCommutative=>true];>>>
<<<M=coker matrix{{x_0^2, x_1^2}};>>>
<<<m=presentation truncate(regularity M,M);>>>
<<<symExt(m,E)>>>

\vskip0.3cm
\noindent
{\texttt{symExt}} is a quick-and-dirty tool which requires little
computation. If it is called on two successive truncations
of a module the maps it produces may NOT compose to zero
because the choice of bases is not consistent.
The second function, {\texttt{bgg}}, makes the computation in such a way
that the bases are consistent, but does more computation
to achieve this end. It takes as input an integer $i$ and 
a finitely generated graded $S$-module  $M$, and 
returns the $i^{\text{th}}$ map in $\RR (M)$,
which is an ``adjoint'' of the multiplication
map between $M_i$ and $M_{i+1}$. 
\vskip0.3cm

<<<bgg = (i,M,E) ->(
     S :=ring(M);
     numvarsE := rank source vars E;
     ev:=map(E,S,vars E);
     f0:=basis(i,M);
     f1:=basis(i+1,M);
     g :=((vars S)**f0)//f1;
     b:=(ev g)*((transpose vars E)**(ev source f0));
     --correct the degrees (which are otherwise
     --wrong in the transpose)
     map(E^{(rank target b):i+1},E^{(rank source b):i}, b));>>>

\vskip0.3cm
\noindent
For instance, in Example 2.2:
\vskip0.3cm

<<<M=cokernel matrix{{x_0^2, x_1^2, x_2^2}};>>>
<<<bgg(1,M,E)>>>

\vskip0.3cm

%\textbf{EISENBUD ENDE}

%{\bf 4. Cohomology and Tate}



%\textbf{EISENBUD}



\section{The cohomology and the Tate resolution of a sheaf}

\index{sheaf cohomology}\index{cohomology!sheaf}
Given a finitely generated graded $S$-module 
$M$ we construct a (doubly infinite)
$E$-free complex $\TT(M)$ with vanishing homology,
called the {\it Tate resolution}
\index{Tate resolution} 
of $M$, as follows:
Let $r$ be the Castelnuovo-Mumford regularity of $M$. The
truncation $\TT^{>r}(M)$, the part of $\TT(M)$ with cohomological
degree $>r$, is $\RR(M_{>r})$. We complete this to an exact
complex by adjoining a minimal projective resolution of the kernel
of $\Hom_K(E,M_{r+1})\to\Hom_K(E,M_{r+2})$. 

If, for example, $M$ has finite length as in Example 2.2, the Tate 
resolution of $M$ is the complex
$$
\cdots\to 0\to 0\to 0 \to\cdots.
$$
At the opposite extreme, take $M=S$, the free module of rank 1.
Since $S$ has regularity 0, it follows that $\RR(S)$ is an injective 
resolution of the residue field $K$ of $E$. Applying the exact functor
$\Hom_K(\hbox{\bf---}, K)$, and using the fact that it carries
$\omega_E=\Hom_K(E,K)$ back to $E$, we see that the 
Tate resolution $\TT(S)$ is the first row of the diagram
$$
\xymatrix{
\cdots \ar[r]& W^*\otimes E \ar[r]& E\ar[dr] \ar[rr]&& \omega_E
\ar[r]& W\otimes\omega_E  \ar[r]&\cdots\\
&&&K\ar[ur]
}
$$      
Another simple example occurs in the case where $M$ is
the homogeneous coordinate ring of a point $p\in\P(W)$. The complex
$\RR(M)$ constructed in Example 2.3 is periodic, so
it may be simply continued to the left, giving
$$
\TT(M):\quad \cdots \rTo^a \omega_E(i)
\rTo^a\omega_E(i-1)\rTo^a\cdots,
$$
where again $a\in V=W^*$ is a non-zero linear functional
vanishing on the linear forms in the ideal of $p$.

For arbitrary $M$, by the results of the previous section,
$\RR(M_{>r})$ has no
homology in cohomological degree $>r+1$, so $\TT(M)$ could be constructed
by a similar recipe from any truncation 
$\RR(M_{>s})$ with $s\geq r$. Thus
the Tate resolution  depends only on the sheaf $\tilde M$ 
on $\P(W)$ corresponding to $M$.
We sometimes write $\TT(M)$ as $\TT(\tilde M)$ to
emphasize this point.

Using the Macaulay2 function {\texttt{symExt}}
of the last section, one can compute any finite piece of the Tate resolution.
\vskip0.3cm

<<<tateResolution = (m,E,loDeg,hiDeg)->(
     M := coker m;
     reg := regularity M;
     bnd := max(reg+1,hiDeg-1);
     mt  := presentation truncate(bnd,M);
     o   := symExt(mt,E);
     --adjust degrees, since symExt forgets them
     ofixed   :=  map(E^{(rank target o):bnd+1},
                E^{(rank source o):bnd},
                o);
     res(coker ofixed, LengthLimit=>max(1,bnd-loDeg+1)));>>>

\vskip0.3cm
\noindent
{\texttt{tateResolution}} takes as input a presentation matrix $m$ of a 
finitely generated graded module $M$ defined 
over some polynomial ring $S=K[x_0,\dots,x_n]$ with variables $x_i$ of 
degree 1, the name of an exterior algebra $E$ with the same number 
$n+1$ of generators, also supposed to be of degree 1, and two integers,
say $l$ and $h$. If $r$ is the regularity of $M$, then
{\tt{tateResolution(m,E,l,h)}} computes the piece
$$
\TT^{\,l}(M)\to \dots \to \TT^{\,\max(r+2,h)}(M)
$$
of $\TT(M)$. For instance, for the homogeneous coordinate ring of a point in
the projective plane:
\vskip0.3cm

<<<m = matrix{{x_0,x_1}};>>>
<<<regularity coker m>>>
<<<T = tateResolution(m,E,-2,4)>>>
<<<betti T>>>
<<<T.dd_1>>>

\vskip0.3cm

For arbitrary $M$ we have $M_i=\H^0\tilde M(i)$ for
large $i$, so the correspon\-ding term of the complex 
$\TT(\tilde M)$ with cohomological degree $i$ is 
$M_i\otimes \omega_E=\H^0(\tilde M(i))\otimes \omega_E$. 
The following result generalizes 
this to a description of all the terms of the Tate resolution,
and gives the formula for the cohomology described in the
introduction.

\begin{theorem}[\cite{EA:eis-sch:sheaf}]\label{tate}
%\vskip0.3cm
%\noindent
%\textbf{Theorem 4.1.}
%{\em 
Let $M$ be a finitely generated graded $S$-module.
The term of the complex $\TT(M)=\TT(\tilde M)$ with cohomological degree $i$ is 
$$
\oplus_j \H^j\tilde M(i-j)\otimes \omega_E \ ,
$$
where $\H^j\tilde M(i-j)$ is regarded as a vector space
concentrated in degree $i-j$, so that the summand
$\H^j\tilde M(i-j)\otimes \omega_E$ is isomorphic to a direct sum
of copies of $\omega_E(j-i)$.
Moreover the subquotient complex 
$$
\cdots \to \H^j\tilde M(i-j)\otimes \omega_E
\to
\H^j\tilde M(i+1-j)\otimes \omega_E
\to \cdots
$$
is 
$\RR(\H^j_*(\tilde M(-j))) (j)$ (up to twists and shifts it is
$\RR(\H^j_* \tilde M).$ )\qed
%}
\end{theorem}

%\vskip0.3cm
Thus each cohomology group of each twist of the sheaf
$\tilde M$ occurs (exactly once) in
a term of $\TT(M)$. When we compute a part of $\TT(M)$,
we are computing the sheaf cohomology of various twists
of the associated sheaf together with maps
which describe the $S$-module structure of $\H^j_*\tilde M$
in the sense that  the linear maps in this complex
are adjoints of the multiplication maps that determine the
module structure (the multiplication maps themselves could
be computed by a function similar to {\texttt{bgg}}). 
The higher degree maps in the complex $\TT(M)$
determine certain higher cohomology operations, which we understand 
only in very special cases (see \cite{EA:Eisenbud-Schreyer:ChowForms}).
 
If $M = \coker\, m$, then {\tt{betti tateResolution(m,E,l,h)}} prints
the dimensions $\h^j \tilde M(i-j) = \dim\,\H^j \tilde M(i-j)$ for
$\max(r+2, h) \geq i\geq l$, where $r$ is the regularity of $M$.
Truncating the Tate resolution if necessary allows one to restrict 
the size of the output.
\index{sheaf cohomology}\index{cohomology!sheaf}
\vskip0.3cm

<<<sheafCohomology = (m,E,loDeg,hiDeg)->(
     T := tateResolution(m,E,loDeg,hiDeg);
     k := length T;
     d := k-hiDeg+loDeg;
     if d > 0 then 
        chainComplex apply(d+1 .. k, i->T.dd_(i))
     else T);>>>


\vskip0.3cm
\noindent
The expression
{\tt{betti sheafCohomology(m,E,l,h)}} prints a cohomology table for
$\tilde M$ of the form
$$
{\setlength{\arraycolsep}{.4cm}
\begin{array}{llcl}
\h^0\tilde M(h)& \dots &\h^0\tilde M(l)\\
\h^1\tilde M(h-1)& \dots & \h^1\tilde M(l-1)\\
\hfil\vdots&&\hfil\vdots\\
\h^n\tilde M(h-n)& \dots &\h^n\tilde M (l-n)\ .
\end{array}}
$$
As a simple example we consider the cotangent bundle on projective 
3-space (see the next section for the Koszul resolution of this bundle):
\vskip0.3cm

<<<S=ZZ/32003[x_0..x_3];>>>
<<<E=ZZ/32003[e_0..e_3,SkewCommutative=>true];>>>
\vskip0.1cm

\noindent
The cotangent bundle is the cokernel of the third
differential of the Koszul complex on the variables of $S$.

\vskip0.1cm
<<<m=koszul(3,vars S);>>>
<<<regularity coker m>>>
<<<betti tateResolution(m,E,-6,2)>>>
<<<betti sheafCohomology(m,E,-6,2)>>>
\vskip0.1cm
\noindent
Of course these two results differ only in the precise point of
truncation.

\index{sheaf cohomology}\index{cohomology!sheaf}
\begin{remark}\label{cohold}  There is also a built-in sheaf cohomology
function {\texttt{HH}} in Macau\-lay2 which is based on the algorithms in 
\cite{EA:MR1484973:eisenbud}.  These algorithms are often much 
slower than {\texttt{sheafCohomology}}. 
To access it, first execute
\begin{verbatim}
     M=sheaf coker m;
\end{verbatim}
\noindent
and pick integers $j$ and $d$. Then
\begin{verbatim}
     HH^j(M(>=d))
\end{verbatim}
\noindent
returns the truncated $j^{\text{th}}$ cohomology module 
$\H^j_{i\geq d} \tilde M$.  In the above example of the cotangent
bundle $\mathcal F$ on projective 3-space we obtain the
Koszul presentation of $H^1 \mathcal F \cong K$ considered as an $S$-module 
sitting in degree 0:

\vskip0.3cm

<<<M=sheaf coker m;>>>
<<<HH^1(M(>=0))>>>
\qed
\end{remark}

The Tate resolutions of sheaves are, as the reader may easily check,
precisely the doubly infinite, graded, exact complexes of finitely-generated
free $E$-modules which are ``eventually linear'' on the right,
in an obvious sense. What about other doubly exact graded free
complexes? For example what if we take the dual of the Tate
resolution of a sheaf? In general it will not be eventually linear.
What is it?

To explain this we must generalize the construction of $\RR(M)$:
If 
$$
M^\bullet:\quad \cdots\rTo M^{i+1}\rTo M^i\rTo M^{i-1}\rTo\cdots
$$
is a complex of $S$-modules, then applying the functor
$\RR$ gives a complex of free complexes over $E$.
By changing some signs we get a double complex. 
In general the associated total complex is not minimal; but at least if 
$M^\bullet$ is a bounded complex then,
just as one produces the unique minimal free
resolution of a module from any free resolution,
we can construct a unique minimal
complex from it. We call this minimal complex $\RR(M^\bullet)$.
(See \cite{EA:eis-sch:sheaf} %Eisenbud and Schreyer [2000]
for more information. This construction is a necessary part of interpreting
the BGG correspondence as an equivalence of derived categories.)

Again
if $M^\bullet$ is a bounded complex of finitely generated
modules, then as before
one shows
that $\RR(M^\bullet)$ is exact from a certain point on, and
so we can form the Tate resolution $\TT(M^\bullet)$ by adjoining
a free resolution of a kernel. Once again, the Tate resolution 
depends only on the bounded complex of coherent sheaves
$\F^\bullet$ associated
to $M^\bullet$, and we write $\TT(\F^\bullet)=\TT(M^\bullet)$.

A variant of the theorem of Bernstein, Gel'fand and Gel'fand
shows that every minimal graded doubly infinite exact sequence
of finitely generated free $E$-modules is of the form $\TT(\F^\bullet)$
for some complex of coherent sheaves $\F^\bullet$, unique up
to quasi-isomorphism. The terms of the Tate resolution can
be expressed using hypercohomology
by a formula like that of Theorem \ref{tate}.

One way that interesting complexes of sheaves arise is through
duality. 
\index{duality of sheaves}
For simplicity,
write $\O$ for the structure sheaf $\O_{\P(W)}$.
If $\F=\tilde M$ is a sheaf on $\P(W)$ then 
the derived functor $RHom(\F, \O)$ may be computed
by applying the functor $Hom(\hbox{\bf ---},\O)$
to a sheafified free resolution of $M$; it's value is thus
a complex of sheaves rather than an individual sheaf.

We can now identify the dual of the Tate resolution:


\begin{theorem}\label{duality}
%\noindent
%\textbf{Theorem 4.2.}
%{\em 
%\theorem{duality} 
$\Hom_K(\TT(\F), K) \iso \TT(RH{\rm om}(\F, \O))[1]$.\qed
%}
\end{theorem}

Here the $[1]$ denotes a shift by one in cohomological degree.
For example, take $\F=\O$. We have $RH{\rm om}(\O, \O)=\O$. The
Tate resolution is given by
$$
\xymatrix @R=0mm @C=6mm{
\TT(\O):\quad\cdots\ar[r]& E \ar[r]& \omega_E\ar[r]&\cdots\\
&-1&0
}
$$
where the number under each term is its  cohomological degree.
Taking into account $\omega_E=\Hom_K(E,K)$,
the dual of the Tate resolution is thus
$$
\xymatrix @R=0mm{
\Hom_K(\TT(\O),K):\quad\cdots&\ar[l]\omega_E&\ar[l]E&\ar[l]\cdots\\
&1&0
}
$$
which is the same as 
$\TT(\O)[1]$.
A completely analogous computation gives the
proof of Theorem \ref{duality} 
if $\F=\O(a)$ for some $a$, and the general case follows
by taking free resolutions.


%\textbf{EISENBUD ENDE}


\section{Cohomology and vector bundles}

\index{vector bundle}
In this section we first recall how vector bundles,
direct sums of line bundles, and  bundles of differentials 
can be characterized among all coherent sheaves on $\PP (W)$ in terms of  
cohomology (as usual we do not distinguish between vector bundles and 
locally free sheaves). Then we describe the homomorphisms between the 
suitably twisted bundles of differentials in terms of the exterior 
algebra $E$. This description plays an important role in the context of 
Beilinson monads.

Vector bundles on $\PP (W)$ are characterized by a criterion of Serre
\index{bundle!Serre's criterion}
\cite{EA:MR16:953c} which can be formulated as follows:
A coherent sheaf $\mathcal F$ on $\PP (W)$ is locally free if and only if 
its module of sections $\H^0_* ({\mathcal F})$ is finitely generated and its 
{\it {intermediate cohomology modules}} \index{cohomology!intermediate}
$\H^j_*\mathcal F$, $1\leq j \leq n-1$, are of finite length.

 From a cohomological point of view, the simplest vector bundles are
the direct sums of line bundles.  Every vector bundle on the projective line
splits into a direct sum of line bundles by Grothendieck's splitting
theorem \index{splitting theorem!of Grothendieck}
(see \cite{EA:MR81b:14001}).  Induction yields Horrocks'
splitting theorem \index{splitting theorem!of Horrocks}
(see \cite{EA:MR80f:14005}): A vector bundle on $\P
(W)$ splits into a direct sum of line bundles if and only if its
intermediate cohomology vanishes (originally, this theorem was proved
as a corollary to a more general result, see \cite{EA:MR30:120} and
\cite{EA:MR99f:14064}).

Just a little bit more complicated are the bundles of differentials.
\index{bundles!of differentials} To fix our notation in this context we write 
$\mathcal O = \mathcal O_{\PP (W)}$, 
$W \otimes \mathcal O$ for the trivial bundle on $\PP (W)$ with fiber $W$,
$U=\Omega_{\P (W)}(1)$ for the cotangent bundle twisted by 1, and
$$
U^i = {\textstyle\bigwedge}^i U = {\textstyle\bigwedge}^i (\Omega_{\P (W)}(1)) = \Omega^i_{\PP (W)}(i)
$$
for the $i^{\text{th}}$ bundle of  differentials twisted by $i$; in particular 
$U^0 = \mathcal O$,  $U^n\cong \mathcal O (-1)$, and $U^i=0$ if $i<0$ or $i>n$.

\begin{remark} \label{dualitybd} For each $0 \leq i \leq n$ the pairing
$$
U^i \otimes U^{n-i} \overset\wedge\longrightarrow U^n \cong \mathcal O (-1)
$$
induces an isomorphism
$$
U^{n-i} \cong (U^{i})^* (-1)\, .\qquad\qed
$$
\end{remark}

The fiber of $U$ at the point of $\PP (W)$ corresponding to the line 
$\langle a \rangle\subset V$ is the subspace $(V/\langle a \rangle)^*\subset W$. 
Thus $U$ fits into the 
short exact sequence
$$
0\rightarrow U \rightarrow W \otimes \mathcal O \rightarrow 
\mathcal O (1) \rightarrow 0\ .
$$
In fact, $U$ is the {\it tautological subbundle\/} 
\index{tautological subbundle}
of $W \otimes \mathcal O$.
Taking exterior powers, we get the short exact sequences
$$
0\rightarrow U^{i+1} \rightarrow {\textstyle\bigwedge}^{i+1}W \otimes \mathcal O \rightarrow 
U^i\otimes \mathcal O (1) \rightarrow 0\ .
$$
Twisting the $i^\th$ sequence by $-i-1$, and gluing them
together we get the exact sequence
$$
\xymatrix@1@C=6mm{
0  \ar[r] & {\bigwedge^{n+1}_{\vbox to 2mm{}}}
W\otimes \mathcal O(-n-1) \ar[r] &\,\cdots\;  \ar[r]& {\bigwedge^{0}_{\vbox to 2mm{}}}
W\otimes \mathcal O \ar[r]& 0\ .
}
$$
This sequence is the sheafification of the Koszul complex, which
is the free resolution of the ``trivial'' graded $S$-module $K=S/(W)$.

\begin{remark} \label{cohbd}
By taking cohomology in the short exact sequences above we find that
$$
\H^j_*U^i =
\begin{cases}
K(i)& j=i,\\
0& j\ne i,
\end{cases}
\qquad 1\leq i,j\leq n-1\ ,
$$
where $K(i)=(S/(W))(i)$. Conversely, every vector bundle $\mathcal F$ on 
$\PP(W)$ with this intermediate cohomo\-logy is {\it {stably equivalent}}
\index{stable equivalence} 
to $U^i$; that is, there exists a direct sum  $\mathcal L$  
of line bundles such that $\mathcal F\cong U^i \oplus\mathcal L.$
This follows by comparing the sheafified Koszul complex with the minimal free resolution
of the dual bundle $\mathcal F^*$.\qed 
\end{remark}

In what follows we describe the homomorphisms bet\-ween the various $U^i$,
$0 \leq i \leq n$. Note that since $U=U^1\subset W\otimes \mathcal O$ each element
of $V=\hom_K(W,K)$ induces a homomorphism $U^1\to U^0$ which is the composite
$$
U^1 \subset W\otimes \mathcal O \to K\otimes\mathcal O = \mathcal O = U^0.
$$
Similarly, using the diagonal map of the exterior algebra
$U^i=\bigwedge^iU \to U\otimes U^{i-1}$, each element of $V$ 
induces a homomorphism $U^i\to U^{i-1}$ which is the composite
$$
U^i \to U\otimes U^{i-1} \to W\otimes U^{i-1} \to K\otimes U^{i-1} = U^{i-1}.
$$
It is not hard to show that these maps induced by elements of $V$
anticommute with each other
(see for example \cite[A2.4.1]{EA:MR97a:13001}).
Thus we get maps $\bigwedge^jV\to \hom(U^i, U^{i-j})$
which together give a graded ring  homomorphism
$\bigwedge V \to \hom(\oplus_i U^i, \oplus_i U^i)$.
In fact this construction gives
all the homomorphisms between the $U^i$:
\noindent
\begin{lemma} \label{hombd} The maps
$$
{\textstyle{\textstyle\bigwedge}}^jV\to \hom(U^i, U^{i-j}),\quad 0 \leq i, i-j \leq n\, ,
$$
described above are isomorphisms. Under these isomorphisms an element
$e\in\bigwedge^jV$ acts by contraction on the fibers of the $U^i$:
$$
\xymatrix{
\bigwedge^i(V/\langle a \rangle)^*\;\ar[d]\ar@{^(->}[r]& \bigwedge^{i} W \ar[d]^{e}\\
\bigwedge^{i-j}(V/\langle a \rangle)^*\;\ar@{^(->}[r] &\bigwedge^{i-j} W\ .
}
$$
\end{lemma}

\begin{proof} Every homomorphism  
$U^i \rightarrow U^{i-j}$ lifts uniquely to a
homomorphism between shifted Koszul complexes:
$$
\xymatrix@1@C=4mm{
0  \ar[r] & {\bigwedge^{n+1}_{\vbox to 2mm{}}}
W\otimes \mathcal O(i-n-1) \ar[d]\ar[r] &\,\cdots\; \ar[r]& {\bigwedge^{j}_{\vbox to 2mm{}}}
W\otimes \mathcal O (i-j) \ar[d]\ar[r]&\cdots\\
 \cdots\,  
\ar[r] & {\bigwedge^{n+1-j}_{\vbox to 2mm{}}}W\otimes \mathcal O(i-n-1) 
\ar[r] &\,\cdots\;  \ar[r]& \mathcal O (i-j) \ar[r]& 0
}
$$
Indeed, the corresponding obstructions vanish by Remarks \ref{dualitybd} and 
\ref{cohbd}. All results follow since the vertical arrows are necessarily 
given by contraction with an element in
$$
\Hom ({\textstyle\bigwedge}^{j} W \otimes \mathcal O (i-j), \mathcal O (i-j))\cong  {\textstyle\bigwedge}^jV\, .\quad\qed
$$
\end{proof} 


In practical terms, these results say that a map
$U^i \overset{e}\longrightarrow U^{i-j}$ is represented as
$$
\xymatrix{
\bigwedge^{i+1} W \otimes \mathcal O (-1) \ar[d]^e \ar@{->>}[r]& U^i \ar[d]\\
\bigwedge^{i-j+1} W \otimes \mathcal O (-1) \ar@{->>}[r]& U^{i-j}\
}
$$
if $0 < i-j \leq i \leq n$, and as the composite
$$
\xymatrix{
\bigwedge^{i+1} W \otimes \mathcal O (-1) \ar@{->>}[r]& U^i\; \ar[d]\ar@{^(->}[r]&
\bigwedge^{i} W \otimes \mathcal O\ar[d]^e\\
&U^0 \ar@{}[r]|{=} &\mathcal O\
}
$$
if $0 = i-j < i \leq n$. 

A map from a sum of copies of various $U^i$ to another
such sum is given by a homogeneous matrix over the exterior algebra $E$. In
general it is an interesting problem to relate properties of
the matrix to properties of the map. Here is one relation
which is easy. We will apply it later on in this chapter.

\noindent
\begin{proposition}\label{critsur} If
$$
r\, U^i\overset B \longrightarrow s\,U^{i-1}
$$
is a homomorphism, that is, if $B$ is an $s\times r$-matrix with entries 
in $V$, then the following condition is 
necessary for $B$ to be surjective: If $(b_1,\dots, b_r)$ is 
a non-trivial linear combination of the rows of $B$, then
$$
\dim\, {\def\span{\operatorname{span}} \span} (b_1,\dots, b_r)\geq i+1.
$$
\end{proposition}
\begin{proof} $B$ is surjective if and only if its dual map is injective on fibers:
$$
s\,{\textstyle\bigwedge}^{i-1}(V/\langle  a \rangle) \overset {\wedge B^t}\longrightarrow r
\,{\textstyle\bigwedge}^{i}(V/\langle  a \rangle)
$$
is injective for any line $\langle a\rangle \subset V$. Consider
a non-trivial linear combination $(b_1,\dots, b_r)^t$ of the columns of ${B^t}$,
and write $d = \dim\, \span (b_1,\dots, b_r)$. 
If $d=i$, then ${B^t}$ is not injective at any point 
of $\P (W)$ corresponding to a vector in ${\def\span{\operatorname{span}} \span} 
(b_1,\dots, b_r)$. If  $d < i$, then ${B^t}$ is not injective at any point of $\P (W)$.\qed
\end{proof}

\section{Cohomology and monads}

\index{monads!applications of}
The technique of monads provides powerful tools for problems such as the 
construction and classification of coherent sheaves with prescribed invariants. 
This section is an introduction to monads. We demonstrate their usefulness, 
which is not obvious at first glance, by reviewing the
classification of stable rank 2 vector bundles on the projective plane
(see \cite{EA:MR57:324}%Barth [1977]
, \cite{EA:MR80m:14012}%Le Potier [1979]
, and \cite{EA:MR80m:14011}%Hulek [1979]
).  Recall that stable bundles admit moduli (see \cite{EA:MR81h:14014}%Gieseker [1977]
, \cite{EA:MR56:8567}, and \cite{EA:MR82h:14011}%Maruyama [1977, 1978]
).

The basic idea behind monads is to represent arbitrary coherent sheaves
in terms of simpler sheaves such as line
bundles or bundles of differentials, and in terms of homomorphisms
between these simpler sheaves. If $M$ is a finitely generated graded $S$-module,
with associated sheaf $\mathcal F = \tilde M$, then the sheafification of
the minimal free resolution of $M$ is a monad for $\mathcal F$ which involves 
direct sums of line bundles and thus homogeneous matrices over $S$. 
The Beilinson monad for $\mathcal F$, which 
will be considered in the next section, involves direct sums of twisted bundles 
of differentials $U^i$, and thus homogeneous matrices over $E$.


\noindent
\begin{definition}\label{monad}
A {\it{monad}}
\index{monad}
on $\P (W)$ is a bounded complex
$$
\cdots\, \longrightarrow {\mathcal K}^{-1} \longrightarrow {\mathcal K}^{0}
\longrightarrow {\mathcal K}^{1} \longrightarrow \, \cdots
$$
of coherent sheaves on $\P (W)$ which is exact except  at
${\mathcal K}^{0}$. The homology $\mathcal F$ at ${\mathcal K}^{0}$ 
is called the {\it {homology of the monad}}, \index{monad!homology of} 
and the monad is said to be a monad for $\mathcal F$. We say that the 
{\it {type of a monad}} \index{monad!type of} is 
determined if the sheaves ${\mathcal K}^{i}$ are determined.\qed
\end{definition}

\noindent
There are different ways of representing a given sheaf as the homology of 
a monad, and the type of the monad depends on the way chosen.

When constructing or classifying sheaves in a given class via 
monads, one typically proceeds along the following lines.

\vskip0.2cm
\noindent
{\it\underline{Step 1.}}\; Compute cohomological information which determines 
the type of the corresponding monads.
\vskip0.1cm
\noindent
{\it\underline{Step 2.}}\; Construct or classify the differentials of the monads.

\vskip0.2cm
\noindent
There are no general recipes for either step and some cases require
sophisticated ideas and quite a bit of intuition (see Example 7.2 below).
If one wants to classify, say, vector bundles, then a third step is needed:

\vskip0.2cm
\noindent
{\it\underline{Step 3.}}\; Determine which monads lead to isomorphic vector bundles.
\vskip0.2cm

\noindent
One of the first successful applications of this approach was the classification
of (Gieseker-)stable rank 2 vector bundles with even first Chern class $c_1\in\Z$ on 
the complex projective plane by  Barth \cite{EA:MR57:324}, who 
detected geometric properties of the corresponding moduli spaces without giving
an explicit description of the differentials in the second step. The same ideas 
apply in the case  $c_1$ odd which we are going to survey in what follows
(see \cite{EA:MR80m:14012}, \cite{EA:MR80m:14011}, and \cite{EA:MR81b:14001} for
full details and proofs).

In general, rank 2 vector bundles enjoy properties which are not shared by
all vector bundles.

\begin{remark}\label{remark-5.2}
%{{
%  {\bf Remark 5.2.}
Every rank 2 vector bundle $\mathcal F$ on $\PP (W)$ is 
{\it {self-dual}},\index{bundle!self-dual} that is, it admits a symplectic structure. 
Indeed, the cano\-nical map 
$$\mathcal F \otimes \mathcal F 
\rightarrow {\textstyle\bigwedge}^2 \mathcal F \cong \mathcal O_{\PP (W)} (c_1)
$$ 
induces an isomorphism 
$\varphi: \mathcal F \overset{\cong}\to \mathcal F^{\ast} (c_1)$ 
with $\varphi = - \varphi^* (c_1)$ (here $c_1$ is the first Chern class
of $\mathcal F$). In particular there are canonical isomorphisms 
$$
(\H^j \mathcal F (i))^*\cong \H^{n-j} \mathcal F (-i-n-1-c_1)
$$
by Serre duality.\index{bundle!Serre duality} \qed
\end{remark}


We will not give a general definition of stability here. For rank 2 vector bundles
stability can be characterized as follows (see \cite{EA:MR81b:14001}).

\begin{remark}\label{remark-5.3} If $\mathcal F$ is a rank 2 vector bundle 
on $\PP (W)$, then the following hold:
\vskip0.1cm
\noindent
(1) \; $\mathcal F$ is stable \index{bundle!stable}
if and only if $\Hom (\mathcal F, \mathcal F)\cong K$.
In this case the symplectic structure on $\mathcal F$ is uniquely determined up to scalars.
\vskip0.1cm
\noindent
(2)\; By tensoring with a line bundle we can {\it {normalize}} $\mathcal F$
\index{bundle!normalized}
so that its first Chern class is $0$ or $-1$. In this case $\mathcal F$ is stable  
\index{bundle!stable} if and only if it has no global sections.\qed
\end{remark}

\begin{example} By the results of the previous section the twisted cotangent bundle 
$U$ on the projective plane is a stable rank 2 vector bundle with Chern classes
$c_1 = -1$ and $c_2 = 1$.\qed
\end{example}

\begin{remark}\label{remark-5.5}
% \noindent
%{{{\bf Remark 5.3.} 
The generalized theorem of Riemann-Roch yields a polynomial
in $\QQ [c_1, \dots ,  c_r]$ which gives the Euler characteristic 
$\chi \mathcal F = \sum_j (-1)^j \h^j \mathcal F$ for e\-very rank $r$ vector bundle 
$\mathcal F$ on $\PP (W)$ with Chern classes $c_1, \dots , c_r$.
This polynomial can be determined by interpreting the generalized theorem of Riemann-Roch
or by computing the Euler characteristic for enough special bundles of rank $r$ (like direct
sums of line bundles). For a rank 2 vector bundle on the projective plane, 
for example, one obtains 
$$\chi (\mathcal F)=(c_1^2-2c_2+3c_1+4)/2\, .\quad\qed$$
\end{remark}


We now focus on stable rank 2 vector bundles on the complex projective plane
$\PP^2(\CC) = \PP(W)$ with first Chern class $c_1 = -1$.
Let $\mathcal F$ be such a bundle.


\begin{remark}\label{remark-5.6} Since $\mathcal F$  is stable and
normalized its second Chern class $c_2$ must be $\geq 1$. Indeed, 
$$\H^2 \mathcal F (i-2) = \H^0 \mathcal F(-i) = 0 \quad {\text{for}}\quad i\geq 0$$
by Remarks 5.2 and 5.3, and $\chi (\mathcal F (i)) = (i+1)^2 -c_2$ by
Riemann-Roch. In particular the dimensions 
$\h^j \mathcal F (i)$ in the range $-2 \leq i \leq 0$ are 
as in the following cohomology table (a zero is represented by an empty box):
\vspace{0.2cm}
%%
%%%%%%%%%%%%%%%%%%%%%%%%%%%%%%%%%%%%%%%%%%%%%%%%%%%%%%%%%%%%%%%%%%%%%%%%%%%%%%%

% Neue L�ngen f�r Abst�nde horizontal und vertikal
% Nur einmal vor dem ersten Auftreten eines Beilinson-Diagramms
%\newlength{\br}
%\newlength{\ho}
%
%%%%%%%%%%%%%%%%%%%%%%%%%%%%%%%%%%%%%%%%%%%%%%%%%%%%%%%%%%%%%%%%%%%%%%%%%%%%%%%

{
$$ %Diagramm zentrieren
%
% W�hle die Einheiten \br horizontal und \ho vertikal
{
\setlength{\br}{10mm}
\setlength{\ho}{6mm}
\fontsize{10pt}{8pt}
\selectfont
\begin{xy}
%%%%%%%%%%%%%%%%%%%%%%%%%%%%%%%%%%%%%%%
%
% Achsenkreuz im Punkte (0,0)
%
% x-Achse von -3\br bis 1\br
% mit einem "j" an 0.98 der L�nge und 3mm unter der Achse:
%
,<-3.5\br,0\ho>;<1\br,0\ho>**@{-}?>*@{>}
?(0.95)*!/^3mm/{i}
%
% y-Achse von 0\ho bis 4\ho
% mit einem "i" an 0,9 der L�nge und 3mm rechts neben der Achse:
%
,<-1\br,0\ho>;<-1\br,4\ho>**@{-}?>*@{>}
?(0.95)*!/^3mm/{j}
%%%%%%%%%%%%%%%%%%%%%%%%%%%%%%%%%%%%%%%
%
% 5 waagrechte Linien von -3\br bis +0\br
% in den H�hen 1\ho,...,5\ho:
%
,0+<-3\br,1\ho>;<0\br,1\ho>**@{-}
,0+<-3\br,2\ho>;<0\br,2\ho>**@{-}
,0+<-3.5\br,3\ho>;<.5\br,3\ho>**@{-}
%%%%%%%%%%%%%%%%%%%%%%%%%%%%%%%%%%%%%%%
%
% 11 senkrechte Linien von 0\ho bis 3\ho
% in den waagrechten Punkten -3\br,...,+0\br:
%
,0+<-3\br,0\ho>;<-3\br,3\ho>**@{-}
,0+<-2\br,0\ho>;<-2\br,3\ho>**@{-}
,0+<0\br,0\ho>;<0\br,3\ho>**@{-}
%
%%%%%%%%%%%%%%%%%%%%%%%%%%%%%%%%%%%%%%%
%
% Eintr�ge in den Mitten der K�sten. Daher die Koordinaten mit .5
%
,0+<-2.5\br,1.5\ho>*{c_2-1}
,0+<-1.5\br,1.5\ho>*{c_2}
,0+<-0.5\br,1.5\ho>*{c_2-1}
%
%%%%%%%%%%%%%%%%%%%%%%%%%%%%%%%%%%%%%%%
,0+<-2.5\br,-0.6\ho>*{-2}
,0+<-1.5\br,-0.6\ho>*{-1}
,0+<-0.5\br,-0.6\ho>*{0}
%%%%%%%%%%%%%%%%%%%%%%%%%%%%%%%%%%%%%%%
,0+<-4.0\br,2.5\ho>*{2}
,0+<-4.0\br,1.5\ho>*{1}
,0+<-4.0\br,0.5\ho>*{0}
\end{xy}
}
$$
\quad\qed
}
%%%%%%%%%%%%%%%%%%%%%%%%%%%%%%%%%%%%%%%%%%%%%%%%%%%%%%%%%%%%%%%%%%%%%%%%%%%%%%%
%\vskip0.1cm
\noindent
\end{remark}

We abbreviate $\mathcal O = \mathcal O_{\PP^2 (\CC)}$ and go through the 
three steps above.

\vskip0.3cm
\noindent
{\it\underline{Step 1.}}\; In this step we show that $\mathcal F$ is the 
homology of a monad of type
$$
0\rightarrow \H^1 \mathcal F (-2)\otimes U^2 
\rightarrow \H^1 \mathcal F (-1) \otimes U \rightarrow 
\H^1 \mathcal F \otimes \mathcal O \rightarrow 0\ ,
$$
where the middle term occurs in cohomological degree 0.
This actually follows from the general construction of Beilinson monads presented
in the next chapter and the fact that $\H^2 \mathcal F (i-2) = \H^0 \mathcal F(-i) = 0$ 
for $2\geq i\geq 0$ (see Remark \ref{remark-5.6}). Here we derive the existence 
of the monad directly with Horrocks' technique of killing cohomology 
\index{killing cohomology} \cite{EA:MR84j:14026},  which 
requires further cohomological information. Such information is typically obtained 
by restricting the given bundles to linear subspaces. In our case we consider 
the Koszul complex on the equations of a point $p\in \PP^2(\CC)$:
$$
\xymatrix @C=11mm{
0\ar[r]&\ \mathcal O(-2)\ar[r]^{\left(\substack{-x'\\x}\right)}
&2\:\! \mathcal O (-1)\ar[r]^{\quad(x\ x^{\prime})} &\mathcal O
\ar[r]&\mathcal O_p\ar[r]&0\ .
}
$$
\noindent
By tensoring with $\mathcal F (i+1)$ and taking cohomology we find that 
$\H^1 \mathcal F$ generates $\H^1_{\geq 0}\, \mathcal F$. Indeed, the composite map 
$${\begin{pmatrix} x & x^{\prime}\end{pmatrix}}: 2 \\\H^1\mathcal F(i) 
\longrightarrow \H^1 (\mathcal J_p\otimes\mathcal F(i+1))
\longrightarrow \H^1\mathcal F(i+1)$$
is surjective if $i\geq -1$. In particular, if $c_2=1$, then  
$\H^1 \mathcal F(i) =  0$ for $i \neq -1$ (apply Serre duality for the
twists $\leq -2$), so $\mathcal F\cong U$ is the twisted cotangent bundle by Remark 
\ref{cohbd} since both bundles have the same rank and intermediate cohomology. 

If $c_2\geq 2$ then $\H^1 \mathcal F\neq 0$, and the identity in 
$$\Hom (\H^1 \mathcal F, \H^1 \mathcal F)\cong\Ext^1(\H^1 \mathcal F 
\otimes \mathcal O, \mathcal F)$$
 defines an extension
$$
0\rightarrow \mathcal F \rightarrow \mathcal G \rightarrow \H^1 \mathcal F
\otimes \mathcal O\rightarrow 0\ , 
$$
where $\H^1_{\geq 0}\,\mathcal G = 0$, and where $\mathcal G$ is a vector bundle 
(apply Serre's criterion in Section 4). Similarly, by taking Serre duality into
account,  we obtain an extension
$$
0\rightarrow \H^1 \mathcal F (-2)\otimes  U^2 \rightarrow \mathcal H
\rightarrow  \mathcal F\rightarrow 0\ , 
$$
where $\mathcal H$ is a vector bundle with $\H^{1}_{\leq -2} \mathcal H = 0$. 
The two extensions fit into a commutative diagram with exact rows and and columns
%%%%%%%%%%%%%%%%%%%%%%%%%%%%%%%%%%%%%%%%%%%%%%%%%%%%%
$$
\xymatrix{
&&0\ar[d]&0\ar[d]\\
0\ar[r]& \H^1 \mathcal F (-2)\otimes U^2\ar@{}[d]|{\scriptscriptstyle ||}\ar[r]&
\mathcal  H\ar[d]\ar[r]&
\mathcal F\ar[r]\ar[d]& 0\\
0\ar[r]&  \H^1 \mathcal F (-2)\otimes U^2\ar[r]^{\qquad \quad \alpha}& 
\mathcal B \ar[d]^{\beta}\ar[r]&\mathcal G\ar[d]\ar[r]& 0\\
&& \H^1 \mathcal F\otimes \mathcal O \ar[d]\ar@{}[r]|{=}&
\H^1 \mathcal F \otimes \mathcal O \ar[d]\\
&&0&0
}
$$
%%%%%%%%%%%%%%%%%%%%%%%%%%%%%%%%%%%%%%%%%%%%%%%%%%%%
since, for example, the extension in the top row lifts uniquely to an extension 
as in the middle row (the obstructions in the corresponding Ext-sequence vanish). 
Then $\mathcal B \cong \H^1 \mathcal F (-1)\otimes U$ 
since by construction these bundles  have the same rank and intermediate cohomology.
What we have is the {\it display}
\index{monad!display of}
 of (the short exact sequences associated to)
a monad 
$$
0\longrightarrow \H^1 \mathcal F (-2)\otimes U^2 \overset{\alpha}
\longrightarrow \H^1 \mathcal F (-1) \otimes U \overset{\beta}\longrightarrow 
\H^1 \mathcal F \otimes \mathcal O \longrightarrow 0
$$
for $\mathcal F$. 

\vskip0.3cm
\noindent
{\it\underline{Step 2.}}\;
Our task in this step is to describe what maps $\alpha$ and $\beta$ could
be the differentials of a monad as above. In fact we give a description in terms of 
linear algebra for which it is enough to deal with one of the differentials,
say $\alpha$, since the self-duality of $\mathcal F$ and the vanishing
of certain obstructions allows one to represent $\mathcal F$ as the
homology of a ``self-dual'' monad. Let us abbreviate $A=\H^1 \mathcal F (-2)$, 
$B=\H^1 \mathcal F (-1)$ and $A^*\cong \H^1 \mathcal F$. By chasing the displays 
of a monad as above and its dual we see that the symplectic structure on 
$\mathcal F$ lifts to a unique isomorphism of monads
$$
\xymatrix{
0\ar[r]& A\otimes U^2 \ar[d]^{\Phi}\ar[r]^{\alpha}&
B \otimes U \ar[d]^{\Psi}\ar[r]^{\beta}&
A^*\otimes\mathcal O \ar[r] \ar[d]^{-\Phi^*(-1)} & 0\\
0\ar[r]&A\otimes \mathcal O(-1) \ar[r]^{{\beta^*}(-1)} & B^* \otimes U^*(-1) 
\ar[r]^{{\alpha^*}(-1)}& A^*\otimes (U^2)^*(-1) \ar[r]& 0
}
$$
with $\Psi = -\Psi^*(-1)$. Indeed, the corresponding obstructions vanish
(see \cite{EA:MR80f:14005} and \cite[II, 4.1]{EA:MR81b:14001} for a discussion of 
this argument in a general context). $\Psi$ is the tensor product of an isomorphism 
$q: B\rightarrow B^*$ and a symplectic form $\iota\in\Hom(U,U^*(-1))\cong\CC$
on $U$.  Note that $q$ is symmetric since
$-(q\otimes\iota) = (q\otimes\iota)^*(-1)=q^*\otimes\iota^*(-1)=-q^*\otimes\iota$.
We may and will now assume that $\mathcal F$ is the homology of a {\it {self-dual monad}},
\index{monad!self-dual}
where self-dual means  that $\beta = \alpha^d:=\alpha^* (-1)\circ(q\otimes\iota)$. 
The monad conditions

\vskip0.2cm
($\alpha_1$) $\alpha^d\circ\alpha = 0$, and
\vskip0.1cm
($\alpha_2$) $\alpha$ is a vector bundle monomorphism (equivalently,
$\alpha^d$ is an epimorphism)

\vskip0.2cm
\noindent
can be rewritten in terms of linear algebra as follows. 
The identifications in Lemma \ref{hombd} allow one to view 
$$\alpha\in\Hom(A\otimes U^2, B \otimes U) \cong V \otimes \Hom(A, B) $$
as a homomorphism $\alpha: W \rightarrow \Hom (A,B)$ operating by
$\xi \otimes (x \wedge x^{\prime}) \rightarrow \alpha (x)(\xi) \otimes x^{\prime}
- \alpha (x^{\prime})(\xi) \otimes x$ on the fibers of 
$A\otimes U^2$. Similarly we consider $\alpha^d$
as the homomorphism $\alpha^d: W \rightarrow \Hom (B,A^*), x\mapsto \alpha^*(x) \circ q$, 
operating by $\eta \otimes x \rightarrow \alpha^d (x)(\eta)$ on the fibers
of $B \otimes U$. Then

\vskip0.2cm
($\alpha_1^{\prime}$)\, $\alpha^d(x) \circ \alpha(x^{\prime})=
\alpha^d(x^{\prime}) \circ\alpha(x)$ for all $x, x^{\prime}\in W$, and
\vskip0.1cm
($\alpha_2^{\prime}$)\, for every $\xi\in A\setminus \{0\}$
the map $W \rightarrow B$, $x \to \alpha(x)(\xi)$ has rank $\geq 2$.


\noindent
% \noindent
%\begin{Example} 
\begin{example}\label{exc22}
If $c_2=2$, then  the monads can be written (non-canonically) as 
$$
\xymatrix @C=9mm{
0\ar[r]&U^2\ar[r]^{\ \left(\substack{a\\b}\right)}
&2\:\!U\ar[r]^{\,(a\ b)} &\mathcal O\ar[r]&0\ ,
}
$$
where $a, b$ are two vectors in $V$.
In this case ($\alpha_1$) gives no extra condition and 
($\alpha_2$) means that $a$ and $b$ are linearly independent.
If $a$ and $b$ are  explicitly given, then we can compute the
homology of the monad with the help of Macaulay2:

\vskip0.3cm
<<<S = ZZ/32003[x_0..x_2];>>>
\vskip0.1cm
\noindent
$U$ is obtained from the Koszul complex resolving $S/(x_0, x_1, x_2)$ by
tensoring the cokernel of the differential $\bigwedge^3 W\otimes S(-3) 
\to \bigwedge^2W \otimes S(-2)$ with $S(1)$ (and sheafifying).

\vskip0.1cm
<<<U = coker koszul(3,vars S) ** S^{1};>>>
\vskip0.1cm

\noindent
For representing $\alpha$ and $\alpha^d$ we also need the differential 
$\bigwedge^2W \otimes S(-2) \to W \otimes S(-1)$ of the Koszul complex.

\vskip0.1cm
<<<k2 = koszul(2,vars S)>>>
\vskip0.1cm

\noindent
The expression
{\tt{koszul(2,vars S)}} computes a matrix representing the differential
with respect to the monomial bases $x_0\wedge x_1, x_0 \wedge x_2, x_1 \wedge x_2$ 
of $\bigwedge^2W$ and $x_0, x_1, x_2$ of $W$.
We pick $(a,b) = (e_1, e_2)$ and represent the corresponding maps $\alpha$ 
and $\alpha^d$ with respect to the monomial bases (see the discussion following
Lemma \ref{hombd}).

\vskip0.1cm
<<<alpha = map(U ++ U, S^{-1}, transpose{{0,1,0,1,0,0}});>>>
<<<alphad = map(S^1, U ++ U, matrix{{0,1,0,0,0,1}} * (k2 ++ k2));>>>
\vskip0.1cm

\noindent
Prune computes a minimal presentation.

\vskip0.1cm
<<<F = prune homology(alphad, alpha);>>>
<<<betti  F>>>

\vskip0.3cm

In the next section we will present a more elegant way of 
computing the homology of Beilinson monads. \qed
\end{example}

We go back to the general case and reverse our construction. 
Let $A$ and $B$ be $\CC$-vector spaces of 
the appropriate dimensions, let $q$ be a non-degenerate quadratic form on $B$,
and let 
$$
\widetilde{\cal{M}}=\{\alpha\in\Hom(W, \Hom(A, B)) \mid \alpha {\text { satisfies }} 
(\alpha_1^{\prime}) {\text { and }} (\alpha_2^{\prime})\}\ .
$$
Then every $\alpha\in \widetilde{\cal{M}}$ defines a self-dual
monad as above whose homology is a stable rank 2 vector bundle on $\PP^2(\CC)$ with 
Chern classes $c_1=-1$ and $c_2$. In this way we obtain a description of
the differentials of  the monads which is not as explicit as we might have 
hoped (with the exception of the case $c_2 = 2$). It is, however,  enough for detecting 
geometric properties of the corresponding moduli spaces.

\vskip0.3cm
\noindent
{\it\underline{Step 3.}}\; Constructing the moduli spaces means to parametrize
the isomorphism classes of our bundles in a convenient way. We very roughly 
outline how to do that. Let ${\text{O}}(B)$ be the orthogonal group of
$(B,q)$, and let $G:= \GL (A)\times {\text{O}}(B)$. Then $G$ acts on 
$\widetilde{\cal{M}}$ by $((\Phi, \Psi),\alpha) \mapsto \Psi\alpha\Phi^{-1}$,
where $\Psi\alpha\Phi^{-1}(x) := \Psi\alpha(x)\Phi^{-1}$. We may consider an 
element $(\Phi, \Psi)\in G$ as an isomorphism bet\-ween 
the monad defined by $\alpha$ and the monad defined by $\Psi\alpha\Phi^{-1}$.
By going back and forth between isomorphisms of bundles
and isomorphisms of mo\-nads one shows that the  stabilizer of $G$ in 
each point is $\{\pm 1\}$, and that our construction induces a bijection
between the set of isomorphism classes of stable rank 2 vector bundles on $\PP^2(\CC)$ 
with Chern classes $c_1=-1$ and $c_2$ and ${\cal{M}}:=\widetilde{\cal{M}}/G_0$, 
where $G_0:=G/\{\pm 1\}$. With the help of a universal monad over 
$\PP^2(\CC)\times \widetilde{\cal{M}}$ one proves that the analytic structure on 
$\widetilde{\cal{M}}$ descends to an analytic structure on ${\cal{M}}$ so that 
${\cal{M}}$ is smooth of dimension $\h^1 \mathcal F^* \otimes \mathcal F = 4c_2-4$ 
in each point (the obstructions for smoothness in the point corresponding to 
$\mathcal F$ lie in $\H^2 \mathcal F^* \otimes \mathcal F$ which is zero). 
Moreover the homology of the universal monad tensored by a suitable line bundle
descends to a universal family over ${\cal{M}}$ (here one needs $c_1=-1$). In other 
words, ${\cal{M}}$ is what one calls a fine moduli space for our bundles. Further 
efforts show that ${\cal{M}}$ is irreducible and rational.

\begin{remark} Horrocks' technique of killing cohomology always yields 3-term monads.
In general, the bundle in the middle can be pretty complicated.\qed
\end{remark}


\section{The Beilinson monad}

%\textbf{EISENBUD}


%\section{beilinson} The Beilinson Monad

\index{monad!Beilinson}
We can use the Tate resolution associated to a sheaf to 
give a construction of a complex first
described by Beilinson \cite{EA:MR80c:14010b}, which gives a 
powerful method for deriving information about a sheaf from
information about a few of its cohomology groups.
The general idea is the following:


Suppose that $\A$ is an additive category and consider
a graded object
$
\oplus_{i=0}^{n+1}U^i
$
in $\A$.
Given a graded ring homomorphism $E\to \End_\A(\oplus_{i=0}^{n+1}U^i)$
we can make an additive functor from the category of 
free $E$-modules to $\A$:
On objects we take
$$
\omega_E(i)\mapsto{ \begin{cases} U^i & \text{for } 0\leq i
\leq {n+1} \text{ and}; \\0 & \text{otherwise.}\end{cases}}
$$
To define the functor on maps, we use 
$$
\begin{aligned}
\Hom_E(\omega_E(i),\omega_E(j))&=
\Hom_E(E(i),E(j))\\
&= E_{j-i}
\longrightarrow\End(\oplus U^i)_{j-i}\longrightarrow \Hom(U^i,U^j)\ .
\end{aligned}
$$
(Note that we could have taken any twist of $E$ in place of 
$\omega_E\cong E(-n-1)$; the choice of $\omega_E$ is made to 
simplify the statement of Theorem \ref{Beilinson-theorem}, below.)
%%%%%%%%% Referenz \ref{Beilinson theorem}, below.)

We shall be interested in the special case
where $\A$ is the category of coherent sheaves on
$\P(W)$ and where $U^i = \Omega_{\P(W)}^i (i)$ as in Section~4.
Further examples may be obtained by taking $U^i$ to be the $i^\th$ exterior
power of the tautological subbundle $U_k$ on the Grassmannian
of $k$-planes in $W$ for any $k$; the case we have taken
here is the case $k=n$. See \cite{EA:Eisenbud-Schreyer:ChowForms} 
for more information on the general case
and applications to the computation of
resultants and more general Chow forms.

Applying the functor just defined to the Tate resolution $\TT(\F)$
of a cohe\-rent sheaf $\F$ on $\P(W)$, 
and using Theorem \ref{tate}, we get 
a complex
$$
\Omega(\F):\quad
\cdots\rTo \oplus_j \H^j\F(i-j)\otimes U^{j-i}\rTo\dots ,
$$
where the term we have written down occurs in cohomological
degree $i$. The resolution $\TT(\F)$ is well-defined up to homotopy, 
so the same is true of $\Omega(\F)$.
Since $U^{k}=0$ unless $0\leq k\leq n$
the only cohomology groups of $\F$ that
are actually involved in $\Omega(\F)$ are $\H^j\F(k)$ with
$-n\leq k\leq 0$; $\Omega(\F)$ is of type
$$
\xymatrix@1@C=3mm{
0\ar[r]& \H^0 \mathcal F (-n)\otimes U^n\ar@{}[d]|{\scriptscriptstyle ||}\ar[r]&
\cdots \ar[r]& 
\oplus_{j=0}^{n}\H^j \mathcal F (-j)\otimes U^j\ar@{}[d]|{\scriptscriptstyle ||}\ar[r]&
\cdots \ar[r]&
\H^n \mathcal F \otimes U^0\ar@{}[d]|{\scriptscriptstyle ||}\ar[r]&
0&\\
0\ar[r]&  \Omega^{-n}(\F)\ar[r]&\cdots \ar[r]& \Omega^0(\F)\ar[r]& 
\cdots \ar[r]& \Omega^n(\F)\ar[r]& 0&.\\
}
$$
For applications it is important to note that instead of working with  
$\Omega(\F)$ one can also work with $\Omega(\F(i))$ for some twist $i$. 
This gives one some freedom in choosing the cohomology groups of $\F$ 
to be involved.


To see a simple example, consider again the structure 
sheaf $\O_p$ of the subvariety consisting of a point
$p\in \P(W)$. Write $I$ for the homogeneous ideal of $p$,
and let $a\in V=W^*$ be a non-zero functional vanishing
on the linear forms in $I$ as before.
The Tate resolution of the homogeneous coordinate ring $S/I$
has already been computed, and we have seen that it depends only
on the sheaf $\widetilde{S/I}=\O_p$. From the computation of
$\TT(S/I)=\TT(\O_p)$ made in Section~3 we see that 
$\Omega(\O_p)$ takes the form
$$
\Omega(\O_p):\quad 0\to U^n\rTo^a U^{n-1}\rTo^a
\cdots \rTo^a U^1\rTo^a U^0\rTo 0\ ,
$$
with $U^i$ in cohomological degree $-i$. 

We have already noted that the map 
$a: U=U^1 \rTo U^0=\O_{\P(W)}$ is the composite of the
tautological embedding $U\subset W\otimes \O_{\P(W)}$ with the
map $a\otimes 1:\ W\otimes \O_{\P(W)} \to \O_{\P(W)}$.
Thus the image of $a: U^1\to \O_{\P(W)}$ is the ideal
sheaf of $p$, and we see that the homology of the complex
$\Omega(\O_p)$ at $U^0$ is $\O_p$. One can check further
that $\Omega(\O_p)$ is the Koszul complex associated with
the map $a: U^1\to \O_{\P(W)}$, and it follows that 
the homology of $\Omega(\O_p)$ at $U^i$ is 0 for $i>0$.
The following result shows that this is typical.

\begin{theorem}[\cite{EA:eis-sch:sheaf}]\label{Beilinson-theorem}
%\theorem{Beilinson theorem} 
%\vskip0.3cm
%\noindent
%\textbf{Theorem 6.1}
%{\em 
If $\F$ is a coherent sheaf on
$\P(W)$, then the only non-vanishing homology of  the
complex $\Omega(\F)$ is 
$$
\H^0(\Omega(\F))=\F\, .\qquad\qed
$$
%}
\end{theorem}

The existence of a complex satisfying the theorem and having the same
terms as $\Omega(\F)$ was first asserted by
Beilinson in \cite{EA:MR80c:14010b},
and thus we will call $\Omega(\F)$ a {\it Beilinson monad\/} 
\index{monad!Beilinson}
\index{Beilinson monad}
for $\F$.
Existence proofs via a somewhat less effective construction than the one
given here may be found in \cite{EA:MR89g:18018} and \cite{EA:MR92g:14013}.

The explicitness of the construction via Tate resolutions allows one to 
detect properties of the differentials of Beilinson monads.
\index{Beilinson monad!differentials of} Let us write
$$
\begin{aligned}
d_{ij}^{(r)}\in \Hom( \H^j \mathcal F (i-j) \otimes U^{j-i}, 
\H^{j-r+1} \mathcal F (i-j+r) \otimes U^{j-i-r})\\
\cong {\textstyle\bigwedge}^r V\otimes  \Hom (\H^j \mathcal F (i-j),  
\H^{j-r+1} \mathcal F (i-j+r))\\
\cong \Hom ({\textstyle\bigwedge}^r W \otimes \H^j \mathcal F (i-j),  
\H^{j-r+1} \mathcal F (i-j+r))
\end{aligned}
$$
for the degree $r$ maps actually occurring in $\Omega(\F)$.

\begin{remark}\label{diff1}\; The constant maps $d_{ij}^{(0)}$ in 
$\Omega(\F)$ are zero since $\TT(\mathcal F)$ is minimal. \qed
\end{remark}

\begin{proposition}[\cite{EA:eis-sch:sheaf}]\label{diff2}\; 
The linear maps $d_{ij}^{(1)}$ in $\Omega(\F)$ 
correspond to the multiplication maps 
$$W \otimes \H^j \mathcal F (i-j)\to \H^{j} \mathcal F (i-j+1)\, .\qquad\qed$$
\end{proposition}

\noindent
This follows from the identification of the linear strands in $\TT (\F)$ 
(see the discussion following Theorem 3.1). The higher degree maps in 
$\TT (\F)$ and $\Omega(\F)$, however, are not yet well-understood.

Since $(\TT(\mathcal F))[1] = \TT(\mathcal F(1))$ we can compare
the differentials in $\Omega(\F)$ with those in $\Omega(\F(1))$:

\begin{proposition}[\cite{EA:eis-sch:sheaf}]\label{diff3}\; If the maps 
$d_{ij}^{(r)}$ in $\Omega(\F)$ and $d_{i-1,j}^{(r)}$ in $\Omega(\F(1))$ 
both actually occur, then they correspond to the same element
in 
$${\textstyle\bigwedge}^r V\otimes  \Hom (\H^j \mathcal F (i-j),  
\H^{j-r+1} \mathcal F (i-j+r))\ .\qquad\qed$$
\end{proposition}


In what follows we present some Macaulay2 code for computing Beilinson monads. 
Our  functions {\tt {sortedBasis}}, {\tt {beilinson1}}, {\tt {U}}, and
{\tt {beilinson}} reflect what we did in Example \ref{exc22} . 

The expression
{\tt {sortedBasis(i,E)}} sorts the monomials of degree $i$ in $E$ to match the order 
of the columns of {\tt {koszul(i,vars S)}}, where our conventions with respect to $S$ and $E$
are as in Section 2, and where we suppose that the monomial order on $E$ is
reverse lexicographic, the Macaulay2 default order.

\vskip0.3cm
<<<sortedBasis = (i,E) -> (
     m := basis(i,E);
     p := sortColumns(m,MonomialOrder=>Descending);
     m_p);>>>

\vskip0.1cm
\noindent
For example:
\vskip0.1cm

<<<S=ZZ/32003[x_0..x_3];>>>
<<<E=ZZ/32003[e_0..e_3,SkewCommutative=>true];>>>
<<<koszul(2,vars S)>>>
<<<sortedBasis(2,E)>>>

\vskip0.1cm
\noindent
If $e\in E$ is homogeneous of degree $j$, then 
{\tt {beilinson1(e,j,i,S)}} computes the map $U^i \overset{e}\longrightarrow U^{i-j}$
on $\PP^n = \Proj \,S$. If $0 < i-j \leq i \leq n$, then the result is a matrix representing 
the map $\bigwedge^{i+1}W \otimes S(-1)\overset{e\otimes 1} \longrightarrow 
\bigwedge^{i-j+1}W\otimes S(-1)$ defined by contraction with $e$. If $0 = i-j < i \leq n$, 
then the result is a matrix representing the composite
of the map $\bigwedge^{i}W \otimes S \overset{e\otimes 1}
\longrightarrow S$ with the Koszul differential 
$\bigwedge^{i+1}W \otimes S(-1) \rightarrow \bigwedge^{i}W \otimes S$.
Note that the degrees of the result are not set correctly since 
the functions {\tt {U}} and {\tt {beilinson}} below are supposed to do that.
\vskip0.1cm

<<<beilinson1=(e,dege,i,S)->(
     E := ring e;
     mi := if i < 0 or i >= numgens E then map(E^1, E^0, 0)
           else if i === 0 then id_(E^1)
           else sortedBasis(i+1,E);
     r := i - dege;
     mr := if r < 0 or r >= numgens E then map(E^1, E^0, 0)
           else sortedBasis(r+1,E);
     s = numgens source mr;
     if i>=0 and i<=numgens E and i === r then
           substitute(e*id_(E^s),S)    
     else if i > 0 and r === 0 then
           (vars S) * substitute(diff(diff(e,mi),transpose mr),S)
     else
           substitute(diff(diff(e,mi),transpose mr),S));>>>

 
\vskip0.1cm
\noindent
For example:
\vskip0.1cm

<<<beilinson1(e_1,1,3,S)>>>
<<<beilinson1(e_1,1,2,S)>>>
<<<beilinson1(e_1,1,1,S)>>>

\vskip0.1cm
\noindent
The function {\tt {U}} computes the bundles $U^i$ on Proj$\,S$:

\vskip0.1cm
\noindent
<<<U = (i,S) -> (
     if i < 0 or i >= numgens S then S^0
     else if i === 0 then S^1
     else cokernel koszul(i+2,vars S) ** S^{i});>>>

\vskip0.1cm
\noindent
Finally, if $o : \oplus E(-a_i) \to \oplus E(-b_j)$ is a homogeneous 
matrix over $E$, then {\tt {beilinson(o,S)}} computes the 
corresponding map $o : \oplus U^{a_i} \to \oplus U^{b_j}$ on Proj$\,S$
by calling {\tt {beilinson1}} and {\tt {U}}.
\vskip0.3cm

<<<beilinson = (o,S) -> (
     coldegs := degrees source o;
     rowdegs := degrees target o;
     mats = table(numgens target o, numgens source o,
              (r,c) -> (
                   rdeg = first rowdegs#r;
                   cdeg = first coldegs#c;
                   overS = beilinson1(o_(r,c),cdeg-rdeg,cdeg,S);
                   -- overS = substitute(overE,S);
                   map(U(rdeg,S),U(cdeg,S),overS)));
     if #mats === 0 then matrix(S,{{}})
     else matrix(mats));>>>

\vskip 0.1cm
\noindent
With these functions the code in Example \ref{exc22} can be rewritten as follows:
\vskip 0.1cm

<<<S=ZZ/32003[x_0..x_2];>>>
<<<E = ZZ/32003[e_0..e_2,SkewCommutative=>true];>>>
<<<alphad = map(E^1,E^{-1,-1},{{e_1,e_2}})>>>
<<<alpha = map(E^{-1,-1},E^{-2},{{e_1},{e_2}})>>>
<<<alphad=beilinson(alphad,S);>>>
<<<alpha=beilinson(alpha,S);>>>
<<<F = prune homology(alphad,alpha);>>>
<<<betti  F>>>

\vskip 0.3cm

\section{Examples}

\index{Beilinson monad!applications of}
In this section we give two examples of explicit constructions of Beilinson 
monads over $\PP^4(\CC) = \PP(W)$ and of classification results based on these monads.
As in Section 5 we proceed in three steps. Let us write $\mathcal O = 
\mathcal O_{\PP^4(\CC)}$.

%\begin{Example}\label{was7.1}
\begin{example}\label{was7.1}
%\noindent
%{\it{Example 7.1.}}\, 
Our first example is taken from the classification
of {\it {conic bundles}} 
\index{conic bundle}
in  $\PP^4(\CC)$, that is, of smooth surfaces 
$X\subset\PP^4(\CC)$ which are ruled in conics in the sense that there exists a 
surjective morphism $\pi:X \rightarrow C$ onto a smooth curve $C$ such that the 
general fiber of $\pi$ is a smooth conic in the given embedding of $X$.
There are precisely three families of such surfaces (see \cite{EA:ES} and 
\cite{EA:BR}). Two families, the Del Pezzo surfaces of 
degree 4 and the  Castelnuovo surfaces, are classical. 
The third family, consisting of {\it {elliptic conic bundles}}
\index{conic bundle!elliptic}
(conic bundles over an elliptic curve) of degree 8, had been falsely 
ruled out in two classification papers in the 1980's
(see  \cite{EA:okgrad8} and \cite{EA:ionescu}). Only recently Abo, Decker, and Sasakura 
\cite{EA:conicbundle} constructed and classified such surfaces  by considering the Beilinson 
monads for the suitably twisted ideal sheaves of the surfaces. Let us explain how
this works.

\vskip0.1cm
\noindent
{\it\underline{Step 1.}}\; In this step we suppose that an elliptic conic
bundle $X$ as above exists, and we determine the type of the Beilinson
monad for the suitably twisted ideal sheaf $\mathcal J_X$.
We know from the classification of smooth surfaces in 
$\PP^4(\CC)$ which are contained in a cubic hypersurface (see \cite{EA:roth} and 
\cite{EA:aure-thesis}) that $\H^0 \mathcal J_X (i) = 0$ for $i\leq 3$. 
It follows from general results such as the theorem of Riemann-Roch that the 
dimensions $\h^j \mathcal J_X(i)$ in range $-2\leq i \leq 3$ are as follows  
(here, again, a zero is represented by an empty box):
\vskip0.2cm
%
%%%%%%%%%%%%%%%%%%%%%%%%%%%%%%%%%%%%%%%%%%%%%%%%%%%%%%%%%%%%%%%%%%%%%%%%%%%%%%%

% Neue L�ngen f�r Abst�nde horizontal und vertikal
% Nur einmal vor dem ersten Auftreten eines Beilinson-Diagramms
% \newlength{\br}
% \newlength{\ho}
%
%%%%%%%%%%%%%%%%%%%%%%%%%%%%%%%%%%%%%%%%%%%%%%%%%%%%%%%%%%%%%%%%%%%%%%%%%%%%%%%

{
$$ %Diagramm zentrieren
%
% W�hle die Einheiten \br horizontal und \ho vertikal
{
\setlength{\br}{9mm}
\setlength{\ho}{6mm}
\fontsize{10pt}{8pt}
\selectfont
\begin{xy}
%%%%%%%%%%%%%%%%%%%%%%%%%%%%%%%%%%%%%%%
%
% Achsenkreuz im Punkte (0,0)
%
% x-Achse von -3\br bis 5\br
% mit einem "j" an 0.95 der L�nge und 3mm unter der Achse:
%
,<-2.5\br,0\ho>;<5\br,0\ho>**@{-}?>*@{>}
?(0.98)*!/^3mm/{i}
%
% y-Achse von 0\ho bis 6\ho
% mit einem "i" an 0,9 der L�nge und 3mm rechts neben der Achse:
%
,<0\br,0\ho>;<0\br,6\ho>**@{-}?>*@{>}
?(0.98)*!/^3mm/{j}
%%%%%%%%%%%%%%%%%%%%%%%%%%%%%%%%%%%%%%%
%
% 5 waagrechte Linien von -2\br bis +4\br
% in den H�hen 1\ho,...,5\ho:
%
,0+<-2\br,1\ho>;<4\br,1\ho>**@{-}
,0+<-2\br,2\ho>;<4\br,2\ho>**@{-}
,0+<-2\br,3\ho>;<4\br,3\ho>**@{-}
,0+<-2\br,4\ho>;<4\br,4\ho>**@{-}
,0+<-2.5\br,5\ho>;<4.5\br,5\ho>**@{-}
%%%%%%%%%%%%%%%%%%%%%%%%%%%%%%%%%%%%%%%
%
% 11 senkrechte Linien von 0\ho bis 5\ho
% in den waagrechten Punkten -2\br,...,+3\br:
%
,0+<-2\br,0\ho>;<-2\br,5\ho>**@{-}
,0+<-1\br,0\ho>;<-1\br,5\ho>**@{-}
%
,0+<1\br,0\ho>;<1\br,5\ho>**@{-}
,0+<2\br,0\ho>;<2\br,5\ho>**@{-}
,0+<3\br,0\ho>;<3\br,5\ho>**@{-}
,0+<4\br,0\ho>;<4\br,5\ho>**@{-}
%
%%%%%%%%%%%%%%%%%%%%%%%%%%%%%%%%%%%%%%%
%
% Eintr�ge in den Mitten der K�sten. Daher die Koordinaten mit .5
%
,0+<-1.5\br,3.5\ho>*{8}
,0+<-0.5\br,3.5\ho>*{4}
%
,0+<0.5\br,2.5\ho>*{1}
,0+<1.5\br,2.5\ho>*{1}
,0+<2.5\br,2.5\ho>*{a}
,0+<3.5\br,2.5\ho>*{b}
,0+<2.5\br,1.5\ho>*{a+1}
,0+<3.5\br,1.5\ho>*{b+1}
%
%%%%%%%%%%%%%%%%%%%%%%%%%%%%%%%%%%%%%%%
,0+<-1.5\br,-0.6\ho>*{-2}
,0+<-0.5\br,-0.6\ho>*{-1}
,0+<0.5\br,-0.6\ho>*{0}
,0+<1.5\br,-0.6\ho>*{1}
,0+<2.5\br,-0.6\ho>*{2}
,0+<3.5\br,-0.6\ho>*{3}
%%%%%%%%%%%%%%%%%%%%%%%%%%%%%%%%%%%%%%%
,0+<-3.0\br,4.5\ho>*{4}
,0+<-3.0\br,3.5\ho>*{3}
,0+<-3.0\br,2.5\ho>*{2}
,0+<-3.0\br,1.5\ho>*{1}
,0+<-3.0\br,0.5\ho>*{0}
\end{xy} 
}
$$
}
%%%%%%%%%%%%%%%%%%%%%%%%%%%%%%%%%%%%%%%%%%%%%%%%%%%%%%%%%%%%%%%%%%%%%%%%%%%%%%%
\vskip0.1cm
\noindent
with $a := \h^2 \mathcal J_X(2)$ and $b := \h^2 \mathcal J_X(3)$ still to
be determined. The Beilinson monad for $\mathcal J_X(2)$ is thus of type
$$
0 \rightarrow 8\:\! \mathcal O (-1) \rightarrow 4\:\! U^3\oplus U^2
\rightarrow  U\oplus (a+1)\:\! \mathcal O \rightarrow a\:\! \mathcal O \rightarrow 0\ ,
$$
where $\,(a+1)\:\! \mathcal O \rightarrow a\:\! \mathcal O\,$ is the zero map
(see Remark \ref{diff1}), 
and where consequently $U$ is mapped surjectively onto $\,a\,\! \mathcal O\,$. By 
Proposition \ref{critsur} this is only possible if $a = 0$. The same idea applied to 
$\mathcal J_X(3)$ shows that then also $b = 0$. 

The cohomological information obtained so far determines the type of the Beilinson 
monad for $\mathcal J_X(2)$ and for $\mathcal J_X(3)$. We decide to concentrate on 
the monad for $\mathcal J_X(3)$ since its differentials are smaller in size than those of the 
monad for $\mathcal J_X(2)$. In order to ease our calculations further we kill
the 4-dimensional space $\H^3 \mathcal J_X(-1)$. Let us write $\omega_X$ for
the dualizing sheaf of $X$. Serre duality on $\PP^4(\CC)$ 
respectively on $X$  yields canonical isomorphisms
$$
\begin{aligned}
Z& :=  \Ext^1(\mathcal J_X(-1),\mathcal O(-5))\\
&\cong (\H^3 \mathcal J_X(-1))^*
\cong (\H^2 \mathcal O_X(-1))^*\cong \H^0(\omega_X(1))\ .
\end{aligned}
$$
The identity in 
$$
\Hom (Z, Z) \cong \Ext^1(\mathcal J_X(-1), Z^* \otimes \mathcal O(-5))
$$
defines an extension which, twisted by 4, can be written as
$$
0 \rightarrow 4\:\! \mathcal O(-1) \rightarrow \mathcal G
\rightarrow \mathcal J_X(3) \rightarrow 0\ .
$$
Let us show that $\mathcal G$ is a vector bundle.
We know from the classification of scrolls in $\PP^4(\CC)$ (see \cite{EA:Lanteri} 
and \cite{EA:aure-thesis}) that $X$ is not a scroll.
Hence adjunction theory implies that $\omega_X(1)$ is generated by the adjoint linear 
system $\H^0(\omega_X(1))$ (see \cite[Corollary 9.2.2]{EA:adj-theory}). It follows by
Serre's criterion (\cite{EA:MR16:953c}, see also \cite[Theorem 2.2]{EA:okreflexiv})
that $\mathcal G$ is locally free. By construction $\mathcal G$ has a 
cohomology table as follows:
\vskip0.2cm
%
%
%%%%%%%%%%%%%%%%%%%%%%%%%%%%%%%%%%%%%%%%%%%%%%%%%%%%%%%%%%%%%%%%%%%%%%%%%%%%%%%

% Neue L�ngen f�r Abst�nde horizontal und vertikal
% Nur einmal vor dem ersten Auftreten eines Beilinson-Diagramms
%\newlength{\br}
%\newlength{\ho}
%
%%%%%%%%%%%%%%%%%%%%%%%%%%%%%%%%%%%%%%%%%%%%%%%%%%%%%%%%%%%%%%%%%%%%%%%%%%%%%%%
{
$$ %Diagramm zentrieren
%
% W�hle die Einheiten \br horizontal und \ho vertikal
{
\setlength{\br}{9mm}
\setlength{\ho}{6mm}
\fontsize{10pt}{8pt}
\selectfont
\begin{xy}
%%%%%%%%%%%%%%%%%%%%%%%%%%%%%%%%%%%%%%%
%
% Achsenkreuz im Punkte (0,0)
%
% x-Achse von -5.5\br bis 1\br
% mit einem "j" an 0.95 der L�nge und 3mm unter der Achse:
%
,<-5.5\br,0\ho>;<1\br,0\ho>**@{-}?>*@{>}
?(0.98)*!/^3mm/{i}
%
% y-Achse von 0\ho bis 6\ho
% mit einem "i" an 0,9 der L�nge und 3mm rechts neben der Achse:
%
,<-1\br,0\ho>;<-1\br,6\ho>**@{-}?>*@{>}
?(0.98)*!/^3mm/{j}
%%%%%%%%%%%%%%%%%%%%%%%%%%%%%%%%%%%%%%%
%
% 5 waagrechte Linien von -5\br bis +0\br
% in den H�hen 1\ho,...,5\ho:
%
,0+<-5\br,1\ho>;<0\br,1\ho>**@{-}
,0+<-5\br,2\ho>;<0\br,2\ho>**@{-}
,0+<-5\br,3\ho>;<0\br,3\ho>**@{-}
,0+<-5\br,4\ho>;<0\br,4\ho>**@{-}
,0+<-5.5\br,5\ho>;<.5\br,5\ho>**@{-}
%%%%%%%%%%%%%%%%%%%%%%%%%%%%%%%%%%%%%%%
%
% 11 senkrechte Linien von 0\ho bis 5\ho
% in den waagrechten Punkten -8\br,...,+3\br:
%
,0+<-5\br,0\ho>;<-5\br,5\ho>**@{-}
,0+<-4\br,0\ho>;<-4\br,5\ho>**@{-}
,0+<-3\br,0\ho>;<-3\br,5\ho>**@{-}
,0+<-2\br,0\ho>;<-2\br,5\ho>**@{-}
,0+<0\br,0\ho>;<0\br,5\ho>**@{-}
%
%%%%%%%%%%%%%%%%%%%%%%%%%%%%%%%%%%%%%%%
%
% Eintr�ge in den Mitten der K�sten. Daher die Koordinaten mit .5
%
,0+<-3.5\br,2.5\ho>*{1}
,0+<-2.5\br,2.5\ho>*{1}
,0+<-1.5\br,1.5\ho>*{1}
,0+<-0.5\br,1.5\ho>*{1}
%
%%%%%%%%%%%%%%%%%%%%%%%%%%%%%%%%%%%%%%%
,0+<-4.5\br,-0.6\ho>*{-4}
,0+<-3.5\br,-0.6\ho>*{-3}
,0+<-2.5\br,-0.6\ho>*{-2}
,0+<-1.5\br,-0.6\ho>*{-1}
,0+<-0.5\br,-0.6\ho>*{0}
%%%%%%%%%%%%%%%%%%%%%%%%%%%%%%%%%%%%%%%
,0+<-6.0\br,4.5\ho>*{4}
,0+<-6.0\br,3.5\ho>*{3}
,0+<-6.0\br,2.5\ho>*{2}
,0+<-6.0\br,1.5\ho>*{1}
,0+<-6.0\br,0.5\ho>*{0}
\end{xy}
}
$$
}
%%%%%%%%%%%%%%%%%%%%%%%%%%%%%%%%%%%%%%%%%%%%%%%%%%%%%%%%%%%%%%%%%%%%%%%%%%%%%%%
\vskip0.1cm
\noindent
So the  Beilinson monad of $\mathcal G$ is of type
$$
0 \rightarrow U^3 \overset\alpha\rightarrow U^2\oplus U
\overset\beta\rightarrow \mathcal O \rightarrow 0\ .
$$

\vskip0.1cm
\noindent
{\it\underline{Step 2.}}\; Now we proceed the other way around. We show that a 
rank 5 bundle $\mathcal G$  as in the first step exists, and that the dependency locus of 
four general sections of $\mathcal G(1)$ is a surface of the desired type. Differentials which 
define a monad as above with a locally free homology  can be easily found.
By Lemma \ref{hombd} $\alpha$ corresponds to a pair of vectors $\alpha=(\alpha_1, \alpha_2)^t
\in V \oplus \bigwedge^2 V$ . It is a vector bundle monomorphism if and only if
the dual map $U^2\oplus U^3\overset{\alpha^t}\longrightarrow U^1$
is an epimorphism (see Remark \ref{dualitybd} for duality). Equivalently, 
$\alpha_1$ is non-zero and $\alpha_2$ considered as a vector in 
$\bigwedge^2 (V/\langle \alpha_1 \rangle)$ is indecomposable (argue as in the proof of
Proposition \ref{critsur}). Taking the other monad conditions into account we see
that we may pick
$$
\alpha=
\begin{pmatrix}
e_4\\
e_0\wedge e_2 + e_1\wedge e_3
\end{pmatrix}
$$
and
$$
\beta=
\begin{pmatrix}
    e_0\wedge e_2 + e_1\wedge e_3\, ,&-e_4
\end{pmatrix} ,
$$
where $e_0 , \dots , e_4$ is a basis of $V$, and that up to isomorphisms of monads 
and up to the choice of the basis this is the only possibility. We fix $\mathcal G$ as
the homology of this monad and compute the syzygies of $\mathcal G$ with Macaulay2.
\vskip0.3cm


<<<S = ZZ/32003[x_0..x_4];>>>
<<<E = ZZ/32003[e_0..e_4,SkewCommutative=>true];>>>
<<<beta=map(E^1,E^{-2,-1},{{e_0*e_2+e_1*e_3,-e_4}})>>>
<<<alpha=map(E^{-2,-1},E^{-3},{{e_4},{e_0*e_2+e_1*e_3}})>>>
<<<beta=beilinson(beta,S);>>>
<<<alpha=beilinson(alpha,S);>>>
<<<G = prune homology(beta,alpha);>>>
<<<betti res G>>>


\vskip0.3cm
\noindent
We see in particular that $\mathcal G (1)$ is globally generated. Hence the
dependency locus of four general sections of $\mathcal G (1)$ is indeed a smooth surface
in $\PP^4(\CC)$ by Kleiman's Bertini-type result \cite{EA:Bertini}. The smoothness can
also be checked with Macaulay2 in an example via the built-in Jacobian criterion
(see \cite{EA:DSJSC} for a speedier method).
\vskip0.3cm

<<<foursect = random(S^4, S^10) * presentation G;>>>
\vskip0.1cm

\noindent
The function
{\tt{trim}} computes a minimal presentation.

\vskip0.1cm
<<<IX = trim minors(4,foursect);>>> 
<<<codim IX>>>
<<<degree IX>>>
<<<codim singularLocus IX>>>

\vskip0.3cm
\noindent
By construction $X$ has the correct invariants and is in fact an elliptic conic bundle
as claimed: Since the adjoint linear system $\H^0(\omega_X(1))$ is base point free 
and 4-dimensional
by what has been  said in the first step, the corresponding adjunction map 
$X \rightarrow \PP^3$ is a morphism which exhibits, as is easy to see, $X$ as a
conic bundle over a smooth elliptic curve in $\PP^3$ 
(see \cite[Proposition 2.1]{EA:conicbundle}). 

\vskip0.1cm
\noindent
{\it\underline{Step 3.}}\; Our discussion in the previous steps gives also a classification
result. Up to projectivities the elliptic conic bundles of degree 8 in $\PP^4(\CC)$ are 
precisely the smooth surfaces arising as the dependency locus of four sections of the 
bundle $\mathcal G(1)$ fixed in Step 2.\qed
%\end{Example}
\end{example}

%\begin{Example}
\begin{example}
%\noindent
%{\it {Example 7.2}}\, 
This example is concerned with the construction and
classification of \ie{abelian surface}s in $\PP^4(\CC)$, and
with the closely related
\ie{Horrocks-Mumford bundle}s \cite{EA:HM}.
\index{bundle!Horrocks-Mumford}
\vskip0.1cm
\noindent
{\it\underline{Step 1.}}\; Horrocks and Mumford  found evidence for the 
existence of a family of abelian surfaces in $\PP^4(\CC)$. 
Suppose that such a surface $X$ exists. Then the dualizing sheaf of $X$ is trivial, 
$\omega_X\cong \mathcal O_X$, and $X$ has degree 10 (see \cite[Example 3.2.15]{EA:fultonit}).
The same arguments as in Example \ref{was7.1} show that $X$ arises as the zero scheme of a 
section of a rank 2 vector bundle: There is an extension 
$$
0 \rightarrow  \mathcal O \rightarrow \mathcal F (3)
\rightarrow \mathcal J_X(5) \rightarrow 0\ ,
$$
where $\mathcal F (3)$ is a rank 2 vector bundle with Chern classes $c_1 = 5$ and 
$c_2 = \deg X = 10$, and where $\mathcal F$ has a cohomology table as follows:
\vskip0.2cm
%
%
%%%%%%%%%%%%%%%%%%%%%%%%%%%%%%%%%%%%%%%%%%%%%%%%%%%%%%%%%%%%%%%%%%%%%%%%%%%%%%%

% Neue L�ngen f�r Abst�nde horizontal und vertikal
% Nur einmal vor dem ersten Auftreten eines Beilinson-Diagramms
%\newlength{\br}
%\newlength{\ho}
%
%%%%%%%%%%%%%%%%%%%%%%%%%%%%%%%%%%%%%%%%%%%%%%%%%%%%%%%%%%%%%%%%%%%%%%%%%%%%%%%
{
$$ %Diagramm zentrieren
%
% W�hle die Einheiten \br horizontal und \ho vertikal
{
\setlength{\br}{9mm}
\setlength{\ho}{6mm}
\fontsize{10pt}{8pt}
\selectfont
\begin{xy}
%%%%%%%%%%%%%%%%%%%%%%%%%%%%%%%%%%%%%%%
%
% Achsenkreuz im Punkte (0,0)
%
% x-Achse von -5.5\br bis 1\br
% mit einem "j" an 0.95 der L�nge und 3mm unter der Achse:
%
,<-5.5\br,0\ho>;<1\br,0\ho>**@{-}?>*@{>}
?(0.98)*!/^3mm/{i}
%
% y-Achse von 0\ho bis 6\ho
% mit einem "i" an 0,9 der L�nge und 3mm rechts neben der Achse:
%
,<-1\br,0\ho>;<-1\br,6\ho>**@{-}?>*@{>}
?(0.98)*!/^3mm/{j}
%%%%%%%%%%%%%%%%%%%%%%%%%%%%%%%%%%%%%%%
%
% 5 waagrechte Linien von -5\br bis +0\br
% in den H�hen 1\ho,...,5\ho:
%
,0+<-5\br,1\ho>;<0\br,1\ho>**@{-}
,0+<-5\br,2\ho>;<0\br,2\ho>**@{-}
,0+<-5\br,3\ho>;<0\br,3\ho>**@{-}
,0+<-5\br,4\ho>;<0\br,4\ho>**@{-}
,0+<-5.5\br,5\ho>;<.5\br,5\ho>**@{-}
%%%%%%%%%%%%%%%%%%%%%%%%%%%%%%%%%%%%%%%
%
% 11 senkrechte Linien von 0\ho bis 5\ho
% in den waagrechten Punkten -8\br,...,+3\br:
%
,0+<-5\br,0\ho>;<-5\br,5\ho>**@{-}
,0+<-4\br,0\ho>;<-4\br,5\ho>**@{-}
,0+<-3\br,0\ho>;<-3\br,5\ho>**@{-}
,0+<-2\br,0\ho>;<-2\br,5\ho>**@{-}
,0+<0\br,0\ho>;<0\br,5\ho>**@{-}
%
%%%%%%%%%%%%%%%%%%%%%%%%%%%%%%%%%%%%%%%
%
% Eintr�ge in den Mitten der K�sten. Daher die Koordinaten mit .5
%
,0+<-4.5\br,3.5\ho>*{5}
,0+<-2.5\br,2.5\ho>*{2}
,0+<-0.5\br,1.5\ho>*{5}
%
%%%%%%%%%%%%%%%%%%%%%%%%%%%%%%%%%%%%%%%
,0+<-4.5\br,-0.6\ho>*{-4}
,0+<-3.5\br,-0.6\ho>*{-3}
,0+<-2.5\br,-0.6\ho>*{-2}
,0+<-1.5\br,-0.6\ho>*{-1}
,0+<-0.5\br,-0.6\ho>*{0}
%%%%%%%%%%%%%%%%%%%%%%%%%%%%%%%%%%%%%%%
,0+<-6.0\br,4.5\ho>*{4}
,0+<-6.0\br,3.5\ho>*{3}
,0+<-6.0\br,2.5\ho>*{2}
,0+<-6.0\br,1.5\ho>*{1}
,0+<-6.0\br,0.5\ho>*{0}
\end{xy}
}
$$
}
%%%%%%%%%%%%%%%%%%%%%%%%%%%%%%%%%%%%%%%%%%%%%%%%%%%%%%%%%%%%%%%%%%%%%%%%%%%%%%%
\vskip0.1cm
\noindent
In particular $\mathcal F$, which has Chern classes $c_1=-1$ and $c_2=4$, is stable by 
Remark 5.3. A discussion as in Section 5 shows that the Beilinson monad for $\mathcal F$ 
is of type 
$$
\xymatrix{
0\ar[r]& A\otimes \mathcal O(-1)\ar[r]^{\alpha}&
B \otimes U^2 \ar[r]^{\alpha^d}& A^*\otimes\mathcal O \ar[r]  & 0
}\ ,
$$
with $\CC$-vector spaces $A$ and $B$ of dimension 5 and 2 respectively,
and with $\alpha^d = \alpha^*(-1)\circ(q\otimes\iota)$,  where $q$ is
a symplectic form on $B$, and where 
$\iota : U^2 \overset{\cong}\longrightarrow (U^2)^*(-1)$ is induced
by the pairing $U^2 \otimes U^2 \longrightarrow U^4\cong\mathcal O(-1)$.
By choosing appropriate bases of $A$ and $B$ we may suppose
that $\alpha$ is a $2\times 5$ matrix with entries in $\bigwedge^2 V$ and that
$\alpha^d = \alpha^t\cdot \begin{pmatrix} 0 & 1\\ -1 & 0\end{pmatrix}$.

\vskip0.1cm
\noindent
{\it\underline{Step 2.}}\; As in Example 7.1 we now proceed the other way around.
But this time it is not obvious  how to define $\alpha$.
Horrocks and Mumford remark that up to projectivities
one may suppose that the abelian surfaces in $\P^4(\CC)$ are invariant under
the action of the {\ie Heisenberg group} $H_5$ in its Schr\"odinger representation, and
they use the representation theory of $H_5$ and its normalizer $N_5$ in 
$\SL(5, \CC)$ to find 
$$
\alpha = 
\begin{pmatrix}
e_2\wedge e_3\;&e_3\wedge e_4\;&e_4\wedge e_0\;&e_0\wedge e_1\;&e_1\wedge e_2\\
e_1\wedge e_4\;&e_2\wedge e_0\;&e_3\wedge e_1\;&e_4\wedge e_2\;&e_0\wedge e_3
\end{pmatrix}\ ,
$$
where $e_0 ,\dots , e_4$ is a basis of $V$. A straightforward computation shows
that with this $\alpha$ the desired monad conditions are indeed satisfied. The resulting
Horrocks-Mumford bundle $\mathcal F_{\text{HM}}$ on $\PP^4(\CC)$ is essentially 
the only rank 2 vector bundle known on $\PP^n(\CC)$, $n\geq 4$, which does not 
split as direct sum of two line bundles. Let us compute the syzygies of 
$\mathcal F_{\text{HM}}$ with Macaulay2.
\vskip0.3cm

<<<alphad1=-matrix{{e_1*e_4,e_2*e_0,e_3*e_1,e_4*e_2,e_0*e_3}};>>>
<<<alphad2=matrix{{e_2*e_3,e_3*e_4,e_4*e_0,e_0*e_1,e_1*e_2}};>>>
<<<alphad=matrix{{alphad1},{alphad2}};>>>
<<<alphad=map(E^5,E^{-2,-2},transpose alphad)>>>
<<<alpha=syz alphad>>>
<<<alphad=beilinson(alphad,S);>>>
<<<alpha=beilinson(alpha,S);>>>
<<<FHM = prune homology(alphad,alpha);>>>
<<<betti res FHM>>>
<<<regularity FHM>>>
<<<betti sheafCohomology(presentation FHM,E,-6,6)>>> 

\vskip0.3cm
\noindent
Since $\H^0 \mathcal F_{\text{HM}}(i) = 0$ for $i<3$ every non-zero section of 
$\mathcal F_{\text{HM}}(3)$ vanishes along a surface (with the desired invariants).
Horrocks and Mumford need an extra argument to show that the general such surface
is smooth (and thus abelian) since Kleiman's Bertini-type result does not apply
($\mathcal F_{\text{HM}}(3)$ is not globally generated). Our explicit construction 
allows one again to check the smoothness with Macaulay2 in an example.
\vskip0.3cm

<<<sect =  map(S^1,S^15,0) | random(S^1, S^4);>>>
\vskip0.1cm

\noindent
We compute the equations of $X$ via a mapping cone.
\vskip0.1cm

<<<mapcone = sect || transpose presentation FHM;>>>
<<<fmapcone = res coker mapcone;>>>
<<<IX =  trim ideal fmapcone.dd_2;>>>
<<<codim IX>>>
<<<degree IX>>>
<<<codim singularLocus IX>>>


\vskip0.3cm
\noindent
{\it\underline{Step 3.}}\; Horrocks and Mumford showed that up to projectivities
every abelian surface in $\PP^4(\CC)$ arises as the zero scheme of a section
of $\mathcal F_{\text{HM}}(3)$. In fact, one can show much more.
By a careful analysis of possible Beilinson monads and their restrictions
to various linear subspaces Decker \cite{EA:uniquenesshm1} proved that every stable rank 2
vector bundle $\mathcal F$ on $\PP^4(\CC)$ with Chern classes $c_1=-1$ and $c_2=4$
is the homology of a monad of the type as in Step 1. From geometric properties of the 
``variety of unstable planes''  of $\mathcal F$  Decker and Schreyer \cite{EA:uniquenesshm2} 
deduced that up to isomorphisms and projectivities the differentials of the monad coincide 
with those of $\mathcal F_{\text{HM}}$. Together with results from \cite{EA:decker24} 
this implies that the moduli space of our bundles is isomorphic to the homogeneous 
space $\SL(5, \CC)/N_5.$\qed
%\end{Example}
\end{example}

%\textbf{EISENBUD ENDE}

% \section*{References}

% use \cite{EA:MR92g:14013} instead.
% \noindent\textbf{Ancona, V. \& Ottaviani, G.}:
%         {\sl An introduction to derived categories and the theorem of Beilinson},
%         Atti Accademia Peloritana dei Pericolanti, Classe I de Scienze 
%         Fis. Mat. et Nat. LXVII, 99-110 (1989)

% use \cite{EA:MR57:324} instead
% \noindent\textbf{Barth, W.}: 
%         {\sl Moduli of vector bundles on the projective plane},
%         Invent. math. {\bf 42}, 63-91 (1977)


% use \cite{EA:MR80f:14005} instead
%\noindent\textbf{Barth, W. \& Hulek, K.}: 
%        {\sl Monads and Moduli of Vector Bundles}, 
%        manuscripta math. {\bf 25}, 323-347 (1978)

% use \cite{EA:MR80c:14010b} instead
% \noindent\textbf{Beilinson, A}:
%         {\sl Coherent sheaves on $\P^n$ and problems of linear algebra},
%         Funct. Anal. and its Appl. {\bf 12}, 214-216 (1978)

% use \cite{EA:MR80c:14010a} instead
% \noindent\textbf{Bernstein, Gel'fand, and Gel'fand}:
%         {\sl Algebraic bundles on $\P^n$ and problems of linear algebra},
%         Funct. Anal. and its Appl. {\bf 12}, 212-214 (1978)

% use \cite{EA:MR89g:13005:appendix} instead
% \noindent\textbf{Buchweitz, R.-O.}:
%         Appendix to Cohen-Macaulay modules on quadrics, by
%         R.-O. Buchweitz, D. Eisenbud, and J. Herzog. In
%         {\sl Singularities, representation of algebras,
%         and vector bundles} (Lambrecht, 1985),
%         Springer-Verlag Lecture Notes in Math, 1273, 96-116 (1987)


% use \cite{EA:MR97a:13001} instead
% \noindent\textbf{Eisenbud,  David}: 
%         {\sl Commutative Algebra with a View Toward Algebraic Geometry},
%         Springer Verlag, 1995

% use \cite{EA:MR1484973:eisenbud} instead
% \noindent\textbf{Eisenbud,  David}: 
%         {\sl Computing cohomology} in Chapter of 
%         ``Computational methods in Commutative Algebra and Algebraic Geometry'' 
%         by ~W.~Vasconcelos, Springer Verlag, Berlin, 1998

% use \cite{EA:Eisenbud-Schreyer:ChowForms} instead
% \noindent\textbf{Eisenbud, D.  \&  Schreyer, F.-O.}: 
%         {\sl Chow forms and free resolution}, in preparation 2001

% use \cite{EA:eis-sch:sheaf} instead
% \noindent\textbf{Eisenbud, D.  \&  Schreyer, F.-O.}: 
%         {\sl Sheaf Cohomology and Free Resolutions over Exterior Algebras},
%         AG/0005055, 2000

% use \cite{EA:MR81h:14014}% instead
% \noindent\textbf{Gieseker, D.}:
%         {\sl On the moduli of vector bundles on an algebraic surface},
%         Ann. of Math. {\bf 106}, 45-60 (1977)

% use \cite{EA:MR80m:14011}% instead
% \noindent\textbf{Hulek, Klaus}:
%         {\sl Stable Rank-2 Vector Bundles on $\P_2$ with $c_1$ odd},
%         Math. Ann. {\bf 242}, 241-266 (1979)

% use \cite{EA:MR30:120}% instead
% \noindent\textbf{Horrocks, G.}: 
%         {\sl Vector bundles on the punctured spectrum of a local ring},
%         Proc. London Math. Soc. (3), {\bf 14}, 689-713 (1964)

% use \cite{EA:MR84j:14026}% instead
% \noindent\textbf{Horrocks, G.}: 
%         {\sl Construction of bundles on $\PP^n$} in ``Les equations de Yang-Mills'',
%         by A. Douady, J.-L. Verdier (eds.), Asterisque {\bf 71-72}, 197-202 (1980)


% use \cite{EA:MR89g:18018}% instead
% \noindent\textbf{Kapranov, M. M.}:
%         {\sl On the derived categories of coherent sheaves on some 
%         homogeneous spaces}, Invent. Math. {\bf 92}, 479-508 (1988)

% use \cite{EA:MR80m:14012} instead
% \noindent\textbf{Le Potier, J.}:
%         {\sl Fibr\'es stables de rang 2 sur $\P_2 (\CC)$},
%         Math. Ann. {\bf 241}, 217-256 (1979)

% use \cite{EA:MR56:8567}% instead
% \noindent\textbf{Maruyama, M.}:
%         {\sl Moduli of stable sheaves I},
%         J. Math. Kyoto Univ. {\bf 17}, 91-126 (1977)

% use \cite{EA:MR82h:14011}% instead
% \noindent\textbf{Maruyama, M.}:
%         {\sl Moduli of stable sheaves II},
%          J. Math. Kyoto Univ. {\bf 18}, 557-614 (1978)


% use \cite{EA:MR81b:14001}% instead
% \noindent\textbf{Okonek, C., Schneider, M. \& Spindler, H.}:
%          {\sl Vector bundles on complex projective spaces}, Boston, 1980


% use \cite{EA:MR16:953c}% instead
% \noindent\textbf{Serre, J.P.}: 
%         {\sl Faisceaux alg\'ebriques coherents},
%         Ann. of Math. {\bf 61}, 197-278  (1955)

% use \cite{EA:MR99f:14064}% instead
% \noindent\textbf{Walter, Charles H.}:
%          {\sl Pfaffian subschemes}, 
%         J. Algebr. Geom. {\bf 5}, 671-704 (1996)

% % Local Variables:
% % mode: latex
% % mode: reftex
% % tex-main-file: "chapter-wrapper.tex"
% % reftex-keep-temporary-buffers: t
% % reftex-use-external-file-finders: t
% % reftex-external-file-finders: (("tex" . "make FILE=%f find-tex") ("bib" . "make FILE=%f find-bib"))
% % TeX-master: "~/M2BUCH/ComputationsBook/chapters/exterior-alge"
% % End:
