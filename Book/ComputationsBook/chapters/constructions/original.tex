\documentclass[12pt,leqno]{amsart}
\usepackage{fullpage,amsmath,amscd,amsthm,amssymb,amsxtra,latexsym}
%\usepackage{epsfig,epic,eepic,graphics,rotating}
\sloppy 
%\setlength{\parindent}{0pt} 
%\setlength{\parskip}{5pt plus  2pt minus 1pt} 
\topmargin+0.5cm
%\input matheb.mac
%\newcommand{\somespace}{\hfill{}\\ \vspace{-0.25cm}}
\pagestyle{headings}
\setcounter{secnumdepth}{3}
\setcounter{tocdepth}{3}

%\documentclass{book}
%\usepackage{amsmath,amscd,amsthm,amssymb,amsxtra,latexsym,epsfig,epic,eepic,graphics}

%\usepackage{amsmath,amscd,amsthm,amssymb,amsxtra,latexsym,epsfig,epic,graphics}
%\usepackage{diagrams}
\usepackage[matrix,arrow,curve]{xy}
\usepackage{pictexwd}


%%%%%%%%%%%%%%%%%%%%%%%%%%%%
%%%The black board font
%%%%%%%%%%%%%%%%%%%%%%%%%%%
\newcommand{\AAA}{{\mathbb A}}
\newcommand{\BB}{{\mathbb B}}
\newcommand{\CC}{{\mathbb C}}
\newcommand{\DD}{{\mathbb D}}
\newcommand{\EE}{{\mathbb E}}
\newcommand{\FF}{{\mathbb F}}
\newcommand{\GG}{{\mathbb G}}
%\newcommand{\HH}{{\mathbb H}}
\newcommand{\II}{{\mathbb I}}
\newcommand{\JJ}{{\mathbb J}}
\newcommand{\KK}{{\mathbb K}}
%\newcommand{\LL}{{\mathbb L}}
\newcommand{\MM}{{\mathbb M}}
\newcommand{\NN}{{\mathbb N}}
%\newcommand{{\mathbb O}}
\newcommand{\PP}{{\mathbb P}}
\newcommand{\QQ}{{\mathbb Q}}
\newcommand{\RR}{{\mathbb R}}
\renewcommand{\SS}{{\mathbb S}}
\newcommand{\Ss}{{\mathbf S}}
\newcommand{\TT}{{\mathbb T}}
\newcommand{\UU}{{\mathbb U}}
\newcommand{\VV}{{\mathbb V}}
\newcommand{\WW}{{\mathbb W}}
\newcommand{\XX}{{\mathbb X}}
\newcommand{\YY}{{\mathbb Y}}
\newcommand{\ZZ}{{\mathbb Z}}

\newcommand{\LL}{{\mathbb L}}
\newcommand{\HH}{{\mathbb H}}
\newcommand{\Syz}{{\rm{Syz}\;}}
\newcommand{\Sym}{{\rm{Sym}\;}}
\newcommand{\SSyz}{{\rm{Syz}}}
\newcommand{\spoly}{{\rm{spoly}}}
\newcommand{\Spe}{{Sp}}
\newcommand{\openC}{{\mathbb C}}
\newcommand{\ms}{{\rm{m}}}
\newcommand{\LS}{{\rm{L}}}
\newcommand{\IS}{{\rm{I}}}
\newcommand{\Loc}{{\rm{Loc}\,}}
\newcommand{\lcm}{{\rm{lcm}}}
\newcommand{\lc}{{\rm{lc}}}
\newcommand{\lm}{{\rm{lm}}}
\newcommand{\con}{{\rm{c}}}
\newcommand{\ext}{{\rm{e}}}
\newcommand{\ec}{{\rm{ec}}}
\newcommand{\ann}{{\rm{ann}}}
\newcommand{\Ext}{{\rm{Ext}}}
\newcommand{\equi}{{\rm{equi}}}
\newcommand{\Tor}{{\rm{Tor}}}
\newcommand{\rad}{{\rm{rad\;}}}
\newcommand{\ini}{{\rm{in}}}
\newcommand{\Hilb}{{\rm{Hilb}}}
\newcommand{\image}{{\rm{image}}}
\newcommand{\cliff}{{\rm{cliff}}}
\newcommand{\Pic}{{\rm{Pic}}}
\newcommand{\PR}{{\KK [x_1, \dots , x_n]}}
%%%%%%%%%%%%%%%%%%%%%%%%%%%%%
%%% new commands for calligraphic characters with amsmath
%%%%%%%%%%%%%%%%%%%%%%%%%%%%

\newcommand{\ka}{{\mathcal A}}
\newcommand{\kb}{{\mathcal B}}
\newcommand{\kc}{{\mathcal C}}
\newcommand{\kd}{{\mathcal D}}
\newcommand{\ke}{{\mathcal E}}
\newcommand{\kf}{{\mathcal F}}
\newcommand{\kg}{{\mathcal G}}
\newcommand{\kh}{{\mathcal H}}
\newcommand{\ki}{{\mathcal I}}
\newcommand{\kj}{{\mathcal J}}
\newcommand{\kk}{{\mathcal K}}
\newcommand{\kl}{{\mathcal L}}
\newcommand{\km}{{\mathcal M}}
\newcommand{\kn}{{\mathcal N}}
\newcommand{\ko}{{\mathcal O}}
\newcommand{\kp}{{\mathcal P}}
\newcommand{\kq}{{\mathcal Q}}
\newcommand{\kr}{{\mathcal R}}
\newcommand{\ks}{{\mathcal S}}
\newcommand{\kt}{{\mathcal T}}
\newcommand{\ku}{{\mathcal U}}
\newcommand{\kv}{{\mathcal V}}
\newcommand{\kw}{{\mathcal W}}
\newcommand{\kx}{{\mathcal X}}
\newcommand{\ky}{{\mathcal Y}}
\newcommand{\kz}{{\mathcal Z}}
%%%%%%%%%%%%%%%%%%%%%%%%%%%%%%
%%%The mathscript for sheaves
%%%%%%%%%%%%%%%%%%%%% %%%%%%%%%
\newcommand{\s}{\mathscr}
\newcommand{\sA}{{\s A}}
\newcommand{\sB}{{\s B}}
\newcommand{\sC}{{\s C}}
\newcommand{\sD}{{\s D}}
\newcommand{\sE}{{\s E}}
\newcommand{\sF}{{\s F}}
\newcommand{\sG}{{\s G}}
\newcommand{\sH}{{\s H}}
\newcommand{\sI}{{\s I}}
\newcommand{\sJ}{{\s J}}
\newcommand{\sK}{{\s K}}
\newcommand{\sL}{{\s L}}
\newcommand{\sM}{{\s M}}
\newcommand{\sN}{{\s N}}
\newcommand{\sO}{{\s O}}
\newcommand{\sP}{{\s P}}
\newcommand{\sQ}{{\s Q}}
\newcommand{\sR}{{\s R}}
\newcommand{\sS}{{\s S}}
\newcommand{\sT}{{\s T}}
\newcommand{\sU}{{\s U}}
\newcommand{\sV}{{\s V}}
\newcommand{\sW}{{\s W}}
\newcommand{\sX}{{\s X}}
\newcommand{\sY}{{\s Y}}
\newcommand{\sZ}{{\s Z}}



\newcommand{\cO}{{\s O}}
\newcommand{\cI}{{\s I}}
\newcommand{\cL}{{\s L}}
\newcommand{\cR}{{\s R}}
\newcommand{\cN}{{\s N}}
\newcommand{\cT}{{\s T}}
\newcommand{\cX}{{\s X}}
%%%%%%%%%%%%%%%%%%%%%%%%%%%%%%%%
%% Arrows
%%%%%%%%%%%%%%%%%%%%%%%%%%%%%%%
\newcommand{\inj}{\hookrightarrow}
%\newcommand{\surj}{\lra}
\newcommand{\lra}{\longrightarrow}
\newcommand{\lla}{\longleftarrow}
\newcommand{\lto}{\leftarrow}
%%%%%%%%%%%%%%%%%%%%%%%%%%%%%%%%%%%%
%\newcommand{\C}{\C}
%\newcommand{\openP}{\P}
\newcommand{\uf}{{\bf F}}
\newcommand{\uc}{{\bf C}}
\newcommand{\tensor}{\otimes}
\newcommand{\mi}{{\bf m}}
\newcommand{\tX}{\widetilde{X}}
\newcommand{\punkt}{\hspace{-.3ex}\raise.15ex\hbox to1ex{\Huge.}}
\newcommand{\tpunkt}{\hspace{-.3ex}\hbox to1ex{\Huge.}}
\newlength{\br}
\newlength{\ho}
\DeclareMathOperator{\GL}{GL}
\DeclareMathOperator{\Aut}{Aut}
\DeclareMathOperator{\Oo}{O}
\DeclareMathOperator{\Spec}{Spec}
\DeclareMathOperator{\Hom}{Hom}
\DeclareMathOperator{\syz}{syz}
\DeclareMathOperator{\ord}{ord}
\DeclareMathOperator{\word}{w\,ord}
\DeclareMathOperator{\supp}{supp}
\DeclareMathOperator{\Ker}{Ker}
\DeclareMathOperator{\im}{im}
%\DeclareMathOperator{\wdeg}{w\,deg}
\DeclareMathOperator{\depth}{depth}
\DeclareMathOperator{\gin}{gin}
\DeclareMathOperator{\Coker}{Coker}
\DeclareMathOperator{\NF}{NF}
\DeclareMathOperator{\pd}{pd}
\DeclareMathOperator{\SL}{SL}
\DeclareMathOperator{\SO}{SO}
\DeclareMathOperator{\Ort}{O}
\DeclareMathOperator{\Spez}{Sp}
\DeclareMathOperator{\PSL}{PSL}
\DeclareMathOperator{\PGL}{PGL}
\DeclareMathOperator{\wdim}{wdim}
\DeclareMathOperator{\cdim}{cdim}
\DeclareMathOperator{\cha}{char}
\DeclareMathOperator{\trdeg}{trdeg}
\DeclareMathOperator{\codim}{codim}
\DeclareMathOperator{\kdim}{kdim}
\DeclareMathOperator{\height}{height}
\DeclareMathOperator{\Ass}{Ass}
\DeclareMathOperator{\Lie}{Lie}
\renewcommand{\labelenumi}{(\arabic{enumi})}
\newcommand{\Ndash}{\nobreakdash--}% for pages 1\Ndash 9
%\newcommand{\somespace}{\hfill{}\\ \vspace{-0.7cm}}
\def\partitle#1{{\medskip\noindent {\bf #1.\hbox to 12pt{}}}} %calls cmsy10

%%%theosdefinitionen
\newcommand{\gm}{\mathfrak m}
\newcommand{\gM}{\mathfrak M}
\newcommand{\integer}{\ZZ}
\newcommand{\proj}{\PP}
\newcommand{\complex}{\CC}
\newcommand{\real}{\mathbb R}
\newcommand{\gp}{\mathfrak p}
\newcommand{\gq}{\mathfrak q}
\newcommand{\go}{\mathfrak so}
%\newcommand{\openF}{\F}

%%%%%%%%%%%%%%%BIBLIOGRAPHY
\newcommand{\by}{}
\newcommand{\paper}{: \begin{it}}
\newcommand{\jour }{, \end{it}}
\newcommand{\vol}{\begin{bf} }
\newcommand{\yr}{\end{bf}(}
\newcommand{\pages}{),}

%\newcommand{\endref}{}
% Local Variables:
% mode: latex
% TeX-master: "tot"
% End:

\title[Algebraic geometry constructions ...]
{Algebraic geometry constructions and investigations with small finite fields}

\author{ Frank-Olaf Schreyer and Fabio Tonoli}
 
\address{Fakult\"at f\"ur Mathematik und Physik,
Universit\"at Bayreuth,
D-95440 Bayreuth,
Germany}
\email{schreyer@btm8x5.mat.uni-bayreuth.de,
tonoli@btm8x5.mat.uni-bayreuth.d } 

\subjclass{14}
\keywords{curves, unirationality, syzygies, Calabi-Yau threefolds, finite fields}
\date{March 30,2000}
\begin{document}

\newtheorem{lemma}{Lemma}[section]
\newtheorem{proposition}[lemma]{Proposition}
\newtheorem{theorem}[lemma]{Theorem}
\newtheorem{corollary}[lemma]{Corollary}
\newtheorem{conjecture}[lemma]{Conjecture}
%\newtheorem{satz}{Satz}[subsection]

\theoremstyle{definition}
\newtheorem{definition}[lemma]{Definition}
\newtheorem{remark}{Remark}[section]
\newtheorem{example}{Example}[section]
\newtheorem{exercise}{Exercise}[section]
\newtheorem{algorithm}{Algorithm}[subsection]
\newtheorem{sub}[subsubsection]{}



\maketitle

\begin{abstract}
\end{abstract}
 
\tableofcontents



\section{Introduction}

The advances in speed of modern computers and computer algebra systems gave 
life to the idea of solving  equations by guessing a solution:
Suppose $\MM \subset \GG$ is a subvariety of a rational variety
of codimension $c$. Then we expect, that the chances for a point 
$p \in \GG(\FF_q)$ to lie in $\MM(\FF_q)$ is about $1:q^c$. Here $\FF_q$
denotes the field with $q$ elements.  

We will discuss this idea in the following setting: $\MM$ will be 
a parameter space for objects in algebraic geometry, eg. $\MM$ could be
a Hilbert scheme or a moduli space, or a space dominating such spaces.

The most basic question we might have in this case is, whether $\MM$ is 
non-empty and whether an open part of $\MM$ corresponds to smooth objects.

Typically in these cases,
we will not have explite equations for $\MM \subset \GG$
but only an implicite algebraic description of $\MM$, and our approach will
be successful, if the cpu time to check $p \notin \MM(\FF_q)$ is sufficiently
small compared to $q^c$.
The first author applied this method first in \cite{Sch1} to construct some
rational surfaces in $\PP^4$, cf. \cite{ElPe,DS} for motivation.


In this first section we describe a program which picks curve of genus
$g \le 14$ at random. The moduli spaces $\gM_g$ are known to be unirational
for $g\le 13$ \cite{Se,CR}. 

Our approach which is based on this result
can viewed as a computer aided proof of the unirationality. 
Many people might object 
that this not a proof because we cannot control every single step in the
computation. We however think that such a proof is much more reliable then
human mind driven proof.
A mistake in a computer aided approach most often leads to an output far 
away from our expectation, hence is easy to spot.  

A substancial
improvement of nowadays computers and computer algebra system would indeed
lead to an unirational parameterization of $\gM_g$ for $g\le 13$.
\medskip

In the second part we apply our ``random curves'' to probe into
the consequences of Green's conjecture on syzygies of canonical curves, 
and compare these results with the correponding statements for 
``Coble self-dual'' sets of $2g-2$ points in $\PP^{g-2}$. 
\medskip

In the last section we exploit our method to prove the existence of three
components of the Hilbert scheme of Calabi-Yau 3-folds of degree $17$ in 
$\PP^6$. This is one of the  main results of the second authors thesis. 
Calabi-Yau threefolds of lower degree in $\PP^6$ are easy to construct, 
using the Pfaffian construction and a study of their Hartshorne-Rao modules.
For degree $17$ the Hartshorne-Rao module has to satisfy a subtle condition. 

 



\section{How to make random curves up to of genus $14$.}

The moduli space of curves $\gM_g$ is known to be of general type for 
$g \ge 24$ and has non-negative Kodaira dimension for $g=23$ by work of 
Harris, Mumford and Eisenbud \cite{HM,EH}. 
For genus $g \le 13  $ unirationality is known \cite{CR,Se}.
In this section we present a Macaulay program which over a finite field $\FF_q$ 
picks a point in $\gM_g(\FF_q)$  for $g \le 14$ at random. 

By Brill-Noether theory \cite{ACGH} every curve of genus $g$ has a linear 
system $g^r_d$ of dimension $r$ and degree $d$ provided the Brill-Noether number
$$\rho := \rho(g,d,r) := g-(r+1)(g-d+r) \ge 0.$$
We utilize this to find appropriate (birational) models for general curves of genus $g$.



\subsection{Plane models, $g\le 10$}

This case was known to Severi, cf \cite{AC}. Choose $d = g+2 - \lfloor \frac g  3 \rfloor$. 
Then $\rho(g,d,2) \ge 0$
ie. a general curve of genus g has a plane model $C'$ of degree d. 
We expect that $C'$ has 
$$\delta= \binom{d-1}{2} - g$$
double points. 
If the double points are in general position, then 
$$s=h^0(\PP^2,\ko(d))-3\delta-1$$
is the expected dimension of the linear system of curves of degree $d$ 
with $\delta$ assigned double points.  
We have the following table:
$$
\begin{tabular}{l|lllll lllll ll}
g        & 1 & 2 & 3 & 4 & 5 & 6 & 7 & 8 & 9 & 10 & 11 & 12 \cr
\hline
$\rho$      & 1 & 2 & 0 & 1 & 2 & 0 & 1 & 2 & 0 & 1  &  2 & 0 \cr
\hline
d        & 3 & 4 & 4 & 5 & 6 & 6 & 7 & 8 & 8 & 9 & 10 & 10 \cr
\hline
$\delta$ & 0 & 1 & 0 & 2 & 5 & 4 & 8 & 13 & 12 & 18 & 25 & 24 \cr
\hline
s        & 9 & 11 & 14 & 14 & 12 & 15 & 11 & 5 & 8 & 0 & -10 & -7 \cr
\end{tabular}
$$
Thus for $g\le 10$ we assume that these double points lie in general position. 
For $g>10$ this cannot be the case because $h^0(\PP^2,\ko(d))-3\delta \le0$. 
Since it is difficult to describe the special locus $H_\delta(g) \subset \Hilb_\delta(\PP^2)$ 
of double points of nodal genus $g$ curves, the plane model approach collapses for $g>10$.

\partitle{Random points}
In our program which picks plane models at random from an Zariski open subspace of $\gM_g$ 
we start by picking the nodes. 
However over a small field $\FF_q$ it is not a good idea to pick points individually, 
because there might be simply too few: $|\PP^2(\FF_q)|=1+q+q^2$.

What we should do is, to pick a collection $\Gamma$ of $\delta$ points defined over $\FF_q$.
General points in $\PP^2$ satisfy the minimal resolution condition, 
that is have expected syzygies.
If the ideal of such $\Gamma$ has generators in minimal degree $k$,
then 
$\binom{k+1}2 \le \delta < \binom{k+2}2$,
which gives $\delta= \binom{k+1}{2} + \epsilon$  with  $0 \le \epsilon \le k$.
Thus $k=\lceil \frac{-3+\sqrt{9+8\delta}}{2} \rceil$.
The Betti table is as follows depending on the sign of $2\epsilon-k$:

\medskip 
$2\epsilon \le k :$ \;
\begin{tabular}{r|lll}\hline
0 & 1 &- & - \cr
1 & - & - & - \cr
\vdots & \vdots & \vdots & \vdots \cr
$k-2$ & - & - & - \cr
$k-1$ & - & $k+1-\epsilon$ & $k-2\epsilon$ \cr
$k$ & - & - & $\epsilon$ \cr
\end{tabular}
$\qquad2\epsilon \ge k :$ \;
\begin{tabular}{r|lll}\hline
0 & 1 &- & - \cr
1 & - & - & - \cr
\vdots & \vdots & \vdots & \vdots \cr
$k-2$ & - & - & - \cr
$k-1$ & - & $k+1-\epsilon$ & - \cr
$k$ & - & $2\epsilon-k$ & $\epsilon$ \cr
\end{tabular}
\medskip

\noindent
So we can specify a collection $\Gamma$ of $\delta$ points by picking 
the Hilbert-Burch matrix of their resolution, cf. \cite{Ei}{Thm 20.15}. 
This is a matrix with linear and quadratic entries only, 
whose minors of size $k-\epsilon$ respectively $\epsilon$ 
generate the homogeneous ideal of $\Gamma$.


{\scriptsize\begin{verbatim}
randomPlanePoints = (delta,R) -> (
     k:=ceiling((-3+sqrt(9.0+8*delta))/2); eps:=delta-binomial(k+1,2);
     if k-2*eps>=0 
     then minors(k-eps,random(R^(k+1-eps),R^{k-2*eps:-1,eps:-2}))
     else minors(eps,random(R^{k+1-eps:0,2*eps-k:-1},R^{eps:-2})));
\end{verbatim}}

\noindent
In unlucky cases these points might be infinitesimally near.

{\scriptsize\begin{verbatim}
distinctPoints = (J) -> (
     singJ:=minors(2,jacobian J)+J;
     codim singJ);
\end{verbatim}}


\medskip \noindent
The procedure that returns the ideal of a random nodal curve is then straightforward: 

{\scriptsize\begin{verbatim}
randomNodalCurve = method();

randomNodalCurve (ZZ,ZZ,Ring) := (d,g,R) -> (
     delta:=binomial(d-1,2)-g;
     K:=coefficientRing R;
     if (delta==0) 
     then (     --no double points
          ideal random(R^1,R^{-d}))
     else (      --delta double points            
          Ip:=randomPlanePoints(delta,R);
          --choose the curve
          Ip2:=saturate Ip^2;
          ideal (gens Ip2 * random(source gens Ip2, R^{-d}))));

isNodalCurve = (I) -> (
          singI:=ideal jacobian I +I;
          delta:=degree singI;d:=degree I;g:=binomial(d-1,2)-delta;
          {distinctPoints(singI),delta,g});
\end{verbatim}}


We next ask if we indeed get in this way points in a parameter space 
which dominates $\gM_g$ for $g \le 10$.
A naive dimension count suggests that this should be true:
indeed the dimension of our parameter space is given by $2\delta +s$, 
which is $3(g-1)+\rho+\dim\PGL(3)$, as it should be.
Anyway to conclude this there is more to verfify: 
It could be that the nodal models of general curve have double points 
in special position, 
while all curve constructed above lie over a subvariety of $\gM_g$. 
One way to exclude this is to prove that $G(g,d,2) \to \gM_g$ is irreducible, 
or to put it differently, that the Severi Conjecture holds true:

\begin{theorem}\cite{Ha1}. The space
of nodal degree d genus g curves in $\PP^2$ is irreducible.
\end{theorem}

Another much easier proof for the few $(d,g)$ we are interested in is to
establish that our parameter space $\MM$ of the construction is smooth of 
expected dimension in our random point $p \in \MM$ as in \cite{AC}.
Consider the following diagram:
$$\xymatrix{\HH = {\Hilb}_{(d,g)}/\Aut(\PP^2) \stackrel{\pi}{\lra} \gM_g}.$$
For a given curve ${\tilde C}\in\gM_g$, the inverse image $\pi^{-1} ({\tilde C})$ 
consists of the variety $W^2_d({\tilde C})\subset\Pic^d({\tilde C})$.
Moreover the choice of a divisor $L\in W^2_d({\tilde C})$ is equivalent to the one 
of $p \in \MM$, modulo $\Aut(\PP^2)$: 
indeed $p$ determines a morphism $\nu\colon {\tilde C} \lra C\subset \PP^2$ 
and a line bundle $L=\nu^{-1}\ko_{\PP^2}(1)$, where ${\tilde C}$ is the 
normaliziation of the (nodal) curve $C$.
Therefore $\MM$ is smooth of expected dimension $3(g-1)+\rho+\dim \PGL(3)$
in $p \in \MM$ 
iff $W^2_d({\tilde C})$ is smooth of expected dimension $\rho$ in $L$.
This is well known to be equivalent to the injectivity of the multiplication map
$\mu_L$
$$
H^0(L)\otimes H^0(K_{\tilde C}\otimes L^{-1}) \stackrel{\mu_L}{\lra}H^0(K_{\tilde C}),
$$
which can be easily checked in our cases,
cf. \cite[p. 189]{ACGH}.
Indeed in our cases $\mu_L$ can be rewritten as 
$$
H^0(\ko_{\PP^2}(1))\otimes H^0(I_\Gamma(d-4)) \stackrel{\mu_L}{\lra}
H^0(I_\Gamma(d-3)).
$$
So we need two conditions: 
\begin{enumerate}
\item $H^0(I_\Gamma(d-5))=0$;
\item there are no linear relations among the generators of $H^0(I_\Gamma(d-4))$ of 
degree $d-3$.
\end{enumerate}
Let us proceed case by case.
For genus $g\leq 5$ this is clear, since $H^0(I_\Gamma(d-4))=0$ for $g=2,3$ and
$H^0(I_\Gamma(d-3))=1$ for $g=4,5$. 
For $g=6$ $H^0(I_\Gamma(d-3))=H^0(I_\Gamma(2))=2$ and 
the Betti numbers of $\Gamma$ 
$$
\begin{tabular}{|lll}\hline
1 & - & - \cr
- & 2 & - \cr
- & - & 1 \cr
\end{tabular}
$$
shows there are no relations 
with linear coefficients in $H^0(I_\Gamma(2))$.
For $7\leq g \leq 10$ the method is similar:
everything is clear once the betti table of resolution of the set of nodal points 
$\Gamma$ is computed. As a further example we do here the case $g=10$:
$H^0(I_\Gamma(d-3))=H^0(I_\Gamma(5))=3$ and the Betti numbers of $\Gamma$ are
$$
\begin{tabular}{|lll}\hline
1 & - & - \cr
- & - & - \cr
- & - & - \cr
- & - & - \cr
- & 3 & - \cr
- & 1 & 3 \cr
\end{tabular}
$$
from which it is clear there are no linear relations between the quintic generators
of $I_\Gamma$.

%% and that the 
%% Kodaira-Spencer map
%% $$T_p(\MM) \to T_C(\gM_g) = H^0(C,\omega^{\tensor 2})^*$$
%% is surjective. This is a first order deformation problem, which can be easily 
%% solved by linear algebra in Macaulay2:
%% 
%% {\scriptsize
%% \begin{verbatim}
%% Kodaira-SpencerMap = (I) -> ???
%% \end{verbatim}}


\subsection{Space models and Hartshorne-Rao modules}\  

\partitle{The case of genus $g=11$}
In this case we have $\rho(11,12,3)=3$. 
Hence every general curve of genus 11 has a space model of degree 12. 
Moreover for a general curve the general space model of this degree 
is linearly normal, because
 $\rho(11,13,4)=-1$ takes smaller value. 
If moreover such curve $C \subset \PP^3$ has maximal rank,
i.e. for each $m \in \ZZ$ the map 
$$H^0(\PP^3,\ko(m)) \to H^0(C,\ko_C(m))$$
has maximal rank,
then the Hartshorne-Rao module $M=H^1_*(\ki_C)=\oplus_m H^1(\PP^3,\ki_C(m))$ 
has Hilbert function $(0,0,4,6,3,0,\dots)$.
Consider the  vector bundle $\kg$ on $\PP^3$ associated to the first syzygy 
module of $I_C$: 
$$ 0 \lto \ki_C \lto \oplus_i \ko(-a_i) \lto \kg \lto 0 \leqno(1)$$
In this set-up $H^2_*(\kg)=H^1_*(\ki_C)$. 
Thus $\kg$ is up to direct sum of line bundles the sheafified second syzygy 
module of $M$.

Expected syzygies of $M$ are
$$
\begin{tabular}{|lllll}\hline
4 & 10 & 3 & - & - \cr
- & - & 8 & 2 & - \cr
- & - & - & 6 & 3 \cr
\end{tabular}
$$
Thus the $\FF$-dual $M^* = \Hom_\FF(M,\FF)$ is presented as 
$\FF[x_0,\dots,x_3]$-module by a 
$3 \times 8$ matrix with linear and quadratic entries, and a general such matrix 
will give a general module
(if the construction works, ie. if the desired space of modules is non-empty), 
because all conditions we put are semi-continuous and open.
Thus $M$ depends on 
$$\dim \GG(6,3*h^0\ko(1))+\dim \GG(2,3*h^0\ko(2)-6*h^0\ko(1))-\dim SL(3)= 6\cdot6+2\cdot4-8=36$$
parameters. 

Assuming that $C$ has minimal possible syzygies:
$$
\begin{tabular}{|llll}\hline
1 & - & - & - \cr
-&-&-&-\cr
-&-&-&-\cr
-&-&-&-\cr
-&6&2&-\cr
-&-&6&3\cr
\end{tabular}
$$
we obtain dualizing the sequence (1) in our case
$$ \kg^* {\lto  } 6\ko(5) \lto \ko \lto 0$$
Thus if every thing is as expected, the entries of the right hand matrix are
the homogeneous polynomials, which generate $I_C$. 
We will compute $I_C$ by determining $ \ker (\phi \colon 6\ko(5) \to \kg^*)$.
Comparing with the syzygies of $M$ we see that
$$\kg^* \cong \ker(2\ko(6)\oplus6\ko(7) \to 3\ko(8)) \cong 
\image(3\ko(4)\oplus8\ko(5) \to 2\ko(6)\oplus6\ko(7))$$
and $\kg^* \lto 6\ko(5)$ factors over $\kg^* \lto 8\ko(5)\oplus3\ko(4)$.
A general $\phi \in \Hom(6\ko(5),\kg^*)$ gives a point in $\GG(6,8)$
and the Hilbert scheme of  desired curves would have dimension 
$36+12=48=4\cdot12=30+3+15$ as expected, cf \cite{Ha2}.

\medskip
Therefore the computation for obtaining a random space curve of genus $11$ 
is done as follows:

{\scriptsize
\begin{verbatim}
randomGenus11Curve = (R) -> (
     correctCodimAndDegree:=false;
     while not correctCodimAndDegree do (
          Mt=coker random(R^{3:8},R^{6:7,2:6});
          M=coker (transpose (res Mt).dd_4);
          Gt:=transpose (res M).dd_3;
          I:=ideal syz (Gt*random(source Gt,R^{6:5}));
          correctCodimAndDegree=(codim I==2 and degree I==11););
     I);
\end{verbatim}}

\medskip
In general for these problems there is rarely an  apriori reason, 
why such construction for general choices will give a smooth curve. 
Kleimann's global generation condition \cite{Klei} is a much too
strong hypothysis for many interesting examples. 
But it is easy to check an example over a finite field with a Computer:

{\scriptsize
\begin{verbatim} 
isSmoothSpaceCurve = (I) -> (
     singI:=I+minors(codim I,jacobian I);
     codim singI==4);
\end{verbatim}}

Hence by semi-continuity this is true over $\QQ$ and the
desired unirationality of $G(11,12,3)/\gM_{11}$ holds for all fields 
with possibly a finite number of
exceptional characteristic of the ground field. 

A calculation of an example over the integers
would bound the number of exceptional characterics, 
which then can be ruled out case by case, 
or by considering sufficiently many integer examples.

As in case of nodal curves, to prove unirationality of $\gM_{11}$ by Computer
aided computations we have to show the injectivity of
$$
H^0(L)\otimes H^0(K_{C}\otimes L^{-1}) \stackrel{\mu_L}{\lra}H^0(K_{C}),
$$
where $L$ is the restriction of $\ko_{\PP^3}(1)$ to the curve $C\subset \PP^3$.
The few following lines do the job:
{\scriptsize
\begin{verbatim} 
K=ZZ/101
R=K[x_0..x_3];
C=randomGenus11Curve(R);
isSmoothSpaceCurve(C)
Omega=prune Ext^2(coker gens C,R^{-4});
betti Omega
\end{verbatim}}

\noindent
We see that there are no linear relations among the two generators
of $H^0_*(\Omega_C)$ in degree -1.

%% to prove uniratioanlity of $\gM_{11}$ by Computer
%% aided computations we have to show that the Kodaira-Spencer map
%%         $$ H^0(C,\kn_C) \to H^1(C,\kt_C) $$
%% is onto in our example. 
%% The following procedure returns the dimension 
%% of the image of the Kodaira-Spencer map for a smooth projective variety
%% (of codimension at least 2,
%% otherwise $\kn_C^*=${\tt I2} is free and {\tt T} is just {\tt coker djt}):
%% 
%% {\scriptsize
%% \begin{verbatim} 
%% kodairaSpencerMap = (I) -> (
%%      S:=R/I;
%%      dIt:=substitute(transpose jacobian I,S);
%%      I2:=substitute(syz gens I,S); --the conormal bundle
%%      T:=prune(kernel transpose I2/image dIt);
%%      hilbertFunction(0,T))
%% \end{verbatim}}
%% THIS SCRIPT FORGETS THE CONTRIBUTION OF H1 IN THE SEQUENCE GIVING T_P|C



\partitle{Betti numbers for genus $g=12,13,14,15$}
The approach in these cases is similar to $g=11$. We choose here $d=g$, 
so $\rho(g,g,3) \ge 0$ (resp. $1,2,3$). 
Under the maximal rank assumption the corresponding space curve have 
a Hartshorn-Rao module with Hilbert function respectively 
$(0,0,g-9,2g-19,3g-34,0,\ldots)$ in case $g=12,13$ 
and $(0,0,g-9,2g-19,3g-34,4g-55,0,\ldots)$ in case $g=14,15$.
Expected syzygies of $M$ are

$$g=12: \;
\begin{tabular}{|lllll}\hline
3 & 7 & - & - & -  \cr
- & - & 10 & 5 & - \cr
- & - & - & 3 & 2  \cr
\end{tabular}
\qquad
g=13: \;
\begin{tabular}{|lllll}\hline
4 & 9 & 1 & - & -  \cr
- & - & 6 & - & - \cr
- & - & 6 & 13 & 5  \cr
\end{tabular}
$$
\medskip
$$g=14: \;
\begin{tabular}{|lllll}\hline
5 & 11 & 2 & - & -  \cr
- & - & 3 & - & - \cr
- & - & 13 & 17 & 4  \cr
- & - & - & - & 1 \cr
\end{tabular}
\qquad
g=15: \;
\begin{tabular}{|lllll}
\hline
6 & 13 & 3 & - & -  \cr
- & - & 3 & - & - \cr
- & - & 8 & 3 & -  \cr
- & - & - & 9 & 5 \cr
\end{tabular}
$$

\medskip
Comparing with the expected syzygies of $C$ 
$$g=12: \;
\begin{tabular}{|llll}\hline
 1 & - & - & -  \cr
 - & - & - & -  \cr
 - & - & - & -  \cr
 - & - & - & -  \cr
 - & 7 & 5 & - \cr
 - & - & 3 & 2  \cr
\end{tabular}
\qquad
g=13: \;
\begin{tabular}{|llll}\hline
 1 & - & - & -  \cr
 - & - & - & -  \cr
 - & - & - & -  \cr
 - & - & - & -  \cr
 - & 3 & - & - \cr
 - & 6 & 13 & 5  \cr
\end{tabular}
$$
\medskip
$$g=14: \;
\begin{tabular}{|llll}\hline
 1 & - & - & -  \cr
 - & - & - & -  \cr
 - & - & - & -  \cr 
- & - & - & -  \cr 
- & - & - & -  \cr
 - & 13 & 17 & 4  \cr
 - & - & - & 1 \cr
\end{tabular}
\qquad
g=15: \;
\begin{tabular}{|llll}
\hline
 1 & - & - & -  \cr 
 - & - & - & -  \cr
- & - & - & -  \cr
 - & - & - & -  \cr
 - & - & - & -  \cr
 - & 8 & 3 & -  \cr
 - & - & 9 & 5 \cr
\end{tabular}
$$
we see that given  $M$ the choice of a curve correspond to a point
in $\GG(7,10)$ or $\GG(3,6)$ for $g=12,13$ respectively, 
while for $g=14,15$ every thing is determined by the Hartshorne-Rao module. 
For $g=12$ Kleiman's result guarantees smoothness for general choices,
in contrast to the more difficult cases $g=14,15$. 
So the construction of $M$ is the crucial step. 




\partitle{Construction of Hartshorne-Rao modules}
In case $g=12$ the construction of $M$ is straight forward. 
It is presented by sufficiently general matrix of linear forms:
$$0 \lto M \lto 3S(-2) \lto 7S(-3).$$

\noindent
The procedure for obtaining a random genus 12 curve is:
{\scriptsize
\begin{verbatim} 
randomGenus12Curve = (R) -> (
     correctCodimAndDegree:=false;
     while not correctCodimAndDegree do (
          M:=coker random(R^{3:-2},R^{7:-3});
          Gt:=transpose (res M).dd_3;
          I:=ideal syz (Gt*random(source Gt,R^{7:5}));
          correctCodimAndDegree=(codim I==2 and degree I==12););
     I);
\end{verbatim}}


\medskip
In case $g=13$ we have to make sure that M has a second linear syzygy. 
Consider the end of the Koszul complex:
$$6R(-2) \stackrel{\kappa}{\lto} 4R(-3) \lto R(-1) \lto 0.$$
Any product of a general map $4R(-2) \stackrel{\alpha}{\lto} 6R(-2)$ with 
the Koszul matrix $\kappa$ yields
$4R(-2) \lto 4R(-3)$ with a linear syzygy, 
and concatenated with a general map $4R(-2) \stackrel{\beta}{\lto} 5R(-3)$
gives a presentation matrix of a module M of desired type:
$$ 0 \lto M \lto 4R(-2) \lto 4R(-3) \oplus 5R(-3).$$

\noindent
The procedure for obtaining a random genus 13 curve is:
{\scriptsize
\begin{verbatim} 
randomGenus13Curve = (R) -> (
     kappa:=koszul(3,vars R);
     correctCodimAndDegree:=false;
     while not correctCodimAndDegree do (
          test:=false;while test==false do ( 
               alpha:=random(R^{4:-2},R^{6:-2});
               beta:=random(R^{4:-2},R^{5:-3});
               M:=coker(alpha*kappa|beta);
               test=(codim M==4 and degree M==16););
          Gt:=transpose (res M).dd_3;
          --up to change of basis we can reduce to this form of phi
          phi:=random(R^6,R^3)++id_(R^6);
          I:=ideal syz(Gt_{1..12}*phi);
          correctCodimAndDegree=(codim I==2 and degree I==13););
     I);
\end{verbatim}}


\medskip
The case of genus $g=14$ is about a magnitude more difficult. 
To start with we can achieve 2 second linear syzygies by the same method as 
in case $g=13$. 
For a general matrix 
$5R(-2) \stackrel{\alpha}{\lto} 12R(-2)$ composed with 
$12R(-2) \stackrel{\kappa \oplus \kappa}{\lto} 8R(-3)$ 
yields the first component of
$$5R(-2) \lto (8+3)R(-3).$$
For a general choice of the second component 
$5R(-2) \stackrel{\beta}{ \lto} 3R(-3)$ the cokernel
will be a module with Hilbert function $(0,0,5,9,8,0,0,\ldots)$ and syzygies
$$
\begin{tabular}{|lllll}\hline
5 & 11 & 2 & - & -  \cr
- & - & 2 & - & - \cr
- & - & 17 & 23 & 8  \cr
- & - & - & - & - \cr
\end{tabular}
$$
What we want is to find $\alpha$ and $\beta$ such that $\dim M_5 = 1$ 
and $\dim Tor_2^R(M,\FF)_5 = 3$.
Taking into account that we ensured $\dim \Tor_2^R(M,\FF)_4 =2$
this amounts to ask that the $100 \times 102$ matrix $m(\alpha,\beta)$ 
obtained from
$$[0 \lto 5R(-2)_5 \lto 11R(-3)_5 \lto 2R(-4)_5 \lto 0] 
\cong [0 \lto 100 \FF \stackrel{m(\alpha,\beta)} \lto 102 \FF \lto 0]$$
drops rank by 1. 
We do not know of a systematic approach how to produce such 
$m(\alpha,\beta)'s$.
However this is a condition of codimension 3, hence in a finite field 
$\FF=\FF_q$ we can hope to find points at a rate of $(1:q^3)$. 
The speed to detect bad modules is rather fast, 
and can be easily computed by the following code:
{\scriptsize
\begin{verbatim} 
testModulesForGenus14Curves = (N,p) ->(
     R:=(ZZ/p)[x_0..x_3];
     i:=0;j:=0;
     kappa:=koszul(3,vars R);
     kappakappa:=kappa++kappa;
     utime:=timing while (i<N) do (
          test:=false;
          alpha:=random(R^{5:-2},R^{12:-2});
          beta:=random(R^{5:-2},R^{3:-3});
          M:=coker (alpha*kappakappa|beta);
          fM:=res (M,DegreeLimit =>3);
          if (tally degrees fM_2)_{5}==3 then (
               --further checks to be sure to pick up the right module
               test=(tally degrees fM_2)_{4}==2 and
               codim M==4 and degree M==23;);
          i=i+1;if test==true then (j=j+1;););
     timeForNModules:=utime#0; numberOfGoodModules:=j;
     {timeForNModules,numberOfGoodModules})     
\end{verbatim}}

\noindent
For timing tests we use a Pentium2 400Mhz with 128Mb of memory running linux.
On such a machine examples can be tested at a rate 
of $0.037 \; \frac{seconds}{example}$.
Hence an approximate estimation of the CPU-time required to find a good example 
is $q^3 \cdot 0.037 \; seconds$.
Comparing this with the time to verify smoothness, 
which is about $12 \; seconds$ for an example of this degree,
we see that up to $|\FF_q|=q \le 13$ we can expect to obtain examples within 
few minutes.
Actually $q=2,3$ take longer then $q=5$ in the average, 
because examples of ``good modules'' tend to give singular curves more often. 
We report here a statistic table which summarizes the situation:
$$
\begin{tabular}{c|rrrrrrr}
p&                              2   &3   &5   &7   &11  &13\cr
\hline
smooth curves &                 100 &100 &100 &100 &100 &100\cr
1-nodal curves &                75  &53  &31  &16  &10  &8\cr
reduced more singular &         1012&142 &24  &11  &2   &0\cr
non reduced curves &            295 &7   &0   &0   &0   &0\cr
\hline
total number of curves &        1482&302 &155 &127 &112 &108\cr
percentage of smooth curves &   6.7\% &33\%  &65\%  &79\%  &89\% &93\%\cr
approx. time (in seconds) &     7400&3100&2700&3400&6500&9500\cr
\end{tabular}
$$

\medskip
\noindent
The procedure for obtaining a random genus 14 curve is
{\scriptsize
\begin{verbatim} 
randomGenus14Curve = (R) -> (
     kappa:=koszul(3,vars R);
     kappakappa:=kappa++kappa;
     correctCodimAndDegree:=false;
     count:=0;while not correctCodimAndDegree do (
          test:=false;
          t:=timing while test==false do (
               alpha=random(R^{5:-2},R^{12:-2});
               beta=random(R^{5:-2},R^{3:-3});
               M:=coker (alpha*kappakappa|beta);
               fM:=res (M,DegreeLimit =>3);
               if (tally degrees fM_2)_{5}==3 then (
               --further checks to be sure to pick up the right module
               test=(tally degrees fM_2)_{4}==2 and
               codim M==4 and degree M==23;);
          count=count+1;);
     Gt:=transpose (res M).dd_3;
     I:=ideal syz (Gt_{5..17});
     correctCodimAndDegree=(codim I==2 and degree I==14););
     <<"    --"<<t#0<<" seconds used for "<<count<<" modules"<<endl;
     I);
\end{verbatim}}


\medskip
For $g=15$ we  do not know of a method along these lines, 
which would give examples over small fields.



\partitle{Counting parameters}
For genus $g=12$ clearly $M$ depends on 
$\dim\GG(7,3*h^0\ko(1))-\dim SL(3)= 7\cdot5-8=36$ parameters, 
and the family of curves has dimension $36+\dim\GG(7,10)=48=4\cdot12=33+0+15$
as expected. 

For genus $g=13$ and $14$ the parameter count is more difficult.
Let us make a careful parameter count for genus $g=14$: 
the case $g=13$ is similar and easier.
The choice of $\alpha$ corresponds to a point in $\GG(5,12)$.
Then $\beta$ correspond to a point $\GG(B_\alpha,3)$ where 
$B_\alpha = U \tensor R_1/<\alpha>$ where $U$ denotes
the universal subbundle on $\GG(5,12)$ and $<\alpha>$ 
the subspace generated by the 8 columns of $\alpha\circ(\kappa \oplus \kappa)$. 
So $\dim B_\alpha = 20-8=12$ and $\GG(B_\alpha,3) \to \GG(5,12)$ 
is a Grassmannian bundle with fiber dimension 27 and total dimension 62. 
In this space the scheme of good modules has codimension 3. 
So we get a 59 dimensional family. 
This is larger than the expected dimension $56=4\cdot 14=39+2+15$ 
of the Hilbert scheme, cf. \cite{Ha2}. 
Indeed the construction gives a curve together
with a basis of $\Tor_2^R(M,\FF)_4$. 
Subtracting the projective (!) coordinate changes we arrive at 
the desired dimension $59-3=56$. 

The unirationality of $\gM_{12}$ and $\gM_{13}$ can be proved by computer 
as in case $\gM_{11}$, 
while in case $g=14$ we don't know the rationality of 
the parameter space of the modules $M$ with 
$\dim M_5=1$ and $\Tor^R_2(M,\FF)_5=3$.






\section{Comparing Green's conjecture for curves and points.}

\subsection{Syzygies of canonical curves.}
One of the most outstanding conjectures about free resolution is 
Green's prediction
for the syzygies of canonical curves. 

A canonical curve
$C \subset \PP^{g-1}$, ie. a linearly normal curve with $\ko_C(1) \equiv 
\omega_C$, the canonical line bundle, is projectively normal by a result of
Max Noether, and hence has a Gorenstein homogenous coordinate ring and is 
3-regular. The
Betti numbers of the free resolution are consequently symmetric and only
two rows of Betti numbers occur.
$$
\beginpicture  
\setlinear
% Koordinatensystem
\setcoordinatesystem units <15mm,6mm> %point at -150 0.5   
\unitlength15mm
\setplotarea x from 0 to 8, y from 0 to 4
% Gitter
\axis bottom 
   ticks andacross length <0pt> unlabeled from 0 to 8 by 1 /
\axis left 
   ticks andacross length <0pt> unlabeled from 0 to 4 by 1 /
%Schattierung
\setplotsymbol(.)
\setshadegrid span <1pt>
\hshade  
3     0     1 
4     0     1 /
\hshade
2     1     5
3     1     5 /
\hshade
1     3     7
2     3     7 /
\hshade
0     7     8
1     7     8 /
% Bettizahlen & Symetriepunkt
\put {$1$} at 0.5 3.5
\put {$\beta_{12}$} at 1.5 2.5
\put {$\dots$} at 2.5 2.5
\put {\circle*{0.1}} [cl]  at 4 2
\put {$\dots$} at 5.5 1.5
\put {$\beta_{g-3,g-1}$} at 6.5 1.5
\put {$1$} at 7.5 0.5
% Beschriftung
\put{$\overbrace{\hspace{120mm}}^{\displaystyle g-1}$} at 4 4.7
\put{$\underbrace{\hspace{30mm}}_{\displaystyle p}$} at 2 -0.6
\endpicture   
$$

\begin{conjecture}\label{GConj} \cite{Gr2} For a smooth canonical curve over $\CC$
 $\beta_{p,p+2} \ne 0$ if and only if $\exists L\in \Pic^d(C)$ with
$h^0(C,L),h^1(C,L) \ge 2$ and $\cliff(L):=d-2(h^0(C,L))-1) \le p.$

In particular $\beta_{p,p+2}= 0$ for $p \le \lfloor \frac{g-3}{2} \rfloor$ for a general 
curve of genus $g$.
\end{conjecture}
 
The ``if'' part is proved in \cite{GL} and holds for arbritary ground fields.
The conjecture is known to be false for some (algebraically closed) fields of
finite characteristics, eg.
genus $g=7$ and characteristic $char(\FF)=2$, cf. \cite{Sch4}.


\subsection{Coble self-dual sets of points.}
The free resolution of a hyperplane section of a Cohen-Macaulay ring 
has the same Betti numbers. Thus we may ask the question about the geometric 
interpretation of syzygies of $2g-2$ points in $\PP^{g-2}$ 
(hyperplane section of a canonical curve), 
or syzygies of a graded artinian Gorenstein algebra with Hilbert function
$(1,g-2,g-2,1,0,\ldots)$ 
(twice an hyperplane section). 
Points obtained as hyperplane section of a canonical curve 
are special $2g-2$ points: 
they impose only $2g-3$ condition on quadrics!
An equivalent condition for points in linearly uniform position is that 
they are Coble (or Gale) self-dual, cf. \cite{EiPo}. 
Thus if we distribute the $2g-2$ points in 2 collections of $g-1$ points, 
say the first the coordinates points
and the second corresponding to the rows of a $(g-1) \times (g-1)$ matrix 
$A=(a_{ij})$ then $A$ can be choosen to be an orthogonal matrix, ie.
$A^t A = 1$, cf. loc cit.

To see what the analog of Green's Conjecture for the general curve means for
orthogonal matrices we recall a result of \cite{RS1}.

Set $n=g-2$. We identify the homogenous coordinate ring of $\PP^n$ with
the ring $S=\FF[\partial_0,\ldots,\partial_n]$ of 
differential operators with 
constant coefficients, $\partial_i=\frac{\partial}{\partial x_i}$. 
$R$ acts on $\FF[x_0,\ldots,x_n]$ by differentation. 
The annihilator of $q=x_0^2+\ldots+x_n^2$ is a homogenous ideal 
$J \subset S$ such that 
$S/J$ is a graded artinian Gorenstein ring with Hilbert function $(1,n+1,1)$ 
and socle the form induced by $q$, 
cf. \cite{RS2}, \cite[Section 21.2 and related exercise 21.7]{Ei}.
The syzygy numbers of S/J are
$$
\begin{matrix}
1 & . &\ldots& . & \ldots &.&.\cr
. &\frac{n}{n+2}\binom{n+3}{2} &\ldots & \frac{p(n+1-p)}{n+2}\binom{n+3}{p+1}
&\ldots &\frac{n}{n+2}\binom{n+3}{n+1} &.&\cr
.&.&\ldots&.&\ldots&.&1\cr  
\end{matrix}
$$ 

A collection $H_0,\ldots,H_n$ of hyperplanes in $\PP^n$ is said to 
form a polar simplex to $q$ 
iff the collection $\Gamma=\{p_0,\ldots,p_n\} \subset \check \PP^n$ 
of the corresponding points in the dual space have a homogenous ideal 
$I_\Gamma \subset S$ contained in $J$.

In particular the set $\Lambda$ consisting of the coordinate points correspond 
to a polar simplex, because 
$\partial_i\partial_j$ annihilates $q$ for $i\ne j$.

For any polar collection of points $\Gamma$  the free resolution
$\Ss_\Gamma$ is a subcomplex of the resolution $\Ss_{S/J}$. Green's conjecture
for the generic curve of genus $g=n+2$ would imply: 

\begin{conjecture}\label{PConj}  For a general
$\Gamma$ and the given  $\Lambda$ the corresponding Tor-groups
$$\Tor^S_k(S/I_\Gamma,\FF)_{k+1} \cap \Tor^S_k(S/I_\Lambda,\FF)_{k+1}
\subset \Tor^S_k(S/J,\FF)_{k+1}$$
intersect transversally.
\end{conjecture} 

\begin{proof}
A zero-dimensional nondegenerate scheme $\Gamma\subset\PP^n$ of degree n+1 
has sysyzies
$$
\begin{matrix}
1 & . &\ldots& . & \ldots &.&.\cr
. &\binom{n+1}{2} &\ldots & k\binom{n+1}{k+1}
&\ldots &n&.&
\end{matrix}
$$

Since both Tor-Groups are contained in $\Tor^S_k(S/J,\FF)_{k+1}$, 
the claim is equivalent to saying that for a general polar simplex $\Gamma$ 
the expected dimension of their intersection is 
$\dim\Tor^S_k(S/_\Gamma,\FF)_{k+1}+\dim\Tor^S_k(S/_\Gamma,\FF)_{k+1}
-\dim\Tor^S_k(S/J,\FF)_{k+1}$, which is
$$
2k\binom{g-1}{k+1}-\frac{k(g-1-k)}{g}\binom{g+1}{k+1}.
$$

On the other side $I_{\Gamma\cup\Lambda}=I_\Gamma\cap I_\Lambda$, hence
$$\Tor^S_k(S/I_\Gamma,\FF)_{k+1} \cap \Tor^S_k(S/I_\Lambda,\FF)_{k+1} 
=\Tor^S_k(S/I_{\Gamma\cup\Lambda},\FF)_{k+1},$$
and Green's conjecture would imply
$$
\dim\Tor^S_k(S/I_\Gamma\cup\Lambda,\FF)_{k+1}=\beta_{k,k+1}(\Gamma\cup\Lambda)
=k\binom{g-2}{k+1}-(g-1-k)\binom{g-2}{k-2}.
$$
Now a calculation shows that the two dimensions above are equal.
\end{proof}


The family of all polar simplices $V$ is dominated by the family defined  
by the ideal of $2 \times 2$ minors of the matrix
$$\begin{pmatrix} 
\partial_0 & \ldots & \partial_i & \ldots &\partial_n \cr
\sum_j b_{0j} \partial_j & \ldots &\sum_j b_{ij} \partial_j & \ldots & \sum_j b_{nj} \partial_j
\end{pmatrix}$$
depending on     
a symmetric matrix $B=(b_{ij})$, ie. $b_{ij}=b_{ji}$ as parameters.
For $B$  a general diagonal matrix
we get $\Lambda$ together with a specific element in 
$\Tor^S_n(S/I_\Lambda,\FF)_{n+1}$. 





\subsection{Comparison and probes}

One of the peculiar consequences of Green's conjecture for odd genus $g=2k+1$ 
is that, if $\beta_{k,k+1} = \beta_{k-1,k+1} \ne 0$, then the curve $C$ lies 
in the closure of the locus of $k+1$-gonal curve. $k+1$-gonal curves 
lie on a rational normal 
scroll $X$ of codimension $k$, which satisfies 
$\beta_{k,k+1}(X) = k$. Hence
$$\beta_{k,k+1}(C) \ne 0 \Rightarrow \beta_{k,k+1}(C) \ge k$$

We may ask whether a result like this is true for the union of two polar simplices $\Lambda \cup \Gamma \subset \PP^{2k-1}$. Define
$$\tilde D=\{ \Gamma \in V | \Gamma \cup \Lambda \hbox{ is syzygy special} \}$$
where as above $V$ denotes the variety of polar simplices and 
$\Lambda$ the coordinate simplex. 
If $\tilde D$ is a proper subvariety, then it is a divisor,
because 
$\beta_{k,k+1}(\Gamma \cup \Lambda) =\beta_{k-1,k+1}(\Gamma \cup \Lambda) $. 


\begin{conjecture}\label{Exceptional locus}  The subscheme 
$\tilde D \subset V$ is an irreducible divisor, 
for $g=n+2=2k+1 \in \{7,9,11\}$. 
The value of $\beta_{k,k+1}$ on a general point of $D$ 
is $3, 2, 1$ respectively.
\end{conjecture} 


We can prove this for $g=7$ by computer algebra. However for
$g=9,11$ the assertion is 
computationally out of reach with our methods. However we can get some
evidence from finite fields examples:

{\bf Evidence.} Since $\tilde D$ is a divisors, we expect that if we pick a symmetric
matrices $B$ over $\FF_q$ at random, we will hit points on every component
of $\tilde D$ at a
rate of $1:q$. For a general point on $ \tilde D$  the corresponding Coble
self-dual set of points will have the generic  number of extra syzygies
of that component. Points with even more syzygies will occur in higher 
codimension, hence only at a rate of $1:q^2$. 
Some evidence for irreducibility can be obtained from the Weil formulas:
For sufficiently large $q$ we should see points on $\tilde D$ at a rate of
$C q^{-1} + O(q^{-3/2})$, where $C$ is the number of components.
 
The following table gives times and number of trials to find at least
twice respectively once a $\Gamma$ with extra syzygies, and the number
of syzygies ibn these cases.
In these tables $q$ denotes the number of elements in the base field.

Genus $g=9$:
$$\begin{tabular}{r|rrrr|r}\hline
$q$ & total & special & 2 extra syz & other syz & time \cr
\hline
101&444&2&2&0&376\cr
101&61&2&2&0&55\cr
101&143&2&2&0&129\cr
101&80&2&2&0&76\cr
101&72&2&2&0&67\cr
\hline
101&800&10&10&0&703
\end{tabular}$$

For $g=11$ the speed goes down:
$$\begin{tabular}{r|rrrr|r}\hline
$q$ & total & special & 1 extra syz & other syz & time \cr
\hline
53&41&1&1&0&6152\cr
53&45&1&1&0&6812\cr
\hline
53&86&2&2&0&12964\cr
\hline
\hline
101&108&1&1&0&18157\cr
101&116&1&1&0&18386\cr
\hline
101&224&2&2&0&36543\cr
\end{tabular}$$

By this finding $\tilde D$ is more likely irreducible than reducible.

 
\medskip
\noindent
{\bf A test of Green's Conjecture for Curves.}
In view of \ref{Exceptional locus} it seems plausible that for a general curve
of odd genus $g \ge 11$ with $\beta_{k,k+1}(C) \ne 0$ the value might
be $\beta_{k,k+1}=1$ contradicting Green's conjecture. 
It is clear that the syzygy 
exceptional loci has codimension 1 in $\gM_g$ for odd genus, if it is proper, 
ie. if Green's 
conjecture holds for the general curve of that genus. So picking points 
at random we might be able to find such curve over a finite field $\FF_q$
with the probability $(1:q)$ roughly.
%%$$<<\hbox{testGreensConjecture}>>$$ 

A script which does this is  straightforward. One makes a loop which
picks up randomly a curve, computes its canonical image, and resolves
its ideal, counting the possible values $\beta_{k,k+1}$
until a certain amount of special curves is reached. 
The result for 10 special curves in $\FF_7$ is as predicted:
$$\begin{tabular}{r|rrr|rrrrrr}
$g$ & seconds & total & special & 
\hbox to 10pt{possible values of $\beta_{k,k+1}$\hss}\cr
&&&&$\le 2$&3&4&5&6&$\ge 7$ \cr\hline
7&148&75&10&0&10&0&0&0&0\cr
9&253&58&10&0&0&9&0&0&1\cr
11&25640&60&10&0&0&0&9&0&1\cr
\end{tabular}$$
(the test for genus 9 and 11 used respectively around 70Mbyte 
and 120Mbyte of memory).


So Green's conjecture passed the test for $g=9,11$. 
Shortly after the first author tried this test for the first time,
a paper of Hirschowitz and Ramanan appeared proving this in general:

\begin{theorem}\cite{HR} If the general curve of odd genus $g=2k+1$ 
satisfies Green's conjecture then the syzygy special curves lie on the divisor
$D=\{ C \in \gM_g | W^1_{k+1}(C) \ne \emptyset \}$ 
\end{theorem}

The theorem gives strong evidence for the full Green's conjecture in
view of our study of Coble self-dual stes of points.  

\medskip
Our findings suggest that the variety of points arizing as 
hyperplane sections of smooth canonical curves has the strange
property that it intersects the divisoe of syzygy special sets of
points $\tilde D$ only in its singular loci. 

\bigskip
The conjecture for general
curves is known us up to $g \le 17$, which is as far as our 
computer  allows to do a ribbon example, cf. \cite{BaEi}.

{\bf The end of the section should be deleted or expanded.}  
The following scripts we do these example and allow to verify
 the generic Green's conjecture
up to genus 17 (in that case the tests took 4556 seconds and 113Mbyte of memory).

{\scriptsize
\begin{verbatim} 
testGreensConjectureForGivenParams = (a,b,k,l,K) -> (
     collectGarbage();
     x=symbol x;X=K[x_1..x_a];
     mult1Withs2=matrix(apply((k-2)+1,i->(apply((a-1-k)+1,j->x_(i+j+1)))));
     mult1Withst=matrix(apply((k-2)+1,i->(apply((a-1-k)+1,j->x_(i+j+2)))));
     mult1Witht2=matrix(apply((k-2)+1,i->(apply((a-1-k)+1,j->x_(i+j+3)))));
     --the Kozsul matrix \wedge^(k-1) Fa ->\wedge^k Fa ** Fa^ is given by
     y=symbol y;Y=K[y_1..y_a];
     alpha=transpose koszul(k,vars Y);
     --the contraction map
     QX=sum(numgens X,i->x_(i+1)^2);
     JX=ideal (symmetricPower(2,vars X)*(syz diff(QX, symmetricPower(2,vars X))));  
     Xquot=X/JX;XY=X**Y;contractMap=map(Xquot,XY,vars Xquot|vars Xquot);
     --Now s^2*(\wedge^(k-1) Fa -> \wedge^k Fa ** Fa^* ** (S_(k-2))^* ** Fa) is symply:
     alpha1=substitute(alpha,XY)**substitute(mult1Withs2,XY);
     --and we just have to contract the x_i's with the y_i's, which are of bidegree (1,1)
     alpha1=substitute(contractMap(alpha1),matrix(K,{toList(numgens X:1)}));
     --and the same for the other 2 terms
     alpha2=-2*substitute(alpha,XY)**substitute(mult1Withst,XY);
     alpha2=substitute(contractMap(alpha2),matrix(K,{toList(numgens X:1)}));
     alpha3=substitute(alpha,XY)**substitute(mult1Witht2,XY);
     alpha3=substitute(contractMap(alpha3),matrix(K,{toList(numgens X:1)}));
     --THE SAME WITH THE S_b, with variables (w,z)
     w=symbol w;W=K[w_1..w_b];
     mult2Withs2=matrix(apply((l-2)+1,i->(apply((a-1-l)+1,j->w_(i+j+1)))));
     mult2Withst=matrix(apply((l-2)+1,i->(apply((a-1-l)+1,j->w_(i+j+2)))));
     mult2Witht2=matrix(apply((l-2)+1,i->(apply((a-1-l)+1,j->w_(i+j+3)))));
     z=symbol z;Z=K[z_1..z_b];
     beta=transpose koszul(l,vars Z);
     QW=sum(numgens W,i->w_(i+1)^2);
     JW=ideal (symmetricPower(2,vars W)*(syz diff(QW, symmetricPower(2,vars W))));  
     Wquot=W/JW;WZ=W**Z;contractMap=map(Wquot,WZ,vars Wquot|vars Wquot);
     beta1=substitute(beta,WZ)**substitute(mult2Withs2,WZ);
     beta1=substitute(contractMap(beta1),matrix(K,{toList(numgens W:1)}));
     beta2=substitute(beta,WZ)**substitute(mult2Withst,WZ);
     beta2=substitute(contractMap(beta2),matrix(K,{toList(numgens W:1)}));
     beta3=substitute(beta,WZ)**substitute(mult2Witht2,WZ);
     beta3=substitute(contractMap(beta3),matrix(K,{toList(numgens W:1)}));
     --FINAL EQUATIONS
     equat=alpha1**beta1+alpha2**beta2+alpha3**beta3;
     <<numgens source equat<<"x"<<numgens target equat<<endl;
     rank equat==min(numgens source equat,numgens target equat))
\end{verbatim}}

{\scriptsize
\begin{verbatim} 
testGreensConjectureForOddGenera = (g,p) -> (
     if p==0 then K=QQ else K=ZZ/p;
     test=true;
     a:=floor((g-1)/2);
     k:=2;while k<=ceiling(a/2) do (
          <<k<<","<<a+1-k<<": ";
          time test=testGreensConjectureForGivenParams(a,a,k,a+1-k,K); 
          if test==false then k=a else k=k+1;);
     test)
\end{verbatim}}









\section{Pfaffian Calabi-Yau threefolds in $\PP^6$}

Calabi-Yau 3-folds  caught the attention of physicists, 
because they can serve as the compact factor 
in a Kaluza-Klein theory of superstring theory. 
One of the remarkable things which grow out of the work in physics
is the discovery of mirror symmetry, which associates to a family of
Calabi-Yau 3-folds $(M_\lambda)$, another family $(W_\mu)$ whose Hodge
diamond is the mirror of the Hodge diamond of the original family.

Although there is an enormous amount of evidence at present, the existence 
of a mirror is still a hypothesis for general Calabi-Yau 3-folds. 
The thousands of cases where this was established all are close to toric
geometry, where through the work of Batyrev and others \cite{Ba,CK} 
rigorous mirror
constructions were given and parts of their conjectured properties proved.


\medskip
>From a commutative algebra point of view the examples studied so far are 
rather trivial, because nearly all are hypersurfaces or complete 
intersections on toric varieties, or zero loci of sections in 
homogeneous bundles on homogenous spaces.  

Of course only  few families of Calabi-Yau 3-folds should be of this kind.
Perhaps the easiest examples beyond the toric/homogeneous range are Calabi-Yau
3-folds in $\PP^6$. 
Here examples can be obtained by the pfaffian
construction of Buchsbaum-Eisenbud \cite{BE} with vector bundles. 
Indeed a recent theorem of Walter \cite{Wa} says 
that any smooth Calabi-Yau in $\PP^6$ can be obtained in this way in principle. 
In this section we report on our construction of such examples. 

As quite usual in this kind of problems, there is a range where a 
construction is still quite easy, 
eg. for surfaces in $\PP^4$ the work in \cite{DES,Po} 
shows that the construction of nearly all the 50 known families 
of smooth non-general type surfaces is straight forward 
and their Hilbert scheme component unirational. 
Only in very few known examples the construction is more difficult
and the unirationality of the Hilbert scheme component an open problem.

Fabio Tonoli did the ``first non-trivial'' case of a construction of 
Calabi-Yau 3-folds in $\PP^6$. 
Although in the end the families  turned out to be unirational, 
the approach utilized small finite field constructions as a research tool.




\subsection{The pfaffian complex}

Let $\kf$ be a vector bundle of odd rank $rk \kf = 2r+1$ on a projective
manifold $M$ and $\kl$ a line bundle. Let $\varphi 
\in H^0(M,\Lambda^2 \kf \otimes \kl)$ be a section. We can 
think of $\varphi$ as a skew-symmetric twisted homomorphism
$$\kf^* \stackrel{\varphi}{\longrightarrow} \kf \otimes \kl.$$
 
The $r^{th}$ devided power of $\varphi$ is a section
$\varphi^{(r)} = \frac{1}{r!}(\varphi \wedge \dots \wedge \varphi) \in 
H^0(M,\Lambda^{2r}\kf \otimes \kl^r)$. Wedge-product with $\varphi^{(r)}$ 
defines a morphism 
$$ \kf \tensor \kl \stackrel{\psi}{\longrightarrow}  \Lambda^{2r+1} \kf \otimes \kl^{r+1}=det(\kf)\tensor\kl^{r+1}.$$

The  twisted image 
$\ki = \image(\psi) \tensor det(\kf^*) \tensor \kl^{-r-1} \subset \ko_M$
is called the {\sl pfaffian ideal} of $\varphi$, because locally working
with frames it is given  the ideal generated by the  $2r\times 2r$ principle
pfaffians of the matrix describing $\varphi$.

\begin{theorem} \cite{BE}.
$$  0 \to det(\kf)^{-2}\tensor\kl^{-2r-1}
\stackrel{\psi^t}{\longrightarrow} det(\kf)^{-1}\tensor\kl^{-r-1}\tensor \kf^* 
\stackrel{\varphi}{\longrightarrow} \kf \tensor det(\kf)^{-1}\tensor 
\kl^{-r} \stackrel{\psi}{\longrightarrow} \ko_M $$
is a complex. 
$X=V(\ki) \subset M$ has codimension $\le 3$ at every point, and in case equality
holds (every where along $X$) then 
this complex is exact and resolves the structure sheaf $\ko_X=\ko_M/\ki$ 
of the  locally Gorenstein subscheme $X$. 
\end{theorem}

We will apply this to construct Calabi-Yau 3-folds in $\PP^6$. 
In that case we want $X$ to be smooth and 
$det(\kf)^{-2}\tensor\kl^{-2r-1} \cong \omega_\PP\cong \ko(-7)$
to conclude $\omega_X \cong \ko_X$. 
A result of Walter for $\PP^n$ guarantees the existence 
of a Pfaffian presentation in $\PP^6$ for every subcanonical embedded 3-fold. 
Moreover Walter choice of $\kf$ for Calabi-Yau 3-folds $X \subset \PP^6$
is the sheafified first syzygy module $H^1_*(\ki_X)$ plus possibly 
a direct sum of line bundles 
(indeed $H^2_*(\ki_X)=0$ because of the Kodaira-vanishing theorem). 
Under the maximal rank assumption for
$$H^0(\PP^6,\ko(m)) \to H^0(X,\ko_X(m))$$
the Hartshorne-Rao module is zero for $d=\deg X \in \{12,13,14\}$ and a 
an arithmetically Cohen-Macaulay $X$ is readily found. For $d \in 
\{15,16,17,18\}$
the Hartshorne-Rao $M$ modules have Hilbert function
$(0,0,1,0,\ldots)$, $(0,0,2,1,0,\ldots)$, $(0,0,3,5,0,\ldots)$ and 
$(0,0,4,9,0,\ldots)$ respectively.

We do not discuss the cases $d\le 16$ further. 
Construction in those cases are quite obvious, cf. \cite{To}.




\subsection{Analysis of the Hartshorne-Rao module for degree 17}

We try to construct $\kf$ as the sheafified first syzygy module of $M$. 
The construct a module with desired Hilbert function is no problem.
The cokernel of $3S(2) \stackrel{m}{\longleftarrow} 16S(1)$ for a general 
matrix of linear forms has this property. However for a general $m$ and
$\kf = ker( 16\ko(1) \stackrel{m}{\longrightarrow} \ko(2))$ the space of 
skew-symmetric maps $\Hom_{skew}(\kf^*,\kf(-1))$ is zero:  $M$ has syzygies
$$
\begin{tabular}{|llllllll}
\hline
3 & 16 & 28 & - & - & - & - \cr
- & - & - & 70 & 112 & 84 & 32 & 5 \cr
\end{tabular}
$$
\noindent
However any map $\varphi \colon \kf^*(1) \to \kf$ induces a map on the free
resolutions:
$$
\xymatrix{
0 &\kf \ar[l] &28\ko \ar[l] &70\ko(-2) \ar[l] &112\ko(-3) \ar[l] \\
0 &\kf^*(1) \ar[l] \ar[u]_\phi &16\ko \ar[l] \ar[u]_{\phi_0}
&3\ko(-1) \ar[l] \ar[u]_{\phi_1} &0 \ar[l]
}
$$
%\begin{diagram}[small] 
%0 &\lTo &\kf &\lTo &28\ko &\lTo &70\ko(-2) &\lTo &112\ko(-3) \cr
%&&\varphi\; \uTo && \varphi_0 \;\uTo && \varphi_1\; \uTo&& \cr
%0&\lTo & \kf^*(1) &\lTo & 16\ko & \lTo &3\ko(-1) &\lTo & 0 \cr 
%\end{diagram}
\noindent
Since $\varphi_1=0$ for degree reasons, $\varphi=0$ as well, and
$\Hom(\kf^*(1),\kf)=0$ for a general module $M$.

What we need are special modules $M$ which has extra syzygies 
$$
\begin{tabular}{|llllllll}
\hline
3 & 16 & 28 & $k$ & - & - & - \cr
- & - & $k$ & 70 & 112 & 84 & 32 & 5 \cr
\end{tabular}
$$
with $k$ at least 3.

In a neighbourhood  $o \in \Spec B$  for $B$ the base space of 
a semi-universal deformation of $M$, 
the resolution above would lift to a complex over $B[x_0,\ldots,x_6]$ 
and in the lifted complex there is a $k\times k$ matrix $\Delta$
with entries in the maximal ideal $o \subset B$. 
By the principal ideal theorem we see that Betti numbers stay constant 
in a subvariety of codimension at most $k^2$. 
To check for second linear syzygyies on a randomly choosen M 
is a computationally rather easy task.
The following procedure tests the computer speed of this task.
{\scriptsize
\begin{verbatim} 
testModulesForDeg17CY = (N,k,p) -> (
     x:=symbol x;R:=(ZZ/p)[x_0..x_6];
     numberOfGoodModules:=0;i:=0;
     usedTime:=timing while (i<N) do (
          b:=random(R^3,R^{16:-1});
          -- We put SyzygyLimit=>60 because we expect k<16 syzygies, so 16+28+k<=60
          fb:=res(coker b, DegreeLimit =>0,SyzygyLimit=>60,LengthLimit =>3);
          if rank fb_3>=k and (dim coker b)==0 then (
               fb=res(coker b, DegreeLimit =>0,LengthLimit =>4);
               if rank fb_4==0 then numberOfGoodModules=numberOfGoodModules+1;);
          i=i+1;);
     collectGarbage();
     timeForNModules:=usedTime#0;
     {timeForNModules,numberOfGoodModules})
\end{verbatim}}

Running this procedure we see that it takes not more than 
$0.64\frac{seconds}{example}$.
Hence we can hope to find examples with $k=3$ within a reasonable time 
for a very small field, say $\FF_3$. 

\medskip
The first surprize is that examples with k-extra syzygies 
are found much more often, 
as can be seen by looking at the second outputted value of {\tt testModulesForDeg17CY()}.

This is not only a ``statistic'' remark, in the sense that 
the result is comfirmed by computing the semi-universal deformations of these modules.
Indeed define $\MM_k=\{ M | Tor^S_3(M,\FF)_5 \ge k \}$ and 
consider a module $M\in \MM_k$:
``generically'' we obtain $\codim(\MM_k)_M=k$ instead of $k^2$
(and in fact one can diagonalize the matrix $\Delta$)!

The procedure is straight forward but a bit long. 
First we pick up an example with $k$-extra syzygies.
{\scriptsize
\begin{verbatim}
randomModuleForDeg17CY = (k,R) -> (
     isGoodModule:=false;i:=0;
     while not isGoodModule do (
          b:=random(R^3,R^{16:-1});
          -- We put SyzygyLimit=>60 because we expect k<16 syzygies, so 16+28+k<=60
          fb:=res(coker b, DegreeLimit =>0,SyzygyLimit=>60,LengthLimit =>3);
          if rank fb_3>=k and (dim coker b)==0 then (
               fb=res(coker b, DegreeLimit =>0,LengthLimit =>4);
               if rank fb_4==0 then isGoodModule=true;);
          i=i+1;);
     <<"    -- Trial n. " << i <<", k="<< rank fb_3 <<endl;
     b)
\end{verbatim}}
Notice that the previous function returns a presentation matrix $b$ of $M$, 
and not $M$.

We compute the tangent codimension of $\MM_k$ in the chosen example $M=\Coker b$
via computing the codimension of the infinitesimal deformations of $M\in o$
which still gives an element in $\MM_k$.
Denote with $b_i$ the maps in the linear strand of a minimal free resolution of $M$, 
and with $b_2'$ the quadratic part in the second map of this resolution.
Over $B=\FF[\epsilon]/{\epsilon^2}$ 
let $b_1+\epsilon f_1$ be an infinitesimal deformation of $b_1$. 
Then $f_1$ lifts to a linear map $f_2\colon 28S \to 16S(1)$ determined by 
$(b_1+\epsilon f_1)\circ(b_2+\epsilon f_2)=0$, and $f_2$ to a map 
$f_3\oplus\Delta\colon k S(-1) \to 28 S\oplus k S(1)$ determined by
$(b_2+b_2')\circ\epsilon (f_3\oplus\Delta)=0$.
Therefore we can determine $\Delta$ as:
{\scriptsize
\begin{verbatim} 
getDeltaForDeg17CY = (b,f1) -> (
     fb:=res(coker b, LengthLimit =>3);
     k:=numgens target fb.dd_3-28; --# of linear syzygies
     b1:=fb.dd_1;b2:=fb.dd_2_{0..27};b2':=fb.dd_2_{28..28+k-1};b3:=fb.dd_3_{0..k-1}^{0..27};
     --the equation for f2 is b1*f2+f1*b2=0, so f2 is a lift of (-f1*b2) through b1 
     f2:=-(f1*b2)//b1;
     --the equation for A=(f3||Delta) is -f2*b3 = (b2|b2') * A
     A:=(-f2*b3)//(b2l|b2');
     Delta:=A^{28..28+k-1})
\end{verbatim}}
Now we just parametrize all possible maps $f_1\colon 16S(1) \to 3S(2)$,
compute their respective maps $\Delta$,
and find of which codimension is the condition that $\Delta$ is the zero map:
{\scriptsize
\begin{verbatim} 
codimInfDefModuleForDeg17CY = (b) -> (
     --We create a parameter ring for the matrices f1's
     R:=ring b;K:=coefficientRing R;
     u:=symbol u;U:=K[u_0..u_(3*16*7-1)];
     i:=0;while i<3 do (
          <<endl<< " " << i+1 <<":" <<endl;
          j:=0;while j<16 do(
               << "    " << j+1 <<". "<<endl;
               k:=0;while k<7 do (
                    l=16*7*i+7*j+k; --index parametrizing the f1's
                    f1:=matrix(R,apply(3,m->apply(16,n->if m==i and n==j then x_k else 0)));
                    Delta:=substitute(getDeltaForDeg17CY(b,f1),U);
                    if l==0 then (equations=u_l*Delta;) else (equations=equations+u_l*Delta;);
                    k=k+1;);
               collectGarbage(); --frees up memory in the stack
               j=j+1;);
          i=i+1;);
     codim ideal equations)
\end{verbatim}}


\medskip
The second surprize is that for $\kf = syz_1(M)$ we find
$$\dim \Hom_{skew}(\kf^*(1),\kf) =k= \dim \Tor_3^S(M,\FF)_5.$$
$\Hom_{skew}(\kf^*(1),\kf)$ is the vector space of skew-symmetric
linear matrices $\varphi$ such that $b \circ \varphi = 0$
The following procedure
gives a $\binom{16}2\times\dim \Hom_{skew}(\kf^*(1),\kf)$ matrix 
whose $i$-th column gives the parameters of a $16\times16$
skew-symmetrix 
matrix 
inducing the $i$-th basis element of $\Hom_{skew}(\kf^*(1),\kf)$.
{\scriptsize
\begin{verbatim} 
skewSymMorphismsForDeg17CY = (b) -> (
     --We create a parameter ring for the morphisms: 
     K:=coefficientRing ring b;
     u:=symbol u;U:=K[u_0..u_(binomial(16,2)-1)];
     --Now we compute the equations for the u_i's:
     UU:=U**ring b;
     equationsInUU:=flatten (substitute(b,UU)*
          substitute(genericSkewMatrix(U,u_0,16),UU));
     uu:=substitute(vars U,UU);
     equations:=substitute(diff(uu,transpose equationsInUU),ring b);
     syz(equations,DegreeLimit =>0))
\end{verbatim}}
A morphism parametrized by a column $\tt skewSymMorphism$ is then recovered by 
{\scriptsize
\begin{verbatim} 
getMorphismForDeg17CY = (SkewSymMorphism) -> (
     n:=16;n1:=binomial(n,2);
     u:=symbol u;U:=K[u_0..u_(n1-1)];
     f=map(ring SkewSymMorphism,U,transpose SkewSymMorphism);
     f genericSkewMatrix(U,u_0,n))
\end{verbatim}}




\partitle{Rank-1 linear syzygies of $M$}
To understand this phenomenon we consider the multiplication tensor of $M$:
$$\mu \colon M_2 \tensor V \to M_3$$
where $V=H^0(\PP^6,\ko(1))\cong \FF^7$. 

\begin{definition} A decomposable element of $M_2 \tensor V$ in the kernel of 
$\mu$ is called a rank 1 linear syyzygy of $M$.
The (projective) space of rank 1 syzygies is
$$Y=(\PP^2 \times \PP^6) \cap \PP^{15} \subset \PP^{20}$$
where $\PP^2=\PP(M_2^*),\PP^6=\PP(V^*)$ and $\PP^{15}=\PP(\ker(\mu)^*)$
inside the Segre space $\PP((M_2 \tensor V)^*)\cong \PP^{20}$. 
\end{definition}

A result of \cite{Gr1} says that for the existence of $j^{th}$ linear syzygies
$\dim Y \ge j$ is necessary. This is automatically satisfied for $j=3$
in our case: $\dim Y \ge 3$ with equality expected. 

The projection $Y \to \PP^2$ has linear fibers, the general fiber is a 
$\PP^1$. However special fibers might have larger dimension. In terms
of the presentation matrix $m$ a special 2-dimensional fiber (defined
over $\FF$) correspond to a block
$$
m=\begin{pmatrix}
0 & 0 & 0 & * & \ldots \cr
0 & 0 & 0 & * & \ldots \cr
l_1 & l_2 & l_3 & *& \ldots \cr 
\end{pmatrix}
$$
with $l_1,l_2,l_3$ 3 linear forms, in the $3\times16$ presentation matrix of $M$.
Such block gives a 
$$
\begin{tabular}{|llllllll}
\hline
1 & 3 & 3 & 1 & - & - & - &- \cr
- & - & - & - & - & - & - & - \cr
\end{tabular}
$$
subcomplex in the free resolution of $M$ and an elemnt 
$s \in H^0(\PP^6,\Lambda^2 \kf \tensor \ko(-1))$
since the syzygy matrix
$$
\begin{pmatrix}
0 & -l_3 & l_2  \cr
l_3 & 0 & -l_1  \cr
-l_2 & l_1 & 0  \cr 
\end{pmatrix}
$$
is skew. 

This answers both surprises: 
We want a module $M$ with at least $k \ge 3$ special
fibers and these satisfy $h^0(\PP^6,\Lambda^2 \kf \tensor \ko(-1)) \ge k$, 
if the $k$ sections are linearly independent. 
The condition for $k$ special fibers is of expected codimension $k$ in the
parameter space $\GG(16,3h^0(\PP^6,\ko(1))$ of the presentation matrices.
In a given point $M^{special}$ the actual codimension can be readily computed
by a first order deformations and that 
$H^0(\PP^6,\Lambda^2 \kf \tensor \ko(-1))$ is k-dimensional, and spanned by the
$k$ sections corresponding to the $k$ special fibers can be checked as well. 

First we check that $M$ has $k$ distinct points in $M_0$ where 
the multiplication map drops rank. 
(Note that this condition is likely to fail over small fields. 
However the check is computationally easy).
{\scriptsize
\begin{verbatim} 
checkBasePtsForDeg17CY = b -> (
     --firstly the number of linear syzygies
     fb:=res(coker b, DegreeLimit=>0, LengthLimit =>4);
     k:=#select(degrees source fb.dd_3,i->i=={3});
     --then the check
     a=symbol a;A=K[a_0..a_2];
     mult:=(id_(A^7)**vars A)*substitute(syz transpose jacobian b,A);
     basePts=ideal mingens minors(5,mult);
     codim basePts==2 and degree basePts==k and distinctPoints basePts)
\end{verbatim}}
Next we check that $H^0(\PP^6,\Lambda^2 \kf \tensor \ko(-1))$ is k-dimensional,
by looking at the numbers of columns of {\tt skewSymMorphismsForDeg17CY}.
Finally we do the computationally hard part of the check, which is to
verify that the k special sections corresponding to the k special
fibers of $Y -to \PP^2$ span $H^0(\PP^6,\Lambda^2 \kf \tensor \ko(-1))$.
{\scriptsize
\begin{verbatim} 
checkMorphismsForDeg17CY = (b,skewSymMorphisms) -> (
     --First the number of linear syzygies
     fb:=res(coker b, DegreeLimit=>0, LengthLimit =>4);
     k:=#select(degrees source fb.dd_3,i->i=={3});
     if (numgens source skewSymMorphisms)!=k then (
          error "the number of skew symmetric morphisms is not the expected one";)
     --We parametrize the morphisms:    
     R:=ring b;K:=coefficientRing R;     
     w:=symbol w;W:=K[w_0..w_(k-1)];
     WW:=R**W;ww:=substitute(vars W,WW);
     genericMorphism:=getMorphismForDeg17CY(substitute(skewSymMorphisms,WW)*transpose ww);
     --We compute the scheme of the 3x3 morphisms:
     equations:=mingens pfaffians(4,genericMorphism);
     equations=diff(substitute(symmetricPower(2,vars R),WW),equations);
     equations=saturate ideal flatten substitute(equations,W);
     CorrectDimensionAndDegree:=(dim equations==1 and degree equations==k);
     isNonDegenerate:=#select((flatten degrees source gens equations),i->i==1)==0;
     collectGarbage();
     isOK:=CorrectDimensionAndDegree and isNonDegenerate;
     if isOK then (
          --In this case we also look for a skew-morphism f which is a linear
          --combination of the special morphisms with all coefficients nonzero.
          isGoodMorphism:=false;while isGoodMorphism==false do (
               evRandomMorphism:=random(K^1,K^k);
               itsIdeal:=ideal (vars W*substitute(syz evRandomMorphism,W));
               isGoodMorphism=isGorenstein intersect(itsIdeal,equations);
               collectGarbage());
          f=map(R,WW,vars R|substitute(evRandomMorphism,R));
          randomMorphism:=f(genericMorphism);
          {isOK,randomMorphism}) else {isOK})
\end{verbatim}}
The script is structured as follows. 
First we parametrize the skew-symmetric morphisms with new variables.
The ideal of $4\times4$ Pfaffians is generated by forms bi-degree $(2,2)$
over $\PP^6 \times \PP^{k-1}$. We are interested in
points $p \in \PP^{k-1}$ such that the whole fiber 
$\PP^6 \times \{ p \}$ is contained in zero loci of the pfaffian
ideal. The next two lines produce the ideal of these points on
$\PP^{k-1}$. Since we already know of $k$ distinct points by the
previous check, it suffices to establish that the set consists of 
collection of k spanning points. Finally, if this is a case, a further
point, i.e. a further skew morhism, is a linear combination with all
coefficients non-zero, iff the union with this point is a Gorenstein set
of $k+1$ points in $\PP^{k-1}$.


{\scriptsize
\begin{verbatim} 
isGorenstein = (I) -> (codim I==length res I and rank (res I)_(length res I)==1)
\end{verbatim}}



It is clear that all 16 relation should take part in the desired
skew homomorphism $\kf^*(1) \stackrel{\varphi}{\longrightarrow} \kf$.
Thus we need $k \ge 6$ to have a chance for a Calabi-Yau.
Since $3\cdot5 <16$ it easy to guarantee 5 special fibers by suitable choice
of the presentation matrix. 
So to get $k \ge 6$ is only of codimension $k-6$ on this subspace, 
and we have chances to obtain a desired modules.
{\scriptsize
\begin{verbatim} 
randomModule2ForDeg17CY = (k,R) -> (
     isGoodModule:=false;i:=0;
     while not isGoodModule do (
          b:=(random(R^1,R^{3:-1})++random(R^1,R^{3:-1})++random(R^1,R^{3:-1})|
               matrix(R,{{1},{1},{1}})**random(R^1,R^{3:-1})|
               random(R^3,R^1)**random(R^1,R^{3:-1})|
                    random(R^3,R^{1:-1}));
          -- We put SyzygyLimit=>60 because we expect k<16 syzygies, so 16+28+k<=60
          fb:=res(coker b, DegreeLimit =>0,SyzygyLimit=>60,LengthLimit =>3);
          if rank fb_3>=k and dim coker b==0 then (
                    fb=res(coker b, DegreeLimit =>0,LengthLimit =>4);
               if rank fb_4==0 then isGoodModule=true;);
          i=i+1;);
     <<"    -- Trial n. " << i <<", k="<< rank fb_3 <<endl;
     b)
\end{verbatim}}


\medskip
Some modules $M^{special}$ with $k=8,9,11$ lead to smooth examples of 
Calabi-Yau 3-folds $X$ of degree 17. 
To check the smoothness via the jacobian criterion 
is computationally too expensive for a nowadays common computer.
For a way to speed up this computation considerably 
and to reduce the required amount of memory to a reasonable value (128MB), 
we refer to \cite{To}.

Since $h^0(\PP^6,\Lambda^2 \kf \tensor \ko(1))=k$
and $\codim \{ M's | Tor^S_3(M,\FF)_5 \ge k \} =k$ all three families
have the same dimension. In particular no family lies in the closure 
of another.

A deformation computation verifies $h^1(X,\kt)=h^1(X,\Omega^2) = 23$. 
Hence a computation of the Hodge numbers $h^q(X,\Omega^p)$ gives the diamond
$$
\begin{matrix}
&&& 1 &&&\cr
&& 0 && 0 &&\cr
& 0 && 1 && 0 &\cr
 1 &&{23} &&{23} && 1\cr
& 0 && 1 && 0 &\cr
&& 0 && 0 &&\cr
&&& 1 &&&\cr 
\end{matrix}
$$



\begin{example}
The following commands gives an example of a Calabi-Yau 3-fold in $\PP^6$:
{\scriptsize
\begin{verbatim} 
K=ZZ/11
R=K[x_0..x_6]
b=randomModule2ForDeg17CY(8,R);
 -- Trial n. 555, k=8
betti res coker b --the value of k
betti (skewSymMorphisms=skewSymMorphismsForDeg17CY b); --there are k skew-symmetric morphisms
checkBasePts b --the base points in M_0 are all distinct?
finalTest=checkMorphismsForDeg17CY(b,skewSymMorphisms); --the k sections span the morphisms?
finalTest#0 --if its true then this is a good module
n=finalTest#1; --we pick up a random morphism involving all the k sections
betti (nn=syz n) --if all the tests are ok there should be a high degree syzygy
n2t=transpose submatrix(nn,{0..15},{3});
b2:=syz b;
j:=ideal mingens ideal flatten(n2t*b2); --the  ideal of the CY threefold in P6
degree j
codim j
\end{verbatim}}
\end{example}

\subsection{Lift to characteristic zero}

At this point we have constructed Calabi-Yau 3-folds $X \subset \PP^6$ over
the finite field $\FF_5$ or $\FF_7$. However our main interest is the field
of complex numbers $\CC$. The existence of lift to characteristic zero 
follows by an easy argument:

$ \MM_k=\{ M | Tor^S_3(M,\FF)_5 \ge k \}$ has codimension at most $k$.
A deformation calculation shows that in our special point
$M^{special} \in \MM(\FF_p)$ the codimension is achieved and that $\MM_k$
is smooth at this point. Thus taking a transversal slice defined over $\ZZ$
through this point we find a number field $K$ and a prime $\gp$ 
in its ring of integers $O_K$ with $O_K/\gp \cong \FF_p$ such
 that $M^{special}$ is the specialisation of an $O_{K,\gp}$ valued point
of $\MM_k$. Over the generic point of $\Spec O_{K,\gp}$ we obtain an 
$K$-valued point. $\varphi^{special}$ extends to $O_{K,\gp}$ as well,
and by semi-continuity we obtain a smooth Calabi-Yau 3-fold defined over
$K \subset \CC$.

\begin{theorem}\cite{To}. The Hilbert scheme of smooth Calabi-Yau 3-folds of 
degree $17$ in 
$\PP^6$ has at least 3 components. These three components are reduced  and 
have dimension $23+48$. The corresponding Calabi-Yau 3-folds differ in the
number of quintic generator of their homogenous ideal 
which are $8,9$ and $11$ respectively.   
\end{theorem}

See \cite{To} for more details.
\medskip


Note that we do not give a bound on the degree $[K:\QQ]$ of the number field,
and certainly we are far from a bound of its discriminant.
Hence  in principle this method allows to find points with a field of 
definition suitable number fields.

This leaves the question open whether these parameter spaces of Calabi-Yau 
3-folds are unirational. Actually they are, as the geometric construction
of modules $M^{special}$ in \cite{To} shows.

A construction of one or several mirror families of these Calabi-Yau 3-folds
is an open problem. 





\begin{thebibliography}{[ACGH85]\ }

\bibitem[AC83]{AC} 
\by E. Arbarello, M. Cornalba
\paper A few remarks about the variety of irreducible plane curve of given degree and genus 
\jour Ann. Sci. \`Ecole Norm. Sup(4)
\vol 16
\yr 1983
\pages 467--488


\bibitem[ACGH85]{ACGH} 
\by E. Arbarello, M. Cornalba, P. Griffith, J. Harris
\paper Geometry of algebraic curves, vol I
\jour Springer  Grundlehren
\vol 267 
\yr 1985
\pages xvi+386

\bibitem[Ba94]{Ba}
\by V. Batyrev
\paper Dual polyhedra and mirror symmetry for Calabi-Yau hypersurfaces in algebraic tori
\jour J. Alg. Geom. 
\vol 3
\yr 1994
\pages 493--535

\bibitem[BaEi95]{BaEi}
\by Bayer, Dave; Eisenbud, David
\paper Ribbons and their canonical embeddings
\jour Trans. Am. Math. Soc. 
\vol 347, 
\yr 1995
\pages 719--756

\bibitem[vB00]{vB}
\by H.-C. v. Bothmer
\paper Geometrische Syzygien kanonischer Kurven
\jour Thesis
\vol Bayreuth
\yr 2000
\pages 123 pp.

\bibitem[BE77]{BE}
\by D. Buchsbaum, D. Eisenbud
\paper Algebra structure on finite free resolutions and some structure 
theorem for ideals of codimension 3
\jour Am. J. Math.
\vol 99
\yr 1977
\pages 447--485

\bibitem[CK99]{CK}
\by D. Cox, S. Katz
\paper Mirror symmetry and algebraic geometry
\jour AMS, Mathematical Surveys and Monographs
\vol 68
\yr 1999
\pages xxii+469

\bibitem[CR84]{CR}
\by M.-C. Chang, Z. Ran
\paper Unirationality of the moduli space of curves genus $11$, $13$ (and $12$)
\jour Invent. Math.
\vol 76
\yr 1984
\pages 41--54

\bibitem[DES93]{DES}
\by W. Decker, L. Ein, F.-O. Schreyer
\paper Construction of surfaces in $\PP^4$
\jour J. of Alg. Geom.
\vol 2
\yr 1993
\pages 185--237

\bibitem[DS00]{DS}
\by W. Decker, F.-O. Schreyer
\paper Non-general type surfaces in $\PP^4$ - Remarks on bounds and
 constructions
\jour J. Symbolic Comp.
\vol to appear
\yr 2000
\pages x1--x3y

\bibitem[Ei95]{Ei}
\by D. Eisenbud
\paper Commutative Algebra. With a view towards algebraic geometry
\jour Springer Graduate Texts in Mathematics 
\vol 150
\yr 1995
\pages xvi+785

\bibitem[EH87]{EH}
\by D. Eisenbud, J. Harris
\paper The Kodaira dimension of the moduli space of curves of genus $g \ge 23$
\jour Invent. Math.
\vol 87
\yr 1987
\pages 495--515

\bibitem[EiPo99]{EiPo}
\by D. Eisenbud, S. Popescu
\paper Gale duality and free resolutions of ideals of points.
\jour Invent. Math.
\vol 136
\yr 1999
\pages 419-449

\bibitem[ElPe89]{ElPe}
\by G. Ellingsrud, C. Peskine
\paper Sur les surfaces de $\PP^4$
\jour Invent. Math.
\vol 95
\yr 1989
\pages 1--11




\bibitem[Gr99]{Gr1}
\by M. Green
\paper The Eisenbud-Koh-Stillman conjecture on linear syzygies
\jour Invent. Math
\vol 163
\yr 1999
\pages 411--418

\bibitem[Gr84]{Gr2}
\by M. Green
\paper Koszul cohomology and the geometry of projective varieties
\jour J. Diff. Geom.
\vol 19
\yr 1984
\pages 125--171

\bibitem[GL83]{GL}
\by M. Green, R. Lazarsfeld
\paper Appendix to Koszul cohomology and the geometry of projective varieties
\jour J. Diff. Geom.
\vol 19
\yr 1984
\pages 1xx--171

\bibitem[Ha86]{Ha1}
\by J. Harris
\paper On the Severi problem
\jour Invent. Math. 
\vol 84
\yr 1986
\pages 445--461

\bibitem[Ha82]{Ha2}
\by J. Harris
\paper Curves in projective space. With collaboration of David Eisenbud
\jour S\'eminaire de Math\'ematiques Sup\'eriores
\vol 85
\yr 1982
\pages Montreal, 138 pp.

\bibitem[HM82]{HM}
\by J. Harris, D. Mumford
\paper On the Kodaira dimension of the moduli space of curves. With an appendix by William Fulton
\jour Invent. Math.
\vol 67
\yr 1982
\pages 23--88

\bibitem[HR97]{HR}
\by A. Hirschowitz, S. Ramanan
\paper New evidence for Green's conjecture on syzgyies of canonical curves
\jour Ann. Sci. \'Ecole Norm. Sup. (4)
\vol 31
\yr 1998 
\pages 145--152

\bibitem[Klei69]{Klei}
\by S. Kleiman
\paper Geometry on Grassmannians and application to splitting bundles and smoothing cycles
\jour Inst. Hautes \'Etudes Sci. Publ. Math. No.
\vol 36
\yr 1969
\pages 281--287


\bibitem[Mu95]{Mu}
\by  S. Mukai
\paper Curves and symmetric spaces, I.
\jour Amer. J. Math.
\vol 117
\yr 1995
\pages 1627--1644

\bibitem[Po94]{Po}
\by  S. Popescu
\paper Some examples of smooth non general type surfaces in $\PP^4$
\jour thesis Saarbr\"ucken
\vol  
\yr 1994
\pages 12x pp.

\bibitem[RS01]{RS1}
\by K. Ranestad, F.-O. Schreyer
\paper On the variety of polar simplices,
\jour manuscript
\vol --
\yr to be completed
\pages --


\bibitem[RS00]{RS2}
\by K. Ranestad, F.-O. Schreyer
\paper Varieties of sums of powers
\jour J. Reine Angew. Math.
\vol xx
\yr 2000
\pages to appear

\bibitem[Sch96]{Sch1}
\by F.-O. Schreyer
\paper Finite fields in constructive algebraic geometry,
\jour in: {\it Moduli of vector bundles (Sanda; Kyoto, 1994} Lecture Notes in Pure and appl. Math. 
\vol 179
\yr 1996
\pages 221--228

\bibitem[Sch91]{Sch2}
\by F.-O. Schreyer
\paper A standard basis approach to syzygies of canonical curves
\jour J. Reine Angew. Math. 
\vol 421
\yr 1991
\pages 83--123

\bibitem[Sch89]{Sch3}
\by F.-O. Schreyer
\paper Green's conjecture for $p$-gonal curves of large genus. Algebraic curves and projective geometry (Trento 1988)
\jour Lecture Notes in Math.
\vol 1396
\yr 1989
\pages 254--260

\bibitem[Sch86]{Sch4}
\by F.-O. Schreyer
\paper Syzygies of canonical curves and special linear series
\jour Math. Ann.
\vol 275
\yr 1986
\pages 105--137



\bibitem[Se81]{Se}
\by E. Sernesi
\paper L' unirazionalit\`a della varieta dei moduli delle curve di genere dodici
\jour Ann. Scuola Norm. Sup. Pisa Cl. Sci. (4)
\vol 8
\yr 1981
\pages 405--439


\bibitem[To00]{To}
\by F. Tonoli
\paper Canonical surfaces in $\PP^5$ and Calabi-Yau threefolds in $\PP^6$
\jour thesis
\vol Padova
\yr 2000
\pages 12x pp.

\bibitem[Vo88]{Vo}
\by C. Voisin
\paper Courbes t\'etragonales et cohomologie de Koszul
\jour J. Reine Angew. Math.  
\vol 387
\yr 1988
\pages 111--121

\bibitem[Wa96]{Wa}
\by C. Walter
\paper Pfaffian subschemes 
\jour J. Alg. Geom.
\vol 5
\yr 1996
\pages 671--704

\end{thebibliography}

\end{document}
