\documentclass{article}
\usepackage{amsmath}
\usepackage{amssymb,latexsym}
\usepackage[matrix,arrow,curve]{xy}
\usepackage{amsthm}
\newtheorem{lemma}{Lemma}
\newtheorem{definition}{Definition}
\begin{document}
\begin{lemma}\label{quis}
  If 
  \begin{equation}
    \xymatrix{
      0 \ar[r] & A  \ar[d]_p \ar[r] & B  \ar[d]_q\ar[r] & C  \ar[d]_r\ar[r] & 0 \\
      0 \ar[r] & A' \ar[r] & B' \ar[r] & C' \ar[r] & 0 
      }
  \end{equation}
  is a map between short exact sequences of chain complexes, and $p$ and $r$
  are quasi-isomorphisms, them so is $q$.
\end{lemma}
\begin{proof}
  \ Apply the 5-lemma to the following diagram.
  \begin{equation}
    \xymatrix{
      H_{n-1} C  \ar[d]_{H_{n-1}r}^\cong \ar[r] &
      H_n     A  \ar[d]_{H_n    p}^\cong \ar[r] &
      H_n     B  \ar[d]_{H_n    q}       \ar[r] &
      H_n     C  \ar[d]_{H_n    r}^\cong \ar[r] &
      H_{n+1} A  \ar[d]_{H_{n+1}p}^\cong
      \\
      H_{n-1} C'         \ar[r] &
      H_n     A'         \ar[r] &
      H_n     B'         \ar[r] &
      H_n     C'         \ar[r] &
      H_{n+1} A' 
      }
  \end{equation}
\end{proof}
\begin{definition} A {\em bounded} ascending filtration of a module $M$ is an ascending
      filtration of $M$ by submodules such that $F_0 M=0$ and $F_n M= M$ for some $n \in \mathbb N$.
\end{definition}
\begin{definition} A {\em bounded} ascending filtration of a chain complex $C$ is an ascending
      filtration by subcomplexes such that for each $k \in \mathbb Z$ the
      induced filtration on the module $C_k$ is bounded.  We say also that
      $C$ is {\em boundedly} filtered.
\end{definition}
\begin{definition}
  A {\em map} of filtered chain complexes $f : C \to D$ is a chain map such that,
  for each $n \in \mathbb Z$ we have $f(F_n C) \subseteq F_n D$.  Such a map
  induces maps of chain complexes $F_n f : F_n C \to F_n D$ and $gr_n f : gr_n C \to gr_n D$.
\end{definition}
\begin{lemma}
  Suppose $f : C \to D$ is a map of boundedly filtered chain complexes, and
  that for each $n \in \mathbb Z$, the induced map $gr_n f$ is a
  quasi-isomorphism.  Then $f$ is a quasi-isomorphism.
\end{lemma}
\begin{proof}
  It is enough to show that for each $n \in \mathbb Z$, the map $F_n f$ is a
  quasi-isomorphism.  For, suppose we are given $k \in \mathbb Z$, and we
  wish to show that $H_k f : H_k C \to H_k D$ is an isomorphism.  Pick $n$ so
  large that $F_n C_{k-1} = C_{k-1}$, $F_n C_{k} = C_{k}$, $F_n C_{k+1} =
  C_{k+1}$, $F_n D_{k-1} = D_{k-1}$, $F_n D_{k} = D_{k}$, and $F_n D_{k+1} =
  D_{k+1}$.  With that choice for $n$, we see that $H_k F_n C = H_k C$ and
  $H_k F_n D = H_k D$.  Assuming as we are that $H_k F_n f : H_k F_n C \to
  H_k F_n D$ has been shown to be an isomorphism, it follows that $H_k f :
  H_k C \to H_k D$ is an isomorphism.
  
  We shall show that for each $n \in \mathbb Z$ the map $F_n f : F_n C \to
  F_n D$ is a quasi-isomorphism by ascending induction on $n$, the statement
  being clear for $n=0$ because $F_0 C = 0$ and $F_0 D = 0$.  For the
  inductive step, we assume that $F_{n-1} f : F_{n-1} C \to F_{n-1} D$ is a
  quasi-isomorphism and show that $F_{n} f : F_{n} C \to F_{n} D$ is, too.
  For this, simply apply the hypothesis and Lemma \ref{quis} to the
  following diagram.
  \begin{equation}
    \xymatrix{
      0 \ar[r] & F_{n-1} C  \ar[d]_\sim \ar[r] & F_n C  \ar[d]\ar[r] & gr_n C  \ar[d]_\sim\ar[r] & 0 \\
      0 \ar[r] & F_{n-1} D \ar[r] & F_n D \ar[r] & gr_n D \ar[r] & 0 
      }
  \end{equation}
\end{proof}
\end{document}
