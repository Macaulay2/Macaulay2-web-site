% Dan: 333-6209 (office) 367-6384 (home)

\CompileMatrices

\newtheorem{sRemark}{Remark}{\bfseries}{\rm}
\newtheorem{notation}[theorem]{Notation}{\bfseries}{\rm}
\newtheorem{construction}[theorem]{Construction}{\bfseries}{\rm}
\numberwithin{sRemark}{subsection}
\newtheorem{sExample}[sRemark]{Example}{\bfseries}{\rm}
\newtheorem{sCode}[sRemark]{Code}{\bfseries}{\rm}

\title{Resolutions and Cohomology \\ over Complete Intersections}
\titlerunning{Complete Intersections}
\toctitle{Resolutions and Cohomology over Complete Intersections}

\author{Luchezar L. Avramov
        %\inst 1
   \and Daniel R. Grayson%
        %\inst 2
        %\fnmsep
        \thanks{Authors supported by the NSF, grants DMS 99-70375 and
        DMS 99-70085.}}
\authorrunning{L. L. Avramov and D. R. Grayson}
% \institute{Purdue University, Department of Mathematics
%         % \endgraf {\tt http://www.math.purdue.edu/\char`\~avramov}
%         % \endgraf {\tt avramov\char`\@math.purdue.edu}
%       \and University of Illinois at Urbana-Champaign,
%         Department of Mathematics
%         % \endgraf {\tt http://www.math.uiuc.edu/\char`\~dan}
%         % \endgraf {\tt dan\char`\@math.uiuc.edu}
% }

\hyphenation{quasi-iso-mor-phism}

\newcommand\ssum[1]{{\underset{#1}{\sum{\vphantom\sum}^{\scriptscriptstyle+}}}}

\def\ann{{\operatorname{ann}}}
\def\inv{{\operatorname{inv}}}
\def\reg{{\operatorname{reg}}}
\def\gr{{\operatorname{gr}}}
\def\bu{{\scriptscriptstyle\bullet}}
\def\HH{{\operatorname{H}}}
\def\Tor{\operatorname{Tor}}
\def\Ext{\operatorname{Ext}}
\def\rExt{\operatorname{ext}}
\def\Deg{\operatorname{Deg}}
\def\Poi{{P}}
\def\Ba{{I}}
\def\gen{{G}}
\def\depth{\operatorname{depth}}
\def\pd{\operatorname{pd}}
\def\cx{\operatorname{cx}}
\def\crdeg{\operatorname{crdeg}}
\def\rank{\operatorname{rank}}
\def\var{\operatorname{V}}
\def\lcontract{\operatorname{\lrcorner}}
\def\Hom{\operatorname{Hom}}
\def\Coker{\operatorname{Coker}}
\def\Ker{\operatorname{Ker}}
\def\Ima{\operatorname{Im}}
\def\C{{\mathbb C}}
\def\F{{\mathbb F}}
\def\N{{\mathbb N}}
\def\Z{{\mathbb Z}}
\def\DD{{\mathsf D}}
\def\SS{{\mathsf S}}
\def\Wedge{{\textstyle\bigwedge\limits}}
\def\a{\alpha}
\def\b{\beta}
\def\g{\gamma}
\def\d{\delta}
\def\e{\epsilon}
\def\o{\otimes}

\maketitle
\begin{abstract}
This chapter contains a new proof and new applications of a theorem of
Shamash and Eisenbud, providing a construction of projective
resolutions of modules over a complete intersection.  The duals of
these infinite projective resolutions are finitely generated
differential graded modules over a graded polynomial ring, so they can
be represented in the computer, and can be used to compute $\Ext$
modules simultaneously in all homological degrees.  It is shown how to
write \Mtwo code to implement the construction, and how to use the
computer to determine invariants of modules over complete intersections
that are difficult to obtain otherwise.
\index{complete intersection}
  \end{abstract}

\section*{Introduction}
\label{sec:introduction}

Let $A=K[x_1,\dots,x_e]$ be a polynomial ring with variables of
positive degree over a field $K$, and $B=A/J$ a quotient ring modulo a
homogeneous ideal.

In this paper we consider the case when $B$ is a {\it\ie{graded
complete intersection}\/}, that is, when the defining ideal $J$ is
generated by a homogeneous $A$-regular sequence.  We set up, describe,
and illustrate a routine {\tt Ext}\indexcmd{Ext}, now implemented in
\Mtwo.  For any two finitely generated graded $B$-modules $M$ and $N$
it yields a presentation of $\Ext^\bu_B(M,N)$ as a bigraded module over
an appropriately bigraded polynomial ring $S=A[X_1,\dots,X_c]$.

A novel feature of our routine is that it computes the modules
$\Ext^n_B(M,N)$ {\it simultaneously in all cohomological degrees\/}
$n\ge0$.  This is made possible by the use of {\it cohomology
operations\/}, a technique usually confined to theoretical
considerations.  Another aspect worth noticing is that, although the
result is over a ring $B$ with nontrivial relations, all the
computations are made over the {\it polynomial ring\/} $S$; this may
account for the effectiveness of the algorithm.

To explain the role of the complete intersection hypothesis, we cast it
into the broader context of homological algebra over graded rings.

Numerous results indicate that the high syzygy modules of $M$ exhibit
`similar' properties.  For an outrageous example, assume that $M$ has
finite projective dimension.  Its distant syzygies are then all equal
to $0$, and so---for trivial reasons---display an extremely uniform
behavior.  However, even this case has a highly nontrivial aspect: due
to the Auslander-Buchsbaum Equality asymptotic information is available
after at most $(e+1)$ steps.  This accounts for the effectiveness of
computer constructions of {\it finite\/} free resolutions.

Problems that computers are not well equipped to handle arise
unavoidably when studying asymptotic behavior of {\it infinite\/}
resolutions.  We describe some, using graded Betti numbers
$\b^B_{ns}(M)=\dim_K\Ext^n_B(M,k)_{-s}$, where $k=B/(x_1,\dots,x_e)B$,
and regularity $\reg_B(M)= \sup_{n,s}\{s-n\,|\, \b^B_{ns}(M)\ne0\}$.
\begin{itemize}
\item[$\bullet$]
{\it Irrationality\/}.
There are rings $B$ for which no recurrent relation with constant
coefficients exists among the numbers $\b^B_n(k)=\sum_{s}\b^B_{ns}(k)$,
see \cite{CI:MR86i:55011a}.
\item[$\bullet$]
{\it Irregularity\/}.
For each $r\ge2$ there exists a ring $B(r)$ with $\b^{B(r)}_{ns}(k)=0$
for $s\ne n$ and $0\le n\le r$, but with $\b^{B(r)}_{r,r+1}(k)\ne0$,
see \cite{CI:MR94b:16040}.
\item[$\bullet$]
{\it Span\/}.
If $B$ is generated over $K$ by elements of degree one and $\reg_B(k)
\ne0$, then $\reg_B(k) =\infty$, see \cite{CI:AP}.
\item[$\bullet$]
{\it Size\/}.
There are inequalities $\b^B_n(k)\ge\beta^n$ for all $n\ge0$ and for
some constant $\beta>1$, unless $B$ is a complete intersection, see
\cite{CI:res}.
 \end{itemize}

These obstructions vanish miraculously when $B$ is a graded complete
intersection:  For each $M$ and all $n\gg0$ the number $\b_{n+1}(M)$ is
a linear combination with constant coefficients of $\b^B_{n-2c}(M),
\dots,\b^B_{n}(M)$. If $B$ is generated in degree one, then
$\reg_B(k)=0$ if and only if the ideal $J$ is generated by quadratic
forms.  There are inequalities $\b^B_n(M)\le\beta(M)n^{c-1}$ for all
$n\ge1$ and for some constant $\beta(M)>0$.

The algebra behind the miracle is a theorem of Gulliksen
\cite{CI:MR51:487}, who proves that $\Ext^\bu_B(M,N)$ is a finitely
generated bigraded module over a polynomial {\it ring of cohomology
operators\/} $S=A[X_1,\dots,X_c]$, where each variable $X_i$ has
cohomological degree $2$.  As a consequence of this result, problems in
Homological Algebra can be answered in terms of Commutative Algebra.

Gulliksen's definition of the operators $X_i$ as iterated connecting
homomorphisms is badly suited for use by a computer.  Other definitions
have been given subsequently by several authors, see Remark
\ref{history}.  We take the approach of Eisenbud \cite{CI:Ei}, who
derives the operators from a specific $B$-free resolution of $M$,
obtained by extending a construction of Shamash \cite{CI:Sh}.

The resolution of Shamash and Eisenbud, and Gulliksen's Finiteness
Theorem, are presented with detailed proofs in Section \ref{Cohomology
operators}.  They are obtained through a new construction---that of an
intermediate resolution of $M$ over the polynomial ring---that encodes
$C$ and all the null-homotopies of $C$ corresponding to multiplication
with elements of $J$; this material is contained in Section
\ref{Universal homotopies}.  It needs standard multilinear algebra,
developed {\sl ad hoc\/} in Section \ref{Graded algebras}.  Rules for
juggling several gradings are discussed in an Appendix.

In Section \ref{Computation of Ext modules} we present and illustrate
the code for the routine {\tt Ext}, which runs remarkably close to the
proofs in Sections \ref{Universal homotopies} and \ref{Cohomology
operators}.  Section \ref{Invariants of modules} contains numerous
computations of popular numerical invariants of a graded module, like
its complexity, Poincar\'e series, and Bass series.  They are extracted
from knowledge of the bigraded modules $\Ext^\bu_B(M,k)$ and
$\Ext^\bu_B(k,M)$, whose computation is also illustrated by examples,
and is further used to obtain explicit equations for the cohomology
variety $\var^*_B(M)$ defined in \cite{CI:MR90g:13027}.  For most
invariants we include some short code that automates their
computation.  In Section \ref{Invariants of pairs of modules} we
extend these procedures to invariants of pairs of modules.

\section{Matrix Factorizations}
\label{Matrix factorizations}

We start the discussion of homological algebra over a complete
intersection with a very special case, that can be packaged
attractively in matrix terms.

Let $f$ be a non-zero-divisor in a commutative ring $A$.

Following Eisenbud \cite[Sect.\ 5]{CI:Ei} we say that a pair $(U,V)$ of
matrices with entries in $A$, of sizes $k\times \ell$ and $\ell\times
k$, is a {\it\ie{matrix factorization}\/} of $-f$ if
\[
U\cdot V= -f\cdot I_k \qquad\text{and}\qquad V\cdot U=-f\cdot I_\ell
\]
where $I_m$ denotes the $m\times m$ unit matrix.  Localizing at $f$, one
sees that $-f^{-1}\cdot U$ and $V$ are inverse matrices over $A_f$; as a
consequence $\ell=k$, and each equality above implies the other, for
instance:
\[
V\cdot U=\big(-f^{-1}\cdot U\big)^{-1}\cdot U= -f\cdot U^{-1}\cdot U=
-f\cdot I_k
\]
Here is a familiar example of matrix factorization, with $f=xy-wz$:
\[
\begin{pmatrix} w & x \\ y & z \end{pmatrix}\cdot
\begin{pmatrix} z & -y \\ -x & w \end{pmatrix}=
-(xy-wz)\cdot
\begin{pmatrix} 1 & 0 \\ 0 & 1 \end{pmatrix}=
\begin{pmatrix} z & -y \\ -x & w \end{pmatrix}\cdot
\begin{pmatrix} w & x \\ y & z \end{pmatrix}
\]

Let now $C_1$ and $C_0$ be free $A$-modules of rank $r$, and let
\[
d_1\colon C_1\to C_0 \qquad\text{and}\qquad s_0\colon C_0\to C_1
\]
be $A$-linear homomorphisms defined by the matrices $U$ and $V$,
respectively, after bases have been tacitly chosen.  

The second condition on the matrices $U$ and $V$ implies that $d_1$ is
injective, while the first condition on these matrices shows that
$fC_0$ is contained in $\Ima(d_1)$.  Setting $L=\Coker(d_1)$, one sees
that the chosen matrix factorization defines a commutative diagram with
exact rows
\[
\xymatrixrowsep{3pc}
\xymatrixcolsep{4pc}
\xymatrix{
0
\ar[r]
& C_1
\ar[r]^-{d_1}
\ar[d]_{-f\cdot 1_{C_1}}
& C_0
\ar[r]
\ar[d]^{-f\cdot 1_{C_0}}
\ar[dl]_-{s_0}
& L
\ar[r]
\ar[d]_{0_L=}^{-f\cdot1_L}
& 0
\\
0
\ar[r]
& C_1
\ar[r]^-{d_1}
& C_0
\ar[r]
& L
\ar[r]
& 0
}
\]
which expresses the following facts: $C=\ 0\to C_1\xrightarrow{d_1}
C_0\to 0$ is a free resolution of the $A$-module $L$,  this module is
annihilated by $f$, and $s_0$ is a homotopy between the maps
$-f\cdot 1_C$ and $0_C$, both of which lift $-f\cdot 1_L$.

Conversely, if an $A$-module $L$ annihilated by $f$ has a free
resolution $(C,d_1)$ of length $1$, then $\rank_AC_1=\rank_AC_0$, and
any choice of homotopy $s_0$ between $-f\cdot 1_C$ and $0_C$ provides a
matrix factorization of $-f$.

When we already have an $A$-module $L$ with a presentation matrix $U$
that defines an injective $A$-linear map, we can use \Mtwo to create a
matrix factorization $(U,V)$ of $-f$.

\begin{Example}
\label{familiar}
We revisit the familiar example from a higher perspective.
\beginOutput
i1 : A = QQ[w,x,y,z]\\
\emptyLine
o1 = A\\
\emptyLine
o1 : PolynomialRing\\
\endOutput
\beginOutput
i2 : U = matrix \{\{w,x\},\{y,z\}\}\\
\emptyLine
o2 = | w x |\\
\     | y z |\\
\emptyLine
\             2       2\\
o2 : Matrix A  <--- A\\
\endOutput
\beginOutput
i3 : C = chainComplex U\\
\emptyLine
\      2      2\\
o3 = A  <-- A\\
\             \\
\     0      1\\
\emptyLine
o3 : ChainComplex\\
\endOutput
\beginOutput
i4 : L = HH_0 C\\
\emptyLine
o4 = cokernel | w x |\\
\              | y z |\\
\emptyLine
\                            2\\
o4 : A-module, quotient of A\\
\endOutput
\beginOutput
i5 : f = -det U\\
\emptyLine
o5 = x*y - w*z\\
\emptyLine
o5 : A\\
\endOutput
Let's verify that $f$ annihilates $L$.
\beginOutput
i6 : f * L == 0\\
\emptyLine
o6 = true\\
\endOutput
We use the {\tt nullhomotopy} function.\indexcmd{nullhomotopy}
\beginOutput
i7 : s = nullhomotopy (-f * id_C)\\
\emptyLine
\          2                     2\\
o7 = 1 : A  <----------------- A  : 0\\
\               \{1\} | z  -x |\\
\               \{1\} | -y w  |\\
\emptyLine
o7 : ChainComplexMap\\
\endOutput
Let's verify that $s$ is a null-homotopy for $-f$, using {\tt
C.dd}\indexcmd{dd} to obtain the differential of the chain complex {\tt
C} as a map of graded modules.
\beginOutput
i8 : s * C.dd + C.dd * s == -f\\
\emptyLine
o8 = true\\
\endOutput
We extract the matrix $V$ from the null-homotopy to get our factorization.
\beginOutput
i9 : V = s_0\\
\emptyLine
o9 = \{1\} | z  -x |\\
\     \{1\} | -y w  |\\
\emptyLine
\             2       2\\
o9 : Matrix A  <--- A\\
\endOutput
\end{Example}

For every $f$ and every $r\ge1$ there exists a trivial matrix
factorization of $-f$, namely, $(f\cdot I_k, -I_k)$; it can be obtained
from the $A$-module $L=A^k/fA^k$.  In general, it may not be clear how
to find an $A$-module $L$ with the properties necessary to obtain an
`interesting' matrix factorization of $-f$.

However, in some cases the supply is plentiful.

\begin{Remark}
\label{factorization}
Let $A$ be a graded polynomial ring in $e$ variables of positive degree
over a field $K$, let $f$ be a homogeneous polynomial in $A$, and set
$B=A/(f)$.  Every $B$-module $M$ of infinite projective dimension
{\it\ie{generates}\/} a matrix factorization $(U,V)$ of $-f$, as follows.

Let $(F,d_F)$ be a minimal graded free resolution of $M$ over $B$, and
set $L=\Coker\big(d_F\colon F_{e+1}\to F_e\big)$.  Since $M$ has
infinite projective dimension, we have $L\ne0$.  By the Depth Lemma,
$\depth_BL=\depth B$.  On the other hand, $\depth_BL=\depth_AL$ and
$\depth B=\depth A-1$.  By Hilbert's Syzygy Theorem, the minimal graded
free resolution $(C,d_C)$ of $L$ over $A$ is finite.  By the
Auslander-Buchsbaum Equality, $C_n=0$ for $n>\depth A-\depth_AL=1$.

The minimality of $F$ ensures that all nonzero entries of the
presentation matrix $U$ of $L$ are forms of positive degree.  On
the other hand, by \cite[Sect.~0]{CI:Ei} the module $L$ has no direct
summand isomorphic to $B$: it follows that all nonzero entries of
the homotopy matrix $V$ are forms of positive degree (this is the
reason for choosing $L$ as above---stopping one step earlier in the
resolution $F$ could have produced a module $L$ with a non-zero free
direct summand).
 \end{Remark}

Our reader would have noticed that \Mtwo can read all the data and
perform all the operations needed to construct a module $L$ by the
procedure described in the preceding remark.  Here is how it does it.

\begin{Example}
\label{square}
We produce a matrix factorization of $-f$, where 
\[
f=x^3 + 3y^3 - 2yz^2 + 5z^3 \in\mathbb Q[x,y,z]=A
\]
generated by the module $M=B/{\mathfrak m}^2$, where $B=A/(f)$ and
${\mathfrak m}=(x,y,z)B$.
\beginOutput
i10 : A = QQ[x,y,z];\\
\endOutput
\beginOutput
i11 : f = x^3 + 3*y^3 - 2*y*z^2 + 5*z^3;\\
\endOutput
\beginOutput
i12 : B = A/f;\\
\endOutput
\beginOutput
i13 : m = ideal(x,y,z)\\
\emptyLine
o13 = ideal (x, y, z)\\
\emptyLine
o13 : Ideal of B\\
\endOutput
Let's take the $B$-module $M$ and compute its minimal $B$-free
resolution.
\beginOutput
i14 : M = B^1/m^2;\\
\endOutput
\beginOutput
i15 : F = resolution(M, LengthLimit=>8)\\
\emptyLine
\       1      6      9      9      9      9      9      9      9\\
o15 = B  <-- B  <-- B  <-- B  <-- B  <-- B  <-- B  <-- B  <-- B\\
\                                                               \\
\      0      1      2      3      4      5      6      7      8\\
\emptyLine
o15 : ChainComplex\\
\endOutput
We introduce a function {\tt restrict1 N}  which accepts a $B$-module
$N$ and restricts scalars to produce an $A$-module.
\beginOutput
i16 : restrict1 = N -> coker(lift(presentation N,A) | f);\\
\endOutput
Now make $L$ as described in Remark \ref{factorization}
\beginOutput
i17 : L = restrict1 cokernel F.dd_4;\\
\endOutput
We proceed as in Example \ref{familiar} to get a matrix factorization.
\beginOutput
i18 : C = res L;\\
\endOutput
\beginOutput
i19 : U = C.dd_1;\\
\emptyLine
\              9       9\\
o19 : Matrix A  <--- A\\
\endOutput
\beginOutput
i20 : print U\\
\{4\} | 0  xy x2       y2    0        0        0        yz-5/2z2 0      |\\
\{4\} | 0  x2 -3y2     xy    yz-5/2z2 0        yz-5/2z2 0        0      |\\
\{4\} | x2 0  -2yz+5z2 0     y2-5/2yz yz-5/2z2 -5/2yz   0        0      |\\
\{5\} | 0  0  0        1/3z  0        0        0        1/2y     x      |\\
\{5\} | 0  0  -z       0     1/2y     0        1/2y     -1/2x    0      |\\
\{5\} | 0  -z 0        0     -1/2x    0        -1/2x    0        3y     |\\
\{5\} | 0  0  0        -1/3x 0        1/2y     -1/3z    0        0      |\\
\{5\} | -z y  x        0     0        -1/2x    0        0        0      |\\
\{5\} | y  0  0        0     0        0        1/3x     0        -2y+5z |\\
\endOutput
\beginOutput
i21 : s = nullhomotopy (-f * id_C);\\
\endOutput
\beginOutput
i22 : V = s_0;\\
\emptyLine
\              9       9\\
o22 : Matrix A  <--- A\\
\endOutput
\beginOutput
i23 : print V\\
\{6\} | 0   0  -x  0         0         -2y2+5yz 0         -2yz+5z2 -3y2 |\\
\{6\} | 0   -x 0   0         0         -2yz+5z2 -3xy      -3y2     -3yz |\\
\{6\} | -x  y  0   0         -2yz+5z2  0        0         0        0    |\\
\{6\} | -3y 0  0   6yz-15z2  0         0        3x2       3xy      3xz  |\\
\{6\} | 0   2z -3y -15xz     -15yz     2x2      6yz-15z2  0        3x2  |\\
\{6\} | -2x 0  2z  0         -4yz+10z2 0        -6y2      2x2      0    |\\
\{6\} | 0   0  3y  -6xy+15xz -6y2+15yz 0        -6yz+15z2 0        -3x2 |\\
\{6\} | 2z  0  0   -6y2      2x2       2xy      0         0        0    |\\
\{6\} | 0   0  0   -x2       -xy       -y2      -xz       -yz      -z2  |\\
\endOutput
\beginOutput
i24 : U*V+f==0\\
\emptyLine
o24 = true\\
\endOutput
\beginOutput
i25 : V*U+f==0\\
\emptyLine
o25 = true\\
\endOutput
\end{Example}

The procedure described above can be automated for more pleasant usage.

\begin{code}
\label{factorization code}
The function {\tt matrixFactorization M} produces a matrix factorization
$(U,V)$ of $-f$ generated by a module $M$ over $B=A/(f)$.
\beginOutput
i26 : matrixFactorization = M -> (\\
\         B := ring M;\\
\         f := (ideal B)_0;\\
\         e := numgens B;\\
\         F := resolution(M, LengthLimit => e+1);\\
\         L := restrict1 cokernel F.dd_(e+1);\\
\         C := res L;\\
\         U := C.dd_1;\\
\         s := nullhomotopy (-f * id_C);\\
\         V := s_0;\\
\         assert( U*V + f == 0 );\\
\         assert( V*U + f == 0 );\\
\         return (U,V));\\
\endOutput
We use the {\tt assert}\indexcmd{assert} command to signal an error in
case the matrices found don't satisfy our requirement for a matrix
factorization.
 \end{code}

Let's illustrate the new code with a slightly bigger module $M$ than
before.

\begin{Example}
\label{cube}
With the same $A$, $f$, $B$, and $\mathfrak m$ as in Example
\ref{square}, we produce a matrix factorization generated by the
$B$-module $M=B/{\mathfrak m}^3$.
\beginOutput
i27 : time (U,V) = matrixFactorization(B^1/m^3);\\
\     -- used 0.21 seconds\\
\endOutput
The parallel assignment statement above provides both variables {\tt U}
and {\tt V} with matrix values.  We examine their shapes without
viewing the matrices themselves by appending a semicolon to the
appropriate command.
\beginOutput
i28 : U;\\
\emptyLine
\              15       15\\
o28 : Matrix A   <--- A\\
\endOutput
\beginOutput
i29 : V;\\
\emptyLine
\              15       15\\
o29 : Matrix A   <--- A\\
\endOutput
\end{Example}

Matrix factorizations were introduced to construct resolutions over the
the residue ring $B=A/(f)$, using the following observation.

\begin{Remark}
\label{periodicity}
If $(U,V)$ is a factorization of $-f$ by $k\times k$ matrices and the
maps $d_1\colon C_1\to C_0$ and $s_0\colon C_0\to C_1$ are
homomorphisms of free $A$-modules defined by $U$ and $V$, respectively,
then the sequence
\[
\cdots \to
C_1\otimes_AB \xrightarrow{d_1\o 1_B}
C_0\otimes_AB \xrightarrow{s_0\o 1_B}
C_1\otimes_AB \xrightarrow{d_1\o 1_B}
C_0\otimes_AB\to0
\]
of $B$-linear maps is a free resolution of the $B$-module
$L=\Coker(d_1)$.

Indeed, freeness is clear, and we have a complex because $d_1s_0=
-f\cdot 1_{C_0}$ and $s_0d_1=-f\cdot 1_{C_1}$.  If $x\in C_1$ satisfies
$\big(d_1\o1_B\big)(x\o 1)=0$, then $d_1(x)=fy$ for some $y\in C_0$,
hence $d_1x=d_1s_0(y)$.  As $d_1$ is injective, we get $x=s_0(y)$, so
$\Ker\big(d_1\o1_B\big)\subseteq \Ima\big(s_0\o1_B\big)$; the reverse
inclusion follows by symmetry.
 \end{Remark}

Pooling Remarks \ref{factorization} and \ref{periodicity} we recover
Eisenbud's result \cite[Sect.\ 6]{CI:Ei}.

\begin{theorem}
Let $A$ be a graded polynomial ring in $e$ variables of positive degree
over a field $K$, and $f$ a homogeneous polynomial in $A$.  The minimal
graded free resolution of every finitely generated graded module over
$B=A/(f)$ becomes periodic of period $2$ after at most $e$ steps.  The
periodic part of the resolution is given by a matrix factorization of
$-f$ generated by $M$.
 \end{theorem}

We illustrate the theorem on an already computed example.

\begin{Example}
Let $A$, $f$, $B$, $M$, and $F$ be as in Example \ref{square}.

To verify the periodicity of $F$ we subtract pairs of differentials and
compare the result with $0$: direct comparison of the differentials 
would not work, because the free modules involved have different degrees.
\beginOutput
i30 : F.dd_3 - F.dd_5 == 0\\
\emptyLine
o30 = false\\
\endOutput
\beginOutput
i31 : F.dd_4 - F.dd_6 == 0\\
\emptyLine
o31 = false\\
\endOutput
\beginOutput
i32 : F.dd_5 - F.dd_7 == 0\\
\emptyLine
o32 = true\\
\endOutput
The first two answers above come as a surprise---and suggest a property
of $F$ that is weaker than the one we already know to be true!  

There is an easy explanation: we checked the syzygy modules for {\it
equality\/}, rather than for {\it isomorphism\/}.  We do not know why
\Mtwo didn't produce an equality at the earliest possible stage, nor why it
eventually produced one.  The program has other strategies for computing
resolutions, so let's try one.
\beginOutput
i33 : M = B^1/m^2;\\
\endOutput
\beginOutput
i34 : G = resolution(M, LengthLimit => 8, Strategy => 0)\\
\emptyLine
\       1      6      9      9      9      9      9      9      9\\
o34 = B  <-- B  <-- B  <-- B  <-- B  <-- B  <-- B  <-- B  <-- B\\
\                                                               \\
\      0      1      2      3      4      5      6      7      8\\
\emptyLine
o34 : ChainComplex\\
\endOutput
\beginOutput
i35 : G.dd_3 - G.dd_5 == 0\\
\emptyLine
o35 = true\\
\endOutput
\beginOutput
i36 : G.dd_4 - G.dd_6 == 0\\
\emptyLine
o36 = true\\
\endOutput
\beginOutput
i37 : G.dd_5 - G.dd_7 == 0\\
\emptyLine
o37 = true\\
\endOutput
The strategy paid off, revealing periodicity at the earliest possible
stage.  However, the algorithm used may be a lot slower that the
default algorithm.
 \end{Example}

\section{Graded Algebras}
\label{Graded algebras}

We describe some standard universal algebras over a commutative ring $A$.

Let $Q$ denote a free $A$-module of rank $c$, and set $Q^*=\Hom_A(Q,A)$.
We assign degree $2$ to the elements of $Q$, and degree $-2$ to those
of $Q^*$.  We let $Q^\wedge$ denote a copy of $Q$ whose elements are
assigned degree $1$; if $x$ is an element of $Q$, then $x^\wedge$
denotes the corresponding element of $Q^\wedge$.

We use $\a = (\a_1 , \dots , \a_c ) \in \Z^c$ as a multi-index, set
$|\a| = \sum_i \a_i$, and order $\Z^c$ by the rule: $\a \ge \b$ means
$\a_i \ge \b_i$ for each $i$.  We let $o$ denote the trivial element
of $\Z^c$, and $\e_i$ the $i$'th element of its standard basis.

\begin{construction}
\label{algebras}
For each integer ${m}\ge0$ we form free $A$-modules
\begin{gather*}
\SS^{m}(Q^*) \quad\text{with basis}\quad \big\{X^{\a} : |\a|={m}\big\}\\
\DD^{m}(Q) \quad\text{with basis}\quad \big\{Y^{(\a)} : |\a|={m}\big\}\\
\Wedge^{m}(Q^\wedge) \quad\text{with basis}\quad 
\big\{Y^{\wedge\a} : |\a|={m} \quad\text{and}\quad\a\le(\e_1+\cdots+\e_c)\big\}
\end{gather*}
For $m<0$ we declare the modules $\SS^{m}(Q^*)$, $\DD^{m}(Q)$,
and $\Wedge^{m}(Q^\wedge)$ to be equal to $0$, and define the symbols
$X^{\a}$, $Y^{(\a)}$, and $Y^{\wedge\a}$ accordingly; in addition,
we set $\Wedge^{m}(Q^\wedge)=0$ and $Y^{\wedge\a}=0$ if $|\a|\not\le
(\e_1+\cdots+\e_c)$, and we set
\begin{gather*}
X_i=X^{\e_i}\qquad Y_i=Y^{(\e_i)}\qquad Y_i^\wedge=Y^{\wedge\e_i}
\qquad\text{for}\qquad i=1,\dots,c
\end{gather*}
Taking $\SS^{m}(Q^*)$, $\DD^{m}(Q)$, and $\Wedge{}^{m}(Q^\wedge)$ as
homogeneous components of degree $-2m$, $2m$, and $m$, respectively,
we introduce graded algebras
\[
S = \SS(Q^*)\qquad
D = \DD(Q)\qquad
E = \Wedge(Q^\wedge)
\]
by defining products of basis elements by the formulas
\begin{gather*}
X^{\a}\cdot X^{\b}=X^{\a+\b}\\
Y^{(\a)}\cdot Y^{(\b)}=
\prod_{i=1}^c\frac{(\a_i+\b_i)!}{\a_i!\b_i!}Y^{(\a+\b)}\\
Y^{\wedge\a}\cdot Y^{\wedge\b}=\inv(\a,\b)Y^{\wedge\,\a+\b}
\end{gather*}
where $\inv(\a,\b)$ denotes the number of pairs $(i,j)$ with
$\a_i=\b_j=1$ and $i>j$.  Thus, $S$ is the {\it\ie{symmetric
algebra}\/} of $Q^*$, with $X^{o}=1$, while $D$ is the {\it\ie{divided
powers algebra}\/} of $Q$, with $Y^{(o)}=1$, and $E$ is the
{\it\ie{exterior algebra}\/} of $Q^\wedge$, with $Y^{\wedge\,o}=1$.  We
identify $S$ and the polynomial ring $A[X_1,\dots,X_c]$.
 \end{construction}

A {\it\ie{homogeneous derivation}\/} of a graded $A$-algebra $W$ is a
homogeneous $A$-linear map $d\colon W\to W$ such that the
{\it\ie{Leibniz rule}\/}
\[
d(x y) = d(x) y + (-1)^{\deg x\cdot\deg d} x d(y)
\]
holds for all homogeneous elements $x,y\in W$.

\begin{construction}
\label{koszul}
Each sequence $f_1,\dots,f_c\in A$ yields a {\it\ie{Koszul map}\/}
\[
\begin{gathered}
d_E \colon E \to E
\qquad\text{defined by the formula}\\
d_E(Y^{\wedge\b}) =
\sum_{i=1}^c (-1)^{\b_1+\cdots+\b_{i-1}}f_iY^{\wedge\, \b-\e_i}
\end{gathered}
\]
It is a derivation of degree $-1$ and satisfies $d_E^2 = 0$.
 \end{construction}

\begin{construction}
\label{actions}
For every $X_i\in\SS^1(Q^*)$ and each $Y^{(\b)} \in \DD^{m}(Q)$ we set
\[
X_i \lcontract Y^{(\b)} = Y^{(\b-\e_i)} \in \DD^{{m}-1}(Q)
\]
Extending this formula by $A$-bilinearity, we define $g\lcontract y$ for 
all $g\in\SS^1(Q^*)$ and all $y\in D$.  It is well known, and easily
verified, that the map $g\lcontract\colon y\mapsto g\lcontract y$ is a
graded derivation $D\to D$ of degree $-2$, and that the derivations
associated with arbitrary $g$ and $g'$ commute.  As a consequence, the
formula
\[
X^\a\lcontract Y^{(\b)} =
(X_1\lcontract)^{\a_1}\cdots(X_c\lcontract)^{\a_c}\big(Y^{(\b)}\big)
\in\DD^{|\b-\a|}(Q)
\]
extended $A$-linearly to all $u\in S$, defines on $D$ a structure of
graded $S$-module.

The usual products on $S\otimes_AE$ and $D\otimes_A E$ and the induced
gradings
\begin{gather*}
(S\otimes_AE)_n=
\bigoplus_{\ell-2k=n}\SS^{k}(Q^*)\otimes_A \Wedge{}^{\ell}(Q^\wedge)\\
(D\otimes_AE)_n=
\bigoplus_{\ell+2k=n}\DD^{(k)}(Q)\otimes_A \Wedge{}^{\ell}(Q^\wedge)
\end{gather*}
turn $S\otimes_AE$ and $D\otimes_A E$ into graded algebras.  The
second one is a graded module over the first, for the action
$(u\o z)\cdot(y\o z')=(u\lcontract v)\o(z\cdot z')$.
  \end{construction}

\begin{construction}
\label{cartan}
The element $w=\sum_{i=1}^c X_i\o Y_i^\wedge$ yields a {\it\ie{Cartan
map}\/} 
\[
\begin{gathered}
d_{DE} \colon D \otimes_A E \to D \otimes_A E
\qquad\text{defined by the formula}\\
d_{DE}( y \o z ) = w\cdot(y \o z)
= \sum_{i=1}^c (X_i \lcontract y) \o (Y_i^\wedge\cdot z)
\end{gathered}
\]
It is an $E$-linear derivation of degree $-1$, and $d_{DE}^2 = 0$
because $w^2=0$.  
 \end{construction}

\begin{lemma}
\label{split}
For each integer $s$ define a complex $G^s$ as follows:
\[
\cdots\to \DD^{k}(Q)\otimes_A \Wedge^{s-k}(Q^\wedge)\xrightarrow{w}
\DD^{k-1}(Q)\otimes_A \Wedge^{s-k+1}(Q^\wedge) \to \cdots 
\]
with $\DD^{0}(Q)\otimes_A \Wedge^{s}(Q^\wedge)$ in degree $s$.  If
$s>0$, then $G^s$ is split exact.
\end{lemma}

\begin{proof}
Note that for each $s\in\Z$ there exist isomorphisms of complexes
$\bigoplus_{s=1}^\infty G^s\cong (D \otimes_A E)_{\ge1} \cong
\big(\bigotimes_{i=1}^c G(i)\big)_{\ge1}$, where $G(i)$ is the complex
\[
\cdots\to
AY^{(k+1)}_i\otimes_A A
\xrightarrow{w_i}(AY^{(k)}_i)\otimes(AY^\wedge_i)
\xrightarrow{0}(AY^{(k)}_i)\otimes_A A
\to\cdots
\]
and $w_i$ is left multiplication with $X_i\o Y_i^\wedge$.  This map
bijective, so each complex $G(i)_{\ge1}$ is split exact.  The assertion
follows.  \qed
 \end{proof}

\section{Universal Homotopies}
\label{Universal homotopies}

This section contains the main new mathematical result of the paper.  

We introduce a universal construction, that takes as input a projective
resolution $C$ of an $A$-module $M$ and a finite set $\boldsymbol f$ of
elements annihilating $M$; the output is a new projective resolution of
$M$ over $A$.  If $\boldsymbol f\ne\varnothing$, then the new
resolution is infinite---even when $C$ is finite---because it encodes
additional data: the null-homotopies for $f\cdot1_C$ for all
$f\in\boldsymbol f$, all compositions of such homotopies, and all
relations between those compositions.  This higher-order information
tracks the transformation of the homological properties of $M$ when its
ring of operators is changed from $A$ to $A/({\boldsymbol f})$.

Our construction is motivated by, and is similar to, one due to Shamash
\cite{CI:Sh} and Eisenbud \cite{CI:Ei}: assuming that the elements of
$\boldsymbol f$ form an $A$-regular sequence, they produce a projective
resolution of $M$ over $A/({\boldsymbol f})$.  By contrast, we make no
assumption whatsoever on $\boldsymbol f$.  With the additional
hypothesis, in the next section we quickly recover the original result
from the one below.  As an added benefit, we eliminate the use of
spectral sequences from the proof.

\begin{theorem}
\label{main}
Let $A$ be a commutative ring, let $f_1 , \dots, f_c$ be a sequence of
elements of $A$, let $M$ be an $A$-module annihilated by $f_i$ for
$i=1,\dots,c$, and let $r \colon  C \to M$ be a resolution of $M$ by
projective (respectively, free) $A$-modules.

There exists a family of homogeneous $A$-linear maps 
\[
\{d_\g \colon C \to C\mid \deg(d_\g)=2|\g| - 1\}_{\g \in \N^c}
\]
satisfying the following conditions
\begin{equation}
\label{family}
\begin{aligned}
d_o &= d_C
\quad\text{is the differential of}\quad C\\
[d_o,d_\g] &=
\begin{cases}
-f_i\cdot 1_C &\text{if\quad} \g=\e_i \text{ for } i=1,\dots,c\\
-\ssum{\a+\b=\g}d_\a d_\b      &\text{if\quad} |\g|\ge2
\end{cases}
\end{aligned}
\end{equation}
where $\sum^{\scriptscriptstyle +}$ denotes a summation restricted to
indices in $\N^c\smallsetminus\{o\}$.

Any family $\{d_\g\}_{\g\in\N^c}$ as above defines an $A$-linear map
of degree $-1$, 
\begin{equation}
\begin{gathered}
\label{dCD}
d_{CD} \colon C\otimes_AD \to C\otimes_AD 
\qquad\text{given by}\\
d_{CD}(x\o y) = \sum_{\g \in\N^c} d_\g(x) \o (X^{\g} \lcontract y)
\end{gathered}
\end{equation}
where $D$ is the divided powers algebra defined in Construction
\ref{algebras}, and the action of $X^{\g}$ on $D$ is defined in
Construction \ref{actions}.

With $d_{E}$ and $d_{DE}$ defined in Constructions \ref{koszul} and
\ref{cartan} and the tensor product of maps of graded modules defined
as in Remark \ref{graded-map-tensor},
the map
\begin{equation}
\begin{gathered}
\label{diff}
d\colon C \otimes_A D \otimes_A E \to C \otimes_A D \otimes_A E
\qquad\text{given by}\\
d = d_{CD}\o 1_E + 1_C\o d_{DE} + 1_C\o 1_D\o d_E 
\end{gathered}
\end{equation}
is an $A$-linear differential of degree $-1$, and the map
\begin{gather*}
q\colon C \otimes_A D \otimes_A E \to M
\qquad\text{given by}\\
q(x\o y\o z)=
\begin{cases}
yz\cdot r(x) &\quad\text{if } \deg(y)=\deg(z)=0\\
0      &\quad\text{otherwise}
\end{cases}
\end{gather*}
is a resolution of $M$ by projective (respectively, free) $A$-modules.
 \end{theorem}

For use in the proof, we bring up a few general homological points.

A {\it\ie{bounded filtration}\/} of a chain complex $F$ is a sequence
\[
0=F^0\subseteq F^1\subseteq\cdots\subseteq F^{s-1}\subseteq
F^s\subseteq\cdots
\]
of subcomplexes such that for each $n$ there exists an $s$ with
$F^s_n=F_n$.  As usual, we let $\gr^s(F)$ denote the complex of
$A$-modules $F^s/F^{s-1}$.

\begin{lemma}
\label{filtration}
Let $q\colon F \to F'$ be a morphism of complexes with bounded
filtrations, such that $q(F^s)\subseteq F'{}^s$ for all $s \in \Z$.  If
for each $s$ the induced map $\gr^s(q)\colon\gr^s(F)\to\gr^s(F')$ is a
quasi-isomorphism, then so is $q$.
 \end{lemma}

\begin{proof}
Denoting $q^s$ the restriction of $q$ to $F^s$, we first show by
induction on $s$ that $\HH_n(q^s)$ is bijective for all $n$.  The
assertion is clear for $s=0$, since $F^0=0$ and $F^{\prime\,0} = 0$.
For the inductive step, we assume that $q^{s-1}$ is a quasi-isomorphism
for some $s\ge1$.  We have a commutative diagram of complexes
\[
    \xymatrix{
      0 \ar[r] & F^{s-1} \ar[d]_{q^{s-1}} \ar[r] & F^s \ar[d]_{q^{s}}\ar[r]
                & \gr^s(F) \ar[d]_{\gr^s(q)}\ar[r] & 0 \\
      0 \ar[r] & F^{\prime\,s-1} \ar[r]     & F^{\prime\, s} \ar[r]
                & \gr^s(F') \ar[r] & 0 
      }
\]
By hypothesis and inductive assumption, in the induced diagram
\[  
\xymatrixcolsep{0.65pc}
    \xymatrix{
      \HH_{n+1}(\gr^s(F)) \ar[d]_{\HH_{n+1}(\gr^s(q))}^\cong \ar[r] &
      \HH_n    (F^{s-1})  \ar[d]_{\HH_n(q^{s-1})}^\cong \ar[r] &
      \HH_n    (F^s)  \ar[d]_{\HH_n(q^s)}       \ar[r] &
      \HH_n    (\gr^s(F)) \ar[d]_{\HH_n(\gr^s(q))}^\cong \ar[r] &
      \HH_{n-1}(F^{s-1})  \ar[d]_{\HH_{n-1}(q^{s-1})}^\cong
      \\
      \HH_{n+1}(\gr^s(F'))        \ar[r] &
      \HH_n    (F^{\prime\,s-1})         \ar[r] &
      \HH_n    (F^{\prime\,s})         \ar[r] &
      \HH_n    (\gr^s{F'})         \ar[r] &
      \HH_{n-1}(F^{\prime\,s-1})
      }
\]  
the four outer vertical maps are bijective.  By the Five-Lemma, so is
$\HH_n(q^s)$.

Now we fix an integer $n \in \Z$, and pick $s$ so large that
\[
F^s_{k}=F_{k} \qquad\text{and}\qquad
F^{\prime\, s}_{k}=F'_{k}\qquad\text{hold for}\qquad
k=n-1,n,n+1\,.
\]
The choice of $s$ implies that $\HH_n(F^{s}) = \HH_n(F)$,
$\HH_n(F^{\prime\,s}) = \HH_n(F')$, and $\HH_n(q^s) = \HH_n(q)$.  Since
we have already proved that $\HH_n(q^s)$ is an isomorphism, we conclude
that $\HH_n(q)\colon \HH_n(F)\to \HH_n(F')$ is an isomorphism.
 \qed \end{proof}

\begin{Remark}
\label{commutator}
If $(F,d_F)$ is a complex of $A$-modules, then $\Hom^{\gr}_A(F,F)$
denotes the graded module whose $n$'th component consists of the
$A$-linear maps $g\colon F\to F$ with $g(F_i)\subseteq F_{i+n}$ for all
$i\in\Z$.

If $g,h$ are homogeneous
$A$-linear maps, then their composition $gh$ is homogeneous of degree
$\deg(g)+\deg(h)$, and so is their graded commutator
\[
[g,h] = g h - (-1)^{\deg g\cdot \deg h} h g
\]
Commutation is a graded derivation: for each homogeneous map $h'$ one has
\[
[g,hh']= [g,h]h'+(-1)^{\deg g\cdot \deg h}h[g,h']
\]

The map $h\mapsto [d_F,h]$ has square $0$, and transforms
$\Hom^{\gr}_A(F,F)$ into a complex of $A$-modules; by definition, its
cycles are the chain maps $F\to F$, and its boundaries are the
null-homotopic maps.
 \end{Remark}

\begin{Remark}\label{graded-map-tensor}
If $p\colon F \to F'$ and $q\colon G \to G'$ are graded maps of graded
modules, we define the tensor product $p \otimes q\colon F \otimes F'
\to G \otimes G'$ by the formula $(p \otimes q)(f \otimes g)=(-1)^{\deg
q \cdot\deg f}( p(f) \otimes q(g) )$.  With this convention, when $F=F'$
and $G=G'$, the graded commutator $[1_F \otimes q,p \otimes 1_G]$
vanishes.
 \end{Remark}

\begin{lemma}
\label{quism}
Let $M$ be an $A$-module and let $r\colon C\to M$ be a free resolution.
If $g\colon C\to C$ is an $A$-linear map with $\deg(g)>0$, and
$[d_C,g]=0$, then $g=[d_C,h]$ for some $A$-linear map $h\colon C\to C$
with $\deg(h)=\deg(g)+1$.
 \end{lemma}

\begin{proof}
The augmentation $r\colon C \to M$ defines a chain map of degree zero
\[
\Hom^\gr_A(C,r)\colon \Hom^\gr_A(C,C)\to\Hom^\gr_A(C,M)
\]
The map induced in homology is an isomorphism: to see this, apply the
`comparison theorem for projective resolutions'.  Since $A$-linear maps
$C\to M$ of positive degree are trivial, the conclusion follows from
Remark \ref{commutator}. \qed
 \end{proof}

\begin{proof}[of Theorem \ref{main}]
Recall that $D$ is the divided powers algebra of a free $A$-module $Q$
with basis $Y_1,\dots, Y_c$, that $X_1, \dots,X_c$ is the dual basis of
the free $A$-module $Q^*$, and $S$ for the symmetric algebra of $Q^*$,
see Construction \ref{algebras} for details.  We set $f=\sum_{i=1}^c
f_iX_i \in\SS^{1}(Q^*)$.

We first construct the maps $d_\g$ by induction on $|\g|$.

If $|\g|=0$, then $\g=o$, so $d_o=d_C$ is predefined.  If $|\g|=1$,
then $\g=\e_i$ for some $i$ with $1\le i\le c$.  Since $f_i$
annihilates the $B$-module $M$, the map $-f_i \cdot 1_C$ lifts the
zero map on $M$, hence is null-homotopic.  For each $i$ we take
$d_{\e_i}$ to be a null-homotopy, that is, $[d_o,d_{\e_i}]=-f_i
\cdot 1_C$.  With these choices, the desired formulas hold for all $\g$
with $|\g|\le1$.

Assume by induction that maps $d_\g$ satisfying the conclusion of the
lemma have been chosen for all $\g\in\N^c$ with $|\g|<n$, for some 
$n\ge2$.  Fix $\g\in\N^c$ with $|\g|=n$.  Using Remark \ref{commutator}
and the induction hypothesis, we obtain
\begin{align*}
\bigg[d_o,\ssum{\a+\b=\g} d_\a d_\b\bigg]
&=\ssum{\a+\b=\g}\big([d_o,d_\a] d_\b -d_\a [d_o,d_\b]\big)\\
&=\ssum{\a+\b=\g}\bigg(
\bigg(\ssum{\a'+\a''=\a} d_{\a'} d_{\a''}\bigg)d_\b
-d_\a\bigg(\ssum{\b'+\b''=\b} d_{\b'} d_{\b''}\bigg)\bigg)\\
&=\ssum{\a'+\a''+\b=\g}d_{\a'} d_{\a''} d_{\b}
-\ssum{\a+\b'+\b''=\g}d_{\a} d_{\b'} d_{\b''}\\
&=0
\end{align*}
The map $-\sum^+_{\a+\b=\g} d_\a d_\b$ has degree $2|\g|-2$, so by
Lemma \ref{quism} it is equal to $[d_o,d_\g]$ for some $A$-linear map
$d_\g\colon C\to C$ of degree $2|\g|-1$.  Choosing such a $d_\g$ for
each $\g\in\N^c$ with $|\g|=n$, we complete the step of the induction.

From the definition of $d$ we obtain an expression
\begin{align*}
d^2 =& d_{CD}^2\o 1_E +
1_C\o[d_{DE}\,,\,1_D\o d_E]+
[d_{CD}\o 1_E\,,\,1_C\o d_{DE}]+
\\
&1_C\o d_{DE}^2 +
1_C\o 1_D\o d_E^2 +
[d_{CD}\o 1_E\,,\,1_C\o 1_D\o d_E]
\end{align*}
Constructions \ref{cartan}, \ref{koszul}, and Remark
\ref{graded-map-tensor} show that the maps in the second row are equal
to $0$, so to prove that $d^2=0$ it suffices to establish the
equalities
\begin{align}
\label{first}
        d_{CD}^2 & = -f\cdot 1_{C\o D} \\
\label{second}
        [d_{DE}\,,\,1_D\o d_E] & = f\cdot 1_{D\o E}\\
\label{third}
        [d_{CD}\o 1_E\,,\,1_C\o d_{DE}] & =  0
\end{align}

A direct computation with formula \eqref{family} proves equality
\eqref{first} above:
\begin{align*}
d_{CD}^2(x\o y)
&=d_{CD}\bigg(\sum_{\b\in\N^c} d_\b(x)\o\big(X^{\b}\lcontract y\big)\bigg)\\
&=\sum_{\b\in\N^c}\bigg
(\sum_{\a\in\N^c}d_\a d_\b(x)\o
\big(X^{\a}\lcontract\big(X^{\b}\lcontract y\big)\big)\bigg)\\
&=\sum_{\a+\b\in\N^c}d_\a d_\b(x)\o \big(X^{\a+\b}\lcontract y\big)\\
&=\sum_{i=1}^c -f_ix\o (X_i\lcontract y)\\
&= -f\cdot (x\o y)
\end{align*}

By Constructions \ref{koszul} and \ref{cartan}, the maps $1_D\o d_E$
and $d_{DE}$ are derivations of degree $-1$, so the commutator
$[d_{DE},1_D\o d_E]$ is a derivation of degree $-2$.  Every element of
$D \otimes_A E$ is a product of elements $1\o Y^\wedge_i$ of degree $1$
and $Y^{(k)}_j\o1$ of degree $2k$, so it suffices to check that the map
on either side of \eqref{third} takes the same value on those
elements.  For degree reasons, both sides vanish on $1\o Y^\wedge_i$.
We now complete the proof of equality \eqref{second} as follows:
\begin{align*}
[d_{DE}\,,\,&1_E\o d_E]\big(Y^{(k)}_j \o 1\big)\\
&=d_{DE}\big((1_D\o d_E)\big(Y^{(k)}_j \o 1\big)\big) +
(1_E\o d_E)\big(d_{DE}\big(Y^{(k)}_j \o 1\big)\big)\\
&= (1_E\o d_E)\big(Y^{(k-1)}_j \o Y_j^\wedge\big)\\
&= Y^{(k-1)}_j \o f_j\\
&= f\cdot\big(Y^{(k)}_j \o 1\big)
\end{align*}

To derive equation \eqref{third} we use Constructions \ref{koszul} and
\ref{cartan} once again:
\begin{align*}
\big((d_{CD}\o 1_E)&(1_C\o d_{DE})\big)(x\o y\o z)\\
=&(-1)^{\deg x} d_{CD}\bigg(
\sum_{i=1}^c x\o(X_i\lcontract y)\o(Y^\wedge_i\cdot z)\bigg)\\
=&(-1)^{\deg x} \sum_{i=1}^c \sum_{\g\in\N^c}
d_\g(x)\o\big(X^{\g}\lcontract(X_i\lcontract y)\big)
\o (Y^\wedge_i\cdot z)\\
=&-\sum_{\g\in\N^c} \sum_{i=1}^c 
(-1)^{\deg(d_\g(x))}d_\g(x)\o\big(X_i\lcontract(X^{\g}\lcontract y)\big)
\o (Y^\wedge_i\cdot z)\\
=&-\big(1_C\o d_{DE}\big)
\bigg(\sum_{\g\in\N^c}d_\g(x)\o(X^{\g}\lcontract y) \o z\bigg)\\
=&-\big((1_C\o d_{DE})(d_{CD}\o 1_E)\big)(x\o y\o z)
\end{align*}

It remains to show $q$ is a quasi-isomorphism.  Setting
\[
F^s = \bigoplus_{k+\ell \le s} C \otimes_A \DD^k(Q)
\otimes_A \Wedge{}^\ell (Q^\wedge)
\qquad\text{for}\qquad s\in\Z
\]
we obtain a bounded filtration of the complex $F= (C \otimes_A D
\otimes_A E,d)$.  On the other hand, we let $F'$ denote the complex
with $F'_0=M$ and $F'_n=0$ for $n\ne0$; the filtration defined by
$F^{\prime\,0}=0$ and $F^{\prime\,s}=F'$ for $s\ge1$ is obviously
bounded, and $q(F^s)\subseteq F^{\prime\,s}$ holds for all $s\ge0$.  By
Lemma \ref{filtration} it suffices to show that the induced map
$\gr^s(q)\colon \gr^s(F)\to\gr^s(F')$ is bijective for all $s$.

Inspection of the differential $d$ of $F$ shows that $\gr^s(F)$ is
isomorphic to the tensor product of complexes $C\otimes_A G^s$, where
$G^s$ is the complex defined in Lemma \ref{split}.  It is established
there that $G^s$ is is split exact for $s>0$, hence $\HH_n(C\otimes_A
G^s)=0$ for all $n\in\Z$.  As $G^0=A$ and $\gr^0(q)=r$, we are done.
 \qed \end{proof}

\section{Cohomology Operators}
\label{Cohomology operators}

We present a new approach to the procedure of Shamash \cite{CI:Sh} and
Eisenbud \cite{CI:Ei} for building projective resolutions over a
complete intersection.  We then use this resolution to prove a
fundamental result of Gulliksen \cite{CI:MR51:487} on the structure of
Ext modules over complete intersections.

A set ${\boldsymbol f}=\{f_1 , \dots, f_c\}\subseteq A$ is
{\it\ie{Koszul-regular}\/} if the complex $(E,d_E)$ of Construction
\ref{koszul}, has $\HH_n(E)=0$ for $n>0$.  A sufficient condition for
Koszul-regularity is that the elements of ${\boldsymbol f}$, in some
order, form a regular sequence.

\begin{theorem}
\label{resolution}
Let $A$ be a commutative ring, ${\boldsymbol f}=\{f_1,\dots,f_c\}
\subseteq A$ a subset, $B=A/({\boldsymbol f})$ the residue ring,
$M$ a $B$-module, and $r \colon  C \to M$ a resolution of
$M$ by projective (respectively, free) $A$-modules.

Let $\{d_\g\colon C\to C\}_{\g\in\N^c}$ be a family of $A$-linear maps
provided by Theorem \ref{main}, set $D'=D\otimes_AB$, and $y'=y\o 1$
for $y\in D$.  The map
\begin{equation}
\begin{gathered}
\label{partial}
\partial \colon C\otimes_A D' \to C\otimes_A D'
\qquad\text{given by}\\
\partial(x\o y') = \sum_{\g \in\N^c} d_\g(x) \o (X^{\g} \lcontract y)'
\end{gathered}
\end{equation}
is a $B$-linear differential of degree $-1$.  If ${\boldsymbol f}$ is
Koszul-regular, then the map 
\begin{gather*}
    q'\colon C \otimes_A D' \to M \qquad \text{given by}\\
    q'(x\o y')= \begin{cases}
                    y \cdot r(x) &\quad\text{if $\deg(y')=0$}\\
                    0            &\quad\text{otherwise}
                \end{cases}
\end{gather*}
is a resolution of $M$ by projective (respectively, free) $A$-modules.
 \end{theorem}

\begin{Remark}
\label{hypersurface}
Assume that in the theorem ${\boldsymbol f}=\{f_1\}$.  The module
$D_\ell$ is then trivial if $\ell$ is odd, and is free with basis
consisting of a single element $Y_1^{(\ell/2)}$ if $\ell$ is even.
Thus, the resolution $C\otimes_AD'$ has the form
\[
\cdots\xrightarrow{\partial_{2n+1}}
\bigoplus_{j=0}^\infty C_{2j}\otimes_A BY_1^{(n-j)}
\xrightarrow{\partial_{2n}}
\bigoplus_{j=1}^\infty C_{2j-1}\otimes_A BY_1^{(n-j)}
\xrightarrow{\partial_{2n-1}}
\cdots
\]

The simplest situation occurs when, in addition, $C$ is a free
resolution with $C_n=0$ for $n\ge 2$.  In this case the differential
$d_o$ has a single non-zero component, $d_1\colon C_1\to C_0$, the
homotopy $d_{\e_1}$ between $-f\cdot 1_C$ and $0_C$ has a single
non-zero component, $s_0\colon C_0\to C_1$, and all the maps $d_\g$
with $\g\in\N^1\smallsetminus\{o,\e_1\}$ are trivial for degree
reasons.  It is now easy to see that the complex above coincides with
the one constructed, {\sl ad hoc\/}, in Remark \ref{periodicity}.
 \end{Remark}

\begin{proof}[of the theorem]
In the notation of Theorem \ref{main}, we have equalities
\[
C \otimes_A D'=(C\otimes_A D \otimes_A E)\otimes_E B
\qquad\text{and}\qquad
\partial= d\o 1_B 
\]
It follows that $\partial^2=0$.  For each $s\ge0$ consider the
subcomplexes
\begin{gather*}
F^s = \bigoplus_{k+\ell \le s} C_k \otimes_A D_\ell \otimes_A E
\qquad\text{of}\qquad F=C \otimes_A D \otimes_A E\\
F^{\prime\,s} = \bigoplus_{k+\ell \le s} C_k \otimes_A D'_\ell
\qquad\text{of}\qquad F'=C \otimes_A D'
\end{gather*}
They provide bounded filtrations of the complexes $F$ and $F'$,
respectively, such that the map $p'=1_C\o 1_D\o p\colon F\to F'$ 
satisfies $p'(F^s)\subseteq F^{\prime\,s}$ for all $s\ge0$.
Setting $G_s=\bigoplus_{k+\ell=s}(C_k \otimes_A D_\ell)$, we obtain
equalities $\gr^s(F)=(G_s\otimes_AE,1_{G_s}\o d_E)$ and $\gr^s(F')=
(G_s\otimes_AB,0)$ of complexes of $A$-modules.

If $\boldsymbol f$ is Koszul regular, then $p\colon E\to B$ is a
quasi-isomorphism, hence so is $1_{G_s}\o p=\gr^s(p')$ for each
$s\ge0$.  Lemma \ref{filtration} then shows that $p'$ is a
quasi-isomorphism.  The quasi-isomorphism $q\colon F\to M$ of Theorem
\ref{main} factors as $q=q'(1_C\o 1_D\o p)$, so we see that $q'$ is
a quasi-isomorphism, as desired.
 \qed\end{proof}

Let $M$ and $N$ be $B$-modules, and let $\Ext^\bu_B(M,N)$ denote the
graded $B$-module having $\Ext^n_B(M,N)$ as component of degree $-n$.
To avoid negative numbers, it is customary to regrade $\Ext^\bu_B(M,N)$
by {\it co\/}homological degree, under which the elements of
$\Ext^n_B(M,N)$ are assigned degree $n$; we do not do it here, in order
not to confuse \Mtwo. Of course, these modules can be computed from any
projective resolution of $M$ over $B$.

The next couple of remarks collect a few innocuous observations.  In
hindsight, they provide some of the basic tools for studying cohomology
of modules over complete intersections: see Remark \ref{history} for
some related material.

\begin{Remark}
\label{action}
The resolution $(C\otimes_A D',\partial)$ provided by Theorem
\ref{resolution} is a graded module over the graded algebra $S$, with
action defined by the formula
\[
u\cdot(x\o y')= x\o (u\lcontract y)'
\]
and this action commutes with the differential $\partial$.  The induced
action provides a structure of graded $S$-module on the complex
$\Hom_B(C\otimes_A D',N)$.

The action of $S$ commutes with the differential $\partial^*=
\Hom_B(\partial,N)$ of this complex, hence passes to its homology,
making it a graded a $S$-module.  Thus, each element $u\in
S_{-2k}=\SS^k(Q)$ determines homomorphisms
\[
\Ext^n_B(M,N)\xrightarrow{\ u\ }\Ext^{n+2k}_B(M,N)
\qquad\text{for all}\qquad n\in\Z
\]
For this reason, from now on we refer to the graded ring $S$ as the
{\it\ie{ring of cohomology operators}\/} determined by the
Koszul-regular set $\boldsymbol f$.
 \end{Remark}

\begin{Remark}
\label{canonical}
The canonical isomorphisms of complexes of $A$-modules
\begin{equation*}
\Hom_B(C \otimes_A D', N)=S \otimes_A \Hom_A(C, N)=
S \otimes_A \Hom_A(C, A) \otimes_A N
\end{equation*}
commute with the actions of $S$.
 \end{Remark}

The following fundamental result shows that in many important cases the
action of the cohomology operators is highly nontrivial.

\begin{theorem}
\label{finiteness}
Let $A$ be a commutative ring, let ${\boldsymbol f}$ be a Koszul
regular subset of $A$, and let $S$ be the graded ring of cohomology
operators defined by $\boldsymbol f$.

If $M$ and $N$ are finitely generated modules over $B=A/({\boldsymbol
f})$, and $M$ has finite projective dimension over $A$ (in particular,
if $A$ is regular), then the graded $S$-module $\Ext^\bu_B(M,N)$ is
finitely generated.
 \end{theorem}

\begin{proof}
Choose a resolution $r\colon C\to M$ with $C_n$ a finite projective
$A$-module for each $n$ and $C_n=0$ for all $n\gg0$.  By Remark
\eqref{canonical}, the graded $S$-module $\Hom_B(C \otimes_A D',N)$ is
finitely generated.  Since $S$ is noetherian, so is the submodule
$\Ker(\partial^*)$, and hence the homology module, $\Ext^\bu_B(M,N)$.
 \qed
  \end{proof}

\begin{Remark}
\label{history}
The resolution of Remark \ref{hypersurface} was constructed by Shamash
\cite[Sect.\ 3]{CI:Sh}, that of Theorem \ref{resolution} by Eisenbud
\cite[Sect.\ 7]{CI:Ei}.  The new aspect of our approach is indicated at
the beginning of Section \ref{Universal homotopies}.

As introduced in Remark \ref{action}, the $S$-module structure on Ext
may appear {\sl ad hoc\/}.  In fact, it is independent of all choices
of resolutions and maps, it can be computed from {\it any\/} projective
resolution of $M$ over $B$, it is natural in both module arguments
and---in an appropriate sense---in the ring argument, and it commutes
with Yoneda products from either side.  These properties were proved by
Gulliksen \cite[Sect.~2]{CI:MR51:487}, Mehta \cite[Ch.~2]{CI:Me},
Eisenbud \cite[Sect.\ 4]{CI:Ei}, and Avramov
\cite[Sect.\ 2]{CI:MR90g:13027}.  However, each author used a different
construction of cohomology operators, and comparison of the
different approaches has turned to be an unexpectedly delicate
problem.  It was finally resolved in \cite{CI:MR2000e:13021}, where
complete proofs of the main properties of the operators can be found.

Gulliksen \cite[Sect.~3]{CI:MR51:487} established a stronger form of
Theorem \ref{finiteness}, without finiteness hypotheses on the ring
$A$: If the $A$-module $\Ext^\bu_A(M,N)$ is noetherian, then the
$S$-module $\Ext^\bu_B(M,N)$ is noetherian; this can be obtained from
the complexes of Remark \eqref{canonical} by means of a spectral
sequence, cf.\ \cite[Sect.~6]{CI:MR1774757}.  The converse of
Gulliksen's theorem was proved in \cite[Sect.~4]{CI:MR99c:13033}.
 \end{Remark}

For the rest of the paper we place ourselves in a situation where \Mtwo
operates best---graded modules over positively graded rings.  This
grading is inherited by the various Ext modules, and we keep careful
track of it.  Our conventions and bookkeeping procedures are discussed
in detail in an Appendix, which the reader is invited to consult as
needed.

For ease of reference, we collect some notation.

\begin{notation}
\label{graded stuff}
The following is assumed for the rest of the paper.
\begin{itemize}
\item[$\bullet$]
$K$ is a field.
\item[$\bullet$]
$\{x_h\,\vert\,\deg'(x_h)>0\}_{h=1,\dots,e}$ is a set of
indeterminates over $K$.
\item[$\bullet$]
$A=K[x_1,\dots,x_e]$, graded by $\deg'(a)=0$ for $a\in K$.
\item[$\bullet$]
$f_1,\dots,f_c$ is a homogeneous $A$-regular sequence
in $(x_1,\dots,x_e)^2$.
%%
%% Lucho, why do we want the f's to be of degree at least 2?
%%
\item[$\bullet$]
$r_i=\deg'(f_i)$ for $i=1,\dots,c$.
\item[$\bullet$]
$\{X_i\,\vert\,\Deg X_i=(-2,-r_i)\}_{i=1,\dots,c}$ is a set
of indeterminates over $A$.
\item[$\bullet$]
$S=A[X_1,\dots, X_c]$, bigraded by $\Deg(a)=(0,\deg'(a))$.
\item[$\bullet$]
$B=A/({\boldsymbol f})$, with degree induced by $\deg'$.
\item[$\bullet$]
$M$ and $N$ are finitely generated graded $B$-modules.
\item[$\bullet$]
$S$ acts as bigraded ring of cohomology operators on
$\Ext^\bu_B(M,N)$.
\item[$\bullet$]
$k=B/(x_1,\dots,x_e)B$, with degree induced by $\deg'$.
\item[$\bullet$]
$R=S\otimes_A k\cong K[X_1,\dots,X_c]$, with bidegree induced by
$\Deg$.
\end{itemize}
 \end{notation}

\begin{Remark}
Under the conditions above, it is reasonable to ask when the $B$-free
resolution $G$ of Theorem \ref{resolution}, obtained from a {\it
minimal\/} $A$-free resolution $C$ of $M$, will itself be minimal.
Shamash \cite[Sect.~3]{CI:Sh} proves that $G$ is minimal if
$f_i\in(x_1,\dots,x_e)\ann_A(M)$ for $i=1,\dots,c$.  An obvious example
with non-minimal $G$ occurs when $M$ has finite projective dimension
over $B$: if $c>0$ then $G$ is infinite.  A more interesting failure of
minimality follows.
 \end{Remark}

\begin{Example}
Let $A$, $f$, $B$, and $M$ be as in Example \ref{cube}.
\beginOutput
i38 : M = B^1/m^3;\\
\endOutput
\beginOutput
i39 : F = resolution(M, LengthLimit=>8)\\
\emptyLine
\       1      10      16      15      15      15      15      15      15\\
o39 = B  <-- B   <-- B   <-- B   <-- B   <-- B   <-- B   <-- B   <-- B\\
\                                                                      \\
\      0      1       2       3       4       5       6       7       8\\
\emptyLine
o39 : ChainComplex\\
\endOutput
Thus, the sequence of Betti numbers $\b^B_n(M)$ is 
$(1,10,16,15,15,15,\dots)$.
\beginOutput
i40 : M' = restrict1 M;\\
\endOutput
\beginOutput
i41 : C = res M'\\
\emptyLine
\       1      10      15      6\\
o41 = A  <-- A   <-- A   <-- A  <-- 0\\
\                                     \\
\      0      1       2       3      4\\
\emptyLine
o41 : ChainComplex\\
\endOutput
By Remark \ref{hypersurface}, the sequence $\rank_B F_n$ is 
$(1,10,16,16,16,16,\dots)$.
 \end{Example}

In a graded context, all cohomological entities discussed so far in the
text acquire an extra grading, discussed in detail in the Appendix.
The notions below are used, but not named, in \cite{CI:AB2} in a local
situation.

\begin{Remark}
\label{reduced ext}
We define the {\it\ie{reduced Ext module}\/} for $M$ and $N$ over $B$ 
by
\[
\rExt^\bu_B(M,N)=\Ext^\bu_B(M,N)\otimes_Ak
\]
With the induced bigrading and action, it is a bigraded module over the
bigraded ring $R$, that we call the {\it\ie{reduced ring of cohomology
operators}\/}.

The dimension of the $K$-vector space $\rExt^n_B(M,N)_{s}$ is equal to
the number of generators of bidegree $(-n,s)$ in any minimal set of
generators of the graded $B$-module $\Ext^n_B(M,N)$.  We define the
graded (respectively, ungraded) {\it\ie{Ext-generator series}\/} of $M$
and $N$ to be the formal power series
\begin{gather*}
\gen_B^{M,N}(t,u)=
\sum_{n\in\N\,,\,s\in\Z}\dim_K \rExt^n_B(M,N)_{s}\, t^n u^{-s}
\in\Z[u,u^{-1}][[t]]\\
\gen_B^{M,N}(t)=
\sum_{n=0}^\infty\dim_K \rExt^n_B(M,N)\, t^n
\in\Z[[t]]
\end{gather*}
There is a simple relation between these series: $\gen_B^{M,N}(t)=
\gen_B^{M,N}(t,1)$.
 \end{Remark}

\begin{corollary}
\label{series}
In the notation above, $\rExt^\bu_B(M,N)$ is a finitely generated
bigraded $R$-module, and $\gen_B^{M,N}(t,u)$ represents a rational
function of the form
\begin{equation*}
\frac{g_B^{M,N}(t,u)}{(1-t^2u^{r_1})\cdots(1-t^2u^{r_c})}
\qquad\text{with}\qquad g_B^{M,N}(t,u)\in\Z[t,u,u^{-1}]
\end{equation*}
\end{corollary}

\begin{proof}
The assertion on finite generation results from Theorem
\ref{finiteness} and the one on bigradings from Remark \ref{graded
action}.  The form of the power series is then given by the
Hilbert-Serre Theorem. \qed
 \end{proof}

\section{Computation of Ext Modules}
\label{Computation of Ext modules}

This section contains the main new computational result of the paper.

We discuss, apply, and present an algorithm that computes, for graded
modules $M$ and $N$ over a graded complete intersection ring $B$, the
graded $B$-modules $\Ext^n_B(M,N)$ simultaneously in all degrees $n$,
along with all the cohomology operators defined in Remark
\ref{action}.

More precisely, the input consists of a field $K$, a polynomial ring
$A=K[x_1,\dots,x_e]$ with $\deg'(x_h)>0$, a sequence $f_1,\dots,f_c$ of
elements of $A$, and finitely generated modules $M$, $N$ over
$B=A/(f_1,\dots,f_c)$.  The program checks whether the sequence
consists of homogeneous elements, whether it is regular, and whether
the modules $M$, $N$ are graded, sending the appropriate error message
if any one of these conditions is violated.  If the input data pass
those tests, then the program produces a presentation of the bigraded
module $H=\Ext^\bu_B(M,N)$, where the elements of $\Ext^n_B(M,N)$ have
homological degree $-n$, over the polynomial ring $A[X_1,\dots,X_c]$,
bigraded by $\Deg(a)=(0,\deg'(a))$ and $\Deg(X_i)=(-2,-\deg'(f_i))$.

The algorithm is based on the proofs of Theorems \ref{resolution} and
\ref{finiteness}, and is presented in Code \ref{master} below.  We
start with an informal discussion.

\begin{Remark}
\label{algorithm}
The routine {\tt resolution}\indexcmd{resolution} of \Mtwo finds the matrices $d_{o,n}\colon
C_n\to C_{n-1}$ of the differential $d_C$ of a minimal free resolution
$C$ of $M$ over $A$.  Matrices $d_{\g,n}\colon C_n\to C_{n+2|\g|-1}$
satisfying equation \eqref{family} for $\g\in\N^c$ with $|\g|>0$ are
computed using the routine {\tt nullhomotopy} of the \Mtwo language in
a loop that follows the first part of the proof of Theorem
\ref{resolution}.  

The transposed matrix $d^*_{\g,n}$ yields an endomorphism of the free
bigraded $A$-module $C^*=\bigoplus_{n=0}^e \Hom_A(C_n,A)$ of rank $m$,
where $m=\sum_{n=0}^e\rank_A C_n$.  The $m\times m$ matrix ${\widetilde
d}^*_{\g,n}$ describing this endomorphism is formed using the routines
{\tt transpose} and {\tt sum}.  The $m\times m$ matrix
\[
\Delta = \sum_{n=0}^e(-1)^{n+2|\g|-1}
\sum_{\substack{\g\in\N^c\\ |\g|\le(e-n+1)/2}}
X^{\g}\cdot\widetilde d_{\g,n}^*
\]
with entries in $S=A[X_1,\dots,X_c]$ defines an endomorphism of the
free bigraded $S$-module $S\otimes_A C^*$.  It induces an endomorphism
$\overline\Delta$ of the bigraded $S$-module $S\otimes_A C^*\otimes_A
N$.  The bigraded $S$-module $H=\Ext^\bu_B(M,N)$ is computed as
$H=\Ker(\overline\Delta)/ \Ima(\overline\Delta)$ using the routine {\tt
homology}\indexcmd{homology}.
 \end{Remark}

In the computations we let $H$ denote the bigraded $S$-module
$\Ext^\bu_B(M,N)$.  As the graded ring $S$ is zero in odd homological
degrees, there is a canonical direct sum decomposition $H =
H^{\text{even}} \oplus H^{\text{odd}}$ of bigraded $S$-modules, where
`even' or `odd' refers to the parity of the {\it first\/} degree in
each pair $\Deg(x)$.

We begin with an example in codimension 1, where it is possible to
construct the infinite resolution and the action of $S$ on it by
hand.

\begin{Example}
Consider the ring $A = K[x]$ where the variable $x$ is assigned degree
$5$, and set $B = A/(x^3)$.  The bigraded ring of cohomology operators
then is $S=A[X,x]$, where $\Deg(X)=(-2,-15)$ and $\Deg x=(0,5)$.

For the $B$-modules $M = B/(x^2)$ and $N = B/(x)$, the bigraded
$S$-module $H=\Ext^\bu_B(M,N)$ is described by the isomorphism
\[
   H\cong (S/(x))\oplus (S/(x))[1,10]
\]

A minimal free resolution of $M$ over $B$ is displayed below.
\[
F = \quad \dots \longrightarrow
 B[-30] \xrightarrow{-x\,} B[-25] \xrightarrow{\ x^2\,} 
 B[-15] \xrightarrow{-x\,} B[-10] \xrightarrow{\ x^2\,} 
 B \longrightarrow 0
\] 
This resolution is actually isomorphic to the resolution $C\otimes_A
D'$ described in Remark \ref{hypersurface}, formed from the free
resolution
\[C
= \quad 0 \longrightarrow A[-10] \xrightarrow{\ x^2\,} A
\longrightarrow 0
\]
of $M$ over $A$ and the nullhomotopy $d_{\e_1}$ displayed in the diagram
\[
\xymatrix{
0 \ar[r] & A[-10] \ar[r]^-{x^2} \ar[d]_{-x^3} & A \ar[r] \ar[d]^{-x^3}
        \ar[dl]_-{-x} & 0 \\
0 \ar[r] & A[-10] \ar[r]^-{x^2}              & A \ar[r]                              & 0
}
\]
The isomorphism of $F$ with $C \otimes_A D'$ endows $F$ with a
structure of bigraded module over $S$, where the action of $X$ on
$F$ is the chain map $F \to F$ of homological degree $-2$ and internal
degree $-15$ that corresponds to the identity map of $B$ in each
component.

The bigraded $S$-module $H=\Ext^\bu_B(M,N)$ is the homology of the
complex
\[
\Hom_R(F,N) = \quad 0 \xrightarrow{\ \ }
N \xrightarrow{\ 0\ } N[10] \xrightarrow{\ 0\,}
N[15] \xrightarrow{\ 0\ } N[25] \xrightarrow{\ 0\,} 
N[30] \longrightarrow\cdots
\]
where multiplication by $x\in S$ is the zero map, and for each $i\ge0$
multiplication by $X\in S$ sends $N[15i]$ to $N[15i+15]$
(respectively, $N[10i+10]$ to $N[10i+25]$) by the identity map.  This
description provides the desired isomorphisms of bigraded $S$-modules.

Here is how to compute $H$ with \Mtwo.
 
Create the rings and modules.
\beginOutput
i42 : K = ZZ/103; \\
\endOutput
\beginOutput
i43 : A = K[x,Degrees=>\{5\}];\\
\endOutput
\beginOutput
i44 : B = A/(x^3);\\
\endOutput
\beginOutput
i45 : M = B^1/(x^2);\\
\endOutput
\beginOutput
i46 : N = B^1/(x);\\
\endOutput
Use the function {\tt Ext} to compute $H = \Ext^\bu_B(M,N)$ (the
semicolon at the end of the line will suppress printing until we have
assigned the name {\tt S} to the ring of cohomology operators
constructed by \Mtwo.)
\beginOutput
i47 : H = Ext(M,N);\\
\endOutput
We may look at the ring.
\beginOutput
i48 : ring H\\
\emptyLine
o48 = K [\$X , x, Degrees => \{\{-2, -15\}, \{0, 5\}\}]\\
\           1\\
\emptyLine
o48 : PolynomialRing\\
\endOutput
\Mtwo has assigned the name {\tt \$X\char`\_1} to the variable $X$.
The dollar sign indicates an internal name that cannot be entered from
the keyboard: if necessary, obtain the variable by entering {\tt
S\char`\_0}; notice that indexing in \Mtwo starts with $0$ rather than $1$.
Notice also the appearance of braces rather than
parentheses in \Mtwo's notation for bidegrees. 
\beginOutput
i49 : degree {\char`\\} gens ring H\\
\emptyLine
o49 = \{\{-2, -15\}, \{0, 5\}\}\\
\emptyLine
o49 : List\\
\endOutput
Assign the ring a name.
\beginOutput
i50 : S = ring H;\\
\endOutput
We can now look at the $S$-module $H$.
\beginOutput
i51 : H\\
\emptyLine
o51 = cokernel \{0, 0\}    | 0 x |\\
\               \{-1, -10\} | x 0 |\\
\emptyLine
\                             2\\
o51 : S-module, quotient of S\\
\endOutput
Each row in the display above is labeled with the bidegree of the
corresponding generator of $H$.  This presentation gives the 
isomorphisms of bigraded $S$-modules, already computed by hand
earlier.
\end{Example}

Let's try an example with a complete intersection of codimension 2.  It
is not so easy to do by hand, but can be checked using the theory in
\cite{CI:MR1774757}.

\begin{Example}
Begin by constructing a polynomial ring $A=K[x,y]$.
\beginOutput
i52 : A = K[x,y];\\
\endOutput
Now we produce a complete intersection quotient ring $B=A/(x^3,y^2)$.
\beginOutput
i53 : J = ideal(x^3,y^2);\\
\emptyLine
o53 : Ideal of A\\
\endOutput
\beginOutput
i54 : B = A/J;\\
\endOutput
We take $N$ to be the $B$-module $B/(x^2,xy)$.
\beginOutput
i55 : N = cokernel matrix\{\{x^2,x*y\}\}\\
\emptyLine
o55 = cokernel | x2 xy |\\
\emptyLine
\                             1\\
o55 : B-module, quotient of B\\
\endOutput
Remark \ref{bigrading} shows that $H=\Ext^\bu_B(N,N)$ is a bigraded module
over the bigraded ring $S=A[X_1,X_2]=K[X_1,X_2,x,y]$ where 
\begin{alignat*}{2}
\Deg(X_1)&=(-2,-3)\quad &\Deg(X_2)&=(-2,-2)\\
\Deg(x)&=(0,1)        &\Deg(y)&=(0,1)
\end{alignat*}
Using \Mtwo (below) we obtain an isomorphism of bigraded $S$-modules
\begin{gather*}
H^{\text{even}}\cong \frac{S}{(x^2,xy,y^2,xX_1,yX_1)}
        \oplus \frac{S}{(x,y)}[2,2]\\
H^{\text{odd}} \cong \left(\frac{S}{(x,y,X_1)}
\oplus \frac{S}{(x,y)}\right)^2[1,1]
\end{gather*}
These isomorphisms also yield expressions for the graded $B$-modules:
\begin{gather*}
\Ext_B^{2i}(N,N)\cong
N\cdot X_2^{i}\oplus
\bigoplus_{h=1}^{i} k\cdot X_1^{h}X_2^{i-h}\oplus
\bigoplus_{h=0}^{i-1}k[2]\cdot X_1^{h}X_2^{i-1-h}\\
\Ext_B^{2i+1}(N,N)\cong
\bigg(k[1]\cdot X_2^{i}\oplus
\bigoplus_{h=0}^{i}k[1]\cdot X_1^{i-h}X_2^{h}\bigg)^2
\end{gather*}

Now we follow in detail the computation of the bigraded $S$-module $H$.
\beginOutput
i56 : time H = Ext(N,N);\\
\     -- used 0.2 seconds\\
\endOutput
\beginOutput
i57 : ring H\\
\emptyLine
o57 = K [\$X , \$X , x, y, Degrees => \{\{-2, -2\}, \{-2, -3\}, \{0, 1\}, \{0, 1\}\}]\\
\           1    2\\
\emptyLine
o57 : PolynomialRing\\
\endOutput
\beginOutput
i58 : S = ring H;\\
\endOutput
One might wish to have a better view of the bidegrees of the variables
of the ring $S$.  An easy way to achieve this, with signs reversed, is
to display the transpose of the matrix of variables.
\beginOutput
i59 : transpose vars S\\
\emptyLine
o59 = \{2, 2\}  | \$X_1 |\\
\      \{2, 3\}  | \$X_2 |\\
\      \{0, -1\} | x    |\\
\      \{0, -1\} | y    |\\
\emptyLine
\              4       1\\
o59 : Matrix S  <--- S\\
\endOutput
The internal degrees displayed for the cohomology operators may come
as a surprise.  To understand what is going on, recall that these
degrees are determined by a choice of minimal generators for $J$.
At this point we do not know what is the sequence of generators that
\Mtwo used, so let's compute those generators the way the program did.
\beginOutput
i60 : trim J\\
\emptyLine
\              2   3\\
o60 = ideal (y , x )\\
\emptyLine
o60 : Ideal of A\\
\endOutput
Notice that \Mtwo has reordered the original sequence of generators.
Now we see that our variable $X_1$, which corresponds to $x^3$, is
denoted {\tt X\char`\_2} by \Mtwo, and that $X_2$, which corresponds
to $y^2$ is denoted {\tt X\char`\_1}.  This explains the bidegrees used
by the program.

Display $H$.
\beginOutput
i61 : H\\
\emptyLine
o61 = cokernel \{-2, -2\} | 0 0 0 0 0 0 0 0 0  0  0  y x 0    0    0     $\cdot\cdot\cdot$\\
\               \{-1, -1\} | y 0 0 0 0 x 0 0 0  0  0  0 0 \$X_1 0    0     $\cdot\cdot\cdot$\\
\               \{-1, -1\} | 0 0 0 y 0 0 0 x 0  0  0  0 0 0    \$X_1 0     $\cdot\cdot\cdot$\\
\               \{-1, -1\} | 0 y 0 0 x 0 0 0 0  0  0  0 0 0    0    0     $\cdot\cdot\cdot$\\
\               \{-1, -1\} | 0 0 y 0 0 0 x 0 0  0  0  0 0 0    0    0     $\cdot\cdot\cdot$\\
\               \{0, 0\}   | 0 0 0 0 0 0 0 0 y2 xy x2 0 0 0    0    \$X_1y $\cdot\cdot\cdot$\\
\emptyLine
\                             6\\
o61 : S-module, quotient of S\\
\endOutput
That's a bit large, so we want to look at the even and odd parts separately.

We may compute the even and odd parts of $H$ as the span of the
generators of $H$ with the appropriate parity.  Since the two desired
functions differ only in the predicate to be applied, we can generate
them both by writing a function that accepts the predicate as its
argument and returns a function.
\beginOutput
i62 : partSelector = predicate -> H -> (\\
\         R := ring H;\\
\         H' := prune image matrix \{\\
\             select(\\
\                 apply(numgens H, i -> H_\{i\}),\\
\                 f -> predicate first first degrees source f\\
\                 )\\
\             \};\\
\         H');\\
\endOutput
\beginOutput
i63 : evenPart = partSelector even; oddPart = partSelector odd;\\
\endOutput
Now to obtain the even part, $H^{\text{even}}$, simply type
\beginOutput
i65 : evenPart H\\
\emptyLine
o65 = cokernel \{-2, -2\} | 0  0  0  y x 0     0     |\\
\               \{0, 0\}   | y2 xy x2 0 0 \$X_1y \$X_1x |\\
\emptyLine
\                             2\\
o65 : S-module, quotient of S\\
\endOutput
Do the same thing for the odd part, $H^{\text{odd}}$.
\beginOutput
i66 : oddPart H\\
\emptyLine
o66 = cokernel \{-1, -1\} | 0 0 y 0 0 0 x 0 0    0    |\\
\               \{-1, -1\} | 0 y 0 0 x 0 0 0 0    0    |\\
\               \{-1, -1\} | 0 0 0 y 0 0 0 x 0    \$X_1 |\\
\               \{-1, -1\} | y 0 0 0 0 x 0 0 \$X_1 0    |\\
\emptyLine
\                             4\\
o66 : S-module, quotient of S\\
\endOutput
These presentations yield the desired isomorphism of bigraded $S$-modules.
 \end{Example}

Here is the source code which implements the routine {\tt Ext}.  It is
incorporated into \Mtwo.

\begin{code}
\label{master}
The function {\tt Ext(M,N)} computes $\Ext_B^\bu(M,N)$ for graded modules
$M$, $N$ over a graded complete intersection ring $B$.  The function {\tt
  code}\indexcmd{code} can be used to obtain a copy of the source code.\indexcmd{Ext}
\beginOutput
i67 : print code(Ext,Module,Module)\\
-- ../../../m2/ext.m2:82-171\\
Ext(Module,Module) := Module => (M,N) -> (\\
\  cacheModule := youngest(M,N);\\
\  cacheKey := (Ext,M,N);\\
\  if cacheModule#?cacheKey then return cacheModule#cacheKey;\\
\  B := ring M;\\
\  if B =!= ring N\\
\  then error "expected modules over the same ring";\\
\  if not isCommutative B\\
\  then error "'Ext' not implemented yet for noncommutative rings.";\\
\  if not isHomogeneous B\\
\  then error "'Ext' received modules over an inhomogeneous ring";\\
\  if not isHomogeneous N or not isHomogeneous M\\
\  then error "'Ext' received an inhomogeneous module";\\
\  if N == 0 then B^0\\
\  else if M == 0 then B^0\\
\  else (\\
\    p := presentation B;\\
\    A := ring p;\\
\    I := ideal mingens ideal p;\\
\    n := numgens A;\\
\    c := numgens I;\\
\    if c =!= codim B \\
\    then error "total Ext available only for complete intersections";\\
\    f := apply(c, i -> I_i);\\
\    pM := lift(presentation M,A);\\
\    pN := lift(presentation N,A);\\
\    M' := cokernel ( pM | p ** id_(target pM) );\\
\    N' := cokernel ( pN | p ** id_(target pN) );\\
\    C := complete resolution M';\\
\    X := local X;\\
\    K := coefficientRing A;\\
\    -- compute the fudge factor for the adjustment of bidegrees\\
\    fudge := if #f > 0 then 1 + max(first {\char`\\} degree {\char`\\} f) // 2 else 0;\\
\    S := K(monoid [X_1 .. X_c, toSequence A.generatorSymbols,\\
\      Degrees => \{\\
\        apply(0 .. c-1, i -> \{-2, - first degree f_i\}),\\
\        apply(0 .. n-1, j -> \{ 0,   first degree A_j\})\\
\        \},\\
\      Adjust => v -> \{- fudge * v#0 + v#1, - v#0\},\\
\      Repair => w -> \{- w#1, - fudge * w#1 + w#0\}\\
\      ]);\\
\    -- make a monoid whose monomials can be used as indices\\
\    Rmon := monoid [X_1 .. X_c,Degrees=>\{c:\{2\}\}];\\
\    -- make group ring, so 'basis' can enumerate the monomials\\
\    R := K Rmon;\\
\    -- make a hash table to store the blocks of the matrix\\
\    blks := new MutableHashTable;\\
\    blks#(exponents 1_Rmon) = C.dd;\\
\    scan(0 .. c-1, i -> \\
\         blks#(exponents Rmon_i) = nullhomotopy (- f_i*id_C));\\
\    -- a helper function to list the factorizations of a monomial\\
\    factorizations := (gamma) -> (\\
\      -- Input: gamma is the list of exponents for a monomial\\
\      -- Return a list of pairs of lists of exponents showing the\\
\      -- possible factorizations of gamma.\\
\      if gamma === \{\} then \{ (\{\}, \{\}) \}\\
\      else (\\
\        i := gamma#-1;\\
\        splice apply(factorizations drop(gamma,-1), \\
\          (alpha,beta) -> apply (0..i, \\
\               j -> (append(alpha,j), append(beta,i-j))))));\\
\    scan(4 .. length C + 1, \\
\      d -> if even d then (\\
\        scan( exponents {\char`\\} leadMonomial {\char`\\} first entries basis(d,R), \\
\          gamma -> (\\
\            s := - sum(factorizations gamma,\\
\              (alpha,beta) -> (\\
\                if blks#?alpha and blks#?beta\\
\                then blks#alpha * blks#beta\\
\                else 0));\\
\            -- compute and save the nonzero nullhomotopies\\
\            if s != 0 then blks#gamma = nullhomotopy s;\\
\            ))));\\
\    -- make a free module whose basis elements have the right degrees\\
\    spots := C -> sort select(keys C, i -> class i === ZZ);\\
\    Cstar := S^(apply(spots C,\\
\        i -> toSequence apply(degrees C_i, d -> \{i,first d\})));\\
\    -- assemble the matrix from its blocks.\\
\    -- We omit the sign (-1)^(n+1) which would ordinarily be used,\\
\    -- which does not affect the homology.\\
\    toS := map(S,A,apply(toList(c .. c+n-1), i -> S_i),\\
\      DegreeMap => prepend_0);\\
\    Delta := map(Cstar, Cstar, \\
\      transpose sum(keys blks, m -> S_m * toS sum blks#m),\\
\      Degree => \{-1,0\});\\
\    DeltaBar := Delta ** (toS ** N');\\
\    assert isHomogeneous DeltaBar;\\
\    assert(DeltaBar * DeltaBar == 0);\\
\    -- now compute the total Ext as a single homology module\\
\    cacheModule#cacheKey = prune homology(DeltaBar,DeltaBar)))\\
\endOutput
\end{code}

\begin{Remark}
The bigraded module $\Tor_\bu^B(M,N)$ is the homology of the complex
$(C \otimes_A D') \otimes_B N$, where $C\otimes_A D'$ is the complex
from Theorem \ref{resolution}.  Observations parallel to Remarks
\ref{action} and \ref{bigrading} show that $\Tor_\bu^B(M,N)$ inherits
from $D'$ a structure of bigraded $S$-module.

It would be desirable also to have algorithms to compute
$\Tor_\bu^B(M,N)$ in the spirit of the algorithm presented above for
$\Ext^\bu_B(M,N)$.  If one of the modules has finite length, then
each $\Tor_n^B(M,N)$ is a $B$-module of finite length, and the
computation of $\Tor_\bu^B(M,N)$ can be reduced to a computation of
$\Ext$ by means of Matlis duality, which here can be realized as vector
space duality over the field $K$.  However, in homology there is no
equivalent for the finiteness property described in Remark
\ref{canonical}; it is an {\bf open problem} to devise algorithms that
would compute $\Tor_\bu^B(M,N)$ in general.
 \end{Remark}

\section{Invariants of Modules}
\label{Invariants of modules}

In this section we apply our techniques to develop effective methods
for computation (for graded modules over a graded complete
intersection) of invariants such as cohomology modules, Poincar\'e
series, Bass series, complexity, critical degree, and support
varieties.  For each invariant we produce code that computes it, and
illustrate the action of the code on some explicit example.

Whenever appropriate, we describe {\bf open problems} on which the
computational power of \Mtwo could be unleashed.

Notation \ref{graded stuff} is used consistently throughout the section.

\subsection{Cohomology Modules}
\label{Computation of cohomology}

We call the bigraded $R$-module $P=\Ext^\bu_B(M,k)$ the
{\it\ie{contravariant cohomology module}\/} of $M$ over $B$, and the
bigraded $R$-module $I=\Ext^\bu_B(k,M)$ the {\it\ie{covariant
cohomology module}\/} of $M$.   Codes that display presentations of the
cohomology modules are presented after a detailed discussion of an example.

\begin{sExample} 
\label{random}
Let us create the ring $B=K[x,y,z]/(x^3,y^4,z^5)$.
\beginOutput
i68 : A = K[x,y,z];\\
\endOutput
\beginOutput
i69 : J = trim ideal(x^3,y^4,z^5)\\
\emptyLine
\              3   4   5\\
o69 = ideal (x , y , z )\\
\emptyLine
o69 : Ideal of A\\
\endOutput
\beginOutput
i70 : B = A/J;\\
\endOutput
We trimmed the ideal, so that we know the generators \Mtwo will use.

This time we want a graded $B$-module $M$ about whose homology we know
nothing {\sl a priori\/}.   One way to proceed is to create $M$ as the
cokernel of some random matrix of forms; let's try a 3 by 2 matrix of
quadratic forms.
\beginOutput
i71 : f = random (B^3, B^\{-2,-3\})\\
\emptyLine
o71 = | 27x2+49xy-14y2-23xz-6yz-19z2 38x2y-34xy2+4y3+x2z+16xyz-y2z-5xz $\cdot\cdot\cdot$\\
\      | -5x2+44xy+38y2+40xz+15yz+4z2 -37x2y+51xy2-36y3+26x2z-38xyz-17y $\cdot\cdot\cdot$\\
\      | 21x2-30xy+32y2-47xz+7yz-50z2 -6x2y-14xy2-26y3-7x2z+41xyz+50y2z $\cdot\cdot\cdot$\\
\emptyLine
\              3       2\\
o71 : Matrix B  <--- B\\
\endOutput
We can't read the second column of that matrix, so let's display it
separately.
\beginOutput
i72 : f_\{1\}\\
\emptyLine
o72 = | 38x2y-34xy2+4y3+x2z+16xyz-y2z-5xz2-6yz2+47z3        |\\
\      | -37x2y+51xy2-36y3+26x2z-38xyz-17y2z+17xz2-11yz2+8z3 |\\
\      | -6x2y-14xy2-26y3-7x2z+41xyz+50y2z+26xz2+46yz2-44z3  |\\
\emptyLine
\              3       1\\
o72 : Matrix B  <--- B\\
\endOutput
Now let's make the module $M$.
\beginOutput
i73 : M = cokernel f;\\
\endOutput

We are going to produce isomorphisms of bigraded modules 
\begin{gather*}
P^{\text{even}} \cong R[4,10] \oplus (X_1,X_2)[2,7]
        \oplus\left(\frac{R}{(X_1,X_2,X_3)}\right)^3\oplus R^4[2,7]\\
P^{\text{odd}} \cong \frac{R}{(X_1,X_2,X_3)}[1,2]
        \oplus\left(\frac{R}{(X_1)}\right)^3[3,9]
        \oplus \frac{R}{(X_1,X_2)}[1,3] \oplus R^6[3,9]
\end{gather*}
over the polynomial ring $R=K[X_1,X_2,X_3]$ over $K$, bigraded by
\[
\Deg(X_1)=(-2,-3)\qquad \Deg(X_2)=(-2,-4) \qquad \Deg(X_3)=(-2,-5)
\]

Let's compute $\Ext^\bu_B(M,B/(x,y,z))$ by the routine from Section 
\ref{Computation of Ext modules}.
\beginOutput
i74 : time P = Ext(M,B^1/(x,y,z));\\
\     -- used 1.64 seconds\\
\endOutput
\beginOutput
i75 : S = ring P;\\
\endOutput
Examine the variables of $S$; due to transposing, their bidegrees
are displayed with the {\it opposite\/} signs.
\beginOutput
i76 : transpose vars S\\
\emptyLine
o76 = \{2, 3\}  | \$X_1 |\\
\      \{2, 4\}  | \$X_2 |\\
\      \{2, 5\}  | \$X_3 |\\
\      \{0, -1\} | x    |\\
\      \{0, -1\} | y    |\\
\      \{0, -1\} | z    |\\
\emptyLine
\              6       1\\
o76 : Matrix S  <--- S\\
\endOutput
The variables $x$, $y$, and $z$ of $A$ annihilate $P$, and so appear in
many places in a presentation of $P$.  To reduce the size of such a
presentation, we pass to a ring which eliminates those variables.
\beginOutput
i77 : R = K[X_1..X_3,Degrees => \{\{-2,-3\},\{-2,-4\},\{-2,-5\}\},\\
\              Adjust => S.Adjust, Repair => S.Repair];\\
\endOutput
\beginOutput
i78 : phi = map(R,S,\{X_1,X_2,X_3,0,0,0\})\\
\emptyLine
o78 = map(R,S,\{X , X , X , 0, 0, 0\})\\
\                1   2   3\\
\emptyLine
o78 : RingMap R <--- S\\
\endOutput
\beginOutput
i79 : P = prune (phi ** P);\\
\endOutput
\beginOutput
i80 : transpose vars ring P\\
\emptyLine
o80 = \{2, 3\} | X_1 |\\
\      \{2, 4\} | X_2 |\\
\      \{2, 5\} | X_3 |\\
\emptyLine
\              3       1\\
o80 : Matrix R  <--- R\\
\endOutput
As we planned, the original variables $x$, $y$, $z$, which act
trivially on the cohomology, are no longer present in the ring. 
Next we compute presentations
\beginOutput
i81 : evenPart P\\
\emptyLine
o81 = cokernel \{-4, -10\} | 0   0   0   0   0   0   0   0   0   0    |\\
\               \{-4, -10\} | 0   0   0   0   0   0   0   0   0   -X_2 |\\
\               \{-4, -11\} | 0   0   0   0   0   0   0   0   0   X_1  |\\
\               \{0, 0\}    | 0   0   X_3 0   0   X_2 0   X_1 0   0    |\\
\               \{0, 0\}    | 0   X_3 0   0   X_2 0   X_1 0   0   0    |\\
\               \{0, 0\}    | X_3 0   0   X_2 0   0   0   0   X_1 0    |\\
\               \{-2, -7\}  | 0   0   0   0   0   0   0   0   0   0    |\\
\               \{-2, -7\}  | 0   0   0   0   0   0   0   0   0   0    |\\
\               \{-2, -7\}  | 0   0   0   0   0   0   0   0   0   0    |\\
\               \{-2, -7\}  | 0   0   0   0   0   0   0   0   0   0    |\\
\emptyLine
\                             10\\
o81 : R-module, quotient of R\\
\endOutput
\beginOutput
i82 : oddPart P\\
\emptyLine
o82 = cokernel \{-1, -2\} | X_3 0   X_2 0   0   0   0   X_1 |\\
\               \{-3, -9\} | 0   0   0   0   0   0   X_1 0   |\\
\               \{-3, -9\} | 0   0   0   0   0   X_1 0   0   |\\
\               \{-3, -9\} | 0   0   0   0   X_1 0   0   0   |\\
\               \{-1, -3\} | 0   X_2 0   X_1 0   0   0   0   |\\
\               \{-3, -9\} | 0   0   0   0   0   0   0   0   |\\
\               \{-3, -9\} | 0   0   0   0   0   0   0   0   |\\
\               \{-3, -9\} | 0   0   0   0   0   0   0   0   |\\
\               \{-3, -9\} | 0   0   0   0   0   0   0   0   |\\
\               \{-3, -9\} | 0   0   0   0   0   0   0   0   |\\
\               \{-3, -9\} | 0   0   0   0   0   0   0   0   |\\
\emptyLine
\                             11\\
o82 : R-module, quotient of R\\
\endOutput
These presentations yield the desired isomorphisms of bigraded
$R$-modules.
 \end{sExample}

The procedure above can be automated by installing a method that will
be run when {\tt Ext} is presented with a module $M$ and the residue
field $k$.  It displays a presentation of $\Ext_B^\bu(M,k)$ as a bigraded
$R$-module.

\begin{sCode}
\label{change}
The function {\tt changeRing H} takes an $S$-module $H$ and tensors it
with $R$.  It does this by constructing $R$ and a ring homomorphism
\[
\varphi\colon A[X_1,\dots,X_c] = S \to R = K[X_1,\dots,X_c]
\]
\beginOutput
i83 : changeRing = H -> (\\
\         S := ring H;\\
\         K := coefficientRing S;\\
\         degs := select(degrees source vars S,\\
\              d -> 0 != first d);\\
\         R := K[X_1 .. X_#degs, Degrees => degs,\\
\              Repair => S.Repair, Adjust => S.Adjust];\\
\         phi := map(R,S,join(gens R,(numgens S - numgens R):0));\\
\         prune (phi ** H)\\
\         );\\
\endOutput
\end{sCode}

\begin{sCode}
\label{cohomology}
The function {\tt Ext(M,k)} computes $\Ext_B^\bu(M,k)$ when $B$ is a
graded complete intersection, $M$ a graded $B$-module, and $k$ is the
residue field of $B$.  The result is presented as a module over the
ring $k[X_1,\dots,X_c]$.
\beginOutput
i84 : Ext(Module,Ring) := (M,k) -> (\\
\         B := ring M;\\
\         if ideal k != ideal vars B\\
\         then error "expected the residue field of the module";\\
\         changeRing Ext(M,coker vars B)\\
\         );\\
\endOutput
\end{sCode}

\begin{sExample}
For a test, we run again the computation for $P^{\text{odd}}$.
\beginOutput
i85 : use B;\\
\endOutput
\beginOutput
i86 : k = B/(x,y,z);\\
\endOutput
\beginOutput
i87 : use B;\\
\endOutput
\beginOutput
i88 : P = Ext(M,k);\\
\endOutput
\beginOutput
i89 : time oddPart P\\
\     -- used 0.09 seconds\\
\emptyLine
o89 = cokernel \{-1, -2\} | X_3 0   X_2 0   0   0   0   X_1 |\\
\               \{-3, -9\} | 0   0   0   0   0   0   X_1 0   |\\
\               \{-3, -9\} | 0   0   0   0   0   X_1 0   0   |\\
\               \{-3, -9\} | 0   0   0   0   X_1 0   0   0   |\\
\               \{-1, -3\} | 0   X_2 0   X_1 0   0   0   0   |\\
\               \{-3, -9\} | 0   0   0   0   0   0   0   0   |\\
\               \{-3, -9\} | 0   0   0   0   0   0   0   0   |\\
\               \{-3, -9\} | 0   0   0   0   0   0   0   0   |\\
\               \{-3, -9\} | 0   0   0   0   0   0   0   0   |\\
\               \{-3, -9\} | 0   0   0   0   0   0   0   0   |\\
\               \{-3, -9\} | 0   0   0   0   0   0   0   0   |\\
\emptyLine
\                                                                       $\cdot\cdot\cdot$\\
o89 : K [X , X , X , Degrees => \{\{-2, -3\}, \{-2, -4\}, \{-2, -5\}\}]-module $\cdot\cdot\cdot$\\
\          1   2   3                                                    $\cdot\cdot\cdot$\\
\endOutput
\end{sExample}

We also introduce code for computing the covariant cohomology modules.

\begin{sCode}
\label{covariant-cohomology}
The function {\tt Ext(k,M)} computes $\Ext_B^\bu(k,M)$ when $B$ is a
graded complete intersection, $M$ a graded $B$-module, and $k$ is the
residue field of $B$.  The result is presented as a module over the
ring $k[X_1,\dots,X_c]$.
\beginOutput
i90 : Ext(Ring,Module) := (k,M) -> (\\
\         B := ring M;\\
\         if ideal k != ideal vars B\\
\         then error "expected the residue field of the module";\\
\         changeRing Ext(coker vars B,M)\\
\         );\\
\endOutput
\end{sCode}

Let's see the last code in action.

\begin{sExample}
For $B$ and $M$ from Example \ref{random} we compute the odd part of
the covariant cohomology module $\Ext^\bu_B(k,M)$.
\beginOutput
i91 : time I = Ext(k,M);\\
\     -- used 14.81 seconds\\
\endOutput
\beginOutput
i92 : evenPart I\\
\emptyLine
o92 = cokernel \{0, 6\} | 37X_2  37X_1  |\\
\               \{0, 6\} | -18X_2 -18X_1 |\\
\               \{0, 6\} | -13X_2 -13X_1 |\\
\               \{0, 6\} | -37X_2 -37X_1 |\\
\               \{0, 6\} | 22X_2  22X_1  |\\
\               \{0, 6\} | 0      0      |\\
\               \{0, 6\} | X_2    X_1    |\\
\emptyLine
\                                                                       $\cdot\cdot\cdot$\\
o92 : K [X , X , X , Degrees => \{\{-2, -3\}, \{-2, -4\}, \{-2, -5\}\}]-module $\cdot\cdot\cdot$\\
\          1   2   3                                                    $\cdot\cdot\cdot$\\
\endOutput
\beginOutput
i93 : oddPart I\\
\emptyLine
o93 = cokernel \{-1, 5\} | -48X_3 13X_3  34X_3  3X_3   0     0      0    $\cdot\cdot\cdot$\\
\               \{-1, 5\} | 3X_3   -40X_3 8X_3   8X_3   0     0      0    $\cdot\cdot\cdot$\\
\               \{-1, 5\} | -X_3   37X_3  -13X_3 -35X_3 0     0      0    $\cdot\cdot\cdot$\\
\               \{-1, 4\} | 4X_2   20X_2  3X_2   -47X_2 4X_1  20X_1  3X_1 $\cdot\cdot\cdot$\\
\               \{-1, 4\} | 0      51X_2  0      -30X_2 0     51X_1  0    $\cdot\cdot\cdot$\\
\               \{-1, 4\} | 0      12X_2  0      -3X_2  0     12X_1  0    $\cdot\cdot\cdot$\\
\               \{-1, 4\} | 42X_2  12X_2  46X_2  25X_2  42X_1 12X_1  46X_ $\cdot\cdot\cdot$\\
\               \{-1, 4\} | 45X_2  24X_2  -14X_2 -35X_2 45X_1 24X_1  -14X $\cdot\cdot\cdot$\\
\               \{-1, 4\} | 0      0      X_2    0      0     0      X_1  $\cdot\cdot\cdot$\\
\               \{-1, 4\} | X_2    0      0      0      X_1   0      0    $\cdot\cdot\cdot$\\
\               \{-1, 4\} | 0      -40X_2 0      10X_2  0     -40X_1 0    $\cdot\cdot\cdot$\\
\               \{-1, 4\} | 0      X_2    0      0      0     X_1    0    $\cdot\cdot\cdot$\\
\               \{-1, 3\} | 0      0      0      X_1    0     0      0    $\cdot\cdot\cdot$\\
\emptyLine
\                                                                       $\cdot\cdot\cdot$\\
o93 : K [X , X , X , Degrees => \{\{-2, -3\}, \{-2, -4\}, \{-2, -5\}\}]-module $\cdot\cdot\cdot$\\
\          1   2   3                                                    $\cdot\cdot\cdot$\\
\endOutput
\end{sExample}

\subsection{Poincar\'e Series}
\label{Poincare series}

The {\it\ie{graded Betti number}\/} of $M$ over $B$ is the number
$\b_{ns}^B(M)$ of direct summands isomorphic to the free module
$B[-s]$ in the $n$'th module of a minimal free resolution of $M$ over
$B$.  It can be computed from the equality
\[
\b_{ns}^B(M)=\dim_K\Ext^n_B(M,k)_{s}
\]
The {\it\ie{graded Poincar\'e series}\/} of $M$ over $B$ is the
generating function
\[
\Poi^B_M(t,u)=\sum_{n\in\N\,,\,s\in\Z}\b_{ns}^B(M)\, t^n u^{-s}
\in\Z[u,u^{-1}][[t]]
\]
It is easily computable with \Mtwo from the contravariant cohomology
module, by using the {\tt hilbertSeries}\indexcmd{hilbertSeries} routine.

\begin{sCode}
The function {\tt poincareSeries2 M} computes the graded Poin\-car\'e
series of a graded module $M$ over a graded complete intersection $B$.

First we set up a ring whose elements can serve as Poincar\'e series.
\beginOutput
i94 : T = ZZ[t,u,Inverses=>true,MonomialOrder=>RevLex];\\
\endOutput
\beginOutput
i95 : poincareSeries2 = M -> (\\
\         B := ring M;\\
\         k := B/ideal vars B;\\
\         P := Ext(M,k);\\
\         h := hilbertSeries P;\\
\         T':= degreesRing P;\\
\         substitute(h, \{T'_0=>t^-1,T'_1=>u^-1\})\\
\         );\\
\endOutput
The last line in the code above replaces the variables in the
Poincar\'e series provided by the {\tt hilbertSeries} function with the
variables in our ring {\tt T}.
 \end{sCode}

The $n$th {\it\ie{Betti number}\/} $\b^B_n(M)$ of $M$ over $B$ is the
rank of the $n$th module in a minimal resolution of $M$ by free
$B$-modules.  The {\it\ie{Poincar\'e series}\/} $\Poi^B_M(t)$ is the
generating function of the Betti numbers.  There are expressions
\[
\b^B_n(M)=\sum_{s=0}^\infty\b^B_{ns}(M)
\qquad\text{and}\qquad
\Poi^B_M(t)=\Poi^B_M(t,1)
\]
Accordingly, the code for $\Poi^B_M(t)$ just replaces
in $\Poi^B_M(t,u)$ the variable $u$ by $1$.

\begin{sCode}
The function {\tt poincareSeries1 M} computes the Poincar\'e series of a
graded module $M$ over a graded complete intersection $B$.
\beginOutput
i96 : poincareSeries1 = M -> (\\
\         substitute(poincareSeries2 M, \{u=>1_T\})\\
\         );\\
\endOutput
\end{sCode}

Now let's use these codes in computations.

\begin{sExample}
\label{cosyzygy}
To get a module whose Betti sequence initially decreases, we form
an artinian complete intersection $B'$ and take $M'$ to be a cosyzygy
in a minimal injective resolution of the residue field $k$.  Since $B'$
is self-injective, this can be achieved by taking a syzygy of $k$, then
transposing its presentation matrix.  Of course, we ask \Mtwo to carry
out these steps.
\beginOutput
i97 : A' = K[x,y,z];\\
\endOutput
\beginOutput
i98 : B' = A'/(x^2,y^2,z^3);\\
\endOutput
\beginOutput
i99 : C' = res(B'^1/(x,y,z), LengthLimit => 6)\\
\emptyLine
\        1       3       6       10       15       21       28\\
o99 = B'  <-- B'  <-- B'  <-- B'   <-- B'   <-- B'   <-- B'\\
\                                                          \\
\      0       1       2       3        4        5        6\\
\emptyLine
o99 : ChainComplex\\
\endOutput
\beginOutput
i100 : M' = coker transpose C'.dd_5\\
\emptyLine
o100 = cokernel \{-5\} | -y 0   0  0  z  0 0 0  0 0  0  0 0  0 0 |\\
\                \{-5\} | -x -y  0  0  0  z 0 0  0 0  0  0 0  0 0 |\\
\                \{-5\} | 0  x   -y 0  0  0 z 0  0 0  0  0 0  0 0 |\\
\                \{-5\} | 0  0   x  -y 0  0 0 z  0 0  0  0 0  0 0 |\\
\                \{-5\} | 0  0   0  -x 0  0 0 0  z 0  0  0 0  0 0 |\\
\                \{-5\} | 0  0   0  0  y  0 0 0  0 0  0  0 0  0 0 |\\
\                \{-5\} | 0  0   0  0  -x y 0 0  0 0  0  0 0  0 0 |\\
\                \{-5\} | 0  0   0  0  0  x y 0  0 0  0  0 0  0 0 |\\
\                \{-5\} | 0  0   0  0  0  0 x y  0 0  0  0 0  0 0 |\\
\                \{-5\} | 0  0   0  0  0  0 0 -x y 0  0  0 0  0 0 |\\
\                \{-5\} | 0  0   0  0  0  0 0 0  x 0  0  0 0  0 0 |\\
\                \{-6\} | 0  0   0  0  0  0 0 0  0 -y 0  z 0  0 0 |\\
\                \{-6\} | 0  0   0  0  0  0 0 0  0 x  -y 0 z  0 0 |\\
\                \{-6\} | 0  0   0  0  0  0 0 0  0 0  -x 0 0  z 0 |\\
\                \{-6\} | z2 0   0  0  0  0 0 0  0 0  0  y 0  0 0 |\\
\                \{-6\} | 0  -z2 0  0  0  0 0 0  0 0  0  x y  0 0 |\\
\                \{-6\} | 0  0   z2 0  0  0 0 0  0 0  0  0 -x y 0 |\\
\                \{-6\} | 0  0   0  z2 0  0 0 0  0 0  0  0 0  x 0 |\\
\                \{-7\} | 0  0   0  0  0  0 0 0  0 0  0  0 0  0 z |\\
\                \{-7\} | 0  0   0  0  0  0 0 0  0 z2 0  0 0  0 y |\\
\                \{-7\} | 0  0   0  0  0  0 0 0  0 0  z2 0 0  0 x |\\
\emptyLine
\                                21\\
o100 : B'-module, quotient of B'\\
\endOutput
Compute the Poincar\'e series in two variables $\Poi^{B'}_{M'}(t,u)$.
\beginOutput
i101 : poincareSeries2 M'\\
\emptyLine
\         -7     -6      -5      -6       -5       -4     2 -5      2 - $\cdot\cdot\cdot$\\
\       3u   + 7u   + 11u   + t*u   + 5t*u   + 9t*u   - 6t u   - 14t u  $\cdot\cdot\cdot$\\
o101 = --------------------------------------------------------------- $\cdot\cdot\cdot$\\
\                                                                       $\cdot\cdot\cdot$\\
\                                                                       $\cdot\cdot\cdot$\\
\emptyLine
o101 : Divide\\
\endOutput
\end{sExample}

\begin{sExample}
We compute $\Poi^B_M(t)$ for the module $M$ from Example \ref{random}.
\beginOutput
i102 : p = poincareSeries1 M\\
\emptyLine
\                  2     3      4    5     6    7\\
\       3 + 2t - 5t  + 4t  + 12t  + t  - 4t  - t\\
o102 = -----------------------------------------\\
\                      2       2       2\\
\                (1 - t )(1 - t )(1 - t )\\
\emptyLine
o102 : Divide\\
\endOutput
We have written some rather na\"\i ve code for simplifying rational
functions as above.  It locates factors of the form $1-t^n$ in the
denominator, factors out $1-t$, and factors out $1+t$ if $n$ is even.
Keeping the factors of the denominator separate, it then cancels as
many of them as it can with the numerator.
\beginOutput
i103 : load "simplify.m2"\\
\endOutput
\beginOutput
i104 : simplify p\\
\emptyLine
\                 2     3     4     5    6\\
\       3 - t - 4t  + 8t  + 4t  - 3t  - t\\
o104 = ----------------------------------\\
\                       2       3\\
\                (1 + t) (1 - t)\\
\emptyLine
o104 : Divide\\
\endOutput
In this case, it succeeded in canceling a factor of $1+t$.
\end{sExample}

\begin{sExample}
  We compute some Betti numbers for $M$.  We use the division operation in
  the Euclidean domain $T' = \mathbb Q[t,t^{-1}]$ with the reverse monomial ordering
  to compute power series expansions.
\beginOutput
i105 : T' = QQ[t,Inverses=>true,MonomialOrder=>RevLex];\\
\endOutput
\beginOutput
i106 : expansion = (n,q) -> (\\
\           t := T'_0;\\
\           rho := map(T',T,\{t,1\});\\
\           num := rho value numerator q;\\
\           den := rho value denominator q;\\
\           n = n + first degree den;\\
\           n = max(n, first degree num + 1);\\
\           (num + t^n) // den\\
\           );\\
\endOutput
Now let's expand the Poincar\'e series up to $t^{20}$.
\beginOutput
i107 : expansion(20,p)\\
\emptyLine
\                  2      3      4      5      6      7      8      9   $\cdot\cdot\cdot$\\
o107 = 3 + 2t + 4t  + 10t  + 15t  + 25t  + 32t  + 46t  + 55t  + 73t  + $\cdot\cdot\cdot$\\
\emptyLine
o107 : T'\\
\endOutput
Just to make sure, let's compare the first few coefficients with the
more pedestrian way of doing the computation, one Ext module at a time.
\beginOutput
i108 : psi = map(K,B)\\
\emptyLine
o108 = map(K,B,\{0, 0, 0\})\\
\emptyLine
o108 : RingMap K <--- B\\
\endOutput
\beginOutput
i109 : apply(10, i -> rank (psi ** Ext^i(M,coker vars B)))\\
\emptyLine
o109 = \{3, 2, 4, 10, 15, 25, 32, 46, 55, 73\}\\
\emptyLine
o109 : List\\
\endOutput
Now we restore {\tt t} to its original use.
\beginOutput
i110 : use T;\\
\endOutput
 \end{sExample}

\subsection{Complexity}

The {\it\ie{complexity}\/} of $M$ is the least $d\in\N$ such that
the function
\[
n \mapsto \dim_K \Ext^n_B(M,k)
\]
is bounded above by a polynomial of degree $d-1$ (with the convention
that the zero polynomial has degree $-1$).  This number, denoted
$\cx_B(M)$, was introduced in
\cite{CI:MR90g:13027} to measures on a polynomial scale the rate of
growth of the Betti numbers of $M$.  It is calibrated so that
$\cx_B(M)=0$ if and only if $M$ has finite projective dimension.
Corollary \ref{series} yields
\[
\Poi^B_M(t)=\frac{p^B_M(t)}{(1-t^2)^c}
\qquad\text{for some}\qquad p^B_M(t)\in\Z[t]
\]
Decomposing the right hand side into partial fractions, one sees that
$\cx_R(M)$ equals the order of the pole of $\Poi^B_M(t)$ at $t=1$; in
particular, $\cx_R(M,N)\le c$.  However, since we get $\Poi^B_M(t)$
from a computation of the $R$-module $P=\Ext^\bu_B(M,k)$, it is natural
to obtain $\cx_R(M)$ as the Krull dimension of $P$.

\begin{sCode}
The function {\tt complexity M} yields the complexity of a graded
module $M$ over a graded complete intersection ring $B$.
\beginOutput
i111 : complexity = M -> dim Ext(M,coker vars ring M);\\
\endOutput
 \end{sCode}

\begin{sExample}
We compute $\cx_B(M)$ for $M$ from Example \ref{random}.
\beginOutput
i112 : complexity M\\
\emptyLine
o112 = 3\\
\endOutput
 \end{sExample}

\subsection{Critical Degree}

The {\it\ie{critical degree}\/} of $M$ is the least integer $\ell$ for
which the minimal resolution $F$ of $M$ admits a chain map $g\colon
F\to F$ of degree $m<0$, such that $g_{m+n}\colon F_{m+n}\to F_n$ is
surjective for all $n>\ell$.  This number, introduced in
\cite{CI:MR99c:13033} and denoted $\crdeg_BM$, is meaningful over every
graded ring $B$.  It is equal to the projective dimension whenever the
latter is finite.

When $B$ is a complete intersection it is proved in
\cite[Sect.~7]{CI:MR99c:13033} that $\crdeg_BM$ is finite and yields
important information on the Betti sequence:
\begin{itemize}
\item[$\bullet$]
if $\cx_BM\le1$, then $\b^B_n(M) =\b^B_{n+1}(M)$ for all $n>\crdeg_BM$.
\item[$\bullet$]
if $\cx_BM\ge2$, then $\b^B_n(M)<\b^B_{n+1}(M)$ for all $n>\crdeg_BM$.
\end{itemize}
Thus, it is interesting to know $\crdeg_BM$, or at least to have a good
upper bound.  Here is what is known, in terms of $h=\depth B-\depth_BM$.
\begin{itemize}
\item[$\bullet$]
if $\cx_BM=0$, then $\crdeg_BM=h$.
\item[$\bullet$]
if $\cx_BM=1$, then $\crdeg_BM\le h$.
\item[$\bullet$]
if $\cx_BM=2$, then $\crdeg_BM\le h+1+\max\{2\beta^B_h(M)-1\,,\,
2\beta^B_{h+1}(M)\}$.
\end{itemize}
The first part is the Auslander-Buchsbaum Equality, the second part is
proved in \cite[Sect.~6]{CI:Ei}, the third is established in
\cite[Sect.~7]{CI:MR1774757}.

These upper bounds are realistic:  there exist examples in complexity
$1$ when they are reached, and examples in complexity $2$ when they are
not more than twice the actual value of the critical degree.  If
$\cx_RM\ge3$, then it is an {\bf open problem} whether the critical
degree of $M$ can be bounded in terms that do not depend on the action
of the cohomology operators.

However, in every concrete case $\crdeg_RM$ can be computed explicitly
by using \Mtwo.  Indeed, it is proved in \cite[Sect.~7]{CI:MR99c:13033}
that $\crdeg_RM$ is equal to the highest degree of a non-zero element
in the socle of the $R$-module $\Ext^\bu_B(M,k)$, that is, the
submodule consisting of elements annihilated by $(X_1,\dots,X_c)$.  The
socle is naturally isomorphic to $\Hom_B(k,\Ext^\bu_B(M,k))$, so it can
be obtained by standard \Mtwo routines.

For instance, for the module $M$ from Example \ref{random}, we get
\beginOutput
i113 : k = coker vars ring H;\\
\endOutput
\beginOutput
i114 : prune Hom(k,H)\\
\emptyLine
o114 = 0\\
\emptyLine
o114 : K [\$X , \$X , x, y, Degrees => \{\{-2, -2\}, \{-2, -3\}, \{0, 1\}, \{0,  $\cdot\cdot\cdot$\\
\            1    2\\
\endOutput
The degrees displayed above show that $\crdeg_RM=1$.
\medskip

Of course, one might prefer to see the number $\crdeg_BM$ directly.

\begin{sCode}
The function {\tt criticalDegree M} computes the critical degree of a
graded module $M$ over a graded complete intersection ring $B$.
\beginOutput
i115 : criticalDegree = M -> (\\
\          B := ring M;\\
\          k := B / ideal vars B;\\
\          P := Ext(M,k);\\
\          k  = coker vars ring P;\\
\          - min ( first {\char`\\} degrees source gens prune Hom(k,P))\\
\          );\\
\endOutput
\end{sCode}

Let's test the new code in a couple of cases.

\begin{sExample}
For the module $M$ of Example \ref{random} we have
\beginOutput
i116 : criticalDegree M\\
\emptyLine
o116 = 1\\
\endOutput
in accordance with what was already observed above.

For the module $M'$ of Example \ref{cosyzygy} we obtain
\beginOutput
i117 : criticalDegree M'\\
\emptyLine
o117 = 5\\
\endOutput
\end{sExample}

\subsection{Support Variety}
\label{Support variety}

Let $\overline K$ denote an algebraic closure of $K$.  The
{\it\ie{support variety}\/} $\var^*_B(M)$ is the algebraic set in
${\overline K}{}^c$ defined by the annihilator of $\Ext^\bu_B(M,k)$
over $R=K[X_1,\dots,X_c]$.  This `geometric image' of the
contravariant cohomology module was introduced in \cite{CI:MR90g:13027}
and used to study the minimal free resolution of $M$.  The dimension of
the support variety is equal to the complexity $\cx_R(M)$, that we can
already compute.  There is no need to associate a variety to the
covariant cohomology module, see \ref{Support variety2}.

Since $\var^*_B(M)$ is defined by homogeneous equations, it is a cone
in ${\overline K}{}^c$.  An important {\bf open problem} is whether
every cone in ${\overline K}{}^c$ that can be defined over $K$ is the
variety of some $B$-module $M$.  By \cite[Sect.~6]{CI:MR90g:13027} all
linear subspaces and all hypersurfaces arise in this way, but little
more is known in general.

Feeding our computation of $\Ext^\bu_B(M,k)$ to standard \Mtwo routines
we write code for determining a set of equations defining $\var^*_B(M)$.

\begin{sCode}
The function {\tt supportVarietyIdeal M} yields a set of polynomial
equations with coefficients in $K$, defining the support variety
$\var^*_B(M)$ in ${\overline K}{}^c$ for a graded module $M$ over a
graded complete intersection $B$.
\beginOutput
i118 : supportVarietyIdeal = M -> (\\
\          B := ring M;\\
\          k := B/ideal vars B;\\
\          ann Ext(M,k)\\
\          );\\
\endOutput
 \end{sCode}

As before, we illustrate the code with explicit computations.  In
view of the open problem mentioned above, we fix a ring and a type of
presentation, then change randomly the presentation matrix in the hope
of finding an `interesting' variety.  The result of the experiment
is assessed in Remark \ref{letdown}.

\begin{sExample}
Let $\F_7$ denote the prime field with $7$ elements, and form the
zero-dimensional complete intersection $B''=\F_7[x,y,z]/(x^7,y^7,z^7)$.
\beginOutput
i119 : K'' = ZZ/7;\\
\endOutput
\beginOutput
i120 : A'' = K''[x,y,z];\\
\endOutput
\beginOutput
i121 : J'' = ideal(x^7,y^7,z^7);\\
\emptyLine
o121 : Ideal of A''\\
\endOutput
\beginOutput
i122 : B'' = A''/J'';\\
\endOutput
We apply the code above to search, randomly, for some varieties.  Using
{\tt scan}\indexcmd{scan} we print the results from several runs with one command.
\beginOutput
i123 : scan((1,1) .. (3,3), (r,d) -> (\\
\               V := cokernel random (B''^r,B''^\{-d\});\\
\               << "--------------------------------------------------- $\cdot\cdot\cdot$\\
\               << endl\\
\               << "V = " << V << endl\\
\               << "support variety ideal = "\\
\               << timing supportVarietyIdeal V\\
\               << endl))\\
------------------------------------------------------------------\\
V = cokernel | -2x+3y+2z |\\
support variety ideal = ideal (X  - 2X , X  + X )\\
\                                2     3   1    3\\
\                        -- 0.7 seconds\\
------------------------------------------------------------------\\
V = cokernel | 3x2-2xy+xz-3yz |\\
support variety ideal = ideal(X  + 3X  + 2X )\\
\                               1     2     3\\
\                        -- 0.48 seconds\\
------------------------------------------------------------------\\
V = cokernel | -2x3+3x2y+y3-x2z-3y2z-xz2-3z3 |\\
support variety ideal = 0\\
\                        -- 1.54 seconds\\
------------------------------------------------------------------\\
V = cokernel | -3y+3z |\\
\             | -2x-2y |\\
support variety ideal = ideal(X  + X  - X )\\
\                               1    2    3\\
\                        -- 0.86 seconds\\
------------------------------------------------------------------\\
V = cokernel | -x2+2y2-xz+yz+3z2 |\\
\             | 2xy-3xz-3yz-2z2   |\\
support variety ideal = 0\\
\                        -- 1.31 seconds\\
------------------------------------------------------------------\\
V = cokernel | -x3-2x2y-xy2-2xyz+3y2z+2xz2-yz2-2z3 |\\
\             | 2xy2+3y3-3x2z-2y2z+2xz2+2yz2        |\\
support variety ideal = 0\\
\                        -- 2.21 seconds\\
------------------------------------------------------------------\\
V = cokernel | 3x-y-z   |\\
\             | -3x-y+2z |\\
\             | x-2y+3z  |\\
support variety ideal = 0\\
\                        -- 1.1 seconds\\
------------------------------------------------------------------\\
V = cokernel | 2x2-2xy+2y2+2xz-3z2   |\\
\             | -x2+2xy+y2+3xz+3yz-z2 |\\
\             | -2xz+2yz+2z2          |\\
support variety ideal = 0\\
\                        -- 1.67 seconds\\
------------------------------------------------------------------\\
V = cokernel | 2x3-x2y+2xy2-y3-2xyz+3y2z+xz2+3yz2+z3  |\\
\             | -3x3-3x2y+3xy2+2x2z+3xyz-3y2z-xz2      |\\
\             | -3x3-2x2y-xy2-2y3-2xyz+y2z+xz2+3yz2-z3 |\\
support variety ideal = 0\\
\                        -- 1.92 seconds\\
\endOutput
\end{sExample}

\begin{sRemark}
\label{letdown}
The (admittedly short) search above did not turn up any non-linear
variety.  This should be contrasted with the known result that
{\it every\/} cone in ${\overline \F_7}{}^3$ is the support variety of
some $B''$-module.

Indeed, $B''$ is isomorphic to the group algebra $\F_7[G]$ of the
elementary abelian group $G=\C_7\times\C_7\times\C_7$, where $\C_7$ is
a cyclic group of order $7$.  It is shown in
\cite[Sect.~7]{CI:MR90g:13027} that $\var^*_{B''}(V)$ is equal to a
variety $\var^*_G(V)$, defined in a different way in
\cite{CI:MR85a:20004} by Carlson.  He proves in \cite{CI:MR86b:20009}
that if $K$ is a field of characteristic $p>0$, and $G$ is an
elementary abelian $p$-group of rank $c$, then every cone in
${\overline K}{}^c$ is the rank variety of a finitely generated module
over $K[G]$.
 \end{sRemark}

\subsection{Bass Series}
\label{Bass series}

The {\it\ie{graded Bass number}\/} $\mu^{ns}_B(M)$ of $M$ over $B$ is
the number of direct summands isomorphic to $U[s]$ in the $n$'th module
of a minimal graded injective resolution of $M$ over $B$, where $U$ 
is the injective envelope of $k$.  It satisfies
\[
\mu^{ns}_B(M)=\dim_K\Ext^n_B(k,M)_{s}
\]
The {\it\ie{graded Bass series}\/} of $M$ over $B$ is the generating
function
\[
\Ba^M_B(t,u)=\sum_{n\in\N\,,\,s\in\Z}\mu^{ns}_B(M)\, t^nu^s
\in\Z[u,u^{-1}][[t]]
\]
It is easily computable with \Mtwo from the covariant cohomology module,
by using the {\tt hilbertSeries}\indexcmd{hilbertSeries} routine.

\begin{sCode}
The function {\tt bassSeries2 M} computes the graded Bass series of a
graded module $M$ over a graded complete intersection $B$.
\beginOutput
i124 : bassSeries2 = M -> (\\
\          B := ring M;\\
\          k := B/ideal vars B;\\
\          I := Ext(k,M);\\
\          h := hilbertSeries I;\\
\          T':= degreesRing I;\\
\          substitute(h, \{T'_0=>t^-1, T'_1=>u\})\\
\          );\\
\endOutput
 \end{sCode}

As with Betti numbers and Poincar\'e series, there are ungraded versions
of Bass numbers and Bass series; they are given, respectively, by
\[
\mu_B^n(M)=\sum_{s=0}^\infty\mu_B^{ns}(M)
\qquad\text{and}\qquad
\Ba_B^M(t)=\Ba_B^M(t,1)
\]

\begin{sCode}
The function {\tt bassSeries1 M} computes the Bass series of a graded
module $M$ over a graded complete intersection $B$.
\beginOutput
i125 : bassSeries1 = M -> (\\
\          substitute(bassSeries2 M, \{u=>1_T\})\\
\          );\\
\endOutput
\end{sCode}

Now let's use these codes in computations.

\begin{sExample}
For $k$, the residue field of $B$, the contravariant and covariant
cohomology modules coincide.  For comparison, we compute side by side
the Poincar\'e series and the Bass series of $k$, when
$B=K[x,y,z]/(x^3,y^4,z^5)$ is the ring defined in Example
\ref{random}.
\beginOutput
i126 : use B;\\
\endOutput
\beginOutput
i127 : L = B^1/(x,y,z);\\
\endOutput
\beginOutput
i128 : p = poincareSeries2 L\\
\emptyLine
\                        2 2    3 3\\
\           1 + 3t*u + 3t u  + t u\\
o128 = ------------------------------\\
\             2 3       2 4       2 5\\
\       (1 - t u )(1 - t u )(1 - t u )\\
\emptyLine
o128 : Divide\\
\endOutput
\beginOutput
i129 : b = bassSeries2 L\\
\emptyLine
\                  -1     2 -2    3 -3\\
\          1 + 3t*u   + 3t u   + t u\\
o129 = ---------------------------------\\
\             2 -3       2 -4       2 -5\\
\       (1 - t u  )(1 - t u  )(1 - t u  )\\
\emptyLine
o129 : Divide\\
\endOutput
The reader would have noticed that the two series are different, and
that one is obtained from the other by the substitution $u\mapsto
u^{-1}$.  This underscores the different meanings of the graded Betti
numbers and Bass numbers.
 \end{sExample}

\begin{sExample}
Here we compute the graded and ungraded Bass series of the $B$-module
$M$ of Example \ref{random}.
\beginOutput
i130 : b2 = bassSeries2 M\\
\emptyLine
\         6      3       4       5    2 2    2 3     3     3      3 2   $\cdot\cdot\cdot$\\
\       7u  + t*u  + 9t*u  + 3t*u  - t u  - t u  - 4t  - 3t u - 3t u  + $\cdot\cdot\cdot$\\
o130 = --------------------------------------------------------------- $\cdot\cdot\cdot$\\
\                                    2 -3       2 -4       2 -5\\
\                              (1 - t u  )(1 - t u  )(1 - t u  )\\
\emptyLine
o130 : Divide\\
\endOutput
\beginOutput
i131 : b1 = bassSeries1 M;\\
\endOutput
\beginOutput
i132 : simplify b1\\
\emptyLine
\                  2     3     4\\
\       7 + 6t - 8t  - 2t  + 3t\\
o132 = ------------------------\\
\                  2       3\\
\           (1 + t) (1 - t)\\
\emptyLine
o132 : Divide\\
\endOutput
 \end{sExample}

\section{Invariants of Pairs of Modules}
\label{Invariants of pairs of modules}

In this final section we compute invariants of a pair $(M,N)$ of graded
modules over a graded complete intersection $B$, derived from the
reduced Ext module $\rExt^\bu_B(M,N)$ defined in Remark \ref{reduced
ext}.  The treatment here is parallel to that in Section
\ref{Invariants of modules}.  When one of the modules $M$ or $N$ is
equal to the residue field $k$, the invariants discussed below reduce
to those treated in that section.

\subsection{Reduced Ext Module}

The reduced Ext module $\rExt^\bu_B(M,N)=\Ext^\bu_B(M,N)\otimes_Ak$
defined in Remark \ref{reduced ext} is computed from our basic
routine {\tt Ext(M,N)} by applying the function {\tt changeRing}
defined in Code \ref{change}.

\begin{sCode}
\label{reduced}
The function {\tt ext(M,N)} computes $\rExt_B^\bu(M,N)$ when $M$
and $N$ are graded modules over a graded complete intersection $B$.
\beginOutput
i133 : ext = (M,N) -> changeRing Ext(M,N);\\
\endOutput
 \end{sCode}

\begin{sExample}
\label{new module}
Using the ring $B=K[x,y,z]/(x^3,y^4,z^5)$ and the module $M$ created
in Example \ref{random}, we make new modules 
\[
N=B/(x^2+z^2\,,\,y^3) \qquad\text{and}\qquad
N'=B/(x^2+z^2\,,\,y^3-2z^3)
\]
\beginOutput
i134 : use B;\\
\endOutput
\beginOutput
i135 : N = B^1/(x^2 + z^2,y^3);\\
\endOutput
\beginOutput
i136 : time rH = ext(M,N);\\
\     -- used 15.91 seconds\\
\endOutput
\beginOutput
i137 : evenPart rH\\
\emptyLine
o137 = cokernel \{-4, -9\} | 0   0   0   0   0   0   0   0   0   0   0   $\cdot\cdot\cdot$\\
\                \{0, 2\}   | 0   0   0   0   0   0   0   X_3 0   0   0   $\cdot\cdot\cdot$\\
\                \{0, 2\}   | 0   0   0   0   0   X_3 0   0   0   0   0   $\cdot\cdot\cdot$\\
\                \{0, 2\}   | 0   0   0   0   0   0   X_3 0   0   0   0   $\cdot\cdot\cdot$\\
\                \{0, 2\}   | 0   0   0   0   X_3 0   0   0   0   0   0   $\cdot\cdot\cdot$\\
\                \{0, 2\}   | 0   0   0   X_3 0   0   0   0   0   0   0   $\cdot\cdot\cdot$\\
\                \{0, 2\}   | 0   0   X_3 0   0   0   0   0   0   0   0   $\cdot\cdot\cdot$\\
\                \{0, 2\}   | 0   X_3 0   0   0   0   0   0   0   0   X_2 $\cdot\cdot\cdot$\\
\                \{0, 2\}   | X_3 0   0   0   0   0   0   0   0   X_2 0   $\cdot\cdot\cdot$\\
\                \{-2, -4\} | 0   0   0   0   0   0   0   0   0   0   0   $\cdot\cdot\cdot$\\
\                \{-2, -4\} | 0   0   0   0   0   0   0   0   0   0   0   $\cdot\cdot\cdot$\\
\                \{-2, -4\} | 0   0   0   0   0   0   0   0   0   0   0   $\cdot\cdot\cdot$\\
\                \{-2, -4\} | 0   0   0   0   0   0   0   0   0   0   0   $\cdot\cdot\cdot$\\
\                \{-2, -4\} | 0   0   0   0   0   0   0   0   0   0   0   $\cdot\cdot\cdot$\\
\                \{-2, -4\} | 0   0   0   0   0   0   0   0   0   0   0   $\cdot\cdot\cdot$\\
\                \{-2, -4\} | 0   0   0   0   0   0   0   0   0   0   0   $\cdot\cdot\cdot$\\
\                \{0, 1\}   | 0   0   0   0   0   0   0   0   X_2 0   0   $\cdot\cdot\cdot$\\
\emptyLine
\                                                                       $\cdot\cdot\cdot$\\
o137 : K [X , X , X , Degrees => \{\{-2, -3\}, \{-2, -4\}, \{-2, -5\}\}]-modul $\cdot\cdot\cdot$\\
\           1   2   3                                                   $\cdot\cdot\cdot$\\
\endOutput
\beginOutput
i138 : oddPart rH\\
\emptyLine
o138 = cokernel \{-3, -6\} | 0      0   0   0   0   0   0   0   0   0    $\cdot\cdot\cdot$\\
\                \{-3, -6\} | 0      0   0   0   0   0   0   0   0   0    $\cdot\cdot\cdot$\\
\                \{-3, -6\} | 0      0   0   0   0   0   0   0   0   0    $\cdot\cdot\cdot$\\
\                \{-3, -6\} | 0      0   0   0   0   0   0   0   0   0    $\cdot\cdot\cdot$\\
\                \{-3, -6\} | 0      0   0   0   0   0   0   0   0   0    $\cdot\cdot\cdot$\\
\                \{-1, -1\} | -39X_3 0   0   0   0   0   0   0   X_2 0    $\cdot\cdot\cdot$\\
\                \{-1, -1\} | 31X_3  0   0   0   0   0   0   X_2 0   0    $\cdot\cdot\cdot$\\
\                \{-1, -1\} | -34X_3 0   0   0   0   0   X_2 0   0   0    $\cdot\cdot\cdot$\\
\                \{-1, -1\} | -35X_3 0   0   0   0   X_2 0   0   0   0    $\cdot\cdot\cdot$\\
\                \{-1, -1\} | -29X_3 0   0   0   X_2 0   0   0   0   0    $\cdot\cdot\cdot$\\
\                \{-1, -1\} | 12X_3  0   0   X_2 0   0   0   0   0   0    $\cdot\cdot\cdot$\\
\                \{-1, -1\} | -8X_3  0   X_2 0   0   0   0   0   0   0    $\cdot\cdot\cdot$\\
\                \{-1, -1\} | X_3    X_2 0   0   0   0   0   0   0   0    $\cdot\cdot\cdot$\\
\                \{-3, -7\} | 0      0   0   0   0   0   0   0   0   X_1  $\cdot\cdot\cdot$\\
\emptyLine
\                                                                       $\cdot\cdot\cdot$\\
o138 : K [X , X , X , Degrees => \{\{-2, -3\}, \{-2, -4\}, \{-2, -5\}\}]-modul $\cdot\cdot\cdot$\\
\           1   2   3                                                   $\cdot\cdot\cdot$\\
\endOutput
\beginOutput
i139 : N' = B^1/(x^2 + z^2,y^3 - 2*z^3);\\
\endOutput
\beginOutput
i140 : time rH' = ext(M,N');\\
\     -- used 20.26 seconds\\
\endOutput
\beginOutput
i141 : evenPart rH'\\
\emptyLine
o141 = cokernel \{-4, -8\} | 0   0   0   0   0   0   0   0   0   0   0   $\cdot\cdot\cdot$\\
\                \{-4, -8\} | 0   0   0   0   0   0   0   0   0   0   0   $\cdot\cdot\cdot$\\
\                \{-4, -9\} | 0   0   0   0   0   0   0   0   0   0   0   $\cdot\cdot\cdot$\\
\                \{-4, -9\} | 0   0   0   0   0   0   0   0   0   0   0   $\cdot\cdot\cdot$\\
\                \{-4, -9\} | 0   0   0   0   0   0   0   0   0   0   0   $\cdot\cdot\cdot$\\
\                \{-4, -9\} | 0   0   0   0   0   0   0   0   0   0   0   $\cdot\cdot\cdot$\\
\                \{-4, -9\} | 0   0   0   0   0   0   0   0   0   0   0   $\cdot\cdot\cdot$\\
\                \{-4, -9\} | 0   0   0   0   0   0   0   0   0   0   0   $\cdot\cdot\cdot$\\
\                \{0, 2\}   | 0   0   0   0   0   0   X_3 0   0   0   0   $\cdot\cdot\cdot$\\
\                \{0, 2\}   | 0   0   0   0   0   X_3 0   0   0   0   0   $\cdot\cdot\cdot$\\
\                \{-2, -4\} | 0   0   0   0   0   0   0   0   0   0   0   $\cdot\cdot\cdot$\\
\                \{-2, -4\} | 0   0   0   0   0   0   0   0   0   0   0   $\cdot\cdot\cdot$\\
\                \{-2, -4\} | 0   0   0   0   0   0   0   0   0   0   0   $\cdot\cdot\cdot$\\
\                \{-2, -4\} | 0   0   0   0   0   0   0   0   0   0   0   $\cdot\cdot\cdot$\\
\                \{-2, -4\} | 0   0   0   0   0   0   0   0   0   0   0   $\cdot\cdot\cdot$\\
\                \{-2, -4\} | 0   0   0   0   0   0   0   0   0   0   0   $\cdot\cdot\cdot$\\
\                \{0, 2\}   | 0   0   0   0   X_3 0   0   0   0   0   0   $\cdot\cdot\cdot$\\
\                \{0, 2\}   | 0   0   X_3 0   0   0   0   0   0   0   X_2 $\cdot\cdot\cdot$\\
\                \{-2, -4\} | 0   0   0   0   0   0   0   0   0   0   0   $\cdot\cdot\cdot$\\
\                \{-2, -4\} | 0   0   0   0   0   0   0   0   0   0   0   $\cdot\cdot\cdot$\\
\                \{-2, -4\} | 0   0   0   0   0   0   0   0   0   0   0   $\cdot\cdot\cdot$\\
\                \{-2, -4\} | 0   0   0   0   0   0   0   0   0   0   0   $\cdot\cdot\cdot$\\
\                \{0, 2\}   | 0   0   0   X_3 0   0   0   0   0   0   0   $\cdot\cdot\cdot$\\
\                \{0, 2\}   | 0   X_3 0   0   0   0   0   0   0   X_2 0   $\cdot\cdot\cdot$\\
\                \{-2, -4\} | 0   0   0   0   0   0   0   0   0   0   0   $\cdot\cdot\cdot$\\
\                \{-2, -4\} | 0   0   0   0   0   0   0   0   0   0   0   $\cdot\cdot\cdot$\\
\                \{-2, -4\} | 0   0   0   0   0   0   0   0   0   0   0   $\cdot\cdot\cdot$\\
\                \{-2, -4\} | 0   0   0   0   0   0   0   0   0   0   0   $\cdot\cdot\cdot$\\
\                \{-2, -4\} | 0   0   0   0   0   0   0   0   0   0   0   $\cdot\cdot\cdot$\\
\                \{-2, -4\} | 0   0   0   0   0   0   0   0   0   0   0   $\cdot\cdot\cdot$\\
\                \{0, 2\}   | X_3 0   0   0   0   0   0   0   0   0   0   $\cdot\cdot\cdot$\\
\                \{0, 1\}   | 0   0   0   0   0   0   0   X_2 0   0   0   $\cdot\cdot\cdot$\\
\                \{0, 1\}   | 0   0   0   0   0   0   0   0   X_2 0   0   $\cdot\cdot\cdot$\\
\emptyLine
\                                                                       $\cdot\cdot\cdot$\\
o141 : K [X , X , X , Degrees => \{\{-2, -3\}, \{-2, -4\}, \{-2, -5\}\}]-modul $\cdot\cdot\cdot$\\
\           1   2   3                                                   $\cdot\cdot\cdot$\\
\endOutput
\beginOutput
i142 : oddPart rH'\\
\emptyLine
o142 = cokernel \{-3, -6\} | 0   0   0   0   0   0   0   0   -42X_2 21X_ $\cdot\cdot\cdot$\\
\                \{-3, -6\} | 0   0   0   0   0   0   0   0   -6X_2  -32X $\cdot\cdot\cdot$\\
\                \{-3, -6\} | 0   0   0   0   0   0   0   0   -8X_2  12X_ $\cdot\cdot\cdot$\\
\                \{-3, -6\} | 0   0   0   0   0   0   0   0   26X_2  -36X $\cdot\cdot\cdot$\\
\                \{-3, -6\} | 0   0   0   0   0   0   0   0   50X_2  18X_ $\cdot\cdot\cdot$\\
\                \{-3, -6\} | 0   0   0   0   0   0   0   0   31X_2  7X_2 $\cdot\cdot\cdot$\\
\                \{-3, -7\} | 0   0   0   0   0   0   0   0   0      0    $\cdot\cdot\cdot$\\
\                \{-3, -7\} | 0   0   0   0   0   0   0   0   0      0    $\cdot\cdot\cdot$\\
\                \{-3, -7\} | 0   0   0   0   0   0   0   0   0      0    $\cdot\cdot\cdot$\\
\                \{-3, -7\} | 0   0   0   0   0   0   0   0   0      0    $\cdot\cdot\cdot$\\
\                \{-3, -7\} | 0   0   0   0   0   0   0   0   0      X_1  $\cdot\cdot\cdot$\\
\                \{-3, -7\} | 0   0   0   0   0   0   0   0   X_1    0    $\cdot\cdot\cdot$\\
\                \{-1, -2\} | 0   0   0   X_2 0   0   0   X_1 0      0    $\cdot\cdot\cdot$\\
\                \{-1, -2\} | 0   0   X_2 0   0   0   X_1 0   0      0    $\cdot\cdot\cdot$\\
\                \{-1, -2\} | 0   X_2 0   0   0   X_1 0   0   0      0    $\cdot\cdot\cdot$\\
\                \{-1, -2\} | X_2 0   0   0   X_1 0   0   0   0      0    $\cdot\cdot\cdot$\\
\emptyLine
\                                                                       $\cdot\cdot\cdot$\\
o142 : K [X , X , X , Degrees => \{\{-2, -3\}, \{-2, -4\}, \{-2, -5\}\}]-modul $\cdot\cdot\cdot$\\
\           1   2   3                                                   $\cdot\cdot\cdot$\\
\endOutput
\end{sExample}

\subsection{Ext-generator Series}

The Ext-generator series $\gen^{M,N}_B(t,u)$ defined in Remark
\ref{reduced ext} generalizes both the Poincar\'e series of $M$
and the Bass series of $N$, as seen from the formulas
\[
\Poi^B_M(t,u)=\gen^{M,k}_B(t,u)
\qquad\text{and}\qquad
\Ba^N_B(t,u)=\gen^{k,N}_B(t,u^{-1})
\]
Similar equalities hold for the corresponding series in one variable.
Codes for computing Ext-generator series are easy to produce.

\begin{sCode}
\label{genseries}
The function {\tt extgenSeries2(M,N)} computes $\gen^{M,N}_B(t,u)$ when
$M$ and $N$ are graded modules over a graded complete intersection $B$,
and presents it as a rational function with denominator
$(1-t^2u^{r_1})\cdots(1-t^2u^{r_c})$.
\beginOutput
i143 : extgenSeries2 = (M,N) -> (\\
\          H := ext(M,N);\\
\          h := hilbertSeries H;\\
\          T':= degreesRing H;\\
\          substitute(h, \{T'_0=>t^-1,T'_1=>u^-1\})\\
\          );\\
\endOutput
\end{sCode}

\begin{sCode}
The function {\tt extgenSeries1(M,N)} computes the Ext-genera\-tor series in
one variable for a pair $(M,N)$ of graded modules over a graded complete
intersection $B$.
\beginOutput
i144 : extgenSeries1 = (M,N) -> (\\
\          substitute(extgenSeries2(M,N), \{u=>1_T\})\\
\          );\\
\endOutput
\end{sCode}

\begin{sExample}
For $M$, $N$, and $N'$ as in Example \ref{new module} we obtain
\beginOutput
i145 : time extgenSeries2(M,N)\\
\     -- used 0.44 seconds\\
\emptyLine
\         -2    -1            2      2 2     2 3     2 4     3 4     3  $\cdot\cdot\cdot$\\
\       8u   + u   + 8t*u - 8t u - 9t u  - 9t u  + 7t u  - 8t u  - 8t u $\cdot\cdot\cdot$\\
o145 = --------------------------------------------------------------- $\cdot\cdot\cdot$\\
\                                                                       $\cdot\cdot\cdot$\\
\                                                                       $\cdot\cdot\cdot$\\
\emptyLine
o145 : Divide\\
\endOutput
\beginOutput
i146 : g=time extgenSeries1(M,N)\\
\     -- used 0.13 seconds\\
\emptyLine
\                   2      3      4     5     6    7\\
\       9 + 8t - 19t  - 11t  + 17t  + 4t  - 7t  - t\\
o146 = --------------------------------------------\\
\                       2       2       2\\
\                 (1 - t )(1 - t )(1 - t )\\
\emptyLine
o146 : Divide\\
\endOutput
\beginOutput
i147 : simplify g\\
\emptyLine
\                 2     3    4\\
\       9 - t - 9t  + 6t  + t\\
o147 = ----------------------\\
\                         2\\
\           (1 + t)(1 - t)\\
\emptyLine
o147 : Divide\\
\endOutput
\beginOutput
i148 : time extgenSeries2(M,N')\\
\     -- used 0.15 seconds\\
\emptyLine
\         -2     -1       2     2      2 2     2 3      2 4     3 5     $\cdot\cdot\cdot$\\
\       7u   + 2u   + 4t*u  - 7t u - 9t u  - 9t u  + 16t u  - 4t u  + 2 $\cdot\cdot\cdot$\\
o148 = --------------------------------------------------------------- $\cdot\cdot\cdot$\\
\                                                                       $\cdot\cdot\cdot$\\
\                                                                       $\cdot\cdot\cdot$\\
\emptyLine
o148 : Divide\\
\endOutput
\beginOutput
i149 : g'=time extgenSeries1(M,N')\\
\     -- used 0.18 seconds\\
\emptyLine
\                  2     3     4     5     6\\
\       9 + 4t - 9t  + 4t  + 8t  - 2t  - 2t\\
o149 = ------------------------------------\\
\                   2       2       2\\
\             (1 - t )(1 - t )(1 - t )\\
\emptyLine
o149 : Divide\\
\endOutput
\beginOutput
i150 : simplify g'\\
\emptyLine
\                  2     3     5\\
\       9 - 5t - 4t  + 8t  - 2t\\
o150 = ------------------------\\
\                  2       3\\
\           (1 + t) (1 - t)\\
\emptyLine
o150 : Divide\\
\endOutput
\end{sExample}

\subsection{Complexity}

The {\it\ie{complexity}\/} of a pair of $B$-modules $(M,N)$
is the least $d\in\N$ such that there exists
a polynomial of degree $d-1$ bounding above the function
\[
n \mapsto \dim_K \rExt^n_B(M,N)
\]
It is denoted $\cx_B(M,N)$ and measures on a polynomial scale the rate
of growth of the minimal number of generators of $\Ext^n_B(M,N)$; it
vanishes if and only if $\Ext^n_B(M,N)=0$ for all $n\gg0$.  Corollary
\ref{series} yields
\[
\gen^{M,N}_B(t)=\frac{h(t)}{(1-t^2)^c}
\qquad\text{for some}\qquad h(t)\in\Z[t]
\]
so decomposition into partial fractions shows that $\cx_R(M,N)$ equals
the order of the pole of $\gen^{M,N}_B(t)$ at $t=1$.  Alternatively,
$\cx_R(M,N)$ can be obtained by computing the Krull dimension of a
reduced Ext module over $R$.

\begin{sCode}
The function {\tt complexityPair(M,N)} yields the complexity of a pair
$(M,N)$ of graded modules over a graded complete intersection ring $B$.
\beginOutput
i151 : complexityPair = (M,N) -> dim ext(M,N);\\
\endOutput
 \end{sCode}

\begin{sExample}
For $M$, $N$, and $N'$ as in Example \ref{new module} we have
\beginOutput
i152 : time complexityPair(M,N)\\
\     -- used 0.39 seconds\\
\emptyLine
o152 = 2\\
\endOutput
\beginOutput
i153 : time complexityPair(M,N')\\
\     -- used 0.12 seconds\\
\emptyLine
o153 = 3\\
\endOutput
\end{sExample}

\subsection{Support Variety}
\label{Support variety2}

Let $\overline K$ be an algebraic closure of $K$.  The {\it\ie{support
variety}\/} $\var^*_B(M,N)$ is the algebraic set in ${\overline K}{}^c$
defined by the annihilator of $\rExt^\bu_B(M,N)$ over
$R=K[X_1,\dots,X_c]$.  It is clear from the definition that
$\var^*_B(M,k)$ is equal to the variety $\var^*_B(M)$ defined in
\ref{Support variety}.  One of the main results of
\cite[Sect.~5]{CI:AB2} shows that $\var^*_B(M,N)=\var^*_B(M)\cap
\var^*_B(N)$, and, as a consequence, $\var^*_B(M)=\var^*_B(M,M)=
\var^*_B(k,M)$.  The dimension of $\var^*_B(M,N)$ is equal to the
complexity $\cx_R(M,N)$, already computed above.

Feeding our computation of $\rExt^\bu_B(M,N)$ to standard \Mtwo
routines we write code for determining a set of equations defining
$\var^*_B(M,N)$.

\begin{sCode}
The function {\tt supportVarietyPairIdeal(M,N)} yields a set of polynomial
equations with coefficients in $K$, defining the variety
$\var^*_B(M,N)$ in ${\overline K}{}^c$ for graded modules $M$, $N$ over
a graded complete intersection $B$.
\beginOutput
i154 : supportVarietyPairIdeal = (M,N) -> ann ext(M,N);\\
\endOutput
 \end{sCode}

\begin{sExample}
For $M$, $N$, and $N'$ as in Example \ref{new module} we have
\beginOutput
i155 : time supportVarietyPairIdeal(M,N)\\
\     -- used 0.97 seconds\\
\emptyLine
o155 = ideal X\\
\              1\\
\emptyLine
o155 : Ideal of K [X , X , X , Degrees => \{\{-2, -3\}, \{-2, -4\}, \{-2, -5\}\}]\\
\                    1   2   3\\
\endOutput
\beginOutput
i156 : time supportVarietyPairIdeal(M,N')\\
\     -- used 1.73 seconds\\
\emptyLine
o156 = 0\\
\emptyLine
o156 : Ideal of K [X , X , X , Degrees => \{\{-2, -3\}, \{-2, -4\}, \{-2, -5\}\}]\\
\                    1   2   3\\
\endOutput
\end{sExample}

\appendix

\section{Gradings}
\label{Gradings}

Our purpose here is to set up a context in which the theory of Sections
\ref{Universal homotopies} and \ref{Cohomology operators} translates
into data that \Mtwo can operate with.

A first point is to develop a {\it flexible\/} and {\it consistent\/}
scheme within which to handle the two kinds of degrees we deal
with---the internal gradings of the input, and the homological degrees
created during computations.

A purely technological difficulty arises when our data are presented to
\Mtwo: it only accepts multidegrees whose first component is positive,
which is {\it not\/} the case for rings of cohomology operators.

A final point, mostly notational, tends to generate misunderstanding
and errors if left unaddressed.  On the printed page, the difference
between homological and cohomological conventions is handled
graphically by switching between sub- and super-indices, and reversing
signs; both authors were used to it, but \Mtwo has so far refused to
read \TeX\ printouts.

The {\sl raison d'\^etre\/} of the following remarks was to debug
communications between the three of us.

\begin{Remark}
\label{bigrading}
Only one degree, denoted $\deg$, appears in Section \ref{Graded
algebras}, and anywhere in the main text before Notation \ref{graded
stuff}; when needed, it will be referred to as {\it homological
degree\/}.

Assume that $A=\bigoplus_{h\in\Z}A_h$ is a graded ring.  Any element
$a$ of $A_h$ is said to be homogeneous of {\it\ie{internal degree}\/}
$h$; the notation for this is $\deg' a=h$.  Let ${\boldsymbol f}=\{f_1,
\dots, f_c\}$ be a Koszul-regular set consisting of homogeneous
elements.  We give the ring $B=A/({\boldsymbol f})$ the induced
grading, and extend the notation for internal degree to all graded
$B$-modules $M$.

Let $M$ be a graded $B$-module.  For any integer $e$, we let $M[e]$
denote the graded module with $M[e]_d = M_{d+e}$.  We take a projective
resolution $C$ of $M$ by graded $A$-modules, with differential $d_C$
preserving internal degrees.  Recall that we have been writing $\deg x
= n$ to indicate that $x$ is an element in $C_n$; we refer to this
situation also by saying that $x$ has {\it\ie{homological degree\/}}
$x$.  We combine both degrees in a single {\it\ie{bidegree\/}}, denoted
$\Deg$, as follows:
\[
\Deg x = (\deg x, \deg' x)
\]
For a bigraded module $H$ and pair of integers $(e, e')$, we let
$H[e,e']$ denote the bigraded module with $H[e,e']_{d,d'} =
H_{d+e,d'+e'}$.

Because $\deg Y_i = 2$, the elements of the free $B$-module $Q$ have
homological degree $2$.  We introduce an internal grading $\deg'$ on
$Q$ by setting $\deg' Y_i = r_i$, where $r_i=\deg' f_i$ for
$i=1,\dots,c$.  With this choice, the homomorphism $f\colon Q\to A$
acquires internal degree $0$ (of course, this was the reason behind our
choice of grading in the first place).  The internal grading on $Q$
defines, in the usual way, internal gradings on all symmetric and
exterior powers of $Q$ and $Q^*$; in particular, $\deg'Y^{(\a)}=\sum
\a_i r_i$ and $\deg' Y^{\wedge\b} = \sum \b_i r_i$.  Thus, the ring
$S=A[X_1,\dots,X_c]$ acquires a bigrading defined by
$\Deg a=(0, h)$ for all elements $a\in A_h$ and $\Deg X_i=(-2,-r_i)$
for $i=1,\dots,c$.

In this context, we call $S$ the {\it\ie{bigraded ring of cohomology
operators}\/}.

Since the differential $d_C$ has internal degree $0$, a null-homotopic
chain map $C\to C$ which is homogeneous of internal degree $r$ will
have a null-homotopy that is itself homogeneous of internal degree
$r$.  In the proof of Theorem \ref{main} we construct maps $d_\g$ as
null-homotopies, so we may arrange for them to be homogeneous maps with
$\deg' d_\g = \sum \g_i d_i$.  Our grading assumptions guarantee that
$d$ is homogeneous with $\Deg d=(-1,0)$.

With these data, the $B$-free resolution $C\otimes_A D'$ provided by
Theorem \ref{resolution} becomes one by graded $B$-modules, and
its differential $\partial$ is homogeneous with $\Deg
\partial=(-1,0)$.  For any graded $B$-module $N$, these properties are
transferred to the complex $\Hom_B(C\otimes_A D',N)$ and its
differential.
 \end{Remark}

We sum up the contents of Remarks \ref{action} and \ref{bigrading}.

\begin{Remark}
\label{graded action}
If $A$ is a graded ring, $\{f_1, \dots, f_c\}$ is a Koszul-regular set
consisting of homogeneous elements, $B$ is the residue ring
$A/({\boldsymbol f})$, and $M,N$ are graded $B$-modules, then
$\Ext^\bu_B(M,N)$ is a bigraded module over the ring
$S=A[X_1,\dots,X_c]$, itself bigraded by setting $\Deg a = (0,
\deg'(a))$ for all homogeneous $a\in A$ and $\Deg X_i =
(-2,-\deg'(f_i))$ for $i=1,\dots,c$.
 \end{Remark}

\begin{Remark}
\label{macaulay grading}
The core algorithms of the program can handle multi-graded rings and
modules, but only if each variable in the ring has positive first
component of its multi-degree.  At the moment, a user who needs a
multi-graded ring {\tt R} which violates this requirement must provide
two linear maps: {\tt R.Adjust}, that transforms the desired
multi-degrees into ones satisfying this requirement, as well as its
inverse map, {\tt R.Repair}.  The routine {\tt Ext}, discussed above,
incorporates such adjustments for the rings of cohomology operators it
creates.  When we wish to create related rings with some of the same
multi-degrees, we may use the same adjustment operator.
 \end{Remark}

% Local Variables:
% mode: latex
% mode: reftex
% compile-command: "make ci-wrapper.dvi"
% tex-main-file: "ci-wrapper.tex"
% reftex-keep-temporary-buffers: t
% reftex-use-external-file-finders: t
% reftex-external-file-finders: (("tex" . "make FILE=%f find-tex") ("bib" . "make FILE=%f find-bib"))
% End:
