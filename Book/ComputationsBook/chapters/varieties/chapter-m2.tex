\title{Ideals, Varieties and \Mtwo}
\titlerunning{Ideals, Varieties and \Mtwo}
\toctitle{Ideals, Varieties and \Mtwo}
\author{Bernd Sturmfels\thanks{Partially supported by
the National Science Foundation (DMS-9970254).}}
\authorrunning{B. Sturmfels}
% \institute{University of California at Berkeley, Department of Mathematics,
%   Berkeley, CA 94720, USA}
\maketitle

\begin{abstract}
This chapter introduces \Mtwo commands for
some elementary computations in algebraic geometry.
Familiarity with Gr\"obner bases is assumed.
\end{abstract}

Many students and researchers alike have their first encounter with
Gr\"ob\-ner bases through the delightful text books \cite{CLO1} and \cite{CLO2}
by David Cox, John Little and Donal O'Shea. This chapter illustrates
the use of \Mtwo for some computations discussed in these books.
It can be used as a supplement for an advanced undergraduate course or 
first-year graduate course in computational algebraic geometry. The
mathematically advanced reader will find this chapter a useful summary
of some basic \Mtwo commands.

\section{A Curve in Affine Three-Space}

Our first example concerns geometric objects in
(complex) affine 3-space. We start by
setting up the ring of polynomial functions with rational coefficients.
\beginOutput
i1 : R = QQ[x,y,z]\\
\emptyLine
o1 = R\\
\emptyLine
o1 : PolynomialRing\\
\endOutput
Various monomial orderings are available in \Mtwo; since we did not specify
one explicitly, the monomials in the ring ${\tt R}$ will be sorted in 
graded reverse lexicographic order  \cite[\S I.2, Definition 6]{CLO1}.
We define an ideal generated by two polynomials
in this ring and assign it to the variable named 
{\tt curve}.

\beginOutput
i2 : curve = ideal( x^4-y^5, x^3-y^7 )\\
\emptyLine
\               5    4     7    3\\
o2 = ideal (- y  + x , - y  + x )\\
\emptyLine
o2 : Ideal of R\\
\endOutput
We compute the reduced Gr\"obner basis of our ideal:
\beginOutput
i3 : gb curve\\
\emptyLine
o3 = | y5-x4 x4y2-x3 x8-x3y3 |\\
\emptyLine
o3 : GroebnerBasis\\
\endOutput
By inspecting leading terms (and using \cite[\S 9.3, Theorem 8]{CLO1}),
we see that our ideal {\tt curve} does indeed 
define a one-dimensional affine variety. This can be tested directly
with the following commands in \Mtwo:
\beginOutput
i4 : dim curve\\
\emptyLine
o4 = 1\\
\endOutput
\beginOutput
i5 : codim curve\\
\emptyLine
o5 = 2\\
\endOutput
The {\it degree} of a curve in complex affine $3$-space is the 
number of intersection points with a general plane. It coincides
with the degree  \cite[\S 6.4]{CLO2} of the projective closure
\cite[\S 8.4]{CLO1} of our curve, which we compute as follows:
\beginOutput
i6 : degree curve\\
\emptyLine
o6 = 28\\
\endOutput
The Gr\"obner basis in {\tt o3} contains two polynomials which are not
irreducible: they contain a factor of $x^3$. This shows that our curve
is not irreducible over ${\bf Q}$. We first extract the components
which are transverse to the plane $x=0$:
\beginOutput
i7 : curve1 = saturate(curve,ideal(x))\\
\emptyLine
\               2       5    4   5    3\\
o7 = ideal (x*y  - 1, y  - x , x  - y )\\
\emptyLine
o7 : Ideal of R\\
\endOutput
And next we extract the component which lies in the plane $x=0$:
\beginOutput
i8 : curve2 = saturate(curve,curve1)\\
\emptyLine
\             3   5\\
o8 = ideal (x , y )\\
\emptyLine
o8 : Ideal of R\\
\endOutput
The second component is a multiple line. Hence our input ideal was not radical.
To test equality of ideals we use the command {\tt ==}\indexcmd{==} .
\beginOutput
i9 : curve == radical curve\\
\emptyLine
o9 = false\\
\endOutput
We now replace our curve by its first component:
\beginOutput
i10 : curve = curve1\\
\emptyLine
\                2       5    4   5    3\\
o10 = ideal (x*y  - 1, y  - x , x  - y )\\
\emptyLine
o10 : Ideal of R\\
\endOutput
\beginOutput
i11 : degree curve\\
\emptyLine
o11 = 13\\
\endOutput
The ideal of this curve is radical:
\beginOutput
i12 : curve == radical curve\\
\emptyLine
o12 = true\\
\endOutput
Notice that the variable ${\bf z}$ does not appear
among the generators of the ideal. Our curve consists of
$13$ straight lines (over {\bf C}) parallel to the {\tt z}-axis.

\section{Intersecting Our Curve With a Surface}

In this section we explore basic operations on ideals,
starting with those described in \cite[\S 4.3]{CLO1}.
Consider the following surface in affine $3$-space:
\beginOutput
i13 : surface = ideal( x^5 + y^5 + z^5 - 1)\\
\emptyLine
\             5    5    5\\
o13 = ideal(x  + y  + z  - 1)\\
\emptyLine
o13 : Ideal of R\\
\endOutput
The union of the curve and the surface is represented by the 
intersection of their ideals:
\beginOutput
i14 : theirunion = intersect(curve,surface)\\
\emptyLine
\              6 2      7      2 5    5    5    5      2       5 5    1 $\cdot\cdot\cdot$\\
o14 = ideal (x y  + x*y  + x*y z  - x  - y  - z  - x*y  + 1, x y  + y  $\cdot\cdot\cdot$\\
\emptyLine
o14 : Ideal of R\\
\endOutput
In this example this coincides with the product of the two ideals:
\beginOutput
i15 : curve*surface == theirunion\\
\emptyLine
o15 = true\\
\endOutput
The intersection of the curve and the surface is represented by the 
sum of their ideals. We get a finite set of points:
\beginOutput
i16 : ourpoints = curve + surface\\
\emptyLine
\                2       5    4   5    3   5    5    5\\
o16 = ideal (x*y  - 1, y  - x , x  - y , x  + y  + z  - 1)\\
\emptyLine
o16 : Ideal of R\\
\endOutput
\beginOutput
i17 : dim ourpoints\\
\emptyLine
o17 = 0\\
\endOutput
The number of points is sixty five:
\beginOutput
i18 : degree ourpoints\\
\emptyLine
o18 = 65\\
\endOutput
Each of the points is multiplicity-free:
\beginOutput
i19 : degree radical ourpoints\\
\emptyLine
o19 = 65\\
\endOutput
The number of points coincides with the number of 
monomials not in the initial ideal \cite[\S 2.2]{CLO2}.
These are called the {\it standard monomials}.
\beginOutput
i20 : staircase = ideal leadTerm ourpoints\\
\emptyLine
\                2   5   5   5\\
o20 = ideal (x*y , z , y , x )\\
\emptyLine
o20 : Ideal of R\\
\endOutput
The {\tt basis} command can be used to list all the standard monomials
\beginOutput
i21 : T = R/staircase;\\
\endOutput
\beginOutput
i22 : basis T\\
\emptyLine
o22 = | 1 x x2 x3 x4 x4y x4yz x4yz2 x4yz3 x4yz4 x4z x4z2 x4z3 x4z4 x3y $\cdot\cdot\cdot$\\
\emptyLine
\              1       65\\
o22 : Matrix T  <--- T\\
\endOutput

The assignment of the quotient ring to the global variable {\tt T} had a side
effect: the variables {\tt x}, {\tt y}, and {\tt z} now have values in that
ring.
To bring the variables of {\tt R} to the fore again, we must say:
\beginOutput
i23 : use R;\\
\endOutput
Every polynomial function on our 65 points can be written uniquely
as a linear combination of these standard monomials. This 
representation can be computed using the normal form command {\tt \%}\indexcmd{\%}.

\beginOutput
i24 : anyOldPolynomial = y^5*x^5-x^9-y^8+y^3*x^5\\
\emptyLine
\       5 5    9    5 3    8\\
o24 = x y  - x  + x y  - y\\
\emptyLine
o24 : R\\
\endOutput
\beginOutput
i25 : anyOldPolynomial {\char`\%} ourpoints\\
\emptyLine
\       4     3\\
o25 = x y - x y\\
\emptyLine
o25 : R\\
\endOutput
Clearly, the normal form is zero if and only the polynomial is in the ideal.
\beginOutput
i26 : anotherPolynomial = y^5*x^5-x^9-y^8+y^3*x^4\\
\emptyLine
\       5 5    9    8    4 3\\
o26 = x y  - x  - y  + x y\\
\emptyLine
o26 : R\\
\endOutput
\beginOutput
i27 : anotherPolynomial {\char`\%} ourpoints\\
\emptyLine
o27 = 0\\
\emptyLine
o27 : R\\
\endOutput


\section{Changing the Ambient Polynomial Ring}

During a \Mtwo session it sometimes becomes necessary to change the
ambient ring in which the computations takes place. Our original
ring, defined in {\tt i1}, is the polynomial ring in three variables
over the field  {\bf Q} of rational numbers
with the graded reverse lexicographic order. In this section 
two modifications are made: first we replace the field of coefficients
by a finite field, and later we replace the  monomial order
by an elimination order.

An important operation in algebraic geometry is 
the decomposition of algebraic varieties
into irreducible components \cite[\S 4.6]{CLO1}.
Algebraic algorithms for this purpose are based on the
{\it primary decomposition} of ideals \cite[\S 4.7]{CLO1}.
A future version of \Mtwo will have an implementation of
primary decomposition over any polynomial ring.
The current version of \Mtwo has a command
{\tt decompose} for finding all the minimal primes of an ideal,
but, as it stands, this works only over a finite field.

Let us change our coefficient field to the field with $101$ elements:
\beginOutput
i28 : R' = ZZ/101[x,y,z];\\
\endOutput

We next move our ideal from the previous section into the new ring
(fortunately, none of the coefficients of its generators have 101 in the
denominator):

\beginOutput
i29 : ourpoints' = substitute(ourpoints,R')\\
\emptyLine
\                2       5    4   5    3   5    5    5\\
o29 = ideal (x*y  - 1, y  - x , x  - y , x  + y  + z  - 1)\\
\emptyLine
o29 : Ideal of R'\\
\endOutput
\beginOutput
i30 : decompose ourpoints'\\
\emptyLine
\                                                                       $\cdot\cdot\cdot$\\
o30 = \{ideal (z + 36, y - 1, x - 1), ideal (z + 1, y - 1, x - 1), idea $\cdot\cdot\cdot$\\
\emptyLine
o30 : List\\
\endOutput
Oops, that didn't fit on the display, so let's print them out one per line.
\beginOutput
i31 : oo / print @@ print;\\
ideal (z + 36, y - 1, x - 1)\\
\emptyLine
ideal (z + 1, y - 1, x - 1)\\
\emptyLine
ideal (z - 6, y - 1, x - 1)\\
\emptyLine
ideal (z - 14, y - 1, x - 1)\\
\emptyLine
ideal (z - 17, y - 1, x - 1)\\
\emptyLine
\        3      2              2                      3    2     2      $\cdot\cdot\cdot$\\
ideal (x  - 46x  + 28x*y - 27y  + 46x + y + 27, - 16x  + x y + x  - 15 $\cdot\cdot\cdot$\\
\emptyLine
\            2                                            2             $\cdot\cdot\cdot$\\
ideal (- 32x  - 16x*y + x*z - 16x - 27y - 30z - 14, - 34x  - 14x*y + y $\cdot\cdot\cdot$\\
\emptyLine
\          2                                         2            2     $\cdot\cdot\cdot$\\
ideal (44x  + 22x*y + x*z + 22x - 26y - 30z - 6, 18x  + 12x*y + y  + 1 $\cdot\cdot\cdot$\\
\emptyLine
\            2                                           2            2 $\cdot\cdot\cdot$\\
ideal (- 41x  + 30x*y + x*z + 30x + 38y - 30z + 1, - 26x  - 10x*y + y  $\cdot\cdot\cdot$\\
\emptyLine
\          2                                            2            2  $\cdot\cdot\cdot$\\
ideal (39x  - 31x*y + x*z - 31x - 46y - 30z + 36, - 32x  - 13x*y + y   $\cdot\cdot\cdot$\\
\emptyLine
\            2                                          2            2  $\cdot\cdot\cdot$\\
ideal (- 10x  - 5x*y + x*z - 5x - 40y - 30z - 17, - 37x  + 35x*y + y   $\cdot\cdot\cdot$\\
\emptyLine
\endOutput
If we just want to see the degrees of the irreducible components, then
we say:
\beginOutput
i32 : ooo / degree\\
\emptyLine
o32 = \{1, 1, 1, 1, 1, 30, 6, 6, 6, 6, 6\}\\
\emptyLine
o32 : List\\
\endOutput
Note that the expressions ${\tt oo}$ 
and ${\tt ooo}$ refer to the previous and
prior-to-previous output lines respectively.

\medskip

Suppose we wish to compute the $x$-coordinates of our sixty five points.
Then we must use an elimination order, for instance, the
one described in \cite[\S 3.2, Exercise 6.a]{CLO1}.
We define a  new polynomial ring with the elimination order
for $\{y,z\} > \{x\}$ as follows:
\beginOutput
i33 : S = QQ[z,y,x, MonomialOrder => Eliminate 2]\\
\emptyLine
o33 = S\\
\emptyLine
o33 : PolynomialRing\\
\endOutput
We move our ideal into the new ring,
\beginOutput
i34 : ourpoints'' = substitute(ourpoints,S)\\
\emptyLine
\              2        5    4     3    5   5    5    5\\
o34 = ideal (y x - 1, y  - x , - y  + x , z  + y  + x  - 1)\\
\emptyLine
o34 : Ideal of S\\
\endOutput
and we compute the reduced Gr\"obner basis in this new order:
\beginOutput
i35 : G = gens gb ourpoints''\\
\emptyLine
o35 = | x13-1 y-x6 z5+x5+x4-1 |\\
\emptyLine
\              1       3\\
o35 : Matrix S  <--- S\\
\endOutput
To compute the elimination ideal we use the following command:
\beginOutput
i36 : ideal selectInSubring(1,G)\\
\emptyLine
\             13\\
o36 = ideal(x   - 1)\\
\emptyLine
o36 : Ideal of S\\
\endOutput

\section{Monomials Under the Staircase}

Invariants of an algebraic variety, such as its dimension
and degree, are computed from an initial monomial ideal.
This computation amounts to the combinatorial task
of analyzing the collection of standard monomials,
that is, the monomials under the staircase \cite[Chapter 9]{CLO1}.
In this section we demonstrate some basic operations on
monomial ideals in \Mtwo.

Let us create a non-trivial staircase in three dimensions
by taking the third power of the initial monomial from line {\tt i20}.
\beginOutput
i37 : M = staircase^3\\
\emptyLine
\              3 6   2 4 5   2 9   7 4     2 10     7 5   6 2 5     12  $\cdot\cdot\cdot$\\
o37 = ideal (x y , x y z , x y , x y , x*y z  , x*y z , x y z , x*y  , $\cdot\cdot\cdot$\\
\emptyLine
o37 : Ideal of R\\
\endOutput
The number of current generators of this ideal equals
\beginOutput
i38 : numgens M\\
\emptyLine
o38 = 20\\
\endOutput
To see all generators we can transpose the matrix of minimal generators:
\beginOutput
i39 : transpose gens M\\
\emptyLine
o39 = \{-9\}  | x3y6   |\\
\      \{-11\} | x2y4z5 |\\
\      \{-11\} | x2y9   |\\
\      \{-11\} | x7y4   |\\
\      \{-13\} | xy2z10 |\\
\      \{-13\} | xy7z5  |\\
\      \{-13\} | x6y2z5 |\\
\      \{-13\} | xy12   |\\
\      \{-13\} | x6y7   |\\
\      \{-13\} | x11y2  |\\
\      \{-15\} | z15    |\\
\      \{-15\} | y5z10  |\\
\      \{-15\} | x5z10  |\\
\      \{-15\} | y10z5  |\\
\      \{-15\} | x5y5z5 |\\
\      \{-15\} | x10z5  |\\
\      \{-15\} | y15    |\\
\      \{-15\} | x5y10  |\\
\      \{-15\} | x10y5  |\\
\      \{-15\} | x15    |\\
\emptyLine
\              20       1\\
o39 : Matrix R   <--- R\\
\endOutput
Note that this generating set is not minimal; see {\tt o48} below.
The number of standard monomials equals
\beginOutput
i40 : degree M\\
\emptyLine
o40 = 690\\
\endOutput
To list all the standard monomials we first create the residue ring
\beginOutput
i41 : S = R/M\\
\emptyLine
o41 = S\\
\emptyLine
o41 : QuotientRing\\
\endOutput
and then we ask for a vector space basis of the residue ring:
\beginOutput
i42 : basis S\\
\emptyLine
o42 = | 1 x x2 x3 x4 x5 x6 x7 x8 x9 x10 x11 x12 x13 x14 x14y x14yz x14 $\cdot\cdot\cdot$\\
\emptyLine
\              1       690\\
o42 : Matrix S  <--- S\\
\endOutput
Let us count how many standard monomials there are of a given degree.
The following table represents the Hilbert function
of the residue ring.
\beginOutput
i43 : tally apply(flatten entries basis(S),degree)\\
\emptyLine
o43 = Tally\{\{0\} => 1  \}\\
\            \{1\} => 3\\
\            \{10\} => 63\\
\            \{11\} => 69\\
\            \{12\} => 73\\
\            \{13\} => 71\\
\            \{14\} => 66\\
\            \{15\} => 53\\
\            \{16\} => 38\\
\            \{17\} => 23\\
\            \{18\} => 12\\
\            \{19\} => 3\\
\            \{2\} => 6\\
\            \{3\} => 10\\
\            \{4\} => 15\\
\            \{5\} => 21\\
\            \{6\} => 28\\
\            \{7\} => 36\\
\            \{8\} => 45\\
\            \{9\} => 54\\
\emptyLine
o43 : Tally\\
\endOutput
Thus the largest degree of a standard monomial is nineteen,
and there are three standard monomials of that degree:
\beginOutput
i44 : basis(19,S)\\
\emptyLine
o44 = | x14yz4 x9yz9 x4yz14 |\\
\emptyLine
\              1       3\\
o44 : Matrix S  <--- S\\
\endOutput
The most recently defined ring involving {\tt x}, {\tt y}, and {\tt z} was
{\tt S}, so all computations involving those variables are done in the
residue ring {\tt S}.
%% The current ring {\tt S} is the residue ring. All calculations
%% are done in {\tt S}. 
For instance, we can also obtain the
standard monomials of  degree nineteen as follows:
\beginOutput
i45 : (x+y+z)^19\\
\emptyLine
\            14   4          9   9         4   14\\
o45 = 58140x  y*z  + 923780x y*z  + 58140x y*z\\
\emptyLine
o45 : S\\
\endOutput
An operation on ideals which will occur frequently throughout this
book is the computation of minimal free resolutions. This is done as follows:
\beginOutput
i46 : C = res M\\
\emptyLine
\       1      16      27      12\\
o46 = R  <-- R   <-- R   <-- R   <-- 0\\
\                                      \\
\      0      1       2       3       4\\
\emptyLine
o46 : ChainComplex\\
\endOutput
This shows that our ideal {\tt M} has sixteen minimal generators.
They are the entries in the leftmost matrix of the chain complex {\tt C}:
\beginOutput
i47 : C.dd_1\\
\emptyLine
o47 = | x3y6 x7y4 x2y9 x2y4z5 x11y2 xy12 x6y2z5 xy7z5 xy2z10 x15 y15 x $\cdot\cdot\cdot$\\
\emptyLine
\              1       16\\
o47 : Matrix R  <--- R\\
\endOutput
This means that four of the twenty generators in {\tt o39} were redundant.
We construct the set consisting of the four redundant generators
as follows:
\beginOutput
i48 : set flatten entries gens M - set flatten entries C.dd_1\\
\emptyLine
\            6 7   10 5   5 10   5 5 5\\
o48 = Set \{x y , x  y , x y  , x y z \}\\
\emptyLine
o48 : Set\\
\endOutput
Here {\tt flatten entries} turns the matrix ${\tt M}$ into a single list.
The command {\tt set} turns that list into a set, to which we
can apply the difference operation for sets.

Let us now take a look at the first syzygies 
(or {\it minimal S-pairs} \cite[\S 2.9]{CLO1})
among  the sixteen minimal generators.
They correspond to the columns of the second matrix in our resolution {\tt C}:
\beginOutput
i49 : C.dd_2\\
\emptyLine
o49 = \{9\}  | -y3 -x4 0   -z5 0   0   0   0   0   0   0   0   0   0   0 $\cdot\cdot\cdot$\\
\      \{11\} | 0   y2  0   0   0   -x4 0   0   -z5 0   0   0   0   0   0 $\cdot\cdot\cdot$\\
\      \{11\} | x   0   -y3 0   0   0   0   0   0   -z5 0   0   0   0   0 $\cdot\cdot\cdot$\\
\      \{11\} | 0   0   0   xy2 -y3 0   -x4 0   x5  y5  0   -z5 0   0   0 $\cdot\cdot\cdot$\\
\      \{13\} | 0   0   0   0   0   y2  0   0   0   0   0   0   0   -x4 0 $\cdot\cdot\cdot$\\
\      \{13\} | 0   0   x   0   0   0   0   -y3 0   0   0   0   0   0   0 $\cdot\cdot\cdot$\\
\      \{13\} | 0   0   0   0   0   0   y2  0   0   0   0   0   0   0   - $\cdot\cdot\cdot$\\
\      \{13\} | 0   0   0   0   x   0   0   0   0   0   -y3 0   0   0   0 $\cdot\cdot\cdot$\\
\      \{13\} | 0   0   0   0   0   0   0   0   0   0   0   xy2 -y3 0   0 $\cdot\cdot\cdot$\\
\      \{15\} | 0   0   0   0   0   0   0   0   0   0   0   0   0   y2  0 $\cdot\cdot\cdot$\\
\      \{15\} | 0   0   0   0   0   0   0   x   0   0   0   0   0   0   0 $\cdot\cdot\cdot$\\
\      \{15\} | 0   0   0   0   0   0   0   0   0   0   0   0   0   0   y $\cdot\cdot\cdot$\\
\      \{15\} | 0   0   0   0   0   0   0   0   0   0   x   0   0   0   0 $\cdot\cdot\cdot$\\
\      \{15\} | 0   0   0   0   0   0   0   0   0   0   0   0   0   0   0 $\cdot\cdot\cdot$\\
\      \{15\} | 0   0   0   0   0   0   0   0   0   0   0   0   x   0   0 $\cdot\cdot\cdot$\\
\      \{15\} | 0   0   0   0   0   0   0   0   0   0   0   0   0   0   0 $\cdot\cdot\cdot$\\
\emptyLine
\              16       27\\
o49 : Matrix R   <--- R\\
\endOutput
The first column represents the S-pair between the
first generator $x^3 y^6 $ and the third generator $x^2 y^9$.
It is natural to form the {\it S-pair graph} with $16$ vertices and
$27$ edges represented by  this matrix. According to the
general theory described in \cite{MS}, this is a planar graph
with $12$ regions. The regions correspond to the $12$ second syzygies,
that is, to the columns of the matrix
\beginOutput
i50 : C.dd_3\\
\emptyLine
o50 = \{12\} | z5  0   0   0   0   0   0   0   0   0   0   0   |\\
\      \{13\} | 0   z5  0   0   0   0   0   0   0   0   0   0   |\\
\      \{14\} | 0   0   z5  0   0   0   0   0   0   0   0   0   |\\
\      \{14\} | -y3 -x4 0   0   0   0   0   0   0   0   0   0   |\\
\      \{14\} | 0   0   -y5 z5  0   0   0   0   0   0   0   0   |\\
\      \{15\} | 0   0   0   0   z5  0   0   0   0   0   0   0   |\\
\      \{15\} | 0   0   0   0   -x5 z5  0   0   0   0   0   0   |\\
\      \{16\} | 0   0   0   0   0   0   z5  0   0   0   0   0   |\\
\      \{16\} | 0   y2  0   0   -x4 0   0   0   0   0   0   0   |\\
\      \{16\} | x   0   -y3 0   0   0   0   0   0   0   0   0   |\\
\      \{16\} | 0   0   0   0   0   0   -y5 z5  0   0   0   0   |\\
\      \{16\} | 0   0   0   -y3 0   -x4 0   0   0   0   0   0   |\\
\      \{16\} | 0   0   0   0   0   0   0   -y5 z5  0   0   0   |\\
\      \{17\} | 0   0   0   0   0   0   0   0   0   z5  0   0   |\\
\      \{17\} | 0   0   0   0   0   0   0   0   0   -x5 z5  0   |\\
\      \{17\} | 0   0   0   0   0   0   0   0   0   0   -x5 z5  |\\
\      \{18\} | 0   0   0   0   y2  0   0   0   0   -x4 0   0   |\\
\      \{18\} | 0   0   x   0   0   0   -y3 0   0   0   0   0   |\\
\      \{18\} | 0   0   0   0   0   y2  0   0   0   0   -x4 0   |\\
\      \{18\} | 0   0   0   x   0   0   0   -y3 0   0   0   0   |\\
\      \{18\} | 0   0   0   0   0   0   0   0   -y3 0   0   -x4 |\\
\      \{20\} | 0   0   0   0   0   0   0   0   0   y2  0   0   |\\
\      \{20\} | 0   0   0   0   0   0   x   0   0   0   0   0   |\\
\      \{20\} | 0   0   0   0   0   0   0   0   0   0   y2  0   |\\
\      \{20\} | 0   0   0   0   0   0   0   x   0   0   0   0   |\\
\      \{20\} | 0   0   0   0   0   0   0   0   0   0   0   y2  |\\
\      \{20\} | 0   0   0   0   0   0   0   0   x   0   0   0   |\\
\emptyLine
\              27       12\\
o50 : Matrix R   <--- R\\
\endOutput
But we are getting ahead of ourselves. Homological algebra and resolutions
will be covered in the next chapter, and monomial ideals
will appear in the chapter of Ho\c{s}ten and Smith.
Let us return to Cox, Little and O'Shea \cite{CLO2}.

\section{Pennies, Nickels, Dimes and Quarters}

We now come to an application of Gr\"obner bases which appears in
\cite[Section 8.1]{CLO2}: {\sl Integer Programming}. This is the problem of
 minimizing a linear objective function over the set of non-negative 
integer solutions of a system of linear equations.  We demonstrate
some techniques for doing this in \Mtwo. Along the way, we learn about
multigraded polynomial rings and how to compute
Gr\"obner bases with respect to monomial orders defined by weights.
Our running example is the linear system defined  by the matrix:
\beginOutput
i51 : A = \{\{1, 1, 1, 1\},\\
\           \{1, 5,10,25\}\}\\
\emptyLine
o51 = \{\{1, 1, 1, 1\}, \{1, 5, 10, 25\}\}\\
\emptyLine
o51 : List\\
\endOutput
For the algebraic study of integer programming problems, a good starting
point is to work in a multigraded polynomial ring, here in four variables:
\beginOutput
i52 : R = QQ[p,n,d,q, Degrees => transpose A]\\
\emptyLine
o52 = R\\
\emptyLine
o52 : PolynomialRing\\
\endOutput
The degree of each variable is the corresponding column vector of the matrix
Each variable represents one of the four coins in the U.S. currency system:
\beginOutput
i53 : degree d\\
\emptyLine
o53 = \{1, 10\}\\
\emptyLine
o53 : List\\
\endOutput
\beginOutput
i54 : degree q\\
\emptyLine
o54 = \{1, 25\}\\
\emptyLine
o54 : List\\
\endOutput
Each monomial represents a collection of coins. For instance, suppose
you own four  pennies, eight nickels, ten dimes, and three quarters:
\beginOutput
i55 : degree(p^4*n^8*d^10*q^3)\\
\emptyLine
o55 = \{25, 219\}\\
\emptyLine
o55 : List\\
\endOutput
Then you have a total of 25 coins worth two dollars and nineteen cents.
There are nine other possible ways of having 25 coins of the same value:
\beginOutput
i56 : h = basis(\{25,219\}, R)\\
\emptyLine
o56 = | p14n2d2q7 p9n8d2q6 p9n5d6q5 p9n2d10q4 p4n14d2q5 p4n11d6q4 p4n8 $\cdot\cdot\cdot$\\
\emptyLine
\              1       9\\
o56 : Matrix R  <--- R\\
\endOutput
For just counting the number of columns of this matrix
we can use the command
\beginOutput
i57 : rank source h\\
\emptyLine
o57 = 9\\
\endOutput
How many ways can you make change for ten dollars using $100$ coins?
\beginOutput
i58 : rank source basis(\{100,1000\}, R)\\
\emptyLine
o58 = 182\\
\endOutput
A typical integer programming problem is this: among all 182 ways of
expressing ten dollars using 100 coins, which one uses the fewest dimes?
We set up the  Conti-Traverso algorithm \cite[\S 8.1]{CLO2} for 
answering this question. We use the following ring with the lexicographic
order and with the variable order:
dimes (d) before pennies (p) before nickels (n) before quarters (q).
\beginOutput
i59 : S = QQ[x, y, d, p, n, q, \\
\          MonomialOrder => Lex, MonomialSize => 16]\\
\emptyLine
o59 = S\\
\emptyLine
o59 : PolynomialRing\\
\endOutput
The option {\tt MonomialSize} advises \Mtwo to use more space to store the
exponents of monomials, thereby avoiding a potential overflow.

We define an ideal with one generator for each column of the matrix A.
\beginOutput
i60 : I = ideal( p - x*y, n - x*y^5, d - x*y^10, q - x*y^25)\\
\emptyLine
\                             5           10           25\\
o60 = ideal (- x*y + p, - x*y  + n, - x*y   + d, - x*y   + q)\\
\emptyLine
o60 : Ideal of S\\
\endOutput
The integer program is solved by normal form reduction with respect
to the following Gr\"obner basis consisting of binomials.
\beginOutput
i61 : transpose gens gb I\\
\emptyLine
o61 = \{-6\}  | p5q-n6     |\\
\      \{-4\}  | d4-n3q     |\\
\      \{-3\}  | yn2-dp     |\\
\      \{-6\}  | yp4q-dn4   |\\
\      \{-4\}  | yd3-pnq    |\\
\      \{-6\}  | y2p3q-d2n2 |\\
\      \{-5\}  | y2d2n-p2q  |\\
\      \{-7\}  | y2d2p3-n5  |\\
\      \{-6\}  | y3p2q-d3   |\\
\      \{-6\}  | y3dp2-n3   |\\
\      \{-5\}  | y4p-n      |\\
\      \{-6\}  | y5n-d      |\\
\      \{-8\}  | y6d2-pq    |\\
\      \{-16\} | y15d-q     |\\
\      \{-7\}  | xq-y5d2    |\\
\      \{-5\}  | xn-y3p2    |\\
\      \{-2\}  | xd-n2      |\\
\      \{-2\}  | xy-p       |\\
\emptyLine
\              18       1\\
o61 : Matrix S   <--- S\\
\endOutput
We fix the quotient ring, so the reduction to normal form
will happen automatically.
\beginOutput
i62 : S' = S/I\\
\emptyLine
o62 = S'\\
\emptyLine
o62 : QuotientRing\\
\endOutput
You need at least two dimes to express one dollar with ten coins.
\beginOutput
i63 : x^10 * y^100\\
\emptyLine
\       2 6 2\\
o63 = d n q\\
\emptyLine
o63 : S'\\
\endOutput
But you can express ten dollars with a hundred coins none of which is a dime.
\beginOutput
i64 : x^100 * y^1000\\
\emptyLine
\       75 25\\
o64 = n  q\\
\emptyLine
o64 : S'\\
\endOutput
The integer program is infeasible if and only if the normal form still
contains the variable $x$ or the variable $y$. For instance, you cannot
express ten dollars with less than forty coins:
\beginOutput
i65 : x^39 * y^1000\\
\emptyLine
\       25 39\\
o65 = y  q\\
\emptyLine
o65 : S'\\
\endOutput
We now introduce a new term order on the polynomial ring, defined
by assigning a weight to each variable. Specifically, we assign
weights for each of the coins. For instance,
let pennies have weight 5, nickels weight 7, 
dimes weight 13 and quarters weight 17.
\beginOutput
i66 : weight = (5,7,13,17)\\
\emptyLine
o66 = (5, 7, 13, 17)\\
\emptyLine
o66 : Sequence\\
\endOutput
We set up a new ring with the resulting weight term order, and work modulo
the same ideal as before in this new ring.
\beginOutput
i67 : T = QQ[x, y, p, n, d, q, \\
\                Weights => \{\{1,1,0,0,0,0\},\{0,0,weight\}\},\\
\                MonomialSize => 16]/\\
\            (p - x*y, n - x*y^5, d - x*y^10, q - x*y^25);\\
\endOutput
One dollar with ten coins:
\beginOutput
i68 : x^10 * y^100\\
\emptyLine
\       5 2 3\\
o68 = p d q\\
\emptyLine
o68 : T\\
\endOutput
Ten dollars with one hundred coins:
\beginOutput
i69 : x^100 * y^1000\\
\emptyLine
\       60 3 37\\
o69 = p  n q\\
\emptyLine
o69 : T\\
\endOutput
Here is an optimal solution which involves all four types of coins:
\beginOutput
i70 : x^234 * y^5677\\
\emptyLine
\       2 4 3 225\\
o70 = p n d q\\
\emptyLine
o70 : T\\
\endOutput
