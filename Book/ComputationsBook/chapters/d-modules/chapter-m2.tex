\title{$D$-modules and Cohomology of Varieties}
\titlerunning{$D$-modules and Cohomology of Varieties}
\toctitle{$D$-modules and Cohomology of Varieties}
\author{Uli Walther}
\authorrunning{U. Walther}
% \institute{Mathematical Sciences Research Institute, Berkeley, CA 94720, USA,
% and\\
% Department of Mathematics, Purdue University, West
% Lafayette, IN 47907, USA}
% % \email{walther@math.umn.edu}

\maketitle

\newtheorem{alg}[theorem]{Algorithm}{\bfseries}{}
% \newtheorem{hyp}[theorem]{Hypothesis}
% \newtheorem{con}[theorem]{Construction}
% \newtheorem{conj}[theorem]{Conjecture}
% \newtheorem{notn}[theorem]{Notation}

\def\ann{\operatorname{ann}}
\def\ass{\operatorname{ass}}
\def\codim{\operatorname{codim}}
\def\cd{\operatorname{cd}}
\def\depth{\operatorname{depth}}
\def\Der{\operatorname{Der}}
\def\endo{\operatorname{End}}
\def\ext{\operatorname{Ext}}
\def\gr{\operatorname{gr}}
\def\hom{\operatorname{Hom}}
\def\height{\operatorname{ht}}
\def\im{\operatorname{im}}
\def\ini{\operatorname{in}}
\def\lcd{\operatorname{lcd}}
\def\lcm{\operatorname{lcm}}
\def\spec{\operatorname{Spec}}
\def\soc{\operatorname{soc}}
\def\supp{\operatorname{supp}}
\def\var{\operatorname{Var}}

\def\curlN{{\mathcal N}}
\def\del{\partial}
\def\eps{\varepsilon}
\def\m{{\mathfrak m}}
\def\into{\hookrightarrow}
\def\onto{\to\hskip-1.7ex\to}
\def\A{{\mathbb A}}
\def\C{{\mathbb C}}
\def\F{{\mathfrak F}}
\def\M{{\mathcal M}}
%\def\MLf{{{\mathcal M}^L_f}}
\def\N{{\mathbb N}}
\def\OO{{\mathcal O}}
\def\P{{\mathbb P}}
\def\Q{{\mathbb Q}}
\def\R{{\mathbb R}}
\def\Z{{\mathbb Z}}


\def\action{\bullet}
\def\order{\prec}




\def\_#1{\underline{#1}}
%\def\mylabel#1{\label{#1} \marginpar{#1}}
\def\mylabel#1{\label{#1}}
%\def\myindex#1{\index{#1} \marginpar{#1}}
\def\myindex#1{\index{#1}}
\def\h{\hspace*{-3pt}}
\def\bar#1{\overline{#1}}

\numberwithin{equation}{section}

% \setcounter{tocdepth}{2}
% \maketitle
% \tableofcontents

% \input{0.tex}

\begin{abstract}
  In this chapter we introduce the reader to some ideas from the world of
  differential operators. We show how to use these concepts in conjunction
  with \Mtwo to obtain new information about polynomials and their algebraic
  varieties.
\end{abstract}

Gr\"obner bases over polynomial rings have been used for many
years in computational algebra, and the other chapters in this book
bear witness to this fact. 
In the mid-eighties some important steps were made in the theory of
Gr\"obner bases in non-commutative rings, notably in rings of differential
operators. This chapter is about some of the applications of this
theory to problems in commutative algebra and algebraic geometry. 

Our interest in rings of differential operators and $D$-modules stems
from the fact that some very interesting objects in algebraic geometry and
commutative algebra have a {\em finite} module structure
over an appropriate ring of differential operators. The prime example
is the ring of regular functions on the complement of an affine
hypersurface. A more general object is the \v Cech complex associated
to a set of polynomials, and its cohomology, the local
cohomology modules of the variety defined by the vanishing of the
polynomials. More advanced topics are restriction functors and de Rham
cohomology. 

With these goals in mind, we shall study 
applications  of Gr\"obner bases theory 
in the simplest
ring of differential operators, the Weyl algebra, and develop
algorithms that compute various invariants associated to a polynomial
$f$. These include the Bernstein-Sato polynomial $b_f(s)$, the set of
differential  operators $J(f^s)$ 
which annihilate the germ of the function $f^s$
(where $s$ is a new variable), and the ring of regular functions on the
complement of the variety of $f$. 


For a family $f_1,\ldots,f_r$ of polynomials we study the associated \v Cech
complex as a complex in the category of modules over the Weyl
algebra. The algorithms are illustrated with examples.
We
also give an indication what other invariants associated to
polynomials or varieties are known to be computable at this point and
list some open problems in the area.


\begin{acknowledgment}
It is with great pleasure that I acknowledge the help of A.\ Leykin,
M.\ Stillman
and H.\ Tsai while writing this chapter. The $D$-module routines
used or mentioned here have all been written by them and I would like
to thank them for this marvelous job.
I also would like to thank D.\ Grayson for help on \Mtwo and D.\
Eisenbud and B.\ Sturmfels for inviting me to contribute to this volume.
\end{acknowledgment}

%\input{1.tex}


\section{Introduction}

\subsection{Local Cohomology -- Definitions} 
Let $R$ be a commutative Noetherian ring (always associative, 
with identity) and $M$ an 
$R$-module. For $f\in R$ one defines a {\em \v Cech complex}\index{Cech complex@\v
Cech complex} of $R$-modules
\begin{eqnarray}
\check
C^\bullet(f)=(0\to \underbrace{R}_{\text{degree}\ 0}\into 
\underbrace{R[f^{-1}]}_{\text{degree}\ 1}\to 0)
\end{eqnarray}
where
the injection is the natural map sending $g\in R$ to ${g}/{1}\in
R[f^{-1}]$ and ``degree'' refers to  cohomological degree. 
For a family $f_1,\ldots,f_r\in R$ one defines 
\begin{eqnarray}
\check
C^\bullet(f_1,\ldots,f_r)=\bigotimes_{i=1}^r\check C^\bullet(f_i),
\end{eqnarray}
and for
an $R$-module $M$ one sets 
\begin{eqnarray}
\check C^\bullet(M;f_1,\ldots,f_r)=M\otimes_R \check
C^\bullet(f_1,\ldots,f_r).
\end{eqnarray}


The $i$-th (algebraic) {\em local
cohomology functor}\index{local cohomology}
 with respect to $f_1,\ldots,f_r$ is the $i$-th 
cohomology functor of $\check C^\bullet(-;f_1,\ldots,f_r)$. If
$I=R\cdot (f_1,\ldots,f_r)$ then this functor  agrees
with the $i$-th right
derived functor of the functor $H^0_I(-)$ which sends $M$ to the
$I$-torsion $\bigcup_{k=1}^\infty (0:_MI^k)$ of $M$ and is denoted by
$H^i_I(-)$. This means in particular, that $H^\bullet_I(-)$ 
depends only on the (radical of the) ideal generated by
the $f_i$. Local 
cohomology was introduced by A.~Grothendieck \cite{DM:lc-notes}
as an algebraic analog of
(classical) relative cohomology. For instance, if $X$ is a scheme, 
$Y$ is
a closed subscheme and $U=X\setminus Y$ then there is a long exact
sequence 
\[
\cdots\to H^i(X,\F)\to H^i(U,\F)\to H^{i+1}_Y(X,\F)\to\cdots
\]
for all quasi-coherent sheaves $\F$ on $X$. (To make sense of this one
has to generalize the definition of local cohomology to be the right
derived functor of $H^0_Y(-): \F\to(U\to\{f\in\F(U):\supp(f)\subseteq
Y\cap U\})$.)
An introduction to algebraic local
cohomology theory may be found in \cite{DM:B-S}. 

The {\em
cohomological dimension of $I$ in $R$}, \index{cohomological
dimension} denoted by $\cd(R,I)$, is the
smallest integer $c$ such that 
$H^i_I(M)=0$ for all $i>c$ and all $R$-modules $M$. If $R$ is the
coordinate ring of an affine variety $X$ and $I\subseteq R$ is the defining
ideal of the Zariski closed subset $Y\subseteq X$ then the {\em local
cohomological dimension of $Y$ in $X$} \index{local cohomological
dimension}  is defined as $\cd(R,I)$.
It is not hard to show that if $X$ is smooth, then the integer
$\dim(X)-\cd(R,I)$ depends only on $Y$ but neither on $X$ nor on the
embedding $Y\into X$. 
%
\subsection{Motivation} 
As one sees from the definition of local cohomology, the
modules $H^i_I(R)$ carry information about the sections of the
structure sheaf on Zariski open sets, and hence about the topology of
these open sets.
This is illustrated by the following examples. Let $I\subseteq
R$ and $c=\cd(R,I)$. Then 
$I$ cannot be generated by fewer than $c$ elements -- in other words,
$\spec(R)\setminus \var(I)$ cannot be covered by fewer than $c$ affine
open subsets (i.e., $\var(I)$ cannot be cut out by fewer than $c$
hypersurfaces).  In fact, no ideal $J$ 
with the same radical as $I$ will be generated by fewer than $c$
elements, \cite{DM:B-S}.

Let $H^i_{{\rm Sing}}(-;\C)$ stand for the $i$-th singular cohomology
functor with complex coefficients. 
The classical
Lefschetz Theorem \index{Lefschetz Theorem} \cite{DM:Gri-Har}
states that if $X\subseteq \P^n_\C$ is a
 variety in projective $n$-space 
and $Y$ a hyperplane section of $X$ such that $X\setminus Y$
is smooth, 
then $H^i_{{\rm Sing}}(X;\C)\to
H^i_{{\rm Sing}}(Y;\C)$ is an isomorphism for $i<\dim (X)-1$ and injective for
$i=\dim (X)-1$. 
The Lefschetz Theorem has generalizations in terms of local
cohomology, called Theorems of Barth Type.
For example, let $Y\subseteq \P_\C^n$ be Zariski 
closed
and $I\subseteq 
R=\C[x_0,\ldots,x_n]$ the 
defining ideal of $Y$. 
Then $H^i_{{\rm Sing}}(\P^n_\C;\C)\to H^i_{{\rm Sing}}(Y;\C)$ is an isomorphism for $i<
n-\cd(R,I)$ and injective if $i=n-\cd(R,I)$ (\cite{DM:DRCAV}, Theorem III.7.1).

A consequence of the work of Ogus and Hartshorne
(\cite{DM:Og}, 2.2, 2.3 and \cite{DM:DRCAV}, Theorem IV.3.1) is the
following. 
If $I\subseteq
R=\C[x_0,\ldots,x_n]$ is the defining 
ideal of a complex smooth variety $Y\subseteq \P^n_\C$ then, for
$i<n-\codim (Y)$,
\[
\dim_\C\soc_R
(H^0_\m(H^{n-i}_I (R)))=\dim_\C H^i_x(\tilde Y;\C)
\]
where
$H^i_x(\tilde 
Y;\C)$ stands for the $i$-th singular cohomology group of the affine
cone $\tilde Y$ over $Y$ with support in the vertex $x$ of $\tilde Y$ and
with coefficients in $\C$ (and $\soc_R(M)$  denotes the socle
$(0:_M(x_0,\ldots,x_n))\subseteq M$ for any 
$R$-module $M$), \cite{DM:L-Dmod}. 
These iterated local cohomology modules have a
special structure (cf.\ Subsection \ref{subsec-lambda}).

Local cohomology relates to the connectedness of the underlying spaces
as is shown by the following facts. If $Y$ is a complete intersection
of positive dimension in $\P^n_\C$, then $Y$ cannot be disconnected by
the removal of closed subsets of codimension 2 in $Y$ or higher,
\cite{DM:Br-R}. 
This is a
consequence of the so-called Hartshorne-Lichtenbaum vanishing theorem,
see \cite{DM:B-S}.

In a similar spirit one can show that if $(A,\m)$ is a complete
local domain of dimension $n$ and $f_1,\ldots,f_r$ are elements of the
maximal ideal with $r+2\le n$, then
$\var(f_1,\ldots,f_r)\setminus\{\m\}$ 
is connected, \cite{DM:Br-R}.

In fact, as we will discuss to some extent in Section \ref{sec-ausblick}, 
over the complex numbers the complex $\check
C^\bullet(R;f_1,\ldots,f_r)$ for $R=\C[x_1,\ldots,x_n]$ determines
the Betti numbers
$\dim_\C(H^i_{{\rm Sing}}(\C^n\setminus\var(f_1,\ldots,f_r);\C))$.
\subsection{The Master Plan} 
The cohomological dimension has been studied by many authors. For an
extensive list of references and some open questions 
we recommend to consult 
the very nice survey article \cite{DM:Hu}. 

It turns out that for the determination of $\cd(R,I)$ it is in
fact enough to find a test to 
decide whether or not the local cohomology module $H^i_I(R)=0$ for
given $i, R, I$. This is because $H^i_I(R)=0$ for all $i>c$ implies
$\cd(R,I)\le c$ (see \cite{DM:CDAV}, Section 1).  

Unfortunately,
calculations are complicated by the fact that $H^i_I(M)$ is rarely
finitely generated as $R$-module, even for very nice $R$ and $M$. 
In this
chapter we show how in an important class of examples one may still
carry out explicit computations, by enlarging $R$. 

We shall assume that 
$I\subseteq R_n=K[x_1,\ldots,x_n]$ where $K$ is a
computable field
containing the rational numbers. (By a
{\em computable field}\index{computable field} 
we mean a subfield $K$ of $\C$ such that $K$ is
described by a finite set of data and for which addition, subtraction,
multiplication and division as well as the test whether the result of
any of these operations is zero in the field can be executed by the
Turing machine. For example, $K$ could be $\Q[\sqrt 2]$ stored as a
2-dimensional vector space over $\Q$ with an appropriate
multiplication table.)

The ring of $K$-linear differential operators $D(R,K)$ of the
commutative $K$-algebra $R$ is defined inductively: one sets
$D_0(R,K)=R$, and for $i>0$ defines 
\[
D_i(R,K)=\left\{P\in\hom_K(R,R):Pr-rP\in D_{i-1}(R,K) \text{ for all } r\in
R\right\}.
\]
Here, $r\in R$ is interpreted as the endomorphism of $R$ that
multiplies by $r$.

The local cohomology
modules $H^i_I(R_n)$ have a natural
structure of finitely generated left $D(R_n,K)$-modules (see for
example \cite{DM:K2,DM:L-Dmod}).
The basic reason for  this finiteness is that in this case 
$R_n[f^{-1}]$ is a cyclic
$D(R_n,K)$-module, generated by $f^{a}$ for $\Z\ni a\ll 0$
(compare \cite{DM:B}): 
\begin{eqnarray}
\label{eqn-loc-iso}
R_n[f^{-1}]=D(R_n,K)\action f^{a}.
\end{eqnarray}

Using 
this finiteness we employ the theory of
Gr\"obner bases in $D(R_n,K)$
to develop 
algorithms that give a presentation of $H^i_I(R_n)$ and $H^i_\m (H^j_I(R_n))$
for all triples  $i,j\in \N$, $I\subseteq R_n$ in terms of generators and
relations over 
$D(R_n,K)$ (where $\m=R_n\cdot(x_1,\ldots,x_n)$), see Section \ref{sec-lc}. 
This also leads to an
algorithm for the computation of the invariants 
\[
\lambda_{i,j}(R_n/I)=\dim_K\soc_{R_n}(H^i_\m(H^{n-j}_I(R_n)))
\]
introduced in
\cite{DM:L-Dmod}.


At the basis for the computation of local cohomology are algorithms that
compute the localization of a $D(R_n,K)$-module at a hypersurface $f\in
R_n$. That means, if the left module $M={D(R_n,K)}^d/L$ is given by means of a
finite number of generators for the left module $L\subseteq {D(R_n,K)}^d$
then we want to compute a finite number of generators for the left
module $L'\subseteq {D(R_n,K)}^{d'}$ which satisfies

\[
{D(R_n,K)}^{d'}/L'\cong 
({D(R_n,K)}^d/L)\otimes_{R_n}R_n[f^{-1}]
,
\]
which we do in Section \ref{sec-loc}.  

Let $L$ be a left ideal of $D(R_n,K)$. 
The computation of the localization of 
$M=D(R_n,K)/L$ at $f\in R_n$ is closely related to the
$D(R_n,K)[s]$-module $\M_f$ generated by 
\begin{eqnarray}
\bar 1\otimes 1\otimes f^s\in M\otimes_{R_n} R_n[f^{-1},s]\otimes f^s
\end{eqnarray}
and the minimal polynomial $b_f(s)$ 
of $s$ on the quotient of $\M_f$ by its submodule $\M_f\cdot f$
generated over $D(R_n,K)[s]$ by $\bar 1\otimes f\otimes f^s$,  
 cf.\ Section \ref{sec-loc}. 
Algorithms for the computation of these objects have been
established by T.\ Oaku in a sequence of papers
\cite{DM:Oa,DM:Oa3,DM:Oa2}.

Astonishingly, the roots of $b_f(s)$ prescribe the exponents $a$ that
can be used in the isomorphism (\ref{eqn-loc-iso}) 
between $R_n[f^{-1}]$ and the
$D(R_n,K)$-module generated by $f^{a}$. Moreover, any
 good exponent $a$ can be used to transform $\M_f$ into
$M\otimes R_n[f^{-1}]$ by a suitable ``plugging in'' procedure. 

Thus the strategy for the computation of local cohomology will be
to compute $\M_f$ and a good $a$ for each $f\in\{f_1,\ldots,f_r\}$,
and then assemble the \v Cech complex.

\subsection{Outline of the Chapter}
The next section is devoted to a short introduction of results on
the Weyl algebra $D(R_n,K)$ and $D$-modules as they apply to our
work. 
We start with some remarks on the theory of Gr\"obner
bases in the Weyl algebra.

In Section \ref{sec-loc} we investigate Bernstein-Sato polynomials,
localizations and the \v Cech complex.
The purpose of
that section is to find a 
presentation of $M\otimes R_n[f^{-1}]$ as a cyclic $D(R_n,K)$-module 
if $M=D(R_n,K)/L$
is a given holonomic $D$-module (for a definition and some properties
of holonomic modules, see Subsection \ref{subsec-D-modules} below).



In
Section \ref{sec-lc} we describe 
algorithms that for
arbitrary $i,j,k,I$ 
determine the structure of
$H^k_I(R), H^i_\m (H^j_I(R))$ and find $\lambda_{i,j}(R/I)$. 
The final section is devoted to comments on 
implementations, efficiency, discussions of other topics, and open problems.



%\input{2.tex}
\section{The Weyl Algebra and Gr\"obner Bases}
\mylabel{sec-weyl}

$D$-modules, that is, rings or sheaves of differential operators and
modules over these, have been around for several decades and played
prominent roles in representation theory, some parts of analysis and
in algebraic geometry. The founding fathers of the theory are 
M.\ Sato, M.\ Kashiwara, T.\ Kawai, J.\ Bernstein, and A.\ Beilinson.
 The area has also benefited much from the work of P.\ Deligne,
J.-E.\ Bj\"ork, J.-E.\ Roos, 
B.\ Malgrange and  Z.\ Mebkhout. The more computational aspects of the
theory have been initiated by T.\ Oaku and N.\ Takayama.

The simplest example of a ring of differential
operators is given by the Weyl algebra, the ring of $K$-linear
differential operators on $R_n$. In characteristic
zero, this is a finitely generated $K$-algebra that resembles the ring
of polynomials in $2n$ variables but fails to be commutative.
\subsection{Notation} Throughout we shall use the following notation:
$K$ will 
denote a computable field 
of characteristic zero and $R_n=K[x_1,\ldots,x_n]$ the ring of polynomials
over $K$ in $n$ variables. The $K$-linear differential operators on
$R_n$ are
then the elements of  
\[D_n=K\langle
x_1,\del_1,\ldots,x_n,\del_n\rangle,
\]
the {\em $n$-th Weyl algebra}\myindex{Weyl algebra}
 over $K$, where the symbol $x_i$ denotes the operator ``multiply by
 $x_i$'' and $\del_i$ denotes the operator ``take partial
derivative with respect to $x_i$''. We therefore have in $D_n$ the relations
\begin{eqnarray*}
x_ix_j&=&x_jx_i\quad \text{ for all } 1\le i,j\le n,\\ 
\del_i\del_j&=&\del_j\del_i \quad \text{ for all } 1\le i,j\le n,\\
x_i\del_j&=&\del_jx_i \quad \text{ for all } 1\le i\not =j\le n,\\
\text{ and }\, x_i\del_i+1&=&\del_ix_i \quad \text{ for all } 1\le i\le n. 
\end{eqnarray*}
The last relation is nothing but
the {\em product} (or {\em Leibniz})\myindex{Leibniz rule} {\em rule}, 
$xf'+f=(xf)'$. 
We shall use multi-index notation: $x^\alpha\del^\beta$ denotes the
monomial 
\[
{x_1}^{\alpha_1}\cdots {x_n}^{\alpha_n}\cdot
{\del_1}^{\beta_1}\cdots{\del_n}^{\beta_n}
\]
and $|\alpha|=\alpha_1+ \dots +\alpha_n$. 
 
In order to keep the product $\del_i x_i\in D_n$ 
and the application of $\del_i\in D_n$ to $x_i\in R_n$ apart, we shall
write $\del_i\action (g)$ to mean the result of 
the action\myindex{action (of a differential operator)} of $\del_i$ on $g\in
R_n$. So for example, $\del_ix_i=x_i\del_i+1\in D_n$ but
$\del_i\action x_i=1\in R_n$. The action of $D_n$ on $R_n$ takes
precedence over the multiplication in $R_n$ (and is of course
compatible with the multiplication in $D_n$), so for
example $\del_2\action (x_1)x_2=0\cdot x_2=0\in R_n$.

The symbol
$\m$ will stand for the maximal ideal $R_n\cdot (x_1,\ldots,x_n)$ of
$R_n$, $\Delta$ will denote the maximal left ideal $D_n\cdot 
(\del_1,\ldots,\del_n)$
of $D_n$ and  $I$ will stand for the ideal $R_n\cdot (f_1,\ldots,f_r)$ in
$R_n$. Every $D_n$-module becomes an $R_n$-module via the embedding
$R_n\into D_n$ as $D_0(R_n,K)$. 

All tensor products in this chapter will be over $R_n$ and all
$D_n$-modules (resp.\ ideals)\myindex{$D_n$-modules}
 will be left modules (resp.\ left ideals) unless specified otherwise.
%
\subsection{Gr\"obner Bases in $D_n$}
\mylabel{subsec-GB}
This subsection is a severely shortened version of Chapter 1 in
\cite{DM:SST} (and we strongly recommend that the reader take a look at
this book). The 
purpose is to see how Gr\"obner basis theory applies to the Weyl algebra.

The elements in $D_n$ allow a {\em normally ordered
expression}\myindex{normally ordered expression}. 
Namely, if $P\in D_n$ then we can write it as 
\[
P=\sum_{(\alpha,\beta)\in E}c_{\alpha,\beta}x^\alpha\del^\beta
\]
where $E$ is a finite subset of $\N^{2n}$.
Thus, as $K$-vector spaces there is an isomorphism
\[
\Psi:K[x,\xi]\to D_n
\]
(with $\xi=\xi_1,\ldots,\xi_n$) 
sending $x^\alpha\xi^\beta$ to $x^\alpha\del^\beta$. We will assume
that every $P\in D_n$ is normally ordered.

We shall say that $(u,v)\in\R^{2n}$ is a {\em weight vector for
$D_n$}\myindex{weight vector for $D_n$}
if $u+v\geq 0$, that is $u_i+v_i\geq 0$ for all $1\le i\le n$. We
set the {\em weight}\myindex{weight}
 of the monomial $x^\alpha\del^\beta$ under
$(u,v)$ to be $u\cdot\alpha+v\cdot\beta$ (scalar product). The weight
of an operator is then the maximum of the weights of the nonzero
monomials appearing in the normally ordered expression of $P$.
 If $(u,v)$
is a weight vector for $D_n$, there is an associated graded ring
$\gr_{(u,v)}(D_n)$ with 
\[
\gr_{(u,v)}^r(D_n)=\frac{\{P\in D_n : w(P)\le r\}}{\{P\in D_n : w(P)<
r\}}.
\]
So $\gr_{(u,v)}(D_n)$ is 
the
$K$-algebra on the symbols $\{x_i:1\le i\le n\}\cup
\{\del_i:u_i+v_i=0\}\cup\{\xi_i:u_i+v_i>0\}$. Here all variables
commute with each other except $\del_i$ and $x_i$ for which the
Leibniz rule holds.

Each $P\in D_n$ has an {\em initial form}\myindex{initial form}
 or {\em symbol} \myindex{symbol} $\ini_{(u,v)}(P)$ 
in $\gr_{(u,v)}(D_n)$ defined by
taking all monomials in the normally ordered expression for $P$ that
have maximal weight, and replacing all $\del_i$ with $u_i+v_i>0$ by
the corresponding $\xi_i$. 

The inequality $u_i+v_i\geq
0$ is 
needed to assure that the product of the initial forms of two
operators equals the initial form of their product: one would not want
to have $\ini(\del_i\cdot x_i)=\ini(x_i\cdot\del_i +1)=1$.



A weight of particular importance is $-u=v=(1,\ldots,1)$, or more
generally $-u=v=(1,\ldots,1,0,\ldots,0)$. In these cases
$\gr_{(u,v)}(D_n)\cong D_n$. On the other hand, if $u+v$ is
componentwise positive, then $\gr_{(u,v)}(D_n)$ is commutative
(compare the initial forms of $\del_ix_i$ and $x_i\del_i$) and
isomorphic to the polynomial ring in $2n$ variables corresponding to
the symbols of $x_1,\ldots,x_n,\del_1,\ldots,\del_n$. 


If $L$ is a left ideal in $D_n$ 
we write $\ini_{(u,v)}(L)$\myindex{$\ini_{(u,v)}(L)$}
 for $\{\ini_{(u,v)}(P):P\in L\}$. This is a
left ideal in $\gr_{(u,v)}(D_n)$. If $G\subset L$ is a finite set we
call it a {\em $(u,v)$-Gr\"obner basis}\myindex{$(u,v)$-Gr\"obner basis}
 if the left ideal of
$\gr_{(u,v)}(D_n)$ generated by the initial forms of the elements of
$G$ agrees with $\ini_{(u,v)}(L)$.

A {\em multiplicative monomial order on $D_n$}\myindex{monomial order
on $D_n$} 
is a total order $\order$
on the normally ordered monomials such that 
\begin{enumerate}
\item $1\order x_i\del_i$ for all $i$, and
\item $x^\alpha\del^\beta\order x^{\alpha'}\del^{\beta'}$ implies 
$x^{\alpha+\alpha''}\del^{\beta+\beta''}\order
x^{\alpha'+\alpha''}\del^{\beta'+\beta''}$ for all
$\alpha'',\beta''\in \N^n$. 
\end{enumerate}
A multiplicative monomial order is a {\em
term order}\myindex{term order on $D_n$}
 if $1$ is the (unique) smallest monomial. 
Multiplicative monomial orders, and more specifically term orders,
clearly abound.

Multiplicative monomial orders (and hence term orders) allow the
construction of initial forms just like weight vectors. Now, however, the
initial forms are always monomials, and always elements of $K[x,\xi]$
(due to the total order requirement on $\order$). One
defines Gr\"obner bases for multiplicative monomial orders analogously
to the weight vector case.

For our algorithms 
we have need to compute weight vector Gr\"obner bases, and this can be
done as follows. Suppose $(u,v)$ is a weight vector on $D_n$ and
$\order$ a term order. Define a
multiplicative monomial order $\order_{(u,v)}$ as follows:
\begin{eqnarray*}
x^\alpha\del^\beta\order_{(u,v)}
x^{\alpha'}\del^{\beta'}&\Leftrightarrow& \left[(\alpha-\alpha')\cdot
u+(\beta-\beta')\cdot 
v<0\right] \text{ or }\\
&&\left[(\alpha-\alpha')\cdot
u+(\beta-\beta')\cdot 
v=0 \text{ and } x^\alpha\del^\beta\order x^{\alpha'}\del^{\beta'}\right].
\end{eqnarray*}
Note that $\order_{(u,v)}$ is a term order precisely when $(u,v)$ is
componentwise nonnegative.
\begin{theorem}[\cite{DM:SST}, Theorem 1.1.6.]
Let $L$ be a left ideal in $D_n$, $(u,v)$ a weight vector for $D_n$, 
$\order$ a term
order and $G$ a Gr\"obner basis for $L$ with respect to
$\order_{(u,v)}$. Then
\begin{enumerate}
\item $G$ is a Gr\"obner basis for $L$ with respect to $(u,v)$.
\item $\ini_{(u,v)}(G)$ is a Gr\"obner basis for $\ini_{(u,v)}(L)$ with
respect to $\order$.\qed
\end{enumerate}
\end{theorem}
We end this subsection with the remarks that Gr\"obner bases with
respect to multiplicative monomial
 orders can be computed using the Buchberger algorithm
adapted to the non-commutative situation (thus, Gr\"obner bases with
respect to  weight vectors are computable according to the theorem), 
and that the computation of
syzygies, kernels, intersections and preimages in $D_n$ 
works essentially as in the commutative algebra $K[x,\xi]$. For precise
statements of the algorithms we refer the reader to \cite{DM:SST}.

\subsection{$D$-modules} \mylabel{subsec-D-modules}
A good introduction to $D$-modules\myindex{$D$-modules}
 are the book by J.-E.\ Bj\"ork,
\cite{DM:B}, the nice introduction \cite{DM:Coutinho} by S.~Coutinho, 
and the lecture notes by J.~Bernstein
 \cite{DM:Bernstein-notes}. 
In this subsection  we list some properties of
localizations of $R_n$ that are important for module-finiteness over
 $D_n$. Most of this section is taken from Section 1 in \cite{DM:B}.

Let $f\in R_n$. Then the $R_n$-module $R_n[f^{-1}]$ has a
structure as left $D_n$-module via the extension of the action
$\action$\myindex{action (of a differential operator)}: 
\begin{eqnarray*}
x_i\action (\frac{g}{f^k})=\frac{x_ig}{f^k},&\quad&
\del_i\action (\frac{g}{f^k})=\frac{\del_i\action(g)f-
   k\del_i\action(f)g}{f^{k+1}}.
\end{eqnarray*} 
This may be thought of as a special case of localizing a $D_n$-module: if
$M$ is a $D_n$-module and $f\in R_n$ then $M\otimes_{R_n} R_n[f^{-1}]$ 
becomes a
$D_n$-module via the {\em product rule}\myindex{product
rule}\myindex{action}
\begin{eqnarray*}
x_i\action (m\otimes \frac{g}{f^k})=m\otimes (\frac{x_ig}{f^k}),
&\quad&
\del_i\action (m\otimes \frac{g}{f^k})=
 m\otimes \del_i\action(\frac{g}{f^k})+\del_i m\otimes \frac{g}{f^k}.
\end{eqnarray*}

Of particular interest are the {\em
holonomic} modules\myindex{holonomic module}
 which are those finitely generated $D_n$-modules $M$
for which $\ext^j_{D_n}(M,D_n)$ vanishes 
unless $j=n$. 
This innocent looking definition has surprising consequences,
some of which we discuss now. 

The holonomic modules form a full Abelian subcategory of
the category of left $D_n$-modules, closed under the formation of
subquotients. 
Our standard example of a holonomic module is\myindex{$R_n$ as a
$D_n$-module} 
\[
R_n=D_n/\Delta.
\] 
This equality may require some thought -- it pictures $R_n$ as a
$D_n$-module generated by $1\in R_n$. It is particularly noteworthy
that not all elements of $R_n$ are killed by $\Delta$ -- quite
impossible if $D_n$ were commutative. 

Holonomic modules are
always cyclic and of finite length over $D_n$. These fundamental
properties are consequences of the {\em Bernstein
inequality}\myindex{Bernstein inequality}. To
understand this inequality we associate with the $D_n$-module
$M=D_n/L$ the Hilbert function $q_L(k)$ with values in the integers which
counts for each $k\in\N$ the number of monomials $x^\alpha\del^\beta$ with
$|\alpha|+|\beta|\le k$ whose cosets in $M$ are $K$-linearly
independent. The filtration $k\mapsto K\cdot \{x^\alpha\del^\beta\mod L:
|\alpha|+|\beta|\le k\}$ is called the \myindex{Bernstein filtration} 
{\em Bernstein filtration}. The Bernstein inequality states that 
$q_L(k)$ is either identically 
zero (in which case $M=0$) or asymptotically a
polynomial in $k$ of degree between $n$ and $2n$.
This degree is called the {\em dimension of $M$}\myindex{dimension (of a
$D$-module)}.
A holonomic module is one of dimension $n$, the minimal possible value
for a nonzero module.

This characterization of holonomicity can be used quite easily to
check with \Mtwo that $R_n$ is holonomic. Namely, let's say $n=3$. 
Start a \Mtwo session with
\beginOutput
i1 : load "D-modules.m2"\\
\endOutput
\beginOutput
i2 : D = QQ[x,y,z,Dx,Dy,Dz, WeylAlgebra => \{x=>Dx, y=>Dy, z=>Dz\}]\\
\emptyLine
o2 = D\\
\emptyLine
o2 : PolynomialRing\\
\endOutput
\beginOutput
i3 : Delta = ideal(Dx,Dy,Dz)\\
\emptyLine
o3 = ideal (Dx, Dy, Dz)\\
\emptyLine
o3 : Ideal of D\\
\endOutput
The first of these commands loads the $D$-module library 
by A.\ Leykin, M.\ Stillman  and H.\ Tsai, \cite{DM:M2D}. 
The second line defines the
base ring $D_3=\Q\langle x,y,z,\del_x,\del_y,\del_z\rangle$, 
while the third command defines the $D_3$-module $D_3/\Delta\cong
R_3$.

As one can see, \Mtwo thinks of {\tt D} as a ring of polynomials. This is
using the vector space isomorphism $\Psi$ from Subsection
\ref{subsec-GB}. Of course, two elements are multiplied according to
the Leibniz rule. 
To see how \Mtwo uses the map $\Psi$, we enter the following expression.
\beginOutput
i4 : (Dx * x)^2\\
\emptyLine
\      2  2\\
o4 = x Dx  + 3x*Dx + 1\\
\emptyLine
o4 : D\\
\endOutput
All Weyl algebra ideals and modules 
are by default left
ideals and left modules in \Mtwo. 

If we don't explicitly specify a monomial ordering to be used in the Weyl
algebra, then \Mtwo uses graded reverse lex ({\tt GRevLex}), as we can see by
examining the options of the ring.
\beginOutput
i5 : options D\\
\emptyLine
o5 = OptionTable\{Adjust => identity                        \}\\
\                 Degrees => \{\{1\}, \{1\}, \{1\}, \{1\}, \{1\}, \{1\}\}\\
\                 Inverses => false\\
\                 MonomialOrder => GRevLex\\
\                 MonomialSize => 8\\
\                 NewMonomialOrder => \\
\                 Repair => identity\\
\                 SkewCommutative => false\\
\                 VariableBaseName => \\
\                 VariableOrder => \\
\                 Variables => \{x, y, z, Dx, Dy, Dz\}\\
\                 Weights => \{\}\\
\                 WeylAlgebra => \{x => Dx, y => Dy, z => Dz\}\\
\emptyLine
o5 : OptionTable\\
\endOutput

To compute the initial ideal of $\Delta$ with respect to the weight
that associates $1$ to each $\del$ and to each variable, execute
\beginOutput
i6 : DeltaBern = inw(Delta,\{1,1,1,1,1,1\}) \\
\emptyLine
o6 = ideal (Dz, Dy, Dx)\\
\emptyLine
o6 : Ideal of QQ [x, y, z, Dx, Dy, Dz]\\
\endOutput
The command {\tt inw} can be used with any weight vector for $D_n$ as
second argument. 
One notes that the output is not an ideal in a Weyl algebra any more,
but in a ring of polynomials, as it should.
The dimension of $R_3$, which is the dimension of the variety
associated to {\tt DeltaBern}, is computed by
\beginOutput
i7 : dim DeltaBern \\
\emptyLine
o7 = 3\\
\endOutput
As this is equal to $n=3$, the ideal $\Delta$ is holonomic.

\bigskip

Let $R_n[f^{-1},s]\otimes f^s$ be the free $R_n[f^{-1},s]$-module
generated by the symbol $f^s$.  
Using the action $\action$ of $D_n$ on $R_n[f^{-1},s]$ we define an
action\myindex{action (of a differential operator)} $\action$ 
of $D_n[s]$ on 
$R_n[f^{-1},s]\otimes f^s$ by setting
\begin{eqnarray*}
s\action \left(\frac{g(x,s)}{f^k}\otimes f^s\right)&=& 
\frac{sg(x,s)}{f^k}\otimes f^s,\\
x_i\action\left(\frac{g(x,s)}{f^k}\otimes f^s\right)&=&
 \frac{x_ig(x,s)}{f^k}\otimes f^s,\\
\del_i\action\left(\frac{g(x,s)}{f^k}\otimes f^s\right)&=&
 \left(\del_i\action\left(\frac{g(x,s)}{f^k}\right)+s\del_i\action(f)\cdot
 \frac{g(x,s)}{f^{k+1}}\right)\otimes f^s.
\end{eqnarray*}
The last rule justifies the choice for the symbol of the generator.

Writing $M=D_n/L$ and denoting by $\bar 1$ the coset of $1\in D_n$ in
$M$, 
this action extends to an action\myindex{action (of a differential
 operator)}
 of $D_n[s]$ on 
\begin{eqnarray}
\M^L_f=D_n[s]\action(\bar 1\otimes 1\otimes f^s)\subseteq 
M\otimes_{R_n} \left(R_n[f^{-1},s]\otimes f^s\right)
\end{eqnarray}
by the product rule\myindex{product rule} for all left $D_n$-modules $M$.
The interesting bit about $\M^L_f$
is the following fact.
If $M=D_n/L$ is holonomic 
then
there is a nonzero polynomial $b(s)$ in $K[s]$ and an
operator $P(s)\in D_n[s]$ such that 
\begin{eqnarray}
\label{def-b-poly}
P(s)\action(\bar 1\otimes f\otimes f^s)=
\bar 1\otimes b(s)\otimes f^s
\end{eqnarray}
in $\M^L_f$.
This entertaining equality, often written as 
\[
P(s)\left( \bar 1\otimes f^{s+1}\right)=\bar{b(s)}\otimes f^s,
\] 
says that $P(s)$ is roughly equal to
division by $f$.
The unique monic polynomial that divides 
all other polynomials $b(s)$ satisfying an identity of this type is called the
{\em Bernstein} (or also {\em Bernstein-Sato}) {\em
polynomial}\myindex{Bernstein (Bernstein-Sato) polynomial} of $L$ and 
$f$ and denoted by $b_f^L(s)$\myindex{$b_f^L(s)$}.
Any operator $P(s)$ that satisfies (\ref{def-b-poly}) with
$b(s)=b_f(s)$ 
 we shall call a 
{\em Bernstein operator}\myindex{Bernstein operator} and refer
to the roots of $b_f^L(s)$ as {\em Bernstein roots}\myindex{Bernstein
root} 
of $f$ on $D_n/L$.
It is clear from (\ref{def-b-poly}) and the definitions that
$b^L_f(s)$ is the minimal polynomial of $s$ on the quotient of
$\M^L_f$ by $D_n[s]\action(\bar 1\otimes f\otimes f^s)$. 

The Bernstein roots of the polynomial $f$ are somewhat mysterious, but
related to other algebro-geometric invariants as, for example, the
monodromy of $f$ (see \cite{DM:M}), the Igusa zeta function (see \cite{Loeser}), and the log-canonical threshold
(see \cite{DM:Kollar}). For a long time it was also unclear how to
compute $b_f(s)$ for given $f$. In \cite{DM:Yano} many interesting examples of
Bernstein-Sato polynomials are worked out by hand, while in
\cite{DM:AK,DM:Brianconetal,DM:Maisonobe,DM:Satoetal}
algorithms were given that compute $b_f(s)$ under certain conditions
on $f$. The general algorithm we are going to explain was given by
T.\ Oaku.  Here is a classical example.
\begin{example}
Let $f=\sum_{i=1}^n {x_i}^2$ and $M=R_n$ with $L=\Delta$. One can check
that 
\[
\sum_{i=1}^n{\del_i}^2\action(\bar 1\otimes 1\otimes f^{s+1})=\bar 1\otimes
4(s+1)(\frac{n}{2}+s)\otimes f^{s}
\]
and hence that 
$\frac{1}{4}\sum_{i=1}^n{\del_i}^2$ is a Bernstein operator while the
Bernstein roots of $f$ are $-1$ and $-{n}/{2}$ and the Bernstein
polynomial is $(s+1)(s+\frac{n}{2})$.
\end{example}
\begin{example}
Although in the previous example the Bernstein operator looked a lot
like the polynomial $f$, this is not often the case and it is
usually hard to guess Bernstein operators. For example, one has
\[
\left(\frac{1}{27}\,{\del_y}^3+
\frac{y}{6}{\del_x}^2\del_y+\frac{x}{8}{\del_x}^3\right)(x^2+y^3)^{s+1}=
(s+\frac{5}{6})(s+1)(s+\frac{7}{6}) (x^2+y^3)^s.
\]
In the case of non-quasi-homogeneous polynomials, there is usually no
resemblance between $f$ and any Bernstein operator.
\end{example}
A very important property of holonomic modules is the
(somewhat counterintuitive) fact that any localization of a
holonomic module $M=D_n/L$ at a single element 
(and hence at any finite number
of elements) of $R_n$ is holonomic (\cite{DM:B}, 1.5.9) and 
in particular cyclic over $D_n$, generated by $\bar 1\otimes f^{a}$ for
sufficiently small $a\in \Z$. 
As a
special case we note that 
localizations of $R_n$ are holonomic, and hence finitely generated 
over $D_n$. 
Coming back to the \v Cech complex we see that the complex $\check
C^\bullet(M;f_1,\ldots,f_r)$   
consists of holonomic $D_n$-modules whenever $M$ is holonomic. 

As a consequence, local cohomology modules of $R_n$ are 
$D_n$-modules and in fact holonomic. To see this it 
suffices to know that the maps
in the \v Cech complex are
$D_n$-linear, which we will explain in Section \ref{sec-lc}. 
Since the category 
of holonomic $D_n$-modules and their $D_n$-linear maps
is closed under subquotients,
holonomicity of $H^k_I(R_n)$
follows. 

For similar reasons, $H^i_\m (H^j_I(R_n))$ is holonomic for
$i,j\in \N$ (since $H^j_I(R_n)$ is holonomic).  
These modules, investigated in Subsections \ref{subsec-lclc} and
\ref{subsec-lambda}, are 
rather special $R_n$-modules and seem to carry some very interesting
information about $\var(I)$, see \cite{DM:G-S,DM:W-lambda}. 

The fact that $R_n$ is holonomic and every localization of a holonomic
module is as well, provides motivation for us to study this class of
modules. There are, however, more occasions where holonomic modules
show up. One such situation arises in the study of linear partial
differential equations. More specifically, the so-called GKZ-systems
(which we will meet again in the final chapter) provide a very
interesting class of objects with fascinating combinatorial and
analytic properties \cite{DM:SST}. 


%\input{3.tex}
\section{Bernstein-Sato Polynomials and Localization}
\mylabel{sec-loc}
We mentioned in the introduction that for the computation
of local cohomology the following is an
important algorithmic problem to solve.

\begin{problem}
\mylabel{prob}
Given $f\in  
R_n$ and a left ideal $L\subseteq D_n$ such that $M=D_n/L$ is holonomic,  
compute the structure of the module 
$D_n/L\otimes R_n[f^{-1}]$ in terms of generators and relations. 
\end{problem}
This section is about solving Problem \ref{prob}.
\subsection{The Line of Attack}
Recall for a given $D_n$-module $M=D_n/L$ the action of $D_n[s]$ on the
tensor product $M\otimes_{R_n}(R_n[f^{-1},s]\otimes f^s)$ from
Subsection \ref{subsec-D-modules}. 
We begin with defining an ideal of operators:
\begin{definition}
Let  $J^L(f^s)$ stand for the ideal in $D_n[s]$
that kills $\bar 1\otimes 1\otimes f^s\in (D_n/L)\otimes_{R_n}
R_n[f^{-1},s]\otimes f^s$. 
\end{definition}
It turns out that it is very useful to know this ideal.
If $L=\Delta$ then there are some obvious candidates for generators of
$J^L(f^s)$. For example, there are $f\del_i-\del_i\action(f)s$ for all
$i$. However, unless the affine hypersurface defined by $f=0$ is smooth, these will not generate
$J^\Delta(f^s)$. For a more general $L$, there is a similar set of
(somewhat less) obvious candidates, but again finding all elements of
$J^\Delta(f^s)$ is far from elementary, even for smooth $f$.

In order to find $J^L(f^s)$, we will consider the module 
$(D_n/L)\otimes R_n[f^{-1},s]\otimes f^s$ over the ring $D_{n+1}=D_n\langle
t,\del_t\rangle$ by defining an appropriate action of $t$ and $\del_t$
on it. It is then not hard to 
compute the ideal $J^L_{n+1}(f^s)\subseteq D_{n+1}$ consisting of all
operators that kill $\bar 1\otimes 1\otimes f^s$, see Lemma
\ref{lem-malgrange}.  
In Proposition \ref{prop-oaku} 
we will then explain how to compute $J^L(f^s)$ from $J_{n+1}^L(f^s)$. 

This construction 
gives an answer to the question of determining
a presentation of  $D_n\action (\bar 1\otimes f^a)$ for ``most'' $a\in
K$, which we make precise as follows.
\begin{definition}
We say that a property depending on $a\in K^m$ {\em holds for $a$
 in
very general position}, if there is a countable set of hypersurfaces
 in $K^m$ such that the property holds for all $a$ not on any of the
 exceptional hypersurfaces. 
\end{definition}
It will turn out that for $a\in K$ in very general position $J^L(f^s)$
``is'' the annihilator for $f^a$: we shall very explicitly 
identify a countable number of
exceptional values
in $K$ such that if $a$ is not equal to one of them, then $J^L(f^s)$
evaluates under $s\mapsto a$ to the annihilator inside $D_n$ of $\bar 1\otimes f^a$.
  
For $a\in\Z$ we have
of course
$D_n\action (\bar 1\otimes f^a)\subseteq M\otimes 
R_n[f^{-1}]$ but the inclusion may be strict
(e.g., for $L=\Delta$ and $a=0$). 
Proposition \ref{prop-kashiwara}
shows how 
$(D_n/L)\otimes R_n[f^{-1}]$ and
$J^L(f^s)$ are related.
\subsection{Undetermined Exponents}
Consider $D_{n+1}=D_n\langle t,\del_t\rangle$, \myindex{$D_{n+1}$} 
the Weyl algebra in
$x_1,\ldots,x_n$ and the new variable $t$. B.\ Malgrange
\cite{DM:M} 
has defined an\myindex{action} 
action $\action$ of $D_{n+1}$ on 
$(D_n/L)\otimes R_n[f^{-1},s]\otimes f^s$ as follows.
We require that $x_i$ acts as multiplication on the first factor, and
for the other variables we set (with $\bar P\in D_n/L$ and  $g(x,s)\in
R_n[s]$) 
\begin{eqnarray*}
\del_i\action(\bar P\otimes \frac{g(x,s)}{f^k}\otimes f^s)&=
  &\left(\bar P\otimes
  \left(\del_i\action(\frac{g(x,s)}{f^k})+\frac{s\del_i\action(f)g(x,s)}{f^{k+1}}\right)  
\right. \\ & & \left. \quad {}
   +\bar{\del_iP}\otimes \frac{g(x,s)}{f^k}\right)\otimes f^s,\\
t\action(\bar P\otimes \frac{g(x,s)}{f^k}\otimes f^s)&=
  &\bar P\otimes \frac{g(x,s+1)f}{f^k}\otimes f^s,\\
\del_t\action(\bar P\otimes \frac{g(x,s)}{f^k}\otimes f^s)&=
  &\bar P\otimes \frac{-sg(x,s-1)}{f^{k+1}} \otimes f^s.
\end{eqnarray*}
One checks that this actually
defines a left $D_{n+1}$-module structure 
 (i.e., $\del_tt$ acts like $t\del_t+1$) 
and that
$-\del_tt$ acts as 
multiplication by $s$. 

\begin{definition}
We denote by $J^L_{n+1}(f^s)$ the ideal in $D_{n+1}$ that annihilates the
element $\bar 1\otimes 1\otimes f^s$ in $(D_n/L)\otimes
R_n[f^{-1},s]\otimes f^s$ with $D_{n+1}$ acting
as defined above. 
Then we have an induced morphism
of $D_{n+1}$-modules $D_{n+1}/J^L_{n+1}(f^s)\to (D_n/L)\otimes
R_n[f^{-1},s]\otimes f^s$ sending
$P+J^L_{n+1}(f^s)$ to 
$P\action (\bar 1\otimes 1\otimes f^s)$. 
\end{definition}

We say that an ideal $L\subseteq D_n$ is {\em
 $f$-saturated}\myindex{saturated@$f$-saturated}
 if $f\cdot
P\in L$ implies $P\in L$ and we say that $D_n/L$ is {\em $f$-torsion
free}\myindex{torsion free@$f$-torsion free} if $L$ is $f$-saturated. 
$R_n$ and all its localizations are examples of $f$-torsion free
modules for arbitrary $f$.

The
following lemma is a modification of Lemma 4.1 in \cite{DM:M} 
where the special case
$L=D_n\cdot (\del_1,\ldots,\del_n), D_n/L=R_n$ is considered (compare also
\cite{DM:W1}).

\begin{lemma}
\mylabel{lem-malgrange}
Suppose that $L=D_n\cdot (P_1,\ldots,P_r)$ is $f$-saturated.
With the above definitions, $J^L_{n+1}(f^s)$ is the
ideal generated by $f-t$ together with the images of the $P_j$ under
the automorphism $\phi$ of $D_{n+1}$ induced by $x_i\mapsto x_i$ for all
$i$, and $t \mapsto t-f$. 
\end{lemma}

\begin{proof}
The automorphism sends $\del_i$ to $\del_i+\del_i\action(f)\del_t$ 
and $\del_t$ to
$\del_t$. So if we write $P_j$ as a polynomial
$P_j(\del_1,\ldots,\del_n)$
in the $\del_i$ 
with coefficients in
$K[x_1,\ldots,x_n]$, then 
\[
\phi
(P_j)=P_j(\del_1+\del_1\action(f)\del_t,\ldots,\del_n+\del_n\action(f)\del_t).
\]

One checks that $(\del_i+\del_i\action(f)\del_t)\action 
(\bar Q\otimes 1\otimes f^s)=
\bar{\del_i
Q}\otimes 1\otimes f^s$ for all $Q\in D_{n+1}$, so that
$\phi(P_j(\del_1,\ldots,\del_n))\action (
\bar 1\otimes 1\otimes f^s)=\bar{P_j(\del_1,\ldots,\del_n)}\otimes 1
\otimes f^s=0$. By
definition, $f\action
(\bar 1\otimes 1\otimes f^s)=t\action (\bar 1\otimes 1\otimes f^s)$. So $t-f\in
J^L_{n+1}(f^s)$ and $\phi(P_j)\in J^L_{n+1}(f^s)$ for $j=1,\ldots,r$.

Conversely let $P\action (\bar 1\otimes 1\otimes f^s)=0$. The proof that
$P\in\phi(J^L_{n+1}+D_{n+1}\cdot t)$ relies on
an elimination idea and has some Gr\"obner basis flavor. 
We have to show that
$P\in D_{n+1}\cdot (\phi(P_1),\ldots,\phi(P_r),t-f)$. 
We may assume, that $P$ does
not contain 
any power of  $t$ since we can eliminate $t$ using $f-t$. Now rewrite $P$ in
terms of $\del_t$ and the $\del_i+\del_i\action(f)\del_t$. Say, 
$P=\sum_{\alpha,\beta}
\del_t^\alpha
x^\beta
Q_{\alpha,\beta}
(\del_1+\del_1\action(f)\del_t,\ldots,\del_n+\del_n\action(f)\del_t)$, 
where the $Q_{\alpha,\beta}\in K[y_1,\ldots,y_n]$ are polynomial
expressions. 
Then
\[
P\action (\bar 1\otimes 1\otimes f^s)=\sum_{\alpha,\beta} 
\del_t^\alpha\action(
\bar{x^\beta
Q_{\alpha,\beta}(\del_1,\ldots,\del_n)}\otimes 1\otimes f^s).
\] 
Let $\bar\alpha$ be the largest $\alpha\in\N$ for which there is a
nonzero $Q_{\alpha,\beta}$ occurring in $P=\sum_{\alpha,\beta}
\del_t^\alpha
x^\beta Q_{\alpha,\beta}(\del_1+\del_1\action(f)\del_t,\ldots,\del_n+\del_n\action(f)\del_t)$.
We show that the sum of terms that contain
$\del_t^{\bar\alpha}$ is in $D_{n+1}\cdot \phi(L)$ as
follows. In 
order for $P\action 
(\bar 1\otimes 1\otimes f^s)$ to vanish, the sum of terms with the
highest $s$-power, namely $s^{\bar\alpha}$, must vanish. Hence  
$\sum_\beta 
x^\beta Q_{\bar\alpha,\beta}(\del_1,\ldots,\del_n)
\otimes (-1/f)^{\bar\alpha}\otimes f^s
\in
L\otimes R_n[f^{-1},s]\otimes f^s$ as $R_n[f^{-1},s]$ is $R_n[s]$-flat.
It follows that $\sum_\beta x^\beta
Q_{\bar\alpha,\beta}(\del_1,\ldots,\del_n)\in L$ ($L$ is
$f$-saturated!) and hence $\sum_\beta
\del_t^{\bar\alpha}
x^\beta
Q_{\bar\alpha,\beta}
(\del_1+\del_1\action(f)\del_t,\ldots,\del_n+\del_n\action(f)  
\del_t)\in
D_{n+1}\cdot \phi(L)$ as announced.

So by the first part,
$P-\sum_\beta \del_t^{\bar\alpha} x^\beta
Q_{\bar\alpha,\beta}(\del_1+\del_1\action(f)\del_t,\ldots,\del_n+\del_n
\action(f)\del_t)$ kills
$\bar 1\otimes 1\otimes f^s$, but is of 
smaller degree in $\del_t$ than $P$ was.

The claim follows by induction on $\bar\alpha$.\qed
\end{proof}

\bigskip

If we identify $D_n[-\del_tt]$ with $D_n[s]$ then $J^L_{n+1}(f^s)\cap
D_n[-\del_tt]$ is identified with  $J^L(f^s)$ since, as we observed
earlier, $-\del_tt$ multiplies by $s$ on $\M^L_f$. 
As we pointed out in the beginning, the crux of our algorithms is to
calculate $J^L(f^s)=J^L_{n+1}(f^s)\cap D_n[s]$. We shall deal with
this computation now.

\mylabel{subsec-oaku}
In Theorem 19 of \cite{DM:Oa2}, T.~Oaku
showed how to construct a generating set for $J^L(f^s)$ in the case 
$L=D_n\cdot(\del_1,\ldots,\del_n)$. 
%According to Subsection 
%\ref{subsec-special-exp}, $J^L(f^s)$ is the
%intersection of $J^L_{n+1}(f^s)$ with $D_n[-\del_tt]$. 
Using his ideas we explain how one may calculate $J\cap
D_n[-\del_tt]$ whenever $J\subseteq D_{n+1}$ is any given ideal, and as a
corollary develop an algorithm that for $f$-saturated $D_n/L$
computes $J^L(f^s)=J^L_{n+1}(f^s)\cap D_n[-\del_tt]$. 

We first review some work
of Oaku. 
On $D_{n+1}$ we define the weight vector\myindex{weight vector}
 $w$ by  $w(t)=1,
w(\del_t)=-1, w(x_i)=w(\del_i)=0$ and we extend it to
$D_{n+1}[y_1,y_2]$ by $w(y_1)=-w(y_2)=1$. If $P=\sum_i P_i\in
D_{n+1}[y_1,y_2]$ and all $P_i$ are monomials, then we will write
$(P)^h$ for the operator $\sum_i P_i\cdot y_1^{d_i}$ where
$d_i=\max_j(w(P_j))-w(P_i)$ and call it the {\em
$y_1$-homogenization}\myindex{$y_1$-homogenization} 
of $P$.

Note that the
Buchberger algorithm preserves homogeneity 
in the following sense: if a set of generators for an ideal is given
and these generators are homogeneous with respect to the weight above,
then any new generator for the ideal constructed with the classical
Buchberger algorithm will also be homogeneous. (This is a consequence
of the facts that the $y_i$ commute with all other variables and that
$\del_t t=t\del_t+1$ is homogeneous of weight zero.) 
This homogeneity is very important for the following 
result of Oaku:
\begin{proposition}
\mylabel{prop-oaku}
Let $J=D_{n+1}\cdot(Q_1,\ldots,Q_r)$.
Let $I$ be the left ideal in $D_{n+1}[y_1]$ generated by the
$y_1$-homogenizations $(Q_i)^h$ of the $Q_i$, relative to the weight
$w$ above, and set $\tilde
I=D_{n+1}[y_1,y_2]\cdot (I,1-y_1y_2)$. Let $G$ be a Gr\"obner basis
for $\tilde I$ under a monomial order that eliminates $y_1,y_2$. For
each $P\in G\cap D_{n+1}$ 
set $P'=t^{-w(P)}P$ if $w(P)<0$ and $P'=\del_t^{w(P)}P$ if
$w(P)\geq 0$. Set $G_0=\{ P': P\in G\cap D_{n+1}\}$. Then
$G_0\subseteq D_n[-\del_tt]$ generates $J\cap D_n[-\del_tt]$.
\end{proposition}

\begin{proof}
This is in essence Theorem 18 of \cite{DM:Oa2}. (See the remarks 
in Subsection \ref{subsec-GB} on how to compute such Gr\"obner
bases.)\qed 
\end{proof}

As a corollary to this proposition we obtain an algorithm for the
computation of $J^\Delta(f^s)$:

\begin{alg}[Parametric Annihilator]
\mylabel{alg-ann-fs}~

\noindent {\sc Input}: $f\in R_n$; $L\subseteq D_n$ such that $L$ is 
$f$-saturated.

\noindent {\sc Output}: Generators for $J^L(f^s)$\myindex{$J^L(f^s)$,
algorithm for}.

\begin{enumerate}
\item For each generator $Q_i$ of $D_{n+1}\cdot (L,t)$ 
compute the image $\phi(Q_i)$
under $x_i\mapsto
x_i$, $t\mapsto t-f$, $\del_i\mapsto \del_i+\del_i\action(f)\del_t$,
$\del_t\mapsto\del_t$.

\item Homogenize all $\phi(Q_i)$ with respect to the new variable
$y_1$ relative to the weight $w$ introduced before Proposition \ref{prop-oaku}.

\item Compute a Gr\"obner basis for the ideal
\[
D_{n+1}[y_1,y_2]\cdot((\phi(Q_1))^h, \ldots, (\phi(Q_r))^h, 1-y_1y_2)
\]
in $D_{n+1}[y_1,y_2]$
using an order that eliminates $y_1,y_2$.

\item Select the operators $\{ P_j\}_1^b$ in this basis which do not
contain $y_1, y_2$. 

\item For each $P_j$, $1\le j\le b$, if $w(P_j)>0$ replace $P_j$ by
$P_j'=\del_t^{w(P_j)}P_j$. Otherwise replace $P_j$ by
$P_j'=t^{-w(P_j)}P_j$. 

\item Return the new operators $\{P_j'\}_1^b$.
\end{enumerate}
End.
\end{alg}
The output will be operators in $D_n[-\del_tt]$ which is naturally
identified with $D_n[s]$ (including the action on $\M^L_f$).
This algorithm is in effect Proposition 7.1 of \cite{DM:Oa3}.

In \Mtwo, one can compute the parametric annihilator ideal (for
$R_n=\Delta$) by the command {\tt AnnFs}:
\beginOutput
i8 : D = QQ[x,y,z,w,Dx,Dy,Dz,Dw, \\
\            WeylAlgebra => \{x=>Dx, y=>Dy, z=>Dz, w=>Dw\}];\\
\endOutput
\beginOutput
i9 : f = x^2+y^2+z^2+w^2\\
\emptyLine
\      2    2    2    2\\
o9 = x  + y  + z  + w\\
\emptyLine
o9 : D\\
\endOutput
\beginOutput
i10 : AnnFs(f)\\
\emptyLine
\                                                                       $\cdot\cdot\cdot$\\
o10 = ideal (w*Dz - z*Dw, w*Dy - y*Dw, z*Dy - y*Dz, w*Dx - x*Dw, z*Dx  $\cdot\cdot\cdot$\\
\                                                                       $\cdot\cdot\cdot$\\
\emptyLine
o10 : Ideal of QQ [x, y, z, w, Dx, Dy, Dz, Dw, \$s, WeylAlgebra => \{x = $\cdot\cdot\cdot$\\
\endOutput
If we want to compute $J^L(f^s)$ 
for more general $L$, we have to use
the command {\tt AnnIFs}:
\beginOutput
i11 : L=ideal(x,y,Dz,Dw)\\
\emptyLine
o11 = ideal (x, y, Dz, Dw)\\
\emptyLine
o11 : Ideal of D\\
\endOutput
\beginOutput
i12 : AnnIFs(L,f)\\
\emptyLine
\                                1        1\\
o12 = ideal (y, x, w*Dz - z*Dw, -*z*Dz + -*w*Dw - \$s)\\
\                                2        2\\
\emptyLine
o12 : Ideal of QQ [x, y, z, w, Dx, Dy, Dz, Dw, \$s, WeylAlgebra => \{x = $\cdot\cdot\cdot$\\
\endOutput
It should be emphasized that saturatedness of $L$ with respect to $f$ is
a must for {\tt AnnIFs}. 

\subsection{The Bernstein-Sato Polynomial}
Knowing $J^L(f^s)$ allows us to get our hands on the Bernstein-Sato
polynomial of $f$ on $M$:

\begin{corollary}
\mylabel{cor-b-poly}
Suppose $L$ is a holonomic ideal in $D_n$ (i.e., $D_n/L$ is holonomic). 
The
Bernstein polynomial $b_f^L(s)$ of $f$ on $(D_n/L)$ satisfies 
\begin{eqnarray}
(b^L_f(s))=\left(D_n[s]\cdot(J^L(f^s),f)\right)\cap K[s].
\end{eqnarray}
Moreover, if $L$ is $f$-saturated then
$b^L_f(s)$ can be computed with Gr\"obner basis computations.
\end{corollary}

\begin{proof}
By definition of $b^L_f(s)$ we have
$(b_f^L(s)-P_f^L(s)\cdot f)\action (\bar 1\otimes 1\otimes f^s)=0$ for
a suitable $P^L_f(s)\in D_n[s]$. Hence
$b_f^L(s)$ is in $K[s]$ and in 
$D_n[s](J^L(f^s),f)$. Conversely, if $b(s)$ is in this intersection
then $b(s)$ satisfies an equality of the type of (\ref{def-b-poly}) and
hence is a multiple of $b^L_f(s)$.

If we use 
an elimination order for which $\{x_i,\del_i\}_1^n\gg s$ in $D_n[s]$, then
if $J^L(f^s)$ is known, 
$b^L_f(s)$ will be (up to a scalar 
factor) the unique element in the reduced Gr\"obner basis for
$D_n[s]\cdot (J^L(f^s),f)$ that 
contains no $x_i$ nor $\del_i$. Since we assume $L$ to be
$f$-saturated,  $J^L(f^s)$ can be computed
according to Proposition \ref{prop-oaku}. 
\qed
\end{proof}

We therefore arrive at the following algorithm for the Bernstein-Sato
polynomial \cite{DM:Oa}.

\begin{alg}[Bernstein-Sato polynomial]~

\mylabel{alg-b-poly-L}
\noindent {\sc Input}: $f\in R_n$; $ L\subseteq D_n$ such that 
$D_n/L$ is holonomic and
$f$-torsion free. 

\noindent {\sc Output}: The Bernstein polynomial\myindex{Bernstein
polynomial, algorithm for} $b^L_f(s)$.
\begin{enumerate}
\item Determine $J^L(f^s)$ following Algorithm \ref{alg-ann-fs}. 

\item Find a reduced Gr\"obner basis for the ideal
$J^L(f^s)+D_n[s]\cdot f$ 
using an elimination order for $x$ and $\del$. 

\item Pick the unique element $b(s)\in K[s]$ contained in that basis and
return it.
\end{enumerate}
End.
\end{alg}

We illustrate the algorithm with two examples. We first recall $f$
which was defined at the end of the previous subsection.
\beginOutput
i13 : f\\
\emptyLine
\       2    2    2    2\\
o13 = x  + y  + z  + w\\
\emptyLine
o13 : D\\
\endOutput
Now we compute the Bernstein-Sato polynomial. 
\beginOutput
i14 : globalBFunction(f)\\
\emptyLine
\        2\\
o14 = \$s  + 3\$s + 2\\
\emptyLine
o14 : QQ [\$s]\\
\endOutput
The routine {\tt globalBFunction} computes the
Bernstein-Sato polynomial of $f$ on $R_n$. We also take a look at the
Bernstein-Sato polynomial of a cubic:
\beginOutput
i15 : g=x^3+y^3+z^3+w^3\\
\emptyLine
\       3    3    3    3\\
o15 = x  + y  + z  + w\\
\emptyLine
o15 : D\\
\endOutput
\beginOutput
i16 : factorBFunction globalBFunction(g)\\
\emptyLine
\                    7       8               4       5\\
o16 = (\$s + 1)(\$s + -)(\$s + -)(\$s + 2)(\$s + -)(\$s + -)\\
\                    3       3               3       3\\
\emptyLine
o16 : Product\\
\endOutput
In \Mtwo one can also find $b^L_f(s)$ for more general $L$. We will
see in the following remark what the appropriate commands are. 
 
\begin{remark}
\mylabel{rem-nonQ-root}
It is clear that $s+1$ is always a factor of any Bernstein-Sato
polynomial on $R_n$, but this is not necessarily the case if $L\not =
\Delta$. For example, 
$b^L_f(s)=s$ for $n=1$, $f=x$ and $L=x\del_x+1$ (in which case
$D_1/L\cong R_1[x^{-1}]$, generated by ${1}/{x}$). 
In particular, it is not true that
the roots of $b_f^L(s)$ are negative for general holonomic $L$. 

If $L$ is equal to $\Delta$, and if
$f$ is nice, then the Bernstein roots are all between $-n$ and $0$
\cite{DM:Varchenko}.  But for  general $f$ very little is known besides
a famous theorem of Kashiwara
 that states that $b^\Delta_f(s)$ factors over
$\Q$ \cite{DM:K} and all roots are negative.

For $L$ arbitrary, the situation is more complicated.
The Bernstein-Sato polynomial of any polynomial $f$ on the
$D_n$-module generated by $\bar 1\otimes f^a$ with $a\in K$ is
related to that of $f$ on $D_n/L$ by a simple shift, and so the
Bernstein roots of $f$ on the $D_n$-module generated by the function
germ $f^a$, $a\in K$,  
are still all
in $K$  by
\cite{DM:K}. Localizing other modules however can easily lead to
nonrational roots. As an example, consider
\beginOutput
i17 : D1 = QQ[x,Dx,WeylAlgebra => \{x=>Dx\}];\\
\endOutput
\beginOutput
i18 : I1 = ideal((x*Dx)^2+1)\\
\emptyLine
\             2  2\\
o18 = ideal(x Dx  + x*Dx + 1)\\
\emptyLine
o18 : Ideal of D1\\
\endOutput
This is input defined over the rationals.
Even localizing $D_1/I_1$ at a very simple $f$ leads to nonrational roots:
\beginOutput
i19 : f1 = x;\\
\endOutput
\beginOutput
i20 : b=globalB(I1, f1)\\
\emptyLine
\                                   2\\
o20 = HashTable\{Boperator => - x*Dx  + 2Dx*\$s + Dx\}\\
\                                 2\\
\                Bpolynomial => \$s  + 2\$s + 2\\
\emptyLine
o20 : HashTable\\
\endOutput
The routine {\tt globalB} is to be used if a Bernstein-Sato polynomial
is suspected to fail to factor over $\Q$. If $b_f^L(s)$ does factor
over $\Q$, one can also use the routine {\tt DlocalizeAll} to be
discussed below. It
would be very interesting to determine rules that govern the splitting field
of $b^L_f(s)$ in general.


\end{remark}

\subsection{Specializing Exponents}
\mylabel{subsec-special-exp}
In this subsection we investigate the result of substituting $a\in K$
for $s$ in $J^L(f^s)$.
Recall that the Bernstein polynomial $b^L_f(s)$  
will exist (i.e., be nonzero) if $D_n/L$ 
is holonomic. As outlined in the previous subsection, $b^L_f(s)$ can
be computed if $D_n/L$ is holonomic and $f$-torsion free. 
The following proposition 
(Proposition 7.3 in \cite{DM:Oa3}, see also Proposition 6.2 in \cite{DM:K})
 shows that replacing $s$ by an
exponent in very general position 
leads to a solution of the localization problem.

\begin{proposition}
\mylabel{prop-kashiwara}
If $L$ is holonomic and $a\in K$ is such that 
no element of $\{a-1,a-2,\ldots\}$ is a Bernstein
root of $f$ on $L$ then we have $D_n$-isomorphisms 
\begin{equation}
%% \label{eqn-s=a-iso}
(D_n/L)\otimes_{R_n}\left( R_n[f^{-1}]\otimes f^a\right)
\cong \left(D_n[s]/J^L(f^s)\right)|_{s=a}\cong D_n\action
(\bar 1 \otimes 1\otimes f^a).
\end{equation}
\qed
\end{proposition}
One notes in particular that
if any $a\in\Z $ satisfies the conditions of the
proposition, then so does every integer smaller than $a$. This
motivates the following 
\begin{definition}
The {\em stable integral exponent of $f$ on $L$} is the smallest
integral root of $b^L_f(s)$, and denoted $a^L_f$.
\end{definition}
In terms of this definition, 
\[
\left(D_n/J^L(f^s)\right)|_{s=a^L_f}\cong
(D_n/L)\otimes_{R_n}R_n[f^{-1}],
\]
and the presentation corresponds to the generator $\bar 1\otimes
f^{a^L_f}$. 
If $L=\Delta$ then Kashiwara's result tells us that $b^L_f(s)$ will
factor over the rationals, and thus it is very easy to find the stable
integral exponent. If we localize a more general module, 
the roots  may not even be
$K$-rational anymore as we saw at the end of the previous subsection. 

The following lemma deals with the question of finding the smallest
integer root of a polynomial. We let $|s|$ denote the complex absolute
value. 
\begin{lemma}
Suppose that in the situation of Corollary \ref{cor-b-poly}, 
\[
b^L_f(s)=s^d+b_{d-1}s^{d-1}+\dots+b_0,
\]
and define 
$B=\max_{i}\{|b_i|^{1/(d-i)}\}$.
The smallest integer root of $b^L_f(s)$ is an integer between $-2B$
and $2B$. 
If in particular $L=D_n\cdot 
(\del_1,\ldots,\del_n)$, it suffices to check the
integers between $-b_{d-1}$ and $-1$.
\end{lemma}

\begin{proof}
Suppose
$|s_0|=2B\rho$ where $B$ is as defined above and $\rho>1$. 
Assume 
also that $s_0$ is a root of $b_f^L(s)$. We find
\begin{eqnarray*}
(2B\rho)^d=|s_0|^d&=&|-\sum_{i=0}^{d-1}b_i{s_0}^i|
%&\le&\sum_0^{d-1}|b_i|\cdot|s_0|^i \\
\le\sum_{i=0}^{d-1}B^{d-i}|s_0|^i\\
&=&B^d\sum_{i=0}^{d-1}(2\rho)^i
\le B^d((2\rho)^d-1),
\end{eqnarray*}
using $\rho\geq 1$.
By contradiction, $s_0$ is not a root.

The
final claim is a consequence of Kashiwara's work
 \cite{DM:K} where
he proves that if $L=D_n\cdot(\del_1,\ldots,\del_n)$ then  all roots of
$b_f^L(s)$ are rational and negative, and hence 
$-b_{n-1}$ is a 
lower bound for each single root.
\qed
\end{proof}

Combining Proposition \ref{prop-kashiwara} 
with Algorithms 
\ref{alg-ann-fs} 
and  
\ref{alg-b-poly-L} 
we therefore obtain
\begin{alg}[Localization]~

\mylabel{alg-D/L-loc-f}
\myindex{localization!algorithm for}
\noindent {\sc Input}: 
 $f\in R_n$; $L\subseteq D_n$ such that $D_n/L$ is holonomic and
$f$-torsion free. 

\noindent {\sc Output}: 
 Generators for an ideal $J$ such that $(D_n/L)\otimes_{R_n}
 R_n[f^{-1}]\cong D_n/J$.

\begin{enumerate}
\item Determine $J^L(f^s)$ following Algorithm \ref{alg-ann-fs}. 
\item Find the Bernstein polynomial $b_f^L(s)$ using Algorithm
\ref{alg-b-poly-L}. 
\item Find the smallest integer root $a$ of $b_f^L(s)$.
\item Replace $s$ by $a$ in all generators for $J^L(f^s)$ and
return these generators.
\end{enumerate}
End.
\end{alg}
Algorithms \ref{alg-b-poly-L} and \ref{alg-D/L-loc-f} are 
Theorems 6.14 and Proposition 7.3 in \cite{DM:Oa3}.

\begin{example}
For $f=x^2+y^2+z^2+w^2$, we found a stable integral exponent of $-2$
in the previous subsection.
To compute the annihilator of 
$f^{-2}$ using \Mtwo, we use the command {\tt
Dlocalize} which automatically uses the stable integral exponent.
We first change the current ring back to the ring {\tt D} which we used 
in the previous subsection:
\beginOutput
i21 : use D\\
\emptyLine
o21 = D\\
\emptyLine
o21 : PolynomialRing\\
\endOutput
Here is the
module to be localized.
\beginOutput
i22 : R = (D^1/ideal(Dx,Dy,Dz,Dw))\\
\emptyLine
o22 = cokernel | Dx Dy Dz Dw |\\
\emptyLine
\                             1\\
o22 : D-module, quotient of D\\
\endOutput
The localization then is obtained by running
\beginOutput
i23 : ann2 = relations Dlocalize(R,f)\\
\emptyLine
o23 = | wDz-zDw wDy-yDw zDy-yDz wDx-xDw zDx-xDz yDx-xDy xDx+yDy+zDz+wD $\cdot\cdot\cdot$\\
\emptyLine
\              1       10\\
o23 : Matrix D  <--- D\\
\endOutput
The output {\tt ann2} is a $1\times 10$ matrix whose entries generate
$\ann_{D_4}(f^{-2})$. 
\end{example}

\begin{remark}
The computation of the annihilator 
of $f^a$ for values of $a$ such that $a-k$ is a
Bernstein root for some $k\in\N^+$ can be achieved by an appropriate
syzygy computation. For example, we saw above
that the Bernstein-Sato
polynomial of $f=x^2+y^2+z^2+w^2$ on $R_4$ is $(s+1)(s+2)$. So
evaluation of $J^L(f^s)$ at $-1$ does not necessarily yield
$\ann_{D_4}(f^{-1})$, as will be documented in the next remark. 
On the other hand, evaluation at $-2$ gives
$\ann_{D_4}(f^{-2})$. It is not hard to see that
$\ann_{D_4}(f^{-1})=\{P\in D_n:Pf\in \ann_{D_n}(f^{-2})\}$ because
$D_4\action f^{-1}=D_4f\action f^{-2}
\subseteq D_4\action f^{-2}$. So we set:
\beginOutput
i24 : F = matrix\{\{f\}\}\\
\emptyLine
o24 = | x2+y2+z2+w2 |\\
\emptyLine
\              1       1\\
o24 : Matrix D  <--- D\\
\endOutput
To find $\ann_{D_4}(f^{-1})$, we use the command {\tt modulo} which
computes relations: {\tt modulo(M,N)} computes for two matrices $M, N$
the set of (vectors of) operators $P$ such that $P\cdot M\subseteq \im(N)$. 
\beginOutput
i25 : ann1 = gb modulo(F,ann2)\\
\emptyLine
o25 = \{2\} | wDz-zDw wDy-yDw zDy-yDz Dx^2+Dy^2+Dz^2+Dw^2 wDx-xDw zDx-xD $\cdot\cdot\cdot$\\
\emptyLine
o25 : GroebnerBasis\\
\endOutput
The generator $\del_2^2+\del_y^2+\del_z^2+\del_w^2$ is particularly interesting. 
To see the quotient of
$D_4\action f^{-2}$ by $D_4\action f^{-1}$ we execute
\beginOutput
i26 : gb((ideal ann2) + (ideal F))\\
\emptyLine
o26 = | w z y x |\\
\emptyLine
o26 : GroebnerBasis\\
\endOutput
which shows that $D_4\action f^{-2}$ is an extension of
$D_4/D_4(x,y,z,w)$ by $D_4\action f^{-1}$. This is not surprising,
since $(0,0,0,0)$ is the only singularity of $f$ and hence the
difference between $D_4\action f^{-2}$ and $D_4\action f^{-1}$ must be
supported at the origin.

It is perhaps interesting to note that for a more complicated (but still
irreducible) polynomial $f$ the
quotient $({D_n\action f^{a}})/({D_n\action f^{a+1}})$ can
be a nonsimple nonzero $D_n$-module. For example, let
$f=x^3+y^3+z^3+w^3$ and $a=a^\Delta_f=-2$. 
A computation similar to the quadric case above
shows that here $({D_n\action f^{a}})/({D_n\action f^{a+1}})$ is a
$(x,y,z,w)$-torsion module (supported at the singular locus of $f$)
isomorphic to $(D_4/D_4\cdot (x,y,z,w))^6$. The socle elements of the
quotient are the degree 2 polynomials in $x,y,z,w$.
\end{remark}

\begin{example}
\mylabel{ex-cubic}
Here we show how with \Mtwo one can get more information
from the localization
procedure. 
\beginOutput
i27 : D = QQ[x,y,z,Dx,Dy,Dz, WeylAlgebra => \{x=>Dx, y=>Dy, z=>Dz\}];\\
\endOutput
\beginOutput
i28 : Delta = ideal(Dx,Dy,Dz);\\
\emptyLine
o28 : Ideal of D\\
\endOutput
We now define a polynomial and compute the localization of $R_3$
at the
polynomial.
\beginOutput
i29 : f=x^3+y^3+z^3;\\
\endOutput
\beginOutput
i30 : I1=DlocalizeAll(D^1/Delta,f,Strategy=>Oaku)\\
\emptyLine
\                                1        1        1             2      $\cdot\cdot\cdot$\\
o30 = HashTable\{annFS => ideal (-*x*Dx + -*y*Dy + -*z*Dz - \$s, z Dy -  $\cdot\cdot\cdot$\\
\                                3        3        3\\
\                                     2      5       4\\
\                Bfunction => (\$s + 1) (\$s + -)(\$s + -)(\$s + 2)\\
\                                            3       3\\
\                              2       3         2       4      1   2   $\cdot\cdot\cdot$\\
\                Boperator => --*y*z*Dx Dy*Dz - --*y*z*Dy Dz + ---*z Dx $\cdot\cdot\cdot$\\
\                             81                81             243      $\cdot\cdot\cdot$\\
\                GeneratorPower => -2\\
\                LocMap => | x6+2x3y3+y6+2x3z3+2y3z3+z6 |\\
\                LocModule => cokernel | 1/3xDx+1/3yDy+1/3zDz+2 z2Dy-y2 $\cdot\cdot\cdot$\\
\emptyLine
o30 : HashTable\\
\endOutput
\beginOutput
i31 : I2=DlocalizeAll(D^1/Delta,f)\\
\emptyLine
o31 = HashTable\{GeneratorPower => -2                                   $\cdot\cdot\cdot$\\
\                                          2        2      2       1\\
\                IntegrateBfunction => (\$s) (\$s + 1) (\$s + -)(\$s + -)\\
\                                                          3       3\\
\                LocMap => | x6+2x3y3+y6+2x3z3+2y3z3+z6 |\\
\                LocModule => cokernel | xDx+yDy+zDz+6 z2Dy-y2Dz z2Dx-x $\cdot\cdot\cdot$\\
\emptyLine
o31 : HashTable\\
\endOutput

%\begin{verbatim}
%i1 : load "Dloadfile.m2"
%i2 : D = QQ[x,y,z,Dx,Dy,Dz, WeylAlgebra => {x=>Dx, y=>Dy, z=>Dz}]
%i3 : Delta=ideal(Dx,Dy,Dz)
%i4 : f=x^3+y^3+z^3
%i5 : I1=DlocalizeAll(D^1/Delta,f,Strategy=>Oaku)
%i6 : I2=DlocalizeAll(D^1/Delta,f)
%\end{verbatim}
The last two commands  both compute the localization of
$R_3$ 
at $f$ but follow different localization algorithms. The
former uses our Algorithm \ref{alg-D/L-loc-f} while the latter follows
\cite{DM:O-T-W}. 

The output of the command {\tt DlocalizeAll} is a hashtable, because
it contains a variety of data that pertain to the map $R_n\into
R_n[f^{-1}]$. {\tt LocMap} gives the element that induces the map on
the $D_n$-module level (by right multiplication). {\tt LocModule} gives
the localized module as cokernel of the displayed matrix. {\tt
Bfunction} is the Bernstein-Sato polynomial and {\tt annFS} the
generic annihilator $J^L(f^s)$. {\tt Boperator} displays a
Bernstein operator and the stable integral exponent is stored in {\tt
GeneratorPower}. 

  %% The warning {\tt Oaku's \ldots saturated} refers
  %% to the fact that 
Algorithm \ref{alg-D/L-loc-f} requires the ideal $L$ to be
$f$-saturated. This property is not checked by \Mtwo, so the user
needs to make sure it holds. For example, this is always the case if
$D_n/L$ is a localization of $R_n$. One can check the saturation
property in \Mtwo, but it is a rather involved computation. This difficulty can
be circumvented by omitting the option {\tt Strategy=>Oaku}, in
which case the localization algorithm of \cite{DM:O-T-W} is used. In
terms of complexity, using the Oaku strategy is much better behaved.

One can address the entries of a hashtable.  For
example, executing
\beginOutput
i32 : I1.LocModule\\
\emptyLine
o32 = cokernel | 1/3xDx+1/3yDy+1/3zDz+2 z2Dy-y2Dz z2Dx-x2Dz y2Dx-x2Dy |\\
\emptyLine
\                             1\\
o32 : D-module, quotient of D\\
\endOutput
one can see that 
$R_3[f^{-1}]$ is isomorphic to the cokernel of the {\tt LocModule}
entry which (for either localization method) is
\begin{eqnarray*}
D_3/&D_3\cdot(&x\del_x+y\del_y+z\del_z+6,\,\,\,z^2\del_y-y^2\del_z,\,\,\,x^3\del_y+y^3\del_y+y^2z\del_z+6y^2,\\ 
         &&      z^2\del_x-x^2\del_z,\,\,\,y^2\del_x-x^2\del_y,\,\,\, 
               x^3\del_z+y^3\del_z+z^3\del_z+6z^2).
\end{eqnarray*}
The first line of the hashtable {\tt I1} shows 
that $R_3[f^{-1}]$ is generated
by $f^{-2}$ over $D_3$, while {\tt I1.LocMap} shows that the natural inclusion
$D_3/\Delta=R_3\into R_3[f^{-1}]=D_3/J^\Delta(f^s)|_{s=a^\Delta_f}$ 
is given by right multiplication by $f^2$, shown as the third entry
of the hashtable {\tt I1}.
%The Bernstein polynomial {\tt I1.Bfunction} can be computed by
%<<factorBFunction(globalBFunction(f))>>
It is perhaps useful to point out that the fourth entry of hashtable
{\tt I2} is a relative of the Bernstein-Sato polynomial of $f$, and is
used for the computation of the so-called restriction functor (compare
with \cite{DM:O-T1,DM:W2}).
\end{example}

\begin{remark}
Plugging in bad values $a$ for $s$ (such that $a-k$ {\em is} a Bernstein
root for some $k\in \N^+$) can have unexpected results. Consider the
case $n=1$, $f=x$. Then $J^\Delta(f^s)=D_1\cdot(s+1-\del_1x_1)$. Hence
$b^\Delta_f(s)=s+1$ and $-1$ is the unique Bernstein root. According
to Proposition \ref{prop-kashiwara}, 
\[
\left(D_1[s]/J^\Delta(f^s)\right)|_{s=a}
\cong R_1[{x_1}^{-1}]\otimes {x_1}^a\cong D_1\action {x_1}^a
\]
for all $a\in K\setminus\N$. For $a\in\N^+$, we also have
$D_1[s]/J^\Delta(f^s)|_{s=a}\cong D_1\action x^a$, but this is of
course not $R_1[{x_1}^{-1}]$ but just $R_1$. 

For $a=0$ however, $\left(D_1[s]/J^\Delta(f^s)\right)|_{s=a}$ 
has $x_1$-torsion! It
equals in fact what is called the Fourier transform of
$R_1[{x_1}^{-1}]$ and fits into an exact sequence 
\[
0\to H^1_{x_1}(R_1)\to{\mathcal F}(R_1[{x_1}^{-1}])\to R_1\to 0.
\]
\end{remark}



\begin{remark}
If $D_n/L$ is holonomic but has $f$-torsion, then 
$(D_n/L)\otimes R_n[f^{-1}]$ 
and $((D_n/L)/H^0_{(f)}(D_n/L))\otimes R_n[f^{-1}]$ are of course
isomorphic.  So if we
knew how to find $M/H^0_f(M)$ for holonomic modules $M$, our localization
algorithm could be generalized to all holonomic modules. There are two
different approaches to the problem of $f$-torsion, presented in
\cite{DM:O-T1} and in \cite{DM:Ts,DM:Ts0}. The former is based on homological
methods and restriction to the diagonal 
while the latter aims at direct computation of
those $P\in D_n$ for which $f^k P\in L$ for some $k$. 

There is also another direct method for localizing $M=D_n/L$ at $f$
that works in the situation where the nonholonomic locus of $M$ is
contained in the variety of $f$ (irrespective of torsion). 
It was proved by Kashiwara, that
$M[f^{-1}]$ is then holonomic, and in \cite{DM:O-T-W} an algorithm based
on integration is
given that computes a presentation for it.
\end{remark}

%\input{5.tex}
\section{Local Cohomology Computations}
\mylabel{sec-lc}
The purpose of this section is to present algorithms
that compute 
for given $i,j,k\in \N, I\subseteq  R_n$ 
the structure of the
local cohomology modules $H^k_I(R_n)$ and  $H^i_\m(H^j_I(R_n))$, and the
invariants $\lambda_{i,j}(R_n/I)$ associated to $I$.
%
In particular, the algorithms 
detect the vanishing of local
cohomology modules. 
\subsection{Local Cohomology}
\mylabel{subsec-lc}
We will first describe an algorithm that takes a finite
set of polynomials $\{f_1,\ldots,f_r\}\subset R_n$ and
returns a 
presentation of $H^k_I(R_n)$ where $I=R_n\cdot(f_1,\ldots,f_r)$. In particular,
if $H^k_I(R_n)$ is zero, then the algorithm will return the zero
presentation. 

\begin{definition}
Let $\Theta^r_k$\myindex{$\Theta^r_k$}
be the set of $k$-element subsets of $\{1,\ldots,r\}$ and
for $\theta\in \Theta^r_k$ write $F_\theta$ for the product $\prod_{i\in
\theta}f_{i}$.
\end{definition}
Consider the  \v Cech complex $\check C^\bullet=\check
C^\bullet(f_1,\ldots,f_r)$ associated to $f_1,\ldots,f_r$ in
$R_n$, 
\begin{equation}
%% \label{cechcomplex}
0\to R_n\to \bigoplus_{\theta\in\Theta^r_1} R_n[{F_\theta}^{-1}]\to 
 \bigoplus_{\theta\in\Theta^r_2}R_n[{F_\theta}^{-1}]
 \to\cdots\to R_n[{(f_1\cdots f_r)}^{-1}]\to 0.
\end{equation}
Its $k$-th cohomology group is 
$H^k_I(R_n)$.
The map 
\begin{equation}
\label{cechmap}
M_k:\left(\check C^k=\bigoplus\limits_{\theta\in\Theta^r_k}
   R_n[{F_\theta}^{-1}]\right)\to 
\left(\bigoplus\limits_{\theta'\in\Theta^r_{k+1}}
   R_n[{F_{\theta'}}^{-1}]=\check C^{k+1} \right)
\end{equation}
 is the sum of maps
\begin{equation}
\label{cechmap-parts}
R_n[{(f_{i_1}\cdots f_{i_k})}^{-1}]\to R_n[{(f_{j_1}\dotsb
f_{j_{k+1}})}^{-1}]
\end{equation}
which are zero if $\{i_1,\ldots,i_k\}\not\subseteq
\{j_1,\ldots,j_{k+1}\}$, or send $\frac{1}{1}$ to
$\frac{1}{1}$ (up to sign). 
With $D_n/\Delta\cong R_n$, identify
$R_n[{(f_{i_1}\cdots f_{i_k})}^{-1}]$ with
$D_n/J^\Delta((f_{i_1}\cdots f_{i_k})^s)|_{s=a}$ and
$R_n[{(f_{j_1}\cdots f_{j_{k+1}})}^{-1}]$ with
$D_n/J^\Delta((f_{j_1}\cdots f_{j_{k+1}})^s)|_{s=a'}$ where
$a,a'$ are sufficiently small integers. By
Proposition \ref{prop-kashiwara} we may assume that $a=a'\le 0$. Then the map
(\ref{cechmap-parts}) is in the nonzero case multiplication from the right by
$(f_l)^{-a}$ where $l=\{j_1,\ldots,j_{k+1}\}\backslash \{i_1,\ldots,i_k\}$,
again up to sign. For example, consider the inclusion 
\[
D_2/D_2\cdot(\del_xx,\del_y)=R_2[x^{-1}]\into
R_2[(xy)^{-1}]=D_2/D_2\cdot (\del_xx,
\del_yy).
\]
Since $\frac{1}{x}
=\frac{y}{xy}$, the inclusion on the level of $D_2$-modules 
maps $P+\ann(x^{-1})$ to $Py+\ann((xy)^{-1})$.  

It follows that the matrix representing the map $\check C^k\to \check 
C^{k+1}$ in
terms of $D_n$-modules is very easy to write down once the annihilator
ideals and Bernstein polynomials for all $k$- and $(k+1)$-fold products
of the $f_i$ are known: the entries are 0 or $\pm f_l^{-a}$ where
$f_l$ is the new factor. These considerations give the following 

\begin{alg}[Local cohomology\index{local cohomology!algorithm for}]~

\mylabel{alg-lc}

\noindent {\sc Input}: $f_1,\ldots,f_r\in R_n; k\in \N$.

\noindent {\sc Output}: $H_I^k(R_n)$ in terms of generators and relations as finitely
generated $D_n$-module where $I=R_n\cdot(f_1,\ldots,f_r)$.
\begin{enumerate}
\item Compute the annihilator ideal $J^\Delta((F_\theta)^s)$
and the Bernstein 
polynomial $b^\Delta_{F_\theta}(s)$ for all $(k-1)$-, $k$- and $(k+1)$-fold
products $F_\theta$ of 
${f_1},\ldots,{f_r}$ following Algorithms \ref{alg-ann-fs} and \ref{alg-b-poly-L} (so
$\theta$ runs through $\Theta^r_{k-1}\cup \Theta^r_k\cup \Theta^r_{k+1}$).

\item Compute the stable integral exponents $a^\Delta_{F_\theta}$, 
 let $a$
be their minimum
and replace $s$ by $a$ in all the annihilator ideals.

\item Compute the two matrices $M_{k-1},M_k$ representing the
$D_n$-linear maps 
$\check C^{k-1}\to \check C^k$ and $\check C^k\to \check 
C^{k+1}$ as explained above.

\item Compute a Gr\"obner basis $G$ for the kernel of the composition
\[
\bigoplus_{\theta\in\Theta_k^r}D_n\onto \bigoplus_{\theta\in\Theta^r_k} 
D_n/J^\Delta({F_\theta}^s)|_{s=a}\stackrel{M_k}{\longrightarrow}
\bigoplus_{\theta'\in
\Theta^r_{k+1}}D_n/J^\Delta({F_{\theta'}}^s)|_{s=a}. 
\]

\item Compute a Gr\"obner basis $G_0$ for the preimage 
in $\bigoplus_{\theta\in\Theta_k^r}D_n$ of the module 
\[
\im(M_{k-1})
\subseteq \bigoplus_{\theta\in\Theta^r_k} 
D_n/J^\Delta((F_\theta)^s)|_{s=a}\leftarrow\hskip-1.7ex\leftarrow \bigoplus_{\theta\in\Theta_k^r}D_n
\]
under the indicated projection.
\item Compute the remainders of all elements of $G$ with
respect to $G_0$. 

\item Return these remainders and $G_0$.
\end{enumerate}
End.
\end{alg}
The nonzero elements of $G$ generate the quotient $G/G_0\cong
H^k_I(R_n)$ so that in particular
$H^k_I(R_n)=0$ if and only if all returned remainders are zero. 

\begin{example}
\mylabel{ex-minors}
Let $I$ be the ideal in $R_6=K[x,y,z,u,v,w]$ that is generated by the
$2\times 2$ minors $f,g,h$ of the matrix
$\left(\begin{array}{ccc}x&y&z\\u&v&w\end{array}\right)$.
Then $H_I^i(R_6)=0$ for $i<2$ and 
$H^2_I(R_6)\ne 0$ because $I$ is a height 2 prime,
 and $H^i_I(R_6)=0$ for $i>3$
because $I$ is  
3-generated, so the only open case is $H^3_I(R_6)$. This module
 in
fact does 
not vanish, and our algorithm provides a proof of this 
fact by direct calculation. The \Mtwo commands are as follows.
\beginOutput
i33 : D= QQ[x,y,z,u,v,w,Dx,Dy,Dz,Du,Dv,Dw, WeylAlgebra =>\\
\                \{x=>Dx, y=>Dy, z=>Dz, u=>Du, v=>Dv, w=>Dw\}];\\
\endOutput
\beginOutput
i34 : Delta=ideal(Dx,Dy,Dz,Du,Dv,Dw);\\
\emptyLine
o34 : Ideal of D\\
\endOutput
\beginOutput
i35 : R=D^1/Delta;\\
\endOutput
\beginOutput
i36 : f=x*v-u*y;\\
\endOutput
\beginOutput
i37 : g=x*w-u*z;\\
\endOutput
\beginOutput
i38 : h=y*w-v*z;\\
\endOutput
These commands define the relevant rings and polynomials. The
following three
compute the localization of $R_6$ at $f$: 
\beginOutput
i39 : Rf=DlocalizeAll(R,f,Strategy => Oaku)\\
\emptyLine
o39 = HashTable\{annFS => ideal (Dw, Dz, x*Du + y*Dv, y*Dy - u*Du, x*Dy $\cdot\cdot\cdot$\\
\                Bfunction => (\$s + 1)(\$s + 2)\\
\                Boperator => - Dy*Du + Dx*Dv\\
\                GeneratorPower => -2\\
\                LocMap => | y2u2-2xyuv+x2v2 |\\
\                LocModule => cokernel | Dw Dz xDu+yDv yDy-uDu xDy+uDv  $\cdot\cdot\cdot$\\
\emptyLine
o39 : HashTable\\
\endOutput
of $R_6[f^{-1}]$ at
$g$:
\beginOutput
i40 : Rfg=DlocalizeAll(Rf.LocModule,g, Strategy => Oaku)\\
\emptyLine
\                                                                       $\cdot\cdot\cdot$\\
o40 = HashTable\{annFS => ideal (Dz*Dv - Dy*Dw, x*Du + y*Dv + z*Dw, z*D $\cdot\cdot\cdot$\\
\                Bfunction => (\$s + 1)(\$s)\\
\                Boperator => - Dz*Du + Dx*Dw\\
\                GeneratorPower => -1\\
\                LocMap => | -zu+xw |\\
\                LocModule => cokernel | DzDv-DyDw xDu+yDv+zDw zDz-uDu- $\cdot\cdot\cdot$\\
\emptyLine
o40 : HashTable\\
\endOutput
and of $R_6[(fg)^{-1}]$ at $h$:   
\beginOutput
i41 : Rfgh=DlocalizeAll(Rfg.LocModule,h, Strategy => Oaku)\\
\emptyLine
\                                                                       $\cdot\cdot\cdot$\\
o41 = HashTable\{annFS => ideal (x*Du + y*Dv + z*Dw, z*Dz - u*Du - v*Dv $\cdot\cdot\cdot$\\
\                Bfunction => (\$s - 1)(\$s + 1)\\
\                Boperator => - Dz*Dv + Dy*Dw\\
\                GeneratorPower => -1\\
\                LocMap => | -zv+yw |\\
\                LocModule => cokernel | xDu+yDv+zDw zDz-uDu-vDv-2 yDy- $\cdot\cdot\cdot$\\
\emptyLine
o41 : HashTable\\
\endOutput
From the output of these commands
one sees that $R_6[(fgh)^{-1}]$ is generated by
${1}/{f^2gh}$. This follows from considering the stable integral exponents
of the three localization procedures, 
encoded in the hashtable entry stored under the key {\tt GeneratorPower}:
for example, 
\beginOutput
i42 : Rf.GeneratorPower\\
\emptyLine
o42 = -2\\
\endOutput
shows that the generator for $R_6[f^{-1}]$ is $f^{-2}$.
Now we compute the annihilator of $H^3_I(R_6)$.
From the \v Cech complex it follows that 
 $H^3_I(R_6)$ is the quotient of the output of {\tt Rfgh.LocModule}
 (isomorphic to $R_6[(fgh)^{-1}]$)
by the submodules generated by $f^2$, $g$ and $h$. (These submodules
 represent $R_6[(gh)^{-1}]$, $R_6[(fh)^{-1}]$ and $R_6[(fg)^{-1}]$
 respectively.)  
\beginOutput
i43 : Jfgh=ideal relations Rfgh.LocModule;\\
\emptyLine
o43 : Ideal of D\\
\endOutput
\beginOutput
i44 : JH3=Jfgh+ideal(f^2,g,h);\\
\emptyLine
o44 : Ideal of D\\
\endOutput
\beginOutput
i45 : JH3gb=gb JH3\\
\emptyLine
o45 = | w z uDu+vDv+wDw+4 xDu+yDv+zDw yDy-uDu-wDw-1 xDy+uDv uDx+vDy+wD $\cdot\cdot\cdot$\\
\emptyLine
o45 : GroebnerBasis\\
\endOutput
So {\tt JH3} is the ideal of $D_3$ generated by
\begin{eqnarray*}
&w,\,\,\, z,\,\,\, u\del_u+v\del_v+w\del_w+4,\,\,\,
x\del_u+y\del_v+z\del_w,\,\, \, 
y\del_y-u\del_u-w\del_w-1,&\\ 
&x\del_y+u\del_v,\,\,\, 
            u\del_x+v\del_y+w\del_z,\,\,\, y\del_x+v\del_u,\,\,\,
x\del_x-v\del_v-w\del_w-1,&\\ & v^2,\,\,\, uv,\,\, yv,\,\,\,
u^2,\,\,\, yu+xv,\,\, \, 
            xu,\,\,\, y^2,\,\, \,xy,&\\ &x^2,\,\,\, xv\del_v+2x,\,\,\,
v\del_y\del_u+w\del_z\del_u-v\del_x\del_v-w\del_x\del_w-3\del_x&
\end{eqnarray*}
which form a Gr\"obner basis.
This proves that $H^3_I(R)\not =0$, because $1$ is not in the 
Gr\"obner basis of {\tt JH3}.
(There are also algebraic and topological proofs to this account. 
Due to Hochster, and Bruns and
Schw\"anzl, they are quite 
ingenious and work only in rather special situations.)


From our output one can see that $H^3_I(R_6)$ is
$(x,y,z,u,v,w)$-torsion as {\tt JH3} contains $(x,y,z,u,v,w)^2$.
The following sequence of commands defines a procedure {\tt testmTorsion}
which as the name suggests tests a module $D_n/L$ for being $\m$-torsion.
We first replace the generators of $L$ with a Gr\"obner basis.
Then we pick the elements of the Gr\"obner basis not using any $\del_i$.
If now the left over polynomials define an ideal of dimension $0$ in
$R_n$, the ideal was $\m$-torsion and otherwise not.
\beginOutput
i46 : testmTorsion = method();\\
\endOutput
\beginOutput
i47 : testmTorsion Ideal := (L) -> (\\
\           LL = ideal generators gb L;\\
\           n = numgens (ring (LL)) // 2;\\
\           LLL = ideal select(first entries gens LL, f->(\\
\                     l = apply(listForm f, t->drop(t#0,n));\\
\                     all(l, t->t==toList(n:0))       \\
\                     ));\\
\           if dim inw(LLL,toList(apply(1..2*n,t -> 1))) == n\\
\           then true\\
\           else false);\\
\endOutput
If we apply {\tt testmTorsion} to {\tt JH3} we obtain
\beginOutput
i48 : testmTorsion(JH3)\\
\emptyLine
o48 = true\\
\endOutput
Further inspection shows that the ideal
{\tt JH3} is in fact the annihilator of the fraction
${f}/({wzx^2y^2u^2v^2})$ in
$R_6[(xyzuvw)^{-1}]/R_6\cong 
D_6/D_6\cdot(x,y,z,u,v,w)$, and that the fraction
generates $D_6/D_6\cdot(x_1,\dots,x_6)$.
Since $D_6/D_6\cdot(x_1,\dots,x_6)$ is
isomorphic to   
$E_{R_6}(R_6/R_6\cdot(x_1,\dots,x_6))$, the injective hull of 
$R_6/R_6\cdot(x_1,\dots,x_6)=K$ in
the category  
of $R_6$-modules, we conclude that  
$H^3_I(R_6)\cong E_{R_6}(K)$. (In the next subsection we will display
a way to use \Mtwo to find the length of an $\m$-torsion module.)



In contrast, let $I$ be defined as generated by the three
minors, but this time over  a field of finite characteristic. Then 
 $H^3_I(R_6)$ is
zero because Peskine and Szpiro proved using the Frobenius functor 
\cite{DM:P-S} that $R_6/I$
Cohen-Macaulay implies that $H^k_I(R_6)$ is nonzero only if $k=\codim(I)$.

Also opposite to the above example, but in any characteristic, is the
following calculation.  Let $I$ be the ideal in $K[x,y,z,w]$
describing the twisted cubic: $I=R_4\cdot (f,g,h)$ with $f=xz-y^2$, 
$g=yw-z^2$,
$h=xw-yz$. 
The projective variety $V_2$ defined by $I$ is isomorphic to the
projective
line. It is of interest to determine whether $V_2$ and other Veronese
embeddings of the projective line are complete
intersections. The set-theoretic complete intersection property can
occasionally be ruled out with local cohomology techniques: if $V$ is
of codimension $c$ in the affine variety $X$ and
$H^{c+k}_{I(V)}(O(X))\not =0$ for any positive $k$ then $V$ cannot be
a set-theoretic complete intersection. 
In the case of the twisted cubic, it turns out hat $H^3_I(R_4)=0$ as
can be seen from the following computation:
\beginOutput
i49 : D=QQ[x,y,z,w,Dx,Dy,Dz,Dw,WeylAlgebra => \{x=>Dx, y=>Dy, z=>Dz,\\
\      w=>Dw\}];\\
\endOutput
\beginOutput
i50 : f=y^2-x*z;\\
\endOutput
\beginOutput
i51 : g=z^2-y*w;\\
\endOutput
\beginOutput
i52 : h=x*w-y*z;\\
\endOutput
\beginOutput
i53 : Delta=ideal(Dx,Dy,Dz,Dw);\\
\emptyLine
o53 : Ideal of D\\
\endOutput
\beginOutput
i54 : R=D^1/Delta;\\
\endOutput
\beginOutput
i55 : Rf=DlocalizeAll(R,f,Strategy => Oaku)  \\
\emptyLine
\                                                         1             $\cdot\cdot\cdot$\\
o55 = HashTable\{annFS => ideal (Dw, x*Dy + 2y*Dz, y*Dx + -*z*Dy, x*Dx  $\cdot\cdot\cdot$\\
\                                                         2             $\cdot\cdot\cdot$\\
\                                   3\\
\                Bfunction => (\$s + -)(\$s + 1)\\
\                                   2\\
\                             1   2\\
\                Boperator => -*Dy  - Dx*Dz\\
\                             4\\
\                GeneratorPower => -1\\
\                LocMap => | y2-xz |\\
\                LocModule => cokernel | Dw xDy+2yDz yDx+1/2zDy xDx-zDz $\cdot\cdot\cdot$\\
\emptyLine
o55 : HashTable\\
\endOutput
        
\beginOutput
i56 : Rfg=DlocalizeAll(Rf.LocModule,g, Strategy => Oaku);\\
\endOutput
          
\beginOutput
i57 : Rfgh=DlocalizeAll(Rfg.LocModule,h, Strategy => Oaku);\\
\endOutput
          
\beginOutput
i58 : Ifgh=ideal relations Rfgh.LocModule;\\
\emptyLine
o58 : Ideal of D\\
\endOutput
\beginOutput
i59 : IH3=Ifgh+ideal(f,g,h);\\
\emptyLine
o59 : Ideal of D\\
\endOutput
\beginOutput
i60 : IH3gb=gb IH3\\
\emptyLine
o60 = | 1 |\\
\emptyLine
o60 : GroebnerBasis\\
\endOutput
It follows that we cannot conclude from local cohomological
considerations that $V_2$ is not a set-theoretic complete
intersection. This is not an accident but typical, as the second
vanishing theorem 
of Hartshorne, Speiser, Huneke and Lyubeznik shows
\cite{DM:CDAV,DM:H-Sp,DM:Hu-L}: if a homogeneous ideal $I\subseteq R_n$
describes an geometrically connected projective variety of positive
dimension then $H^{n-1}_I(R_n)=H^{n}_I (R_n)=0$.
\end{example}


\subsection{Iterated Local Cohomology}
\mylabel{subsec-lclc}

Recall that $\m=R_n\cdot(x_1,\ldots,x_n)$. 
As a second application of Gr\"ob\-ner basis computations over the
Weyl algebra we
show now how to compute the $\m$-torsion modules $H^i_\m (H^j_I(R_n))$.
Note that we cannot apply Lemma \ref{lem-malgrange} to $D_n/L=H^j_I(R_n)$
since $H^j_I(R_n)$ may well 
contain some torsion.

$\check C^j(R_n;f_1,\ldots,f_r)$ denotes the $j$-th
module in the 
\v Cech complex to $R_n$ and $\{f_1,\ldots,f_r\}$. 
Let $\check C^{\bullet,\bullet}$ be the double complex 

\[
\check C^{i,j}=\check C^i(R_n;x_1,\ldots,x_n)\otimes_{R_n} \check
 C^j(R_n;f_1,\ldots,f_r), 
\] 
with
vertical maps $\phi^{\bullet,\bullet}$ induced by the identity on the
first factor and the 
usual \v Cech maps on the second, and  horizontal maps
$\xi^{\bullet,\bullet} $  induced
by the \v Cech maps on the first factor and the identity on the
second. 
Now $\check C^{i,j}$ is a direct sum of modules $R_n[g^{-1}]$ where 
$g=x_{\alpha_1}\cdots x_{\alpha_i}\cdot
f_{\beta_1}\cdots f_{\beta_j}$. So the whole double complex
can be rewritten in terms of $D_n$-modules and $D_n$-linear maps using
Algorithm \ref{alg-D/L-loc-f}: 
\[
\diagram
{\,\check C^{i-1,j+1}\,}{\rto^{\,\,\xi^{i-1,j+1}}}&
        \check C^{i,j+1}\rto^{\xi^{i,j+1}}&
                \check C^{i+1,j+1}\\
\check C^{i-1,j}\rto^{\xi^{i-1,j}}\uto_{\phi^{i-1,j}}& 
        \check C^{i,j}\rto^{\xi^{i,j}}\uto_{\phi^{i,j}}& 
                \check C^{i+1,j}\uto_{\phi^{i+1,j}}\\
\check C^{i-1,j-1}\rto^{\xi^{i-1,j-1}}\uto_{\phi^{i-1,j-1}}&
        \check C^{i,j-1}\rto^{\xi^{i,j-1}}\uto_{\phi^{i,j-1}}& 
                \check C^{i+1,j-1}\uto_{\phi^{i+1,j-1}}
\enddiagram
\]
Since $\check C^i(R_n;x_1,\ldots,x_n)$ is $R_n$-flat, the column
co\-ho\-mo\-lo\-gy of $\check C^{\bullet,\bullet}$ at $(i,j)$ is
$\check C^i(R_n;x_1,\ldots,x_n)\otimes_{R_n}H^j_I(R_n)$ and the induced horizontal maps
in the $j$-th row are
simply the maps in the \v Cech complex $\check C^\bullet(H^j_I(R_n);x_1,\ldots,x_n)$. 
It follows that
the row cohomology of the column cohomology at $(i_0,j_0)$ is
$H^{i_0}_\m(H^{j_0}_I(R_n))$, the object of our interest.

We have, denoting by $X_{\theta'}$
in analogy to $F_\theta$ the product $\prod_{i\in
\theta'}x_i$,
 the following 
\begin{alg}[Iterated local cohomology]~

\mylabel{alg-lclc}
\myindex{local cohomology!algorithm for}
\myindex{local cohomology!iterated}
\noindent {\sc Input}: $f_1,\ldots,f_r\in R_n; i_0,j_0\in \N$.

\noindent {\sc Output}: 
$H^{i_0}_\m (H^{j_0}_I(R_n))$ in terms of generators and relations as
finitely generated $D_n$-module where $I=R_n\cdot(f_1,\ldots,f_r)$.

\begin{enumerate}
\item For $i=i_0-1, i_0, i_0+1$ and $j=j_0-1,j_0,j_0+1$ compute the
annihilators $J^\Delta((F_\theta\cdot X_{\theta'})^s)$, Bernstein
polynomials $b^\Delta_{F_\theta\cdot X_{\theta'}}(s)$, and stable
integral exponents $a^\Delta_{F_\theta\cdot X_{\theta '}}$  
of $F_\theta\cdot
X_{\theta'}$ 
where $\theta \in \Theta^r_j, \theta'\in \Theta^n_i$.

\item Let $a$ be the minimum of all $a^\Delta_{F_\theta\cdot X_{\theta '}}$
and replace $s$ by $a$ in all the annihilators
computed in the previous step.

\item Compute the matrices to the $D_n$-linear maps
 $\phi^{i,j}:\check C^{i,j}\to 
\check C^{i,j+1}$ 
and $\xi^{k,l}:\check C^{k,l}\to \check C^{k+1,l}$, for
$(i,j)\in\{(i_0,j_0),(i_0+1,j_0-1),(i_0,j_0-1),(i_0-1,j_0)\}$ and 
$(k,l)\in \{(i_0,j_0),(i_0-1,j_0)\}$. 
\item Compute a Gr\"obner basis $G$ for the module 
\[
D_n\cdot G=\ker(\phi^{i_0,j_0})\cap
\left[ 
(\xi^{i_0,j_0})^{-1}(\im(\phi^{i_0+1,j_0-1}))\right]+\im(\phi^{i_0,j_0-1})
\] 
and a Gr\"obner basis $G_0$ for the module
\[
D_n\cdot
G_0=\xi^{i_0-1,j_0}(\ker(\phi^{i_0-1,j_0}))+\im(\phi^{i_0,j_0-1}).
\]
\item Compute the remainders of all elements of $G$ with
respect to $G_0$. 

\item Return these remainders together with $G_0$.
\end{enumerate}
End.
\end{alg}

Note that $(D_n\cdot G)/(D_n\cdot G_0)$ is isomorphic to
\[\frac{
 \ker\left(\frac{\displaystyle\ker(\phi^{i_0,j_0})}
            {\displaystyle\im(\phi^{i_0,j_0-1})}
      \stackrel{\xi^{i_0,j_0}}{\longrightarrow}
      \frac{\displaystyle\ker(\phi^{i_0+1,j_0})}
            {\displaystyle\im(\phi^{i_0+1,j_0-1})}\right)}
 {\xi^{i_0-1,j_0}\left(
      \frac{\displaystyle\ker(\phi^{i_0-1,j_0})}
           {\displaystyle\im(\phi^{i_0-1,j_0-1})}\right)}\cong H^{i_0}_\m(H^{j_0}_I(R_n)).
\]
The elements of $G$ will be generators for $H^{i_0}_\m
(H^{j_0}_I(R_n))$ and 
the elements of $G_0$ generate the extra relations that are not
syzygies.

The algorithm can of course be modified to compute any iterated local
cohomology group $H^j_J(H^i_I(R_n))$ for $J\supseteq I$ by replacing
the generators $x_1,\ldots,x_n$ for $\m$ by those for $J$. Moreover,
the iteration depth can also be increased by considering
``tricomplexes'' etc.\ instead of bicomplexes.

Again we would like to point out that with the methods of \cite{DM:O-T1}
or \cite{DM:O-T-W} one could actually compute first $H^i_I(R_n)$ and from
that $H^j_J(H^i_I(R_n))$, but probably that is quite a bit more
complex a computation. 
%
\subsection{Computation of Lyubeznik Numbers}
\mylabel{subsec-lambda}
G.\ Lyubeznik proved in \cite{DM:L-Dmod} that if $K$ is a field, 
$R=K[x_1,\ldots,x_n]$, 
$I\subseteq R$, $\m=R\cdot (x_1,\ldots,x_n)$ 
and $A=R/I$ then $\lambda_{i,j}(A)=\dim_K\soc_R H^i_\m(H^{n-j}_I(R))$ is
invariant under change of presentation of $A$. 
In other words, it only
depends on $A$ and $i,j$ but not the projection $R\onto A$. 
Lyubeznik proved that $H^i_\m (H^j_I(R_n))$ is in fact an injective
$\m$-torsion $R_n$-module of finite socle dimension $\lambda_{i,n-j}(A)$
and so
isomorphic to $(E_{R_n}(K))^{\lambda_{i,n-j}(A)}$ where $E_{R_n}(K)$ is the
injective hull of $K$ over $R_n$. We
are now in a position to compute these invariants of $R_n/I$ in
characteristic zero..

\begin{alg}[Lyubeznik numbers\index{Lyubeznik numbers}]~

\mylabel{alg-lambda}
\noindent {\sc Input}: $f_1,\ldots,f_r\in R_n; i,j\in \N$.

\noindent {\sc Output}: $\lambda_{i,n-j}(R_n/R_n\cdot(f_1,\ldots,f_r))$.

\begin{enumerate}
\item Using Algorithm \ref{alg-lclc} find $g_1,\ldots,g_l\in {D_n}^d$
and $h_1,\ldots,h_e\in {D_n}^d$ such that $H^i_\m(H^j_I(R_n))$ is
isomorphic to $D_n\cdot (g_1,\ldots,g_l)$ modulo $H=D_n\cdot (h_1,\ldots,h_e)$.
\item Assume that after a suitable renumeration 
$g_1$ is not in $H$. If such a $g_1$ cannot be
chosen, quit. 
\item Find a monomial $m\in R_n$ such that $m\cdot g_1\not\in H$ but
$x_img_1\in H$ for all $x_i$.
\item Replace $H$ by $D_nmg_1+H$ and reenter at Step 2.
\item Return $\lambda_{i,n-j}(R_n/I)$, the number of times Step 3 was
executed. 
\end{enumerate}
End.
\end{alg}
The reason that this works is as follows.
We know that
$(D_n\cdot g_1+H)/H$ is $\m$-torsion (as $H^i_\m(H^j_I(R_n))$ is) 
and so it is possible (with trial
and error, or a suitable syzygy computation) 
to find the monomial $m$ in Step 3.
The element $mg_1 \mod H\in D_n/H$ has annihilator equal
to $\m$ over $R_n$ and therefore generates a $D_n$-module isomorphic to
$D_n/D_n\cdot \m\cong E_{R_n}(K)$. The injection 
\[
(D_n\cdot mg_1+H)/H\into
(D_n\cdot(g_1,\ldots,g_l)+H)/H
\]
splits as map of $R_n$-modules because $E_{R_n}(K)$ is  
injective and so the cokernel 
$D_n\cdot (g_1,\ldots,g_l)/D_n\cdot (mg_1,h_1,\ldots,h_e)$ is isomorphic to
$(E_{R_n}(K))^{\lambda_{i,n-j}(A)-1}$. 

Reduction of the $g_i$ with respect to a Gr\"obner basis of the new
relation module and repetition will lead to
the determination of $\lambda_{i,n-j}(A)$.
%

Assume that $D_n/L$ is an $\m$-torsion module. For example, we could
have $D_n/L\cong H^i_\m(H^j_I(R_n))$.  
Here is a procedure that finds by trial and error 
the monomial socle element $m$ of Step
3 in Algorithm \ref{alg-lclc}.
\beginOutput
i61 : findSocle = method();\\
\endOutput
\beginOutput
i62 : findSocle(Ideal, RingElement):= (L,P) -> (\\
\           createDpairs(ring(L));\\
\           v=(ring L).dpairVars#0;\\
\           myflag = true;\\
\           while myflag do (\\
\                w = apply(v,temp -> temp*P {\char`\%} L);\\
\                if all(w,temp -> temp == 0) then myflag = false\\
\                else (\\
\                     p = position(w, temp -> temp != 0);\\
\                     P = v#p * P;)\\
\                );\\
\           P);\\
\endOutput
For example, if we want to apply this socle search to the ideal
{\tt JH3} describing $H^3_I(R_6)$ of Example \ref{ex-minors} we do
\beginOutput
i63 : D = ring JH3\\
\emptyLine
o63 = D\\
\emptyLine
o63 : PolynomialRing\\
\endOutput
(as {\tt D} was most recently the differential operators on $\Q[x,y,z,w]$)
\beginOutput
i64 : findSocle(JH3,1_D)\\
\emptyLine
o64 = x*v\\
\emptyLine
o64 : D\\
\endOutput
One can then repeat the socle search and kill the newly found element
as suggested in the explanation above:
\beginOutput
i65 : findLength = method();\\
\endOutput
\beginOutput
i66 : findLength Ideal := (I) -> (   \\
\           l = 0;\\
\           while I != ideal 1_(ring I) do (\\
\                l = l + 1;\\
\                s = findSocle(I,1_(ring I));\\
\                I = I + ideal s;);\\
\           l);\\
\endOutput
Applied to {\tt JH3} of the previous subsection this yields
\beginOutput
i67 : findLength JH3\\
\emptyLine
o67 = 1\\
\endOutput
and hence {\tt JH3} does indeed describe a module isomorphic to $E_{R_6}(K)$.
\nocite{DM:W1}

%\input{6.tex}
\section{Implementation, Examples, Questions}
\mylabel{sec-ausblick}
\subsection{Implementations and Optimizing}
The Algorithms \ref{alg-ann-fs}, \ref{alg-b-poly-L} and \ref{alg-D/L-loc-f} 
have first been implemented by
T.\ Oaku
  and N.\ Takayama 
using the package Kan \cite{DM:T} which
is a postscript language for computations in the Weyl algebra and in
polynomial rings. 
In \Mtwo Algorithms  \ref{alg-ann-fs},
\ref{alg-b-poly-L} and \ref{alg-D/L-loc-f} as well as Algorithm 
\ref{alg-lc}
have been implemented by A.~Leykin, M.~Stillman and H.~Tsai. They additionally
implemented a wealth of $D$-module routines that
relate to topics which we cannot all cover in this chapter. These include
homomorphisms between holonomic modules and extension functors,
restriction functors to linear subspaces, integration (de Rham)
functors to quotient spaces and others. For further theoretical 
information 
the reader is referred to \cite{DM:O-T1,DM:O-T2,DM:O-T-T,DM:SST,DM:T-W,DM:W2,DM:W4,DM:W3}. 

\mylabel{efficiency}
Computation of Gr\"obner bases in many variables is in general a time
and space consuming enterprise. In commutative polynomial
rings the worst case performance for the number of elements in reduced
Gr\"obner bases 
is doubly exponential in the number of variables and the degrees of
the generators. In the (relatively) small 
Example \ref{ex-minors} above $R_6$ is of dimension 6,
so that the intermediate ring $D_{n+1}[y_1,y_2]$ contains 16
variables. In view of these facts the following idea 
has proved useful. 

The general context in which Lemma \ref{lem-malgrange} and Proposition
\ref{prop-kashiwara} were stated allows successive localization of
$R_n[(fg)^{-1}]$ 
in the following way. First one computes $R_n[f^{-1}]$ according to
Algorithm \ref{alg-D/L-loc-f} as quotient $D_n/J^\Delta(f^s)|_{s=a}$,
$\Z\ni a\ll 0$.  
Then $R_n[(fg)^{-1}]$ may be
computed using  Algorithm 
\ref{alg-D/L-loc-f} again since $R_n[(fg)^{-1}]\cong
R_n[g^{-1}]\otimes_{R_n} 
D_n/J^\Delta(f^s)|_{s=a}$. (Note that all
localizations of $R_n$ are automatically $f$-torsion free for $f\in R_n$
so that Algorithm \ref{alg-D/L-loc-f} can be used.) This process
may be iterated for products with any finite number of factors. 
Of course the exponents for the various factors might be different. 
This requires some care 
%as the following situations illustrate. Assume
%first that $-1$ is the smallest integer root of the Bernstein The
%Bernstein polynomials
%of $f$ and $g$ (both in $R_n$) with respect to the holonomic module
%$R_n$. Assume further that $R_n[(fg)^{-1}]\cong D_n\action
%( f^{-2}g^{-1})
%\supsetneq
%D_n\action( (fg)^{-1})$. Then $R_n[f^{-1}]\to R_n[(fg)^{-1}]$ can be written as
%$D_n/\ann(f^{-1})\to D_n/\ann(f^{-2}\cdot g^{-1})$ sending $P\in D_n$ to
%$P\cdot f\cdot g$. 
%
%Suppose on the other hand that we are interested in $H^2_I(R_n)$ where
%$I=(f,g,h)$ and we know that $R_n[f^{-1}]=D_n\action
%( f^{-2})\supsetneq 
%D_n\action(
%f^{-1}), R_n[g^{-1}]=D_n\action( g^{-2})$ and $R_n[(fg)^{-1}]=D_n\action( f^{-1}g^{-2})$. (In
%fact, the degree 2 part of the \v Cech complex of Example
%\ref{ex-minors} consists of such localizations.) We cannot write
%the embedding $R_n[f^{-1}]\to R_n[(fg)^{-1}]$ with the use of a
%Bernstein operator for $s=-2$ since $f^{-1}$ is not a
%generator for $R_n[f^{-1}]$. So we must
%write $R_n[(fg)^{-1}]$ as $D_n/\ann((fg)^{-2})$ and then send $P\in
%\ann(f^{-2})$ to $P\cdot g^2$. 
%
%The two examples indicate how to write the \v Cech complex in terms
%of generators and relations over $D_n$ while making 
%
%
when setting up the \v Cech complex. In particular one needs to make 
sure that the maps
$\check C^k\to \check C^{k+1}$ can be made explicit using the $f_i$.
% -- the exponents
%used in $C^k$ have to be at least as big as those in $C^{k-1}$ (for the
%same $f_i$). 
(In our Example \ref{ex-minors}, this is precisely how we proceeded
when we found {\tt Rfgh}.)
\begin{remark}
\label{remark}
One might hope that for all holonomic $fg$-torsion free 
modules $M=D_n/L$ 
we have  (with $M\otimes R_n[g^{-1}]\cong D_n/L'$):
\begin{eqnarray}
\label{eq-ineq}
a^L_f=\min\{s\in\Z:b_f^L(s)=0\}\le \min\{s\in\Z:b_f^{L'}(s)=0\}=a^{L'}_f.
\end{eqnarray}
This hope is unfounded. Let $R_5=K[x_1,\ldots,x_5]$,
$f=x_1^2+x_2^2+x_3^2+x_4^2+x_5^2$. One may check that then
$b^\Delta_f(s)=(s+1)(s+5/2)$. Hence $R_5[f^{-1}]=D_5\action f^{-1}$, let
$L=\ker(D_5\to D_5\action f^{-1})$. Set $g=x_1$. Then
$b^\Delta_g(s)=s+1$, let $L'=\ker(D_5\to D_5\action g^{-1})$.

Then
$b^{L'}_f(s)=(s+1)(s+2)(s+5/2)$ and $b^L_g(s)=(s+1)(s+3)$ because of 
the following computations.
\beginOutput
i68 : erase symbol x; erase symbol Dx;\\
\endOutput
These two commands essentially clear the history of the variables {\tt
x} and {\tt Dx} and make them available for future computations.
\beginOutput
i70 : D = QQ[x_1..x_5, Dx_1..Dx_5, WeylAlgebra =>\\
\           apply(toList(1..5), i -> x_i => Dx_i)];\\
\endOutput
\beginOutput
i71 : f = x_1^2 + x_2^2 + x_3^2 + x_4^2 +x_5^2;\\
\endOutput
\beginOutput
i72 : g = x_1;\\
\endOutput
\beginOutput
i73 : R = D^1/ideal(Dx_1,Dx_2,Dx_3,Dx_4,Dx_5);\\
\endOutput
As usual, these commands defined the base ring, two polynomials and
the $D_5$-module $R_5$. Now we compute the respective localizations.
\beginOutput
i74 : Rf =DlocalizeAll(R,f,Strategy => Oaku);\\
\endOutput
\beginOutput
i75 : Bf = Rf.Bfunction\\
\emptyLine
\            5\\
o75 = (\$s + -)(\$s + 1)\\
\            2\\
\emptyLine
o75 : Product\\
\endOutput
\beginOutput
i76 : Rfg = DlocalizeAll(Rf.LocModule,g,Strategy => Oaku);\\
\endOutput
\beginOutput
i77 : Bfg = Rfg.Bfunction\\
\emptyLine
o77 = (\$s + 1)(\$s + 3)\\
\emptyLine
o77 : Product\\
\endOutput
\beginOutput
i78 : Rg = DlocalizeAll(R,g,Strategy => Oaku);\\
\endOutput
\beginOutput
i79 : Bg = Rg.Bfunction\\
\emptyLine
o79 = (\$s + 1)\\
\emptyLine
o79 : Product\\
\endOutput
\beginOutput
i80 : Rgf = DlocalizeAll(Rg.LocModule,f,Strategy => Oaku);\\
\endOutput
\beginOutput
i81 : Bgf = Rgf.Bfunction\\
\emptyLine
\                            5\\
o81 = (\$s + 2)(\$s + 1)(\$s + -)\\
\                            2\\
\emptyLine
o81 : Product\\
\endOutput

%\begin{verbatim}
%i20 : D = QQ[x_1..x_5, Dx_1..Dx_5, WeylAlgebra => 
%    apply(toList(1..5), i -> x_i => Dx_i)]
%i21 : f = x_1^2 + x_2^2 + x_3^2 + x_4^2 +x_5^2
%i22 : g = x_1
%i23 : R = D^1/ideal(Dx_1,Dx_2,Dx_3,Dx_4,Dx_5)
%i24 : M1 =DlocalizeAll(R,f,Strategy => Oaku)
%i25 : M12 = DlocalizeAll(M1.LocModule,g,Strategy => Oaku)
%i26 : M2 = DlocalizeAll(R,g,Strategy => Oaku)
%i27 : M21 = DlocalizeAll(M2.LocModule,f,Strategy => Oaku)
%\end{verbatim}
The output shows
that $R_n[(fg)^{-1}]$ is generated by $f^{-2}g^{-1}$ or $f^{-1}g^{-3}$ but not
by $f^{-1}g^{-2}$ and in particular not by $f^{-1}g^{-1}$. 
This can be seen from the various Bernstein-Sato polynomials: as for
example the smallest integral root of {\tt Bf} is $-1$ and that of
{\tt Bfg} is $-3$, $R_3[f^{-1}]$ is generated by $f^{-1}$ and
$R_3[(fg)^{-1}]$ by $f^{-1}g^{-3}$.
This example not only disproves the above inequality (\ref{eq-ineq})
but also shows
the inequality to be wrong if $\Z$ is replaced by $\R$ (as
$-3<\min(-5/2,-1)$). 
\end{remark}


Nonetheless, localizing $R_n[(fg)^{-1}]$ as $(R_n[f^{-1}])[g^{-1}]$ is
heuristically advantageous, apparently for two reasons. For 
one, it allows the exponents of the various factors to be distinct
which is useful for the subsequent cohomology computation: it helps
to keep the degrees of the maps small. So in
Example \ref{ex-minors} we can write $R_6[(fg)^{-1}]$ as $D_6\action 
(f^{-1} g^{-2})$ instead
of $D_6\action (fg)^{-2}$. 
Secondly,  
since the computation of Gr\"obner bases is potentially 
doubly exponential it
seems to be advantageous to break a big problem (localization at a
product) into several ``easy'' problems (successive localization).

An interesting case of this behavior is our Example \ref{ex-minors}. If we
compute $R_n[(fgh)^{-1}]$ as $((R_n[f^{-1}])[g^{-1}])[h^{-1}]$, 
the calculation uses
approximately 6MB and lasts a few seconds 
using \Mtwo. If one tries to localize $R_n$ at the
product of the three generators at once, \Mtwo 
runs out of memory on all machines the author has 
tried this computation on.
\subsection{Projects for the Future}
This is a list of theoretical and implementational questions that the
author finds 
important and interesting.
\subsubsection{Prime Characteristic} In \cite{DM:L-Fmod},
G.~Lyubeznik gave an
algorithm for deciding whether or not 
$H^i_I(R)=0$ for any given  $I\subseteq R=K[x_1,\ldots,x_n]$ where
$K$ is a computable field of positive characteristic. His algorithm is built on
entirely different methods than the ones used in this chapter 
and relies on the Frobenius functor. The
implementation of this algorithm would be quite worthwhile.
\subsubsection{Ambient Spaces Different from ${\mathbb A}^n_K$}
\mylabel{singular-spaces}
If $A$ equals $K[x_1,\ldots,x_n]$, 
$I\subseteq A$, $X=\spec (A)$ and $Y=\spec(A/I)$,
knowledge of $H^i_I(A)$ for all $i\in \N$ answers of course the
question about the local cohomological dimension of $Y$ in $X$. 
If $W\subseteq X$ is a smooth variety containing $Y$
then Algorithm \ref{alg-lc} for the computation of $H^i_I(A)$ also
leads to a determination of the local cohomological dimension of $Y$
in $W$. Namely, if $J$ stands for the
defining ideal of $W$ in $X$ so that $R=A/J$ is the affine
coordinate ring of $W$  and if we set $c=\height(J)$, then it can be
shown that 
$H^{i-c}_{I}(R)=\hom_A(R,H^i_I(A))$ for all $i\in\N$.
As $H^i_I(A)$ is 
$I$-torsion (and hence $J$-torsion), $\hom_A(R,H^i_I(A))$ is zero if
and only if 
$H^i_I(A)=0$. It follows that the local cohomological dimension of $Y$
in $W$ equals $\cd(A,I)-c$ and in fact $\{i\in \N:H^i_I(A)\not =0\}=\{i\in
\N:H^{i-c}_I(R)\not =0\}$. 

If however $W=\spec(R)$ is not smooth, no algorithms for the computation of
either $H^i_I(R)$ or $\cd(R,I)$ are known, irrespective of the
characteristic of the base field. It would be very interesting to have
even partial ideas for computations in that case.
\subsubsection{De Rham Cohomology} In \cite{DM:O-T1,DM:W2} algorithms are given
 to compute de Rham (in this case equal to singular) 
cohomology of complements of complex affine
hypersurfaces and more general varieties. In \cite{DM:W3} an algorithm
is given to compute the multiplicative (cup product) structure, and in
\cite{DM:W4} the computation of the de Rham cohomology of open and
closed sets in projective
space is explained. Some of these
algorithms have been implemented while others are still waiting. 

For
example, de Rham cohomology of complements of hypersurfaces, and
partially the cup product routine,  are implemented.
\begin{example}
Let $f=x^3+y^3+z^3$ in $R_3$. One can compute with \Mtwo
the de Rham cohomology of the complement of $\var(f)$, and 
it turns out that the
cohomology in degrees 0 and 1 is 1-dimensional, in degrees 3 and 4
2-dimensional and zero otherwise -- here is the input:
\beginOutput
i82 : erase symbol x;\\
\endOutput
Once {\tt x} gets used as a subscripted
variable, it's hard to use it as a nonsubscripted variable.  So let's just
erase it.
\beginOutput
i83 : R = QQ[x,y,z];\\
\endOutput
\beginOutput
i84 : f=x^3+y^3+z^3;\\
\endOutput
\beginOutput
i85 : H=deRhamAll(f);\\
\endOutput
%\begin{verbatim}
%i28 ; R = QQ[x,y,z]
%i29 ; f=x^3+y^3+z^3
%i30 : deRhamAll(f)
%\end{verbatim}
{\tt H} is a hashtable with the entries 
{\tt Bfunction}, 
{\tt LocalizeMap}, 
{\tt VResolution}, 
{\tt TransferCycles}, 
{\tt PreCycles}, 
{\tt OmegaRes} and 
{\tt CohomologyGroups}. For example,
we have
\beginOutput
i86 : H.CohomologyGroups\\
\emptyLine
\                       1\\
o86 = HashTable\{0 => QQ \}\\
\                       1\\
\                1 => QQ\\
\                       2\\
\                2 => QQ\\
\                       2\\
\                3 => QQ\\
\emptyLine
o86 : HashTable\\
\endOutput
showing that the dimensions are as claimed above. One can also extract
information on the generator of $R_3[f^{-1}]$ used to represent the
cohomology classes by
\beginOutput
i87 : H.LocalizeMap\\
\emptyLine
o87 = | \$x_1^6+2\$x_1^3\$x_2^3+\$x_2^6+2\$x_1^3\$x_3^3+2\$x_2^3\$x_3^3+\$x_3^6 |\\
\emptyLine
o87 : Matrix\\
\endOutput
which proves that the generator in question is $f^{-2}$. The cohomology
classes that \Mtwo computes are differential forms:
\beginOutput
i88 : H.TransferCycles\\
\emptyLine
o88 = HashTable\{0 => | -1/12\$x_1^4\$x_2^3\$D_1-1/3\$x_1\$x_2^6\$D_1-1/12\$x_ $\cdot\cdot\cdot$\\
\                1 => | 2/3\$x_1^5+2/3\$x_1^2\$x_2^3+2/3\$x_1^2\$x_3^3  |\\
\                     | -2/3\$x_1^3\$x_2^2-2/3\$x_2^5-2/3\$x_2^2\$x_3^3 |\\
\                     | 2/3\$x_1^3\$x_3^2+2/3\$x_2^3\$x_3^2+2/3\$x_3^5  |\\
\                2 => | 48\$x_1\$x_2\$x_3^2 600\$x_3^4     |\\
\                     | 48\$x_1\$x_2^2\$x_3 600\$x_2\$x_3^3 |\\
\                     | 48\$x_1^2\$x_2\$x_3 600\$x_1\$x_3^3 |\\
\                3 => | -\$x_1\$x_2\$x_3 -\$x_3^3 |\\
\emptyLine
o88 : HashTable\\
\endOutput
So, for example, the left column of the three rows that correspond to
$H^2_{{\rm dR}}(\C^3\setminus\var(f),\C)$ represent the form
$({xyz(48zdxdy+48ydzdx+48xdydz)})/{f^2}$. 
If we apply the above elements of $D_3$ to $f^{-2}$ and equip the
results with appropriate differentials one arrives (after dropping
unnecessary integral factors) at 
the results displayed in the table in Figure \ref{Figure}.
\begin{figure}
\[
\def\sp#1{\vcenter{\vskip 2pt \hbox{$#1$}\vskip 2pt}}
\begin{array}{|c|c|c|}
\hline
\text{Group}&\text{Dimension}&\text{Generators}\\
\hline
\hline
H^0_{{\rm dR}}&1&\sp{e:=\frac{\displaystyle f^2}{\displaystyle f^2}}\\
\hline
H^1_{{\rm dR}}&1&\sp{o:=\frac{\displaystyle (x^2dx-y^2dy+z^2dz)f}{\displaystyle f^2}}\\
\hline
H^2_{{\rm dR}}&2&\sp{t_1:=\frac{\displaystyle xyz(zdxdy+ydzdx+xdydz)}{\displaystyle f^2}}\\
        & &\sp{t_2:=\frac{\displaystyle (zdxdy+ydzdx+xdydz)z^3}{\displaystyle f^2}}\\
\hline
H^3_{{\rm dR}}&2&\sp{d_1:=\frac{\displaystyle xyzdxdydz}{\displaystyle f^2}}\\
        & &\sp{d_2:=\frac{\displaystyle z^3dxdydz}{\displaystyle f^2}}\\
\hline
\end{array}
\]
\caption{}\label{Figure}
\end{figure}
In terms of de Rham cohomology, the cup product is given by the wedge
product of differential forms. So 
from the table one reads off the cup  product
relations
$o\cup t_1=d_1$, $o\cup t_2=d_2$, $o\cup
o=0$, and that $e$ operates as the identity. 
\end{example}
For more involved examples, and the algorithms in \cite{DM:W4}, an
actual implementation would be  necessary since paper and pen are
insufficient tools then.
\begin{remark}
The reader should be warned: if $f^{-a}$ generates $R_n[f^{-1}]$ over
$D_n$, then it is not necessarily the case that each de Rham
cohomology class of $U=\C^n\setminus\var(f)$ can be written as a form
with a pole of order at most $a$. A counterexample is given by
$f=(x^3+y^3+xy)xy$, where $H^1_{{\rm dR}}(U;\C)$ has a class that requires a
third order pole, although $-2$ is the smallest integral Bernstein root
of $f$ on $R_2$.
\end{remark}

\subsubsection{Hom and Ext}
In \cite{DM:O-T-T,DM:Ts0,DM:T-W} algorithms are explained that compute homomorphisms
between holonomic systems. In particular, rational and polynomial
solutions can be found because, for example, a polynomial solution to the system
$\{P_1,\ldots,P_r\}\in D_n$ corresponds to an element of
$\hom_{D_n}(D_n/I,R_n)$ where $I=D_n\cdot(P_1,\ldots,P_r)$.
\begin{example}
Consider the GKZ system in 2 variables associated to the matrix
$(1,2)\in \Z^{1\times 2}$ and the parameter vector $(5)\in\C^1$. 
Named after
Gelfand-Kapranov-Zelevinski \cite{DM:GKZ}, this 
is the following system of
differential equations:
\begin{eqnarray*}
(x\del_x+y\del_y)\action f&=&5f,\\
(\del_x^2-\del_y)\action f&=&0.
\end{eqnarray*}
With \Mtwo one can solve systems of this sort as follows:
\beginOutput
i89 : I = gkz(matrix\{\{1,2\}\}, \{5\})\\
\emptyLine
\              2\\
o89 = ideal (D  - D , x D  + 2x D  - 5)\\
\              1    2   1 1     2 2\\
\emptyLine
o89 : Ideal of QQ [x , x , D , D , WeylAlgebra => \{x  => D , x  => D \}]\\
\                    1   2   1   2                   1     1   2     2\\
\endOutput
This is a simple command to set up the GKZ-ideal associated to a
matrix and a parameter vector. The polynomial solutions are obtained
by
\beginOutput
i90 : PolySols I\\
\emptyLine
\        5      3          2\\
o90 = \{x  + 20x x  + 60x x \}\\
\        1      1 2      1 2\\
\emptyLine
o90 : List\\
\endOutput

This means that there is exactly one polynomial solution to the given
GKZ-system, and it is 
\[
x^5+20x^3y+60xy^2.
\]
\end{example}
The algorithm for $\hom_{D_n}(M,N)$ is implemented and
can be used to check whether two given $D$-modules are
isomorphic. 
Moreover, there are algorithms (not implemented yet) 
to compute the ring structure of
$\endo_D(M)$ for a given $D$-module $M$ of finite holonomic
rank which can be used to split a given holonomic module into its
direct summands. 
Perhaps an adaptation of these methods can be used to construct
Jordan-H\"older sequences for holonomic $D$-modules.

\subsubsection{Finiteness and Stratifications}
Lyubeznik pointed out in \cite{DM:L-Bpoly} the following curious
fact. 
\begin{theorem}
Let $P(n,d;K)$ denote the set of polynomials of degree at most $d$ in
at most $n$ variables over the field $K$ of characteristic zero. Let
$B(n,d;K)$ denote the set  of Bernstein-Sato polynomials
\[
B(n,d;K)=\{b_f(s):f\in P(n,d;K)\}.
\]
Then $B(n,d;K)$ is finite.\qed
\end{theorem}
So $P(n,d;K)$ has a finite decomposition into strata with constant
Bernstein-Sato polynomial.
A.\ Leykin proved in \cite{DM:Ley} that this decomposition is independent
of $K$ and computable
in the sense that membership in  each stratum can be tested by the
vanishing of a
finite set of algorithmically computable polynomials over $\Q$ in the
coefficients of the given polynomial in $P(n,d;K)$. In particular,
the stratification is algebraic and for each $K$ induced by base
change from $\Q$ to $K$. It makes thus sense to define $B(n,d)$ which
is the finite set of Bernstein polynomials that can occur for $f\in
P(n,d;K)$ (where $K$ is in fact irrelevant).
\begin{example}
Consider $P(2,2;K)$, the set of quadratic binary forms over $K$. With
\Mtwo, Leykin showed that there are
precisely 4 different Bernstein polynomials possible:
\def\labelitemi{$\bullet$}
\begin{itemize}
\item $b_{f}(s)=1$
iff $f\in V_{1}=V_{1}'\setminus V_{1}''$,
where $V_{1}'=V(a_{1,1},a_{0,1},a_{0,2},a_{1,0},a_{2,0}) $, while
$V_{1}''=V(a_{0,0})$
\item $ b_{f}(s)=s+1 $
iff $ f\in V_{2}=(V_{2}'\setminus V_{2}'')\cup (V_{3}'\setminus V_{3}'') $,
where $ V_{2}'=V(0) $, $ V_{2}''=V(\gamma _{1}) $,
$V_{3}'=V\left( \gamma _{2},\gamma _{3},\gamma _{4}\right)$, 
$V_{3}''=V\left( \gamma _{3},\gamma _{4},\gamma _{5},\gamma _{6},\gamma _{7},\gamma _{8}\right)  $,
\item $ b_{f}(s)=(s+1)^{2} $
iff $ f\in V_{4}'\setminus V_{4}'' $,
where $ V_{4}'=V(\gamma _{1}) $, 
$ V_{4}''=V\left( \gamma _{2},\gamma _{3},\gamma _{4}\right)  $, 
\item $ b_{f}(s)=(s+1)(s+\frac{1}{2}) $
iff $ f\in V_{5}'\setminus V_{5}'' $,
where $ V_{5}'=V\left( \gamma _{3},\gamma _{4},\gamma _{5},\gamma _{6},\gamma _{7},\gamma _{8}\right)  $,
$ V_{5}''=V(a_{1,1},a_{0,1},a_{0,2},a_{1,0},a_{2,0}) $.
\end{itemize}
Here we have used the abbreviations
\begin{itemize}
\item $ \gamma_{1}=
a_{0,2}a_{1,0}^{2}-a_{0,1}a_{1,0}a_{1,1}+a_{0,0}a_{1,1}^{2}+ 
a_{0,1}^{2}a_{2,0}-4a_{0,0}a_{0,2}a_{2,0}$, 
\item$ \gamma _{2}=2a_{0,2}a_{1,0}-a_{0,1}a_{1,1} $,

\item $ \gamma _{3}=a_{1,0}a_{1,1}-2a_{0,1}a_{2,0} $, 

\item $ \gamma _{4}=a_{1,1}^{2}-4a_{0,2}a_{2,0} $,

\item $ \gamma _{5}=2a_{0,2}a_{1,0}-a_{0,1}a_{1,1} $,

\item $ \gamma _{6}=a_{0,1}^{2}-4a_{0,0}a_{0,2} $,

\item $ \gamma _{7}=a_{0,1}a_{1,0}-2a_{0,0}a_{1,1} $, 

\item $ \gamma _{8}=a_{1,0}^{2}-4a_{0,0}a_{2,0} $.
\end{itemize}
Similarly, Leykin shows that there are 9 possible Bernstein
polynomials for $f\in B(2,3;K)$:
\begin{eqnarray*}
B(2,3)&=&\left\{\,
 (s+1)^{2}(s+\frac{2}{3})(s+\frac{4}{3}),\,\,\,(s+1)^{2}(s+\frac{1}{2}),\,\,\,(s+1),\,\,\,1,\right.\\ 
 &  &\phantom{x}(s+1)(s+\frac{2}{3})(s+\frac{1}{3}),\,\,\,(s+1)^{2},\,\,\,(s+1)(s+\frac{1}{2}),\\
 &  &\phantom{x}\left.(s+1)(s+\frac{7}{6})(s+\frac{5}{6}),\,\,\,(s+1)^{2}(s+\frac{3}{4})(s+\frac{5}{4})\right\}.
\end{eqnarray*}
\end{example}
It would be very interesting to study the nature of the stratification
in larger cases, and its restriction to hyperplane arrangements.

A generalization of this stratification result is obtained in
\cite{DM:W5}. There it is shown that there is an algorithm to give
$P(n,d;K)$ an algebraic
stratification defined over $\Q$ 
such that the algebraic de Rham cohomology groups of the
complement of $\var(f)$ do not vary on the stratum in a rather strong
sense. 
Again, the study
and explicit computation of this stratification should be very
interesting.

\subsubsection{Hodge Numbers}
If $Y$ is a projective variety in $\P^n_\C$ then algorithms outlined in \cite{DM:W4}
show how to compute the dimensions not only of the de Rham cohomology
groups of $\P^n_\C\setminus Y$ but also of $Y$ itself. 
Suppose now that $Y$ is in fact a smooth  projective variety.  
An interesting set of invariants are the Hodge numbers, defined by
$h^{p,q}=\dim H^p(Y,\Omega^q)$, where $\Omega^q$ denotes the sheaf of
$\C$-linear differential $q$-forms with coefficients in $\OO_Y$. 
At present we do not know how to
compute them. Of course there is a spectral sequence $
H^p(Y,\Omega^q)\Rightarrow H^{p+q}_{{\rm dR}}(Y,\C)$ and we know the abutment
(or at least its dimensions), but the technique for computing the
abutment does not seem to be usable 
to compute the $E^1$ term because on an affine
patch $H^p(Y,\Omega^q)$ is either zero or an infinite dimensional
vector space.

Hodge structures and Bernstein-Sato polynomials are related as is for
example shown in \cite{DM:Varchenko}. 

\subsection{Epilogue}
In this chapter we have only touched a few highlights of the theory of
computations in $D$-modules, most of them related to homology and
topology. Despite this we hardly touched on the topics of integration
and restriction, which are the $D$-module versions of a pushforward
and pullback, \cite{DM:K2,DM:Mebkhout,DM:O-T1,DM:W2}. 

A very different aspect of $D$-modules is discussed in
\cite{DM:SST} where at the center  of investigations is the combinatorics of
solutions of hypergeometric differential equations. The combinatorial
structure is used to find series solutions for the differential
equations which are polynomial in certain logarithmic functions and 
power series with respect to the variables.

Combinatorial elements can also be found in the work of Assi, Castro
and Granger, see \cite{DM:ACG1,DM:ACG2}, 
on Gr\"obner fans in rings of differential
operators. An important (open) question in this direction is the
determination of the set of ideals in $D_n$ that are initial ideals
under {\em some} weight.



Algorithmic $D$-module theory promises to be an active
area of research for many years to come, and to have interesting
applications to various other parts of mathematics.


%\begin{example}
%In the case $n=d=2$ one has clearly 2 cases of the de Rham cohomology ring
%$DR(f)$ in $P_h(n,d;K)$:
%\begin{itemize}
%\item $DR(f)=K\oplus K\cdot e$ with $e^2=0$ iff $4a_{2,0}a_{02}=a_{1,1}^2$,
%\item $DR(f)=K\oplus K\cdot e_1\oplus K\cdot e_2$ with
%$e_1^2=e_2^2=e_1e_2=0$ iff $4a_{20}a_{0,2}\not =a_{1,1}^2$.
%\end{itemize}
%\end{example}


% Local Variables:
% mode: latex
% mode: reftex
% tex-main-file: "chapter-wrapper.tex"
% reftex-keep-temporary-buffers: t
% reftex-use-external-file-finders: t
% reftex-external-file-finders: (("tex" . "make FILE=%f find-tex") ("bib" . "make FILE=%f find-bib"))
% End:
